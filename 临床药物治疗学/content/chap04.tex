\chapter{药物相互作用}

\section{概述}

药物相互作用(Drug-Drug
Interaction,DDI)是指同时或相继使用两种或两种以上药物时,由于药物之间的相互影响而导致其中一种或几种药物作用的强弱、持续时间甚至性质发生不同程度改变的现象。

药物相互作用有广义和狭义之分。广义药物相互作用是指联合用药时所发生的疗效变化。疗效变化虽然有多种多样表现,但结果只有两种可能,即作用加强或作用减弱。从临床角度考虑,作用加强可表现为疗效提高,也可表现为毒性加大;作用减弱可表现为疗效降低,也可表现为毒性减轻。虽然多药联用的情况非常普遍,但药物相互作用常常只在对患者造成有害影响时才引起充分注意。狭义的药物相互作用通常是指两种或两种以上药物同时或相继使用时产生的不良影响,可以是药效降低甚至治疗失败,也可以是毒性增加,这种不良影响是单一药物应用时所没有的。

一个典型的药物相互作用对(interaction
pair)由两个药物组成:药效发生变化的药物称为目标药(object drug或index
drug),引起这种变化的药物称为相互作用药或促发药(interacting
drug或precipitating
drug)。一个药物可以在某一相互作用对中是目标药(如苯妥英钠-西咪替丁),而在另一相互作用对中是相互作用药(如多西环素-苯妥英钠)。

\subsection{按发生机制分类}

\subsubsection{体外药物相互作用}

体外药物相互作用是指在患者用药之前(即药物尚未进入机体以前),药物相互间发生化学或物理性相互作用,使药性发生变化。即一般所称化学配伍禁忌或物理配伍禁忌,故又称之为物理化学性相互作用。

\subsubsection{药动学相互作用}

药物在其吸收、分布、代谢和排泄过程的任一环节发生相互作用,均可影响药物在血浆或其作用靶位的浓度,最终使其药效或不良反应发生相应改变。

\subsubsection{药效学相互作用}

两种或两种以上的药物作用于同一受体或不同受体,产生疗效的协同、相加或拮抗作用,而对药物的血浆或作用靶位的浓度可无明显影响。

应当注意的是,有时药物相互作用的产生可以是几种机制并存。

\subsection{按严重程度分类}

\subsubsection{轻度药物相互作用}

造成的影响临床意义不大,无须改变治疗方案。如对乙酰氨基酚能减弱呋塞米的利尿作用,但并不会显著影响临床疗效,也无须改变剂量。

\subsubsection{中度药物相互作用}

药物联用虽会造成确切的不良后果,但临床上仍会在密切观察下使用。如异烟肼与利福平合用,利福平是肝药酶诱导剂,会促进异烟肼转化为具有肝毒性的代谢物乙酰异烟肼,而利福平本身也有肝功能损害作用,两者合用会增强肝毒性作用,但两药联用对结核杆菌有协同抗菌作用,所以这一联合用药对肝功能正常的结核病患者仍是首选用药方案之一,但在治疗过程中应定期检查肝功能。

\subsubsection{重度药物相互作用}

药物联用会造成严重的毒性反应,需要重新选择药物,或须改变用药剂量及给药方案。如特非那定与许多药物(大环内酯类、咪唑类、H{2}
受体阻断药、口服避孕药等)合用时代谢过程受阻,其原形对心脏毒性较大,可致患者室性心动过速而死亡。骨骼肌松弛药与氨基糖苷类抗生素庆大霉素等合用,可能增强及延长骨骼肌松弛作用,甚至引起呼吸肌麻痹。

此外,按药物相互作用发生的概率大小可分为:肯定、很可能、可能、可疑、不可能等几个等级。这主要是根据已发表的临床研究或体外研究、病例报告、临床前研究等文献结果进行判断。按发生的时间过程,有的药物相互作用可立即发生,如四环素类抗生素与含钙、铝、镁的抗酸药发生络合反应,可使四环素的吸收立即下降。另一些药物相互作用的影响可能需要数小时或几天后才表现出来,如华法林的抗凝作用可被合用的维生素K逐渐减弱。

\section{体外药物相互作用}

体外药物相互作用是指在患者用药之前(即药物尚未进入机体以前),药物相互间发生化学或物理性相互作用,使药性发生变化。即一般所称化学配伍禁忌或物理配伍禁忌。

\subsection{分类}

\subsubsection{可见配伍变化}

包括溶液混浊、产气、沉淀、结晶及变色。可见配伍变化,应在混合后仔细观察,大多数是可以避免的。有些可见配伍变化不是立即发生的,而是在使用过程中逐渐出现的,更应该引起足够重视。如20%磺胺嘧啶钠注射液(pH值为9.5~11)加入10%的葡萄糖注射液(pH值为3.2~5.5)中,由于pH值的改变,可使磺胺嘧啶微结晶析出,这种结晶输入血管可造成栓塞。

\subsubsection{不可见配伍变化}

包括水解反应、效价下降、聚合变化及肉眼不能直接观察到的直径50μm以下的微粒等,潜在的影响药物对人体的安全性和有效性。如在氨基酸注射液中不能加入对酸不稳定的药物,因为该类药物在氨基酸营养液中容易降解;维生素C(pH值为5.8~6.9)与偏碱性的氨茶碱(pH值为9.0~9.5)溶液混合时,外观无变化,但效价降低。

\subsection{常见注射剂配伍变化产生的原因}

\subsubsection{沉淀}
\paragraph{注射液溶媒组成改变}

因改变溶媒的性质而析出沉淀。某些注射剂内含非水溶剂,目的是使药物溶解或制剂稳定,若把这类药物加入水溶液中,由于溶媒性质的改变而析出药物产生沉淀。如氯霉素注射液(含乙醇、甘油等)加入5%葡萄糖注射液或0.9%氯化钠注射液中,可析出氯霉素沉淀。
\paragraph{电解质的盐析作用}

主要是对亲水胶体或蛋白质药物自液体中被脱水或因电解质的影响而凝集析出。如氟罗沙星注射剂与0.9%氯化钠注射液合用可发生盐析作用而出现沉淀。
\paragraph{pH值改变}

pH值发生改变时,药物的溶解性也会发生改变,会导致药物的析出。5%硫喷妥钠10mL加入5%葡萄糖注射液500mL中,由于溶液pH值下降导致产生沉淀。
\paragraph{形成配合物}

如米诺环素与\ce{Ca^2+} 、\ce{Mg^2+} 等金属离子形成难溶性配合物而析出沉淀。

\subsubsection{变色}

出现新的颜色,或原有颜色消失。酚类化合物、水杨酸及其衍生物以及含酚羟基的药物如肾上腺素与铁盐发生配合反应,或受空气氧化,都能产生有色物质。

\subsubsection{产气}

碳酸盐、碳酸氢盐与酸类药物配伍,铵盐与碱类药物配伍,均可产生气体。

\subsubsection{效价下降}

某些药物在水溶液中不稳定,易分解失效,与其他药物合用,可加速分解,致药物活性下降。如氨苄西林在含乳酸根的复方氯化钠注射液中,由于乳酸根可加速氨苄西林的水解,4h效价损失20%。

\subsubsection{聚合反应}

氨苄西林1%({w/v}
)的储备液在放置期间,会发生变色、溶液变黏稠、形成沉淀,这是由于形成聚合物所致。

\subsection{注射剂配伍变化的预测}

根据注射药物的理化性质,将预测符号分为7类。

AI类为水不溶性的酸性物质制成的盐,与pH值较低的注射液配伍时易产生沉淀。如青霉素类、头孢菌素类、苯妥英钠等。

BI类为水不溶性的碱性物质制成的盐,与pH值较高的注射液配伍时易产生沉淀。如红霉素乳糖酸盐、盐酸氯丙嗪、盐酸普鲁卡因等。

AS类为水溶性的酸性物质制成的盐,其本身不因pH值变化而析出沉淀。如维生素C、氨茶碱、葡萄糖酸钙、甲氨蝶呤(MTX)等。

BS类为水溶性的碱性物质制成的盐,其本身不因pH值变化而析出沉淀。如硫酸阿托品、硫酸多巴胺、硫酸庆大霉素、盐酸林可霉素等。

N类为水溶性无机盐或水溶性不成盐的有机物,其本身不因pH值变化而析出沉淀,但可导致AS、BI类药物产生沉淀。如氯化钾、葡萄糖、碳酸氢钠、氯化钠等。

C类为有机溶媒或增溶剂制成不溶性注射液(如氢化可的松),与水溶性注射剂配伍时,常由于溶解度改变而析出沉淀。如氯霉素、维生素K{1}
、地西泮等。

P类为水溶性的具有生理活性的蛋白质(如胰岛素),pH值变化、重金属盐、乙醇等均可影响其活性或使其产生沉淀。如抗利尿激素、透明质酸酶、催产素、肝素等。

\section{药动学方面的相互作用}

药物代谢动力学(pharmacokinetics,PK)简称药动学,是研究药物在体内变化规律的一门学科。药动学的研究内容主要包括:一是药物的体内过程,包括吸收、分布、代谢和排泄;二是药物在体内随时间变化的速率过程。前者主要描述药物在体内变化过程的一般特点;后者主要以数学公式定量地描述药物随时间改变的变化过程。

机体对药物的处理是药物与机体相互作用的一个重要组成部分,药动学过程包括药物在其吸收、分布、代谢和排泄过程的任一环节发生相互作用,均可影响药物在血浆或其作用靶位的浓度,最终使其药效或不良反应发生相应改变。

\subsection{影响药物吸收的相互作用}

药物由给药部位进入血液循环的过程称为吸收。除静脉注射和静脉滴注给药外,其他血管外给药途径都存在吸收过程。临床常用的血管外给药途径可分为消化道给药、注射给药、呼吸道给药及皮肤黏膜给药,口服是最常用的给药途径。药物在胃肠道吸收时相互影响的因素有如下几个方面。

\subsubsection{pH值的影响}

药物在胃肠道的吸收主要通过被动转运。药物的脂溶性愈大、非解离型比值越大,越易吸收。胃肠道的pH值可通过影响药物的溶解度和解离度,进而影响药物的吸收。如酸性药物在酸性环境以及碱性药物在碱性环境下解离度低,非解离型药物占大多数,因而药物脂溶性较高,较易透过生物膜被吸收;反之,酸性药物在碱性环境或碱性药物在酸性环境下解离度高,因而药物脂溶性低,扩散透过生物膜的能力差,吸收减少。药物与能改变胃肠道pH值的其他药物合用,其吸收将会受到影响。如水杨酸类药物在酸性环境下吸收较好,若同时服用抗酸药碳酸氢钠,将减少水杨酸类药物的吸收。

\subsubsection{配合作用与吸附作用的影响}

含有2、3价的阳离子(\ce{Ca^2+} 、\ce{Al^3+} 、\ce{Mg^2+}
等)能与四环素类抗生素、异烟肼、喹诺酮类抗菌药物等形成不溶性或难以吸收的配合物,从而影响药物吸收。如口服的四环素与金属离子(\ce{Ca^2+}
、\ce{Al^3+} 、\ce{Mg^2+} 等)配合,使其吸收减少。

阴离子交换树脂如考来烯胺、考来替泊,对酸性分子如阿司匹林、地高辛、华法林、环孢素、甲状腺素等有很强的亲和力,妨碍了这些药物的吸收。药用炭、白陶土等吸附剂也可使一些与其一同服用的药物吸收减少,如林可霉素与白陶土同服,其血药浓度只有单独服用时的10%。

这些药物相互作用可采用增加给药时间间隔的方法来避免。

\subsubsection{胃肠运动的影响}

大多数口服药物主要在小肠上部吸收,因此改变胃排空和肠蠕动速度的药物能影响目标药物到达小肠吸收部位的时间和在小肠滞留的时间,从而影响目标药物吸收程度和起效时间。

一般而言,胃肠蠕动加快,药物起效快,但在小肠滞留时间短,可能吸收不完全;胃肠蠕动减慢,药物起效慢,吸收可能完全。这在溶解度低和难吸收的药物中表现得比较明显。如地高辛片剂在肠道内溶解度较低,与促进胃肠蠕动的甲氧氯普胺等合用,地高辛的血药浓度可降低约30%,有可能导致治疗失败;而与抑制胃肠蠕动的溴丙胺太林合用,地高辛的血药浓度可提高30%左右,如不调整地高辛剂量,就可能中毒;而口服快速溶解的地高辛溶液或胶囊,则溴丙胺太林对其吸收的影响相对较小。但是,对那些在胃的酸性环境中会被灭活的药物如左旋多巴,抑制胃肠蠕动的药物可增加其在胃黏膜脱羧酶的作用下转化为多巴胺(DA),从而降低其口服生物利用度。

\subsubsection{肠吸收功能的影响}

抗肿瘤药物如环磷酰胺、长春碱以及对氨基水杨酸、新霉素等能破坏肠壁黏膜,引起吸收不良。如环磷酰胺可使合用的地高辛吸收减少,血药浓度降低,疗效下降。

\subsubsection{食物的影响}

一般情况下食物可减少药物的吸收。如利福平、异烟肼等可因进食而吸收缓慢,但对药物吸收总量未有影响。但某些脂溶性药物,如灰黄霉素与高脂肪的食物同服,可明显增加吸收量。

\subsubsection{肠道菌群的影响}

消化道的菌群主要位于大肠内,胃和小肠内数量极少。因此,主要在小肠内吸收的药物较少受到肠道菌群的影响。口服地高辛后,在部分患者的肠道中,地高辛能被肠道菌群大量代谢灭活,如同时服用红霉素等能抑制这些肠道菌群的抗生素,可使地高辛血浆浓度增加一倍。

部分药物结合物经胆汁分泌,在肠道细菌的作用下可水解为有活性的原药而重吸收,形成肠肝循环。抗菌药物通过抑制细菌可抑制这些药物的肠肝循环。如抗生素可抑制口服避孕药中炔雌醇的肠肝循环,导致循环血中雌激素水平下降。

\subsubsection{其他因素的影响}

消化液是某些药物重要的吸收条件。硝酸甘油片舌下含服,需要充分的唾液帮助其崩解和吸收,如同服抗胆碱药,则由于唾液分泌减少而使之降效。

某些药物合并用药可影响胃肠道黏膜内外酶和酶系统,从而影响药物的吸收。如秋水仙碱能抑制肠黏膜中多种酶系统(如蔗糖酶、麦芽糖酶、乳酸酶等),导致维生素B{12}
的吸收不良。

另外,口服以外的给药途径也有可能因相互作用而影响吸收。如应用局麻药时,常加入微量肾上腺素以收缩血管,延缓局麻药的吸收,达到延长局麻药作用时间、减少不良反应的效果。

\subsection{影响药物分布的相互作用}

药物吸收后,通过各种生理屏障经血液转运到组织器官的过程称为分布(distribution)。分布过程中的药物相互作用方式,可表现为相互竞争血浆蛋白结合部位,改变游离型药物的比例,或改变药物在某些组织的分布量,从而影响它们在靶部位的浓度。

\subsubsection{竞争血浆蛋白结合部位}

药物经吸收进入血液循环后,大部分药物或其代谢产物均不同程度地与血浆蛋白发生可逆性结合,称结合型药物;另一部分为游离型药物。

当药物合用时,它们可在蛋白结合部位发生竞争,结果是与蛋白亲和力较强的药物可将另一种亲和力较弱的药物从血浆蛋白结合部位上置换出来,使后一种药物的游离型增多。由于游离型的药物分子才能跨膜转运,产生生物活性,并能被分布、代谢与排泄,因此这种蛋白结合的置换可对被置换药物的药动学和药效学产生一定的影响。

通过体外试验很容易证明,许多药物间均存在这种蛋白结合的置换现象。因此,过去一度认为它是临床上许多药物相互作用的一个重要机制。但近年来,更严谨的研究得出结论:大多数置换性相互作用并不产生严重的临床后果,因为置换使游离型药物增多的同时,相应分布、消除的比例也增加,仅引起血药浓度的短暂波动。

保泰松与华法林的相互作用研究是对蛋白结合置换现象的临床意义进行重新认识的典型例子。保泰松可以增强华法林的抗凝作用而致出血不止。过去一直认为,保泰松将华法林从其血浆蛋白结合部位置换出来,游离型华法林浓度升高导致出血。并据此认为任何非甾体抗炎药(NSAID)均以这种方式增强华法林的抗凝作用。现在的研究认识到,华法林是R和S两种异构体的混合物,S异构体的活性较R强5倍;保泰松除了竞争置换出华法林外,还可抑制S-华法林的代谢(由CYP2C9/18催化)而促进R-华法林代谢(由CYP1A2、CYP3A4催化),这样表面上药物总的半衰期不变,但血浆中活性高的S-华法林的比例增大,因而抗凝作用增强。

药物在蛋白结合部位的置换反应能否产生明显的临床后果,取决于目标药的药理学特性,那些蛋白结合率高、分布容积小、半衰期长和安全范围小的药物被置换下来后,往往发生药物作用的显著增强而导致不良的临床后果。表\ref{tab4-1}列出了一些常见的通过血浆蛋白置换而发生药物相互作用的实例。

\begin{longtable}[]{@{}lll@{}}
    \caption{血浆蛋白置换引起的药物相互作用}
    \label{tab4-1}\\
    \toprule
目标药(被置换药物) & 相互作用药 & 临床后果\tabularnewline
\midrule
甲苯磺丁脲 & 水杨酸、保泰松、磺胺药 & 低血糖\tabularnewline
华法林 & 水杨酸、水合氯醛 & 出血倾向\tabularnewline
MTX & 水杨酸、呋塞米、磺胺药 & 粒细胞缺乏症\tabularnewline
硫喷妥钠 & 磺胺药 & 麻醉时间延长\tabularnewline
卡马西平、苯妥英钠 & 维拉帕米 & 两药毒性增强\tabularnewline
\bottomrule
\end{longtable}

\subsubsection{改变组织分布}
\paragraph{改变组织血流量}

某些作用于心血管系统的药物可通过改变组织血流而影响与其合用药物的组织分布。如去甲肾上腺素减少肝脏血流量,使得利多卡因在肝脏的分布量减少,导致代谢减慢、血药浓度增高;而异丙肾上腺素增加肝脏血流量,增加利多卡因在肝脏中的分布及代谢,使其血药浓度降低。
\paragraph{组织结合位点上的竞争置换}

与药物在血浆蛋白上的置换一样,类似的反应也可发生于组织结合位点上,而且置换下来的游离型药物可返回到血液中,使血药浓度升高。由于组织结合位点的容量一般都很大,通常对血药浓度影响不大,但有时也能产生有临床意义的药效变化。例如奎尼丁能将地高辛从骨骼肌的结合位点上置换下来,可使90%患者的地高辛血药浓度升高约1倍,两药合用时,地高辛用量应减少30%~50%。

\subsection{影响药物代谢的相互作用}

药物在体内发生化学结构的改变称为代谢,或称为生物转化。药物代谢的主要场所是肝脏,肝脏进行药物代谢主要依赖于微粒体中的多种酶系。药物经代谢后可转化为无活性物质;或使原来无药理活性的药物转变为有活性的代谢产物;或将活性药物转化为其他活性物质;或产生有毒物质。影响药物代谢的相互作用占药动学相互作用的40%,是一种具有重要临床意义的药动学相互作用。

\subsubsection{酶诱导}

某些药物能增加肝药酶的合成或提高肝药酶的活性,称之为酶诱导。酶诱导使目标药的代谢加快,一般是导致作用减弱或作用时间缩短。具有酶诱导作用的常见药物如表\ref{tab4-2}所示。如口服抗凝血药双香豆素期间加服苯巴比妥,后者使血中双香豆素的浓度下降,抗凝作用减弱,表现为凝血酶原时间缩短。因此,如果这两类药物合用,必须应用较大剂量才能维持其治疗效应。

\begin{longtable}{ccc}
    \caption{常见的酶诱导及相互作用}
    \label{tab4-2}\\
    \toprule
    药物种类 & 受影响药物 & 相互作用结果\tabularnewline
\midrule
巴比妥类 & 巴比妥类、洋地黄毒苷、类固醇激素& \multirow{4}*{血药浓度下降、药效减弱或不良反应减轻}\tabularnewline             
保泰松、苯妥英钠 & 口服降血糖药、氢化可的松、茶碱 & ~\tabularnewline
利福霉素 & 口服抗凝药、地高辛、普萘洛尔、美托洛尔等 & ~\tabularnewline
灰黄霉素 & 口服抗凝药 & ~\tabularnewline
\bottomrule
\end{longtable}





需要指出的是,酶诱导促使药物代谢增加,但不一定均导致药物疗效下降,因为有些药物的药效是由其活性代谢物引起的。如环磷酰胺在体外无活性,只有经肝药酶代谢活化生成磷酰胺氮芥,才能与DNA烷化发挥药理作用,抑制肿瘤细胞的生长增殖。另外,如果药物经代谢生成毒性代谢产物,与酶诱导剂合用就可能会导致不良反应增加。如异烟肼产生肝毒性代谢物乙酰异烟肼,若与利福平合用,后者的酶诱导作用将加重异烟肼的肝毒性。

\subsubsection{酶抑制}

一些药物能减少肝药酶的合成或者降低肝药酶的活性,称之为酶抑制。临床上因肝药酶的抑制而引起的药物相互作用较肝药酶诱导所引起的药物相互作用常见。肝药酶被抑制,将使另一药物的代谢减少,因而加强或延长其作用。具有酶抑制作用的常见药物如表\ref{tab4-3}所示。如氯霉素与双香豆素合用,明显加强双香豆素的抗凝血作用,这是由于氯霉素抑制肝药酶,使双香豆素的半衰期延长2~4倍。

\begin{longtable}{ccc}
    \caption{常见的酶抑制及相互作用}
    \label{tab4-3}\\
    \toprule
    药物种类 & 受影响药物 & 相互作用结果\tabularnewline
\midrule
西咪替丁、阿司匹林 & 苯二氮䓬类药物& \multirow{4}*{血药浓度上升、药效增强或出现毒性反应}\tabularnewline
氯霉素、异烟肼 & 苯妥英钠、口服降血糖药 & ~\tabularnewline
别嘌醇 & 口服抗凝药、AZA\footnote{AZA表示硫唑嘌呤(azathioprine)} & ~\tabularnewline
肾上腺皮质激素 & 三环类抗抑郁药、环磷酰胺 & ~\tabularnewline
\bottomrule
\end{longtable}

有些药物在体内通过各自的灭活酶而被代谢,若这些酶被抑制,将加强相应药物的作用。食物中的酪胺在吸收过程中被肠壁和肝脏的单胺氧化酶所灭活,因而不呈现作用。但在服用单胺氧化酶抑制剂期间,若食用酪胺含量高的食物如奶酪、红葡萄酒等,由于肠壁及肝脏的单胺氧化酶已被抑制,被吸收的酪胺不经破坏,大量到达去甲肾上腺素能神经末梢,引起末梢中的去甲肾上腺素大量释放出来,使动脉血压急剧升高,产生高血压危象,危及患者生命。

虽然酶抑制可导致相应目标药自机体的清除减慢,体内药物浓度升高,但酶抑制能否引起有临床意义的药物相互作用取决于多种因素。
\paragraph{目标药的毒性及治疗窗的大小}

药物相互作用能产生临床意义的药物通常其治疗窗很窄,即治疗剂量和中毒剂量之间的范围很小;或其剂量-反应曲线陡峭,药物浓度虽然只有轻微改变,但是其效果差异变化显著。如抗过敏药阿司咪唑具有心脏毒性,与酮康唑、红霉素等酶抑制剂合用时,由于代谢受阻血药浓度显著上升,可出现致死性的心脏毒性。而酮康唑抑制舍曲林的代谢则不会引起严重的心血管不良反应。
\paragraph{是否存在其他代谢途径}

如果目标药可由多种肝药酶催化代谢,当其中一种酶受到抑制时,药物可代偿性经由其他途径消除,药物代谢速率所受影响可不大。但对主要由某一种肝药酶代谢的药物,如果代谢酶受到抑制,则容易产生明显的药物浓度和效应的变化。
\paragraph{与能抑制多种肝药酶的药物合用}

有些药物能抑制多种肝药酶,在临床上容易发生与其他药物的相互作用。如H{2}
受体阻断剂西咪替丁,其结构中的咪唑环可与肝药酶中的血红素部分紧密结合,故能抑制多种肝药酶而影响许多药物在体内的代谢。目前已报道有70多种药物的肝清除率在与西咪替丁合用后,出现不同程度的下降。临床上当药物与西咪替丁合用时,应注意调整剂量,必要时可用雷尼替丁代替西咪替丁。

酶抑制引起的药物相互作用常常导致药物作用的增强及不良反应的发生,但也有例外。如奎尼丁是酶抑制剂,而可待因须经肝药酶代谢生成吗啡产生镇痛作用,两者合用可使可待因的镇痛作用明显减弱,药效降低。

\subsection{影响药物排泄的相互作用}

药物及其代谢产物经机体的排泄器官或分泌器官排出体外的过程称为排泄。大多数影响药物排泄的相互作用发生在肾脏。当一种药物改变肾小管液的pH值、干扰肾小管的主动转运过程或重吸收过程或影响到肾脏的血流量时,就能影响一些其他药物的排泄,尤其对以原形排出的药物影响较大。

\subsubsection{改变尿液pH值}

尿液的pH值通过影响解离型/非解离型药物的比例,改变进入肾小管内药物的重吸收。这主要是因为大多数药物为有机弱电解质,在酸性尿液中,弱酸性药物(pKa为3.0~7.5)大部分以非解离型存在,脂溶性高,易通过肾小管上皮细胞重吸收;而弱碱性药物(pKa为7.5~10)的情况相反,大部分以解离型存在,随尿液排出多。临床上可通过碱化尿液增加弱酸性药物的肾清除率,如苯巴比妥多以原形自肾脏排泄,当过量中毒时,可用碳酸氢钠碱化尿液,减少重吸收,促进苯巴比妥的排泄而解毒。同理,酸化尿液可促进碱性药物的排泄。

但在药物的相互作用中,尿液pH值改变的临床意义甚小,因为除小部分药物直接以原形排出,大多数药物经代谢失活后,最终从肾脏消除;同时能大幅度改变尿液pH值的药物在临床上也很少使用。

\subsubsection{干扰肾小管分泌}

肾小管分泌是一种主动转运过程,要通过肾小管的特殊转运载体,包括酸性药物载体和碱性药物载体。当两种酸性药物合用时(或两种碱性药物合用),可相互竞争酸性(或碱性)载体,竞争力弱的药物,经由肾小管分泌的量减少,肾脏排泄减慢,有可能增强其疗效或毒性。如痛风患者合用丙磺舒和吲哚美辛,两者竞争酸性载体,可使吲哚美辛的分泌减少,排泄减慢,不良反应发生率明显增加。

但是有些药物间的这种竞争可被用于产生有益的治疗目的。如丙磺舒和青霉素竞争肾小管上的酸性转运系统,可延缓青霉素的经肾排泄过程,使其发挥持久的治疗作用。

\subsubsection{改变肾脏血流量}

减少肾脏血流量的药物可妨碍药物的经肾排泄,但这种情况在临床上并不多见。肾脏的血流量部分受到肾组织中扩血管的前列腺素生成量的调控。有报道指出,如果这些前列腺素的合成被吲哚美辛等药物抑制,则锂的肾排泄量会降低,并伴有血清锂水平的升高。这提示合用锂盐和NSAIDs的患者,应密切监测血清锂水平。

\section{药效学方面的相互作用}

药效学方面的药物相互作用是指不同药物通过与疾病相关药物靶点的影响,使一种药物增强或减弱另一种药物的效应或不良反应的现象。相互作用结果可分为药物效应的相加、协同和拮抗。

\subsection{相加或协同作用}

相加作用(addition effect)或协同作用(synergistic
effect)是指作用于疾病相关靶点的两种药物合用的效果等于(相加)或大于(协同)单用效果之和。相加或协同作用是临床用药的主要目的。

\subsubsection{表现为药理作用的增强}

如磺胺甲噁唑(SMZ)和甲氧苄啶(TMP)通过双重阻断机制(SMZ抑制二氢叶酸合成酶,TMP抑制二氢叶酸还原酶),协同阻断敏感菌的四氢叶酸合成,抗菌活性是两药单独等量应用时的数倍至数十倍,甚至呈现杀菌作用,且抗菌谱扩大,并减少细菌耐药性的产生。常将SMZ与TMP按5∶1的比例制成复方磺胺甲噁唑(SMZco)用于临床。另外,临床上常用青霉素和庆大霉素联用抗感染、异烟肼和利福平联用抗结核,这些联用都表现为治疗效应的增强。

\subsubsection{表现为药理作用的相加}

如应用一般治疗剂量的巴比妥类药物或其他具有中枢神经系统抑制作用的药物时,饮用少量酒即可引起昏睡,因为乙醇具有非特异性中枢神经系统的抑制作用,致使药理作用的相加。

\subsubsection{表现为增加药物不良反应的风险}

如治疗帕金森病的抗胆碱药物,与具有抗胆碱作用的其他药物(如氯丙嗪、H{1}
受体阻断药、三环类抗抑郁药)合用时可产生性质协同的相互作用,常可出现过度的抗胆碱能效应,在老年患者甚至可能出现抗胆碱危象。口服广谱抗生素抑制肠道菌群后,可使维生素K合成减少,从而增加香豆素类抗凝药的活性,应适当减少抗凝药的剂量。临床常见的药物相加或协同作用如表\ref{tab4-4}所示。

\begin{longtable}{cc}
    \caption{临床常见的药物相加或协同作用}
    \label{tab4-4}\\
\toprule
\endhead
相互作用药物 & 药理效应\tabularnewline
\midrule
NSAIDs和华法林 & 增加出血的风险\tabularnewline
血管紧张素转换酶抑制剂和氨苯蝶啶 & 增加高血钾的风险\tabularnewline
维拉帕米和β受体拮抗剂 & 心动过缓和停搏\tabularnewline
氨基糖苷类和呋塞米 & 增加耳、肾毒性\tabularnewline
骨骼肌松弛药和氨基糖苷类 & 增加骨骼肌松弛作用\tabularnewline
乙醇与苯二氮䓬类 &
增强镇静作用\tabularnewline
MTX与复方磺胺甲噁唑 & 骨髓巨幼红细胞症\tabularnewline
\bottomrule
\end{longtable}

\subsection{拮抗作用}

拮抗作用是指两种或两种以上药物合用所产生的效应小于其中一种药物单用的效应。在临床上,通常要尽量避免药物治疗作用的相互拮抗。根据作用机制,可将药物的拮抗作用分为两类。

\subsubsection{竞争性拮抗}

两种药物在共同的作用部位或受体上产生了拮抗作用。本类相互拮抗作用可发挥治疗作用,如在治疗虹膜炎时,交替使用毛果芸香碱和阿托品,可防止虹膜粘连;也可产生药理性拮抗作用,在药物中毒时抢救患者的生命。
如用苯二氮䓬
类受体拮抗剂氟马西尼抢救苯二氮䓬
类过量中毒;用α-肾上腺素受体激动剂去甲肾上腺素对抗氯丙嗪过量引起的低血压。

\subsubsection{非竞争性拮抗}

作用物与拮抗物不是作用于同一受体或同一部位,也可出现拮抗作用。如较大剂量的氯丙嗪用于治疗精神分裂症时,因阻断黑质-纹状体通路的多巴胺受体,使中枢乙酰胆碱作用相对增强,可引起锥体外系反应,而苯海索具有中枢抗胆碱作用,可减轻锥体外系反应;氨茶碱可因兴奋中枢而引起失眠,常合用催眠药加以对抗;维生素B{6}
能增加外周多巴脱羧酶活性,加速左旋多巴在外周部位脱羧,减少左旋多巴进入中枢的量,降低左旋多巴的疗效,产生对抗左旋多巴的作用。

