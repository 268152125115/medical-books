\chapter{神经系统疾病的药物治疗}

\section{急性脑血管病}

\subsection{定义}

急性脑血管病又称脑卒中,主要包括出血性脑血管病和缺血性脑血管病,后者又包括短暂性脑缺血发作、脑血栓形成和脑栓塞。

\subsection{病因及发病机制}

\subsubsection{病因}
\paragraph{出血性脑血管病的病因}

高血压是脑出血的主要原因,其他少见原因包括脑血管畸形、动脉瘤、脑动脉炎、血液病、应用溶栓抗凝药后及脑肿瘤等。
\paragraph{缺血性脑血管病的病因}

(1)短暂性缺血发作是一种多病因的综合征,其主要病因是主动脉-脑动脉系统的动脉粥样硬化。

(2)脑血栓形成是发病率最高的一种缺血性卒中,占全部卒中的70%~80%,由脑动脉粥样硬化、脑动脉炎、血液高凝状态和脑供血不足所致。

(3)脑栓塞又称栓塞性脑梗死,多数与心脏病有关,由脑外来源的栓子、栓塞所致。

\subsubsection{发病机制}
\paragraph{出血性脑血管病发病机制}

长期高血压出现小动脉平滑肌透明变性,局部管壁变薄,在高血流压力下膨出,形成微小动脉瘤或纤维素性坏死,从而在血压突然升高时发生破裂而引起脑出血。
\paragraph{短暂性脑缺血发作的发病机制}

短暂性脑缺血与多种促发因素及其组合可能有关,包括微栓子形成、盗血现象和其他疾病,如颈椎病、血压过低、动脉痉挛、血高凝状态等。
\paragraph{脑血栓形成}

引起脑血栓主要有以下5个方面的因素:①动脉管壁病变,最常见的是动脉粥样硬化;②脑动脉炎;③血流动力学的改变;④动脉畸形;⑤血液成分变化。
\paragraph{脑栓塞}

脑栓塞是由于来自身体各部位的各种栓子(如心内膜炎的炎性栓子、风湿性心脏瓣膜病栓子、心肌梗死或动脉硬化附壁血栓等),通过颈动脉或椎动脉阻断脑血管,致使其供血区域缺血、坏死,从而发生脑梗死和脑功能障碍。

\subsection{临床表现}

\subsubsection{出血性脑血管病}

本病好发于50岁以上,没有系统治疗或血压控制不好的高血压者,常在体力活动或情绪激动时突然发病,出现昏迷、偏瘫、呕吐等。发病时多有血压明显升高,意识障碍程度是判断病情轻重的主要指标。

\subsubsection{短暂性脑缺血发作}

本病好发于中老年人,男性多于女性。发作突然,症状在1min内达到高峰,少数于数分钟内进行性发展,一般持续时间不超过15min,个别可达2h。发作停止后,神经症状完全消失,但常有反复发作的趋势。临床上将短暂性脑缺血发作分为两类。
\paragraph{颈内动脉系统短暂性脑缺血发作}

最常见的症状为对侧上肢或下肢无力,也可只限于一只手无力,但少累及面部。感觉障碍多为部分肢体麻木,感觉很少完全丧失。可产生感觉性或运动性失语。单侧视力丧失为特有症状,发作时,在眼底可见到动脉栓子,本类较多见。
\paragraph{椎-基底动脉系统短暂性脑缺血发作}

最常见的症状为眩晕,伴视野缺损和复视,很少有耳鸣。可出现言语不清、单侧共济失调、双眼视物模糊、声音嘶哑、呃逆、呕吐。一侧脑神经麻痹伴对侧肢体瘫痪或感觉障碍为典型表现。跌倒发作为特有表现,患者突然跌倒在地,而无可觉察的意识障碍,虽有很短暂的四肢无力,但患者可以立即自行站起。

\subsubsection{脑血栓形成}

本病多发生于中老年人,多有高血压、动脉粥样硬化史。起病突然,但症状体征进展较缓慢,常需数小时,甚至1~2d达到高峰。不少患者在睡眠中发病,清晨醒来时发现偏瘫或单瘫,以及失语等。部分患者发病前有短暂性脑缺血发作病史。多数患者清醒,如果起病时即意识不清,要考虑椎-基底动脉系统脑梗死可能。大脑半球较大区域梗死,缺血、水肿影响间脑和脑干功能,可于起病后不久出现意识障碍。

\subsubsection{脑栓塞}

脑栓塞的起病年龄不一,因多数与心脏病有关,所以发病年龄以中青年居多。起病前无先兆,起病急骤,数秒或数分钟内症状发展到高峰,是所有脑血管病中起病最急者。个别患者可在数日内呈阶梯式进行性恶化,由反复栓塞所致。半数患者起病时有意识丧失,但意识丧失的时间远比脑出血短。常有突发的面瘫、上肢瘫、偏瘫、失语、偏盲、局限性癫痫
发作,或偏身感觉障碍等局部脑病症状。多数抽搐为局部性,如为全身性大发作,提示栓塞范围广泛,病情较重。

\subsection{治疗}

\subsubsection{出血性脑血管病的药物治疗原则}

急性期主要治疗原则是降低颅内压和脑代谢、控制血压,尽量减少不必要的搬动。

\subsubsection{缺血性脑血管病的药物治疗原则}

早期溶栓治疗,恢复血氧供应。改善脑循环,降低脑代谢,减轻脑水肿。全身治疗要纠正高血糖,降低血黏度,维持水电解质平衡。

\subsubsection{常用药物}
\paragraph{溶栓药}

\textbf{尿激酶:}

【适应证】 主要用于脑血栓形成的溶栓治疗。溶栓的疗效均需用后续的肝素抗凝治疗加以维持。

【用法和用量】 现用现配,临用前用灭菌的生理盐水配制,每日20000~40000IU,分两次静脉注射,疗程7~10d。

【不良反应】 出血,严重可致脑出血。少数出现过敏反应。

【禁忌证】 禁用于近期有出血倾向疾病、手术、外伤及出血性脑卒中病史者。
\paragraph{抗凝药}

\textbf{肝素:}

【适应证】 用于防治血栓形成或栓塞性疾病。

【用法和用量】 静脉注射:成人初始剂量5000IU加入5%~10%葡萄糖溶液或生理盐水100mL中,30~60min滴完。之后每4h按100IU/kg溶于生理盐水中持续滴注,肝素静滴每日20000~40000IU。

【不良反应】 偶见过敏反应,表现为哮喘、荨麻疹、结膜炎和发热以及一过性脱发和腹泻等;短暂的血小板减少症。

【禁忌证】 禁用于对肝素钠过敏;不能控制的活动性出血;有出血性疾病及凝血机制障碍者;外伤或术后渗血;胃及十二指肠溃疡;严重肝肾功能不全;重症高血压。

\textbf{华法林:}

【适应证】 适用于需长期持续抗凝的患者,如用于防治血栓栓塞性疾病,能防止血栓的形成与发展。

【用法和用量】 口服给药,避免冲击治疗。第1~3日,每日3~4mg;3d后可2.5~5mg/d维持。

【不良反应】 使用过量易致各种出血。出血可发生在任何部位,特别是泌尿道和消化道。偶见恶心、呕吐、腹泻、过敏反应等。

【禁忌证】 禁用于肝肾功能损害、严重高血压、凝血功能障碍伴有出血倾向、活动性溃疡、外伤、先兆流产、近期手术者和妊娠期。
\paragraph{抗血小板药}

\textbf{阿司匹林:}

【适应证】 小剂量用于预防缺血性脑血管病。

【用法和用量】 口服,每次80~300mg,每日1次。

【不良反应】 胃肠道反应。

\textbf{氯吡格雷:}

【适应证】 用于脑血管疾病、脑中风、脑供血不全等。

【用法和用量】 口服:每次50~75mg,每日1次。

【不良反应】 不良反应较少、较轻,主要表现为上腹不适,偶见中性粒细胞减少。
\paragraph{降纤药}

\textbf{巴曲酶:}

【适应证】 适用于急性脑梗死,包括脑血栓、脑栓塞,短暂性脑缺血发作及脑梗死再复发的预防。

【用法和用量】 成人用量首次量为10BU,以后的维持量可减为5BU,隔日1次。

【不良反应】 少数患者有轻度不良反应。主要表现为注射部位出血,偶见消化道出血。可见头痛、头晕等症状,个别患者可能出现少量红斑、瘙痒及荨麻疹,偶见过敏性休克。
\paragraph{脱水药}

\textbf{甘露醇:}

【适应证】 用于治疗各种原因引起的脑水肿,降低颅内压,防止脑疝。

【用法和用量】 推荐静脉滴注20%甘露醇(4h内0.25~0.5g/kg,疗程<5d)。

【不良反应】 快速大量静脉滴注甘露醇可引起体内甘露醇积聚,血容量迅速大量增多,导致心力衰竭,稀释性低钠血症,偶可致高钾血症。
\paragraph{血容量扩充药}

\textbf{低分子右旋糖酐:}

【适应证】 可增加血容量,降低血液黏稠度,改善病灶区微循环。

【用法和用量】 静脉滴注:每次250~500mL,每日或隔日1次,7~14次为1疗程。

【不良反应】 偶见过敏反应,如发热、胸闷、呼吸困难、荨麻疹等。
\paragraph{钙通道阻滞剂}

\textbf{尼莫地平:}

【适应证】 适用于各种原因的蛛网膜下腔出血后的脑血管痉挛和急性脑血管病恢复期的血液循环改善。

【用法和用量】 口服,每次20~60mg,每日3次。静脉滴注,须单独输注,每分钟滴注0.5μg/kg,随时监测血压,病情稳定后改口服。

【不良反应】 常见肝功能异常,可有血小板计数减少、皮疹、瘙痒、恶心、呕吐及出血,偶见一过性头晕、头痛、面潮红等。

\textbf{桂利嗪:}

【适应证】 预防和治疗脑血液循环障碍,如脑梗死、脑血栓形成、短暂性脑缺血发作等。

【用法和用量】 口服,每次25~50mg,每日3次,饭后服。静脉注射:每次20~40mg,缓慢注入。

【不良反应】 最为常见的不良反应为嗜睡和疲惫;长期服用者可出现抑郁症,女性较为常见。

【禁忌证】 禁用于孕妇和哺乳期妇女。慎用于帕金森病、驾驶员和机械操作者。

\section{癫痫}

\subsection{定义}

癫痫
是一组反复发作的脑神经元异常放电所致的暂时性中枢神经系统功能失常的慢性疾病。目前以药物治疗为主。

\subsection{病因及发病机制}

按有无明确病因将癫痫
分为原发性癫痫
和继发性癫痫 两大类:

\subsubsection{原发性癫痫}

原发性癫痫
又称“特发性”或“隐源性”癫痫
,病因不清或暂时未能确定脑内有器质性疾病。多与遗传因素有关,起病多在儿童期和青春期(5~20岁)。

\subsubsection{继发性癫痫}

继发性癫痫
又称症状性癫痫
或获得性癫痫
,脑内有明确的致病因素引起的癫痫
发作,占癫痫
的大多数,可发生于各个年龄段。病因复杂,常见的原因如下。

(1)脑部疾病:先天性疾病如结节性硬化、脑畸形等,脑肿瘤,脑外伤,颅内感染,脑血管病如脑出血、脑梗死等。

(2)全身或系统性疾病:窒息缺氧、一氧化碳中毒,低血糖、低血钙、尿毒症等,糖尿病,高血压脑病,有机磷中毒,中枢兴奋性及某些重金属中毒等。

\subsection{临床表现}

癫痫
的临床表现取决于癫痫
发作的类型,主要分为两大类:部分性发作和全面性发作。其中部分性发作主要有单纯部分性发作和复杂部分性发作,全面性发作则主要包括失神发作、全身强直阵挛发作等。

\subsubsection{全身强直阵挛发作(大发作)}

典型的发作以一声尖叫开始,随即意识丧失,跌倒在地口吐白沫、双眼上翻、凝视;继而出现全身强直性抽搐,面色发青,持续10~20s;随后转为阵挛性抽搐,通常持续1~2min停止,可发生大、小便失禁;抽搐停止后患者进入昏迷或昏睡状态,醒后对发作过程不能回忆,常述头痛和疲乏无力。约50%的患者有发作先兆,为恐惧感、麻木等不适,持续时间短。患者一般会有发作时跌伤的经历。

\subsubsection{失神发作(小发作)}

临床表现为突然发生和突然停止的短暂和频发的意识障碍。患者突然静止不动,持续5~20s,很少超过30s,无语,双目凝神或上视,眼球可有细微颤动,有时面色苍白,发作后继续原来的活动,但对发作全无记忆,每日数次或数十次,甚至达百次以上。

\subsubsection{单纯部分性发作}

单纯部分性发作也称局限性发作。发作时表现可分为运动性、感觉性、精神性或自律性。发作大多短促,抽搐常自一侧上肢远端扩展至近端,面部和同侧下肢,患者意识不丧失。

\subsubsection{复杂部分性发作}

复杂部分性发作也称颞叶癫痫
或精神运动性发作,主要见于继发性癫痫
,是有意识障碍的部分性发作。常有味、嗅、视、听幻觉。味幻觉常感口中苦味,嗅幻觉可闻及特殊气味,视幻觉常视物变形,物体忽然变大、变小,听幻觉可闻及噪声、话语声、音乐声等。患者常先表现为一些自主神经症状,如面色潮红或苍白,然后做出无意识的动作如咀嚼、流涎、吞咽等进食性动作;有时表现为反复腹痛伴面红。

\subsection{治疗}

\subsubsection{治疗原则}

明确诊断,尽早治疗;依照类型,对症选药;合适剂量,单药治疗;长期用药,慎重停药。

\subsubsection{药物治疗原则}
\paragraph{早期治疗}

一旦癫痫
诊断成立,就应给予治疗,治疗越早越好,但对以下情况可暂缓给药:①首次发作,有明显环境因素,脑电图正常;②每次发作间隔大于12个月以上者。
\paragraph{药物的选择}

原则上应根据发作类型来选择疗效高、毒性小、价格低廉的药物。常以单一用药为主,单药治疗疗效可靠,便于观察不良反应,又能减少慢性中毒。当单药治疗增量后效果不满意时,或确认为难治性癫痫
、非典型小发作及混合性发作,可考虑联合用药。合并用药一般限于两种,最好不要超过3种药物。
\paragraph{药物剂量的调整及使用方法}

从低剂量开始,耐受后再缓慢加量,直至完全控制发作或产生毒性反应。药物显效时间一般为1~2周,常需监测血药浓度,当药量增至有效浓度上限仍无效时,应更换新药。
\paragraph{药物更换原则}

当某种抗癫痫
药经过一定时间应用(不少于1~2个月)确认无效,或毒性反应明显而需要换用另一种药物时,宜逐步替换,过渡时间一般5~7倍于药物的半衰期,至少要3~7d。切忌突然停药和更换药物,否则会使癫痫
发作加频,甚至诱发癫痫 持续状态。
\paragraph{减量或停药原则}

减量或停药原则:①原发性大发作和简单部分性发作,在完全控制2~5年后;失神发作在完全控制1年后可考虑停药。而复杂部分性发作多需长期或终身服药。②脑电图异常无改善或脑部病变处于活跃期不停药。③青春期应持续至青春期以后再考虑停药。有器质性病因的癫痫
患者,则需终身服药。停药前应缓慢减量,病程越长,剂量越大,用药越多,减量越要缓慢。
\paragraph{长期坚持,定期复查}

让患者及家属了解规律性服药和长期治疗的重要性,随意停药或换药是造成难治性癫痫
持续状态的原因之一。服药应定时、定量,用药期间应定期做血、尿常规及肝、肾功能检查,有条件可做血药浓度监测,防止药量过大引起毒性反应。

\subsubsection{常用药物}

\textbf{苯妥英钠:}

【适应证】 主要适用于强直-阵挛性发作、癫痫
持续状态,也可用于复杂部分性发作,对失神发作无效。

【用法和用量】 成人口服通常每日200~300mg,1次顿服(入睡前),或分2次服用。儿童刚开始服药每日3~5mg/kg,最大量为7mg/kg,每日总量不超过300mg,分2~3次服用。

【不良反应】 有神经系统反应,牙龈结缔组织增生,多毛,痤疮,鼻、唇变粗厚等。开始服药数周内可有皮疹,伴发热及淋巴结肿,停药后消失。

\textbf{苯巴比妥:}

【适应证】 是一种有效、低毒、价廉的抗癫痫
药,主要适用于强直-阵挛性发作、癫痫
持续状态。

【用法和用量】 成人维持量为每日1~3mg/kg,开始先用小剂量,每次15~30mg,每日3次。老年人应减量,儿童用量为每日2~4mg/kg。治疗癫痫
持续状态时,每次静脉缓慢注射0.1~0.2g。

【不良反应】 神经精神系统反应如头晕、共济失调、眼震、构音障碍等;过敏性皮疹多轻微,停药消失,罕见剥脱性皮炎等严重不良反应。

\textbf{卡马西平:}

【适应证】 是安全、有效、广谱的抗癫痫
药,主要用于复杂部分性发作(精神运动性发作)。

【用法和用量】 成人口服,每次100~200mg,每日1~2次,逐渐增加至每次400mg,每日2~3次。儿童每日10~20mg/kg,分次服用。

【不良反应】 可有胃肠反应(腹痛、腹泻、口干)和皮肤反应(瘙痒、光敏、脱发、多汗、皮疹),偶见心律失常,肝功能损害。

\textbf{乙琥胺:}

【适应证】 治疗失神发作的首选药物。

【用法和用量】 成人开始每日口服500mg,维持量每日15~30mg/kg,最大用量每日1.5g。儿童开始用量每日250mg,维持量每日5~40mg/kg,分2~4次服用。

【不良反应】 胃肠道症状,偶见嗜睡、头痛、共济失调、头晕。

\textbf{丙戊酸钠:}

【适应证】 对小发作疗效优于乙琥胺,但因其肝脏毒性较大,常不作为首选药物。

【用法和用量】 成人口服每次200~400mg,每日3或4次,在饭后和入睡前服用。儿童每日5~15mg/kg开始,以后每周增加5~10mg/kg,最高剂量可达每日50~60mg/kg。

【不良反应】 神经系统反应和胃肠道刺激症状,最严重的不良反应为肝脏受损。

\textbf{地西泮:}

【适应证】 属苯二氮䓬
类,适用于癫痫 持续状态。

【用法和用量】 成人用10~20mg不稀释做静脉注射,速度每分钟不超过2mg,直到发作终止或总量达30mg。小儿静注用量:出生30d至5岁每2~5min
0.2~0.5mg,最大限量5mg;5岁以上每2~5min静注1mg,最大限量10mg,必要时在2~4h内可重复使用。

【注意事项】 需密切观察呼吸、心率、血压,注意翻身和吸痰。

\textbf{拉莫三嗪:}

【适应证】 属叶酸拮抗剂,为强效抗癫痫
药,对难治性癫痫 有显著疗效。

【用法和用量】 成人口服每天50mg,2周后增加为每次50~100mg,每日2次;小儿2岁以上,开始每日2mg/kg,维持量5~15mg/kg。

【不良反应】 常见不良反应为皮疹、头晕、头痛、复视及恶心等,大多数不需要特殊处理或停药。

\textbf{奥卡西平:}

【适应证】 为卡马西平的衍生物,吸收迅速而安全,是一种新型治疗难治性癫痫
药物。

【用法和用量】 适合于单独或与其他的抗癫痫
药联合使用。在单药治疗和联合用药中,本品可从临床有效剂量开始用药,一日内2次给药,根据患者的临床反应增加剂量。

【不良反应】 皮肤过敏、头晕、发音困难、复试、疲劳、嗜睡、恶心等,多与高剂量和长时间用药有关。

\textbf{加巴喷丁:}

【适应证】 结构类似γ-氨基丁酸,可控制各种癫痫
发作。

【用法和用量】 第1次睡前服300mg。以后每日增加300mg,用量可以高达每日3600mg,上述剂量需分3次服用。

【不良反应】 有嗜睡、头晕、共济失调、疲劳、眼球震颤、复视、恶心、呕吐等。

\section{帕金森病}

\subsection{定义}

帕金森病(parkinson
disease,PD)又称震颤麻痹,是一种多发于中老年人的慢性退行性神经系统疾病。其主要病变在黑质-纹状体多巴胺神经通路,黑质多巴胺能神经元变性,导致纹状体内的多巴胺不足。临床表现为静止性震颤、肌强直、运动减少和姿势反射减少。

\subsection{病因及发病机制}

\subsubsection{病因}

PD的病因不明,目前认为PD是多种因素所致。遗传使其易感性增加,在环境和年龄因素的共同促进下,通过多种病理生理机制导致发病。疾病发生可能与下列因素有关。
\paragraph{脑老化}

本病主要发生于中、老年患者,并随年龄的增长而增多,40岁前少见。提示年龄老化与其发病有关。正常人每10年有13%的黑质多巴胺神经元死亡,当80%的神经元死亡时就可出现PD症状。
\paragraph{环境因素}

20世纪80年代发现一种吡啶类衍生物1-甲基-4-苯基-1,2,3,6-四氢吡啶(MPTP),在动物模型或误服MPTP的PD患者,在许多方面如病理变化、行为症状、生化改变、药物治疗反应与原发性PD患者的改变十分相似。经研究认为,环境中与MPTP结构类似的工业或农业毒素如除草剂百草枯、异喹啉等是PD发病的危险因素。
\paragraph{遗传因素}

有报道约10%PD患者有家族史,部分呈常染色体显性或隐性遗传。细胞色素P450酶2D6亚型基因变异可能是其易感因素之一。

\subsubsection{发病机制}

发病机制十分复杂,是由纹状体内缺乏多巴胺所致,主要病变在黑质-纹状体多巴胺能神经通路。黑质中多巴胺能神经元发出上行纤维到达纹状体(尾状核及壳核),其末梢与尾状核中也有胆碱能神经元,与尾状-壳核神经元所形成的突触以乙酰胆碱为递质,对脊髓前角运动神经元起兴奋作用。正常时,两种递质处于平衡状态,共同调节运动功能。PD患者因黑质有病变,多巴胺合成减少,使纹状体内含量降低,造成黑质-纹状体通路多巴胺能神经功能减弱,而胆碱能神经功能相对占优势,因而产生PD的肌张力增高症状。

\subsection{临床表现}

多于50~60岁起病,男性略多于女性。起病缓慢,症状逐渐加重,主要症状有震颤、肌强直、运动迟缓和姿势反射减少。

\subsubsection{震颤}

往往是本病的首发症状,肢体和头面部不自主震颤,这种震颤在安静时尤为明显,故又称静止性震颤。病情严重时震颤呈持续性,只有在睡眠后消失。

\subsubsection{肌强直}

四肢、躯干、颈部、面部的肌肉均可发生强直,患者表现出一种特殊姿势:头部前倾,躯干俯屈,前臂内收,下肢髋及膝关节略为弯曲,手指内收,腕关节和指间关节伸直,拇指对掌,称“帕金森手”。

\subsubsection{运动减少}

主要表现为随意运动减少,始动困难和动作缓慢,如转弯和行走困难。不能完成精细动作,表现为写字越写越小,称为“小写症”。日常工作不能自理,系纽扣和鞋带及洗脸等困难。面肌运动减少,表情呆板,称为“面具脸”。

\subsubsection{姿势反射减少}

走路时双上肢前后摆动的“联合动作”减少,甚至不摆动。步态障碍表现为起步较难,一旦迈步,即以碎步向前冲,不能及时停步,称为“慌张步态”。姿势转变也有障碍。

\subsection{治疗}

药物治疗用药原则如下:①最小剂量,最佳效果;②对症用药,酌情加减;③长期服药,减轻症状;④权衡利弊,联合用药。

\subsection{常用药物}

\textbf{左旋多巴:}

【适应证】 对运动减少和强直的疗效最佳,但对震颤的治疗效果不肯定。

【用法和用量】 口服:震颤麻痹,开始每次0.25~0.5g,每日3或4次,每隔3或4日增加0.125~0.5g。维持量每日3~6g,分3或4次服,餐前0.5h或餐后1.5h服用,餐前服用比餐后疗效好。

【不良反应】 近期有胃肠道症状、心血管症状、直立性低血压等外周性不良反应,远期有运动功能波动、睡眠障碍、精神症状等中枢性不良反应。

\textbf{金刚烷胺:}

【适应证】 用于不能耐受左旋多巴治疗的震颤麻痹患者。

【用法和用量】 口服:成人每次0.1g,早晚各1次,最大剂量每日400mg。小儿用量酌减,可连用3~5d,最多10d。

【不良反应】 不良反应少,少数患者服后可有嗜睡、眩晕、抑郁、食欲减退等,也可出现四肢皮肤青斑,踝部水肿等。

\textbf{溴隐亭:}

【适应证】 多巴胺受体激动剂,早期单用于未经左旋多巴治疗的帕金森病,以延迟左旋多巴的使用;晚期帕金森病患者长期使用左旋多巴,出现疗效减退,并发异常不自主运动,可减轻并发症的程度,减少每日左旋多巴的用量。

【用法和用量】 每次1.25~2.5mg,每日2次;2~4周内,每周增加每日2.5mg,直到每日10~20mg的最合适剂量。

【不良反应】 食欲不振、恶心、呕吐、头晕、口干、便秘或腹泻、幻觉、心律失常等。

\textbf{司来吉兰:}

【适应证】 预防早期帕金森患者,可阻滞多巴胺的氧化反应,从而减少自由基的产生。

【用法和用量】 口服:每次2.5mg,每日1次,逐渐增加至每次2.5mg,每日2次,再加至每次5mg,每日2次。

【不良反应】 出现精神障碍、意识模糊、智能减退、幻觉等症状;还有恶心、口干、失眠、头晕、直立性低血压等症状,因其有兴奋作用,应避免晚间服用。

\textbf{托卡朋:}

【适应证】 用于长期使用复方多巴胺制剂后疗效减退、开关现象明显的帕金森病患者。

【用法和用量】 口服:每次100~200mg,每日3次。

【不良反应】 运动障碍、恶心、肌痉挛、失眠、转氨酶升高等。

\textbf{苯海索:}

【适应证】 用于帕金森病、帕金森综合征;也可用于药物引起的锥体外系疾患。

【用法和用量】 帕金森病、帕金森综合征:开始每日1~2mg,以后每3~5d增加2mg,至疗效最好而又不出现不良反应为止,一般每日不超过10mg,分3或4次服用,极量每日20mg。药物诱发的锥体外系疾患,每日2~4mg,分2或3次服用,以后视需要及耐受情况逐渐增加至5~10mg。

【不良反应】 常见口干、视物模糊等,偶见心动过速、恶心、呕吐、尿潴留、便秘等。长期服用可出现嗜睡、抑郁、记忆力下降、幻觉等。

\section{阿尔茨海默病}

\subsection{定义}

阿尔茨海默病(Alzheimer's
disease,AD)即所谓的老年痴呆症,是一种进行性发展的致死性神经退行性疾病,临床表现为认知和记忆功能不断恶化,日常生活能力进行性减退,并有各种神经精神症状和行为障碍。目前已广泛应用的抗痴呆药物有乙酰胆碱酯酶抑制剂和N-甲基-D-天门冬氨酸(NMDA)受体拮抗剂等。

\subsection{病因及发病机制}

阿尔茨海默病患者的大脑表现出脑萎缩现象,中枢神经系统内神经元和神经突触明显减少或消失,这种改变在与认知能力相关区域如海马及相关皮质部位尤为明显。脑组织布满神经元内纤维缠结、衰老斑并沉积大量淀粉样β蛋白。许多神经递质,如乙酰胆碱、5-羟色胺、去甲肾上腺素、多巴胺、P物质等减少也与阿尔茨海默病发病有关。在复杂的阿尔茨海默病病因研究中,发现高龄老化及遗传因素明确与阿尔茨海默病发病有关。

\subsection{临床表现}

阿尔茨海默病起病隐匿,为特征性、进行性病程,无缓解,由发病至死亡平行病程约8~10年,但也有些患者病程可持续15年或以上。阿尔茨海默病的临床症状分为两方面,即认知功能减退症和非认知性精神症状。

\subsection{治疗}

控制行为改变是治疗阿尔茨海默病的重要目标,因为大约90%的患者行为异常,用药可有抗精神病、抗抑郁症和狂躁症等药物。

\subsection{常用药物}

\textbf{多奈哌齐:}

多奈哌齐是脑内AchE的可逆性抑制药,使脑内Ach量增加,改善脑细胞功能。

【适应证】 轻度或中度阿尔茨海默症痴呆症状。

【用法和用量】 口服:开始时每日睡前服用5mg,如需要一个月后可增加到最大剂量每日10mg。

【不良反应】 常见的不良反应有恶心、腹泻、疲劳和肌肉痉挛,这些反应轻微、短暂,连续服药2~3周后自行消失。

【注意事项】 轻中度肝功能不全者宜适当调整剂量;病窦综合征或其他室上性心脏传导阻滞、消化道溃疡者以及哮喘、慢性阻塞性肺病者慎用。

【禁忌证】 孕妇及对本品过敏者。

\textbf{石杉碱甲:}

石杉碱甲是从天然植物中提取的一种生物碱,是一种高选择性胆碱酯酶抑制药。

【适应证】 良性记忆障碍,对痴呆患者和脑器质性病变引起的记忆障碍也有改善作用。

【用法和用药】 口服:常用量每次0.1或0.2mg,每日2次,每日最大剂量0.45mg。

【不良反应】 偶见头晕、恶心、胃肠道不适、乏力、视力模糊。

【注意事项】 心动过缓、支气管哮喘者慎用;治疗应从小剂量开始,逐渐增量。

【禁忌证】 癫痫
、肾功能不全、机械性肠梗阻、心绞痛者。

\textbf{美金刚:}

美金刚是N-甲基-D-天门冬氨酸受体拮抗药,影响谷氨酰胺传递。

【适应证】 中到重度阿尔茨海默症。

【用法和日用量】 口服:起始剂量为每早5mg,每周增加5mg直至最大剂量每次10mg,每日2次;一旦剂量超过每日5mg,则应分为2次服用。

【不良反应】 常见便秘、高血压、头痛、眩晕、嗜睡。

【注意事项】 肌酐清除率在10~60mL/min者,应减量至每日10mg,建议肌酐清除率小于每分钟10mL的患者应避免使用本品。癫痫
患者、惊厥史患者、孕妇慎用。


