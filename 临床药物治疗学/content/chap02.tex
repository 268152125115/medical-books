\chapter{影响药物作用的常见因素}

药物进入人体被吸收后发挥作用是受多种因素影响和制约的,尤其是在临床治疗过程中,影响的因素更为复杂和多变,如患者个体的因素、社会心理因素、药物因素和给药方法等,都可能增强或减弱药物的疗效,甚至还会产生不良反应。因此,了解和掌握这些影响因素,可以更好地进行个体化用药,充分发挥药物的治疗效应,减少不良反应,达到安全、有效地防治疾病的目的。

\section{机体因素}

\subsection{生理因素}

\subsubsection{年龄}

由于儿童和老年人的生理功能与成人有较大差异,因此,国家药典规定年龄在14岁以下患儿的用药剂量为儿童剂量,14~60岁患者用成人剂量,60岁以上患者用老年人剂量。儿童剂量和老年人剂量应以成人剂量为参考酌情减量。
\paragraph{儿童}

儿童的各个器官和组织正处于生长、发育阶段,年龄越小器官和组织的发育越不完全。药物使用不当会造成器官和组织发育障碍,甚至会造成严重损伤,并可能产生后遗症。例如,使用吗啡、哌替啶极易出现呼吸抑制,而对尼可刹米、氨茶碱、麻黄碱等又容易出现中枢兴奋而致惊厥。氨基糖苷类抗生素对第8对脑神经的毒性作用极易造成听觉损害。据有关资料报道,国内聋哑患者病因调查结果表明,由此类药物应用不当所致约占60%。氟喹诺酮类药物因含氟可影响骨骼和牙齿的发育生长,故对婴幼儿应慎重使用。儿童体液占体重比例较大而对水盐代谢的调节能力差。如高热时使用解热药不当引起出汗过多极易造成脱水。此外,儿童还对利尿药特别敏感,易致电解质平衡紊乱。
\paragraph{老年人}

老年人的组织器官及其功能随年龄增长而出现生理性衰退,对药物的药效学和药动学产生影响。老年人体液相对减少,脂肪增多,蛋白质合成减少。如丙戊酸钠在老年人血液中游离药物浓度明显高于年轻人,其原因一是白蛋白含量减少,二是白蛋白对药物的亲和力明显降低,三是器官清除能力下降。肝、肾功能随着年龄增长而逐渐衰退,药物代谢和排泄速率相应减慢。因此,老年人使用抗生素时,应根据肝、肾功能状况调整给药剂量。老年人除生理功能逐渐衰退外,多数伴有不同程度的老年性疾病,如心脑血管病、糖尿病、痴呆症、骨代谢疾病、前列腺肥大、胃肠疾病等,对作用于中枢神经系统药物、心血管系统药物等比较敏感。如有心脑血管病的老年人在拔牙时禁用含肾上腺素的局部麻醉(局麻)药。苯丙醇胺易诱发老年人卒中、心肌梗死、肾功能不全等,说明老年人有心脑血管病、肾病者不宜使用含有这类药物的复方制剂。

\subsubsection{体重}

年龄差异导致体重存在明显差别,即使在同年龄段内体重也会有一定的差别,这种差别可影响药物作用。如果服药者的体形差别不大而体重相差较大时,给予同等剂量药物则体重较轻者血药浓度明显高于重体重者。反之,当体重相近而体形差别明显时,则药物的水溶性和脂溶性在两者的体内分布情况就有差别。因此,比较科学的给药剂量应以体表面积为计算依据,它既考虑了体重因素又考虑了体形因素,如婴幼儿用药一般均采用体表面积来计算。

\subsubsection{性别}

男女性别不同对药物的反应在正常情况下无明显差别,但女性在特殊生理期间,如月经期、妊娠期和哺乳期对药物作用的反应与男性有很大差别。女性在月经期,子宫对泻药、刺激性较强的药物、引起子宫收缩的药物敏感,容易引起月经过多、痛经等反应。在妊娠期使用上述药物还容易引起流产、早产等。此外,有些药物还能通过胎盘进入胎儿体内,对胎儿的生长发育和活动造成影响,严重的可导致畸胎,故妊娠期用药应十分慎重。在分娩前用药应注意药物在母体内的维持时间,一旦胎儿离开母体,则药物无法被母体消除,引起药物在新生儿体内滞留而产生不良反应。在哺乳期的妇女,有些药物可通过乳汁被婴儿摄入体内而引起药物反应。

\subsubsection{个体差异}

有些个体对药物反应非常敏感,所需药量低于常用量,称为高敏性。反之,有些个体需使用高于常用量的药量方能出现药物效应,称为低敏性。

某些过敏体质的人用药后发生过敏反应,又称变态反应。变态反应是机体将药物视为一种外来物所发生的免疫反应。这种反应与药理效应无关,且无法预先知道,仅发生于少数个体。轻度的可引起发热、药疹、局部水肿,严重的可发生剥脱性皮炎(如磺胺药)、过敏性休克(如青霉素)。对于易产生严重过敏反应的药物用药前应做皮肤试验,阳性者禁用,即使阴性者也应小心应用。

还有一类特异体质的人对某些药物发生特异性反应,称为特异质反应。特异质反应是由于这类人的遗传异常所致。如骨骼肌松弛药琥珀胆碱引起的特异质反应是由于先天性缺乏血浆胆碱酯酶所致。

\subsection{心理因素}

心理因素主要指患者心理活动变化可对药物治疗效果产生影响。其显著特点是:①患者受外界环境、医师和护士的语言、表情、态度、信任程度、技术操作熟练程度、工作经验、暗示性等的影响产生心理活动变化,从而影响药物治疗效果。心理因素对药物治疗效果的影响大约占35%~40%。②心理因素的影响主要发生在慢性病、功能性疾病及较轻的疾病中,在重症和急症治疗中影响程度很小。例如对轻微疼痛采用一般的安慰性措施效果明显,而对剧烈疼痛无效。③心理因素的影响往往与患者的心理承受能力有关。对承受能力强的患者影响相对较小,对承受能力弱的患者影响则较大。④心理因素还有先入为主的特点。如果一个医师告诉患者某药物对其的病情治疗效果不理想时,无论其他医师反复说明也不容易被接受,从而影响该药的效果。⑤心理因素的影响不仅发生在人,在动物身上也存在近似的现象。

除了心理活动变化以外,患者对药物效应的反应能力、敏感程度、耐受程度也对药物治疗效果产生一定的影响。如对疼痛敏感者和不敏感者在应用镇痛药后产生的效果就有很大差异。另外,患者与医护人员的医疗合作是否良好对药物疗效也有着重要的影响。

\subsection{病理因素}

\subsubsection{心脏疾病}

心力衰竭时药物在胃肠道的吸收下降、分布容积减少、消除速率减慢。如普鲁卡因胺的达峰时间由正常时的1h延长至5h,生物利用度减少50%,分布容积减少25%,血药浓度相对升高。清除率由正常时的400~600mL/min降至50~100mL/min,半衰期{t}
{1/2} 由3h延长至5~7h。

\subsubsection{肝脏疾病}

有些药物需在肝脏转化成活性物质发挥疗效。肝功能不全时这种转化作用减弱,致使血药浓度降低,疗效下降。故肝功能障碍时宜选用氢化可的松或泼尼松龙而不选用可的松或泼尼松。

\subsubsection{肾脏疾病}

卡那霉素主要经肾排泄,正常人半衰期$t_{1/2}$
为1.5h,而肾衰患者延长至25h。若不改变给药剂量或给药间隔,势必会造成药物在体内的蓄积,还会造成第8对脑神经的损害,引起听力减退,甚至导致药源性耳聋。

\subsubsection{胃肠疾病}

胃肠道内的pH值改变可对弱酸性和弱碱性药物的吸收带来影响。胃排空时间延长或缩短也可使在小肠吸收的药物延长或缩短吸收时间。腹泻时常使药物吸收减少,而便秘可使药物吸收增加。

\subsubsection{营养不良}

如血浆蛋白含量下降可使血中游离药物浓度增加,引起药物效应增加。

\subsubsection{酸碱平衡失调}

主要影响药物在体内的分布。当呼吸性酸中毒时血液pH值下降,可使血中苯巴比妥(弱酸性药物)解离度减少,易于进入细胞内液。

\subsubsection{电解质紊乱}

\ce{Na^+} 、\ce{K^+} 、\ce{Ca^2+} 、\ce{Cl^-}
是细胞内、外液中的主要电解质,当发生电解质紊乱时它们在细胞内、外液的浓度将发生改变,影响药物的效应。如当细胞内缺\ce{K^+}
时,心肌细胞最易对强心苷类药物产生心律失常的不良反应。\ce{Ca^2+}
在心肌细胞内减少时,使用强心苷类药物加强心肌收缩力的作用降低;若\ce{Ca^2+}
浓度过高时该类药物易致心脏毒性。胰岛素降低血糖时也需要\ce{K^+}
协助使血中葡萄糖易于进入细胞内。

\subsection{遗传因素}

药物作用的差异有些是由遗传因素引起的。如前述的高敏性、低敏性和特异质反应皆与遗传因素有关。许多药物如安替比林、双香豆素、保泰松、苯妥英、去甲替林、异烟肼、对氨基水杨酸、磺胺、普鲁卡因胺、硝基地西泮、肼屈嗪、甲基硫氧嘧啶、华法林、伯氨喹、阿司匹林、对乙酰氨基酚、呋喃类等其作用均受到遗传因素的影响。

\subsection{时间因素}

人体的生理生化活动往往随着不同季节及时间的改变而发生有规律的周期性变化,从而对药物疗效产生影响。很多药物如中枢神经系统药物、心血管系统药物、内分泌系统药物、抗肿瘤药物、抗菌药物、平喘药物等均有昼夜时间节律变化。例如,相同剂量的镇痛药分别于白天和夜间给人用药,其镇痛效果为白天高、夜间低。胃酸的分泌高峰在夜间,某些患胃溃疡的患者易在夜间发病,将H{2}
受体阻断药西咪替丁在夜间用药能有效抑制胃酸分泌,减少发病。根据药物的时间节律变化来调整给药方案具有重要的临床意义。如肾上腺皮质激素分泌高峰出现在清晨,血浆浓度在08∶00左右最高,而后逐渐下降,直至00∶00左右达最低。临床上根据这种节律变化将皮质激素药物由原来的每日分次用药改为每日08∶00一次给药,提高了疗效,大大减轻了不良反应。

\subsection{生活习惯与环境}

饮食对药物的影响主要表现在饮食成分、饮食时间和饮食数量。一般来说,药物应在空腹时服用,有些药物因对消化道有刺激,在不影响药物吸收和药效的情况下可以饭后服用,否则应在饭前服用或改变给药途径。食物成分对药物也有影响,如高蛋白饮食可使氨茶碱和安替比林代谢加快;低蛋白饮食可使肝药酶含量降低,导致多数药物代谢速率减慢,还可使血浆蛋白含量降低,血中游离药物浓度升高。吸烟对药物的影响主要是烟叶在燃烧时产生的多种化合物可使肝药酶活性增强,使药物代谢速率加快。经常吸烟者对药物的耐受性明显增强。长期小量饮酒可使肝药酶活行增强,药物代谢速率加快;急性大量饮酒使肝药酶活性饱和或降低,导致其他药物的代谢速率减慢。饮茶主要影响药物的吸收,茶叶中的鞣酸可与药物结合减少其吸收。

\section{药物因素}

\subsection{药物理化性质}

因药物的溶解性各不相同,故根据临床需要将药物制备成不同的剂型。每种药物都有保存期限,超过期限药物性质可发生改变而失效。如青霉素G在干粉状态下有效期为3年,而在水溶液中极不稳定,需临用前配制。有些药物需在常温下干燥、密闭、避光保存;个别药物还需要在低温下保存,保存不当易挥发、潮解、氧化或光解。如乙醚易挥发、易燃;维生素C、硝酸甘油易氧化;肾上腺素、去甲肾上腺素、硝普钠、硝苯地平易见光分解等。

\subsection{药物剂型}

每种药物都有其适宜的剂型给药以产生理想的药效。同种药物的不同剂型对药物的疗效亦有不同的影响,如片剂、胶囊、口服液等均可口服给药,但因药物崩解、溶解速率不同,吸收快慢和吸收量就会不同。注射剂中水剂、乳剂、油剂在注射部位释放速率不同,药物起效快慢和维持时间也就不同。不同厂家生产的同种药物制剂由于制剂工艺配方不同,药物的吸收情况和药效情况也有差别。随着生物制剂学的发展,近年来为临床提供了一些新的制剂,如缓释剂、控释剂。这些制剂的特点是能够缓慢持久释放药物,保持血药浓度的基本稳定,从而产生持久药效。透皮贴剂就是其中的一种,如硝酸甘油透皮贴剂、芬太尼透皮贴剂等。

\subsection{给药方法}

\subsubsection{给药剂量}

剂量指用药量。随着剂量的加大,效应逐渐增强,若超出最大治疗剂量时,便会产生药物的不良反应或毒性反应。如镇静催眠药在小剂量时出现镇静效应,随着剂量的增加,可依次出现催眠、麻醉甚至死亡。

\subsubsection{给药途径}

给药途径不同,药物的吸收和分布也就不同,药物作用效应就会产生差异。个别药物甚至出现药物效应方面的改变,如硫酸镁。
\paragraph{消化道给药}

(1)口服给药。这是最常用的给药方法,药物经胃肠黏膜吸收。其优点为方便、经济,较注射给药相对安全,无感染发生。其缺点是许多药物易受胃肠内容物影响而延缓或减少吸收,有的药物可发生首过消除,使生物利用度降低,有的药物甚至根本不能吸收。另外,口服给药不适合昏迷、呕吐、抽搐等急重症患者及不合作者。

(2)舌下给药。药物通过口腔舌下黏膜丰富的毛细血管吸收,可避免胃肠道刺激、吸收不全和首过消除,但要求药物溶解快,无异味,用量少。如硝酸甘油片舌下给药缓解心绞痛急性发作。

(3)直肠给药。将药栓或药液导入直肠内由直肠黏膜血管吸收,可避免胃肠道刺激及首过消除。此法较适宜小儿给药,可以避免小儿服药时的困难及胃肠刺激。目前国内适于小儿直肠给药的药物栓剂很少,限制了其使用。
\paragraph{注射给药}

(1)肌肉注射。药物在注射部位通过肌肉丰富的血管吸收入血,吸收较完全,起效迅速,其中吸收速度为水溶液>混悬液>油溶液。

(2)皮下注射。药物经注射部位的毛细血管吸收,吸收较快且完全,但对注射容量有限制,且仅适用于水溶性药物,如肾上腺素皮下注射抢救青霉素过敏性休克。

(3)静脉注射或静脉滴注。因药物直接进入血液循环而迅速起效,适用于急重症患者的治疗。但静脉给药对剂量、配伍禁忌和给药速度有较严格的规定。

(4)椎管内给药。将药物注入蛛网膜下腔的脑脊液中产生局部作用,如有些外科手术需要做蛛网膜下腔麻醉(腰麻)。也可将某些药物注入脑脊液中产生疗效,如抗菌药物等。
\paragraph{呼吸道给药}

即吸入给药。某些挥发性或气雾性药物常采用此种给药方法,主要是通过肺泡扩散进入血液而迅速起效。如全身麻醉药用于外科手术,异丙肾上腺素气雾剂治疗支气管哮喘急性发作等,缺点是对呼吸道有刺激性。
\paragraph{皮肤黏膜用药}

将药物施放于皮肤、黏膜局部发挥局部疗效,如外用擦剂、滴眼剂、滴鼻剂等。有的药物可通过透皮吸收发挥全身疗效,如硝酸甘油贴膜剂贴敷于心前区,药物通过透皮缓慢吸收,从而起到预防心绞痛发作的作用。

\subsubsection{给药时间}

不同的药物给药时间有可能不同。有的药物对胃刺激性强,应于饭后服。催眠药应在临睡前服;胰岛素应在饭前注射;有明显生物节律变化的药物应按其节律用药,如糖皮质激素类药。

\subsubsection{给药间隔}

一般以药物的半衰期($t_{1/2}$
)为参考依据。但有些药物例外,如青霉素的$t_{1/2}$
为30min,由于该药对人毒性极低,大剂量给药后经过数个$t_{1/2}$
后血药浓度仍在有效范围以内,加之抗菌药物大多都有抗菌后效应,在此期间细菌尚未恢复活力,因此其给药间隔可适当延长。另外,肝、肾功能不全者可适当调整给药间隔时间。给药间隔时间短易致累积中毒;反之,给药间隔时间延长血药浓度波动加大。

\subsubsection{疗程}

疗程指给药持续时间。对于一般疾病和急重症患者,症状消失后即可停止用药;对于某些慢性病及感染性疾病应按规定的持续时间用药,以避免疾病复发或加重。

\subsection{药物相互作用}

药物相互作用是指两种或两种以上药物同时或先后应用所出现的药物效应增强或减弱的现象。

药物在体外发生相互影响称为配伍禁忌。是指将药物混合在一起发生的物理或化学反应,尤其容易发生在几种药物合在一起静脉滴注时。如氨基糖苷类抗生素与β内酰胺类抗生素合用时二者不能放在同一针管或同一溶液中混合,因为β内酰胺可使氨基糖苷类失去抗菌活性。红霉素只能在葡萄糖溶液中静脉滴注,若在生理盐水溶液中易析出结晶和沉淀。

药物在体内发生相互影响称为相互作用。主要表现在药动学和药效学方面。药物相互作用的结果只有两种,或使原有的效应增强称为协同作用,或使原有的效应减弱称为拮抗作用。在药动学方面的影响主要发生在吸收、分布、代谢和排泄过程。如服用抗酸药改变胃液pH值可减少弱酸性药物吸收。吗啡、阿托品减弱肠蠕动可延长药物在肠道中停留时间而增加吸收。若食物中重金属离子(Mg{2+}
、\ce{Ca^2+} 、Al{3+} 、Fe{2+}
)较多时易与某些药物形成配合物而减少吸收。华法林和保泰松可发生血浆蛋白竞争性结合,从而使华法林血浆游离浓度增加,导致抗凝血效应加强。能改变尿液pH值的药物可以减少或增加弱酸性或弱碱性药物的重吸收。共同通过肾小管主动分泌排泄的药物联合用药也会发生竞争性抑制,使药效时间延长。药效学方面的影响主要发生在药物作用部位。如受体激动药和受体拮抗药可在同一受体部位产生竞争性拮抗效应。氢氯噻嗪和螺内酯均为利尿药,合用后氢氯噻嗪排钾的不良反应可以被螺内酯拮抗,利尿效应增强。磺胺嘧啶与甲氧苄啶合用后,通过对细菌叶酸代谢的双重阻断作用,使抗菌效应增强。

\subsection{长期用药}

某些疾病需要长期用药,机体会相应产生一些反应。主要表现在以下三个方面。

(1)耐受性:即连续用药后出现的药物反应性下降。若在很短时间内产生称为急性耐受性,停药后可以恢复,如麻黄碱、硝酸甘油、垂体后叶素等。反之,若在长期用药后产生则称为慢性耐受性,如抗高血压药、降血糖药、苯巴比妥等。胰岛素既可产生急性耐受性又可产生慢性耐受性。病原体和肿瘤细胞在长期用药后产生的耐受性称为耐药性。

(2)依赖性:指长期用药后患者对药物产生主观和客观上需要连续用药的现象。若仅产生精神上的依赖性,停药后患者只表现为主观上的不适,没有客观上的体征表现,称为习惯性,如镇静催眠药。若患者对药物不但产生精神依赖性,还有躯体依赖性,一旦停药后,患者产生精神和躯体生理功能紊乱的戒断症状,称为成瘾性,如吗啡类镇痛药。

(3)撤药症状:指长期用药后突然停药出现的症状,又称停药症状。如长期应用肾上腺皮质激素突然停药不但产生停药症状(肌痛、关节痛、疲乏无力、情绪消沉等),还可使疾病复发或加重,称为反跳现象。可采取逐渐减量停药的方法避免发生撤药症状和反跳现象。

\section{患者的依从性和用药指导}

依从性(compliance)也称顺从性、顺应性,是指患者按医师规定进行治疗,与医嘱一致的行为,反映了患者对其医疗行为的配合程度,是药物治疗有效性的基础。患者能遵守医师推荐的治疗方案及服从医药人员对其进行健康指导时,就认为患者具有依从性;反之,称为非依从性。正确的药物治疗方法是治愈疾病的前提,若患者不服从治疗,不能按规定用药,则不能达到预期的目的和效果。所以,患者的依从性与患者的治疗和康复有着密切的联系,是保证药物治疗质量的一个重要条件。非依从性的危害是多种多样的,轻者贻误病情,不良反应增加,耐药性增强,导致治疗失败;重者将发生严重中毒,甚而危及生命。另外,非依从性也可能加重患者及社会的经济负担,从而使患者产生对医疗行为的不信任。

\subsection{非依从的主要类型}

(1)不按处方取药。如由于种种原因,患者擅自取舍处方中的药物。

(2)不按医嘱用药。如忘记服用;擅自更改药物剂量、用药次数、用药途径、用药时间或用药顺序及疗程等;认为疗效不好而拒服,嫌药物太贵而不服,急于求成而滥用药物等。

(3)不当的自行用药。如患者凭经验或直觉用药。

(4)重复就诊。如患者先后就诊于不同医疗机构、科室,或同时正在使用其他药物而不告知就诊医师,导致相同或者相似药物重复使用。

\subsection{产生非依从性的主要原因}

患者产生非依从性的原因较多,主要与以下因素有关。

\subsubsection{医药人员因素}

缺少与患者的沟通,对患者缺乏指导或提供的用药指导不清楚。在日常医疗工作中,常因医药人员对患者联系和指导不力而使患者出现非依从。如在用药过程中,医药人员未向患者说明药物的作用、用法用量、不良反应及注意事项,则患者可能因自我感觉疗效不佳而加大剂量,或出现不良反应而停用,也可能发生用药途径错误,如将栓剂口服或片剂当作栓剂用等。此外,医师在开具处方或书写标签时对用法说明不恰当,如“必要时服用”“遵医嘱”“同前”等均会使患者发生理解错误而造成非依从。

\subsubsection{患者因素}

患者因求治心切而盲目地超剂量用药、病情好转而中断用药、年迈残障或健忘而不能及时准确用药或重复用药、久病成医或相信他人经验而自行下药或停药、对医师缺乏信任而自行更改用药方案、担心药物不良反应或不良反应难以忍受等。同时,患者的心理因素是产生非依从性的一个重要因素。有的患者对药物治疗期望过高,健康保健要求过强,害怕受疾病折磨的痛苦,要求治疗效果快速,因而出现乱投医、乱用药,听信不规范的药品广告宣传误导,不遵医嘱,盲目自购药品服用,均会对治疗产生一定的影响。

\subsubsection{疾病因素}

有些疾病本身症状不明显,或经过一段时间治疗后症状减轻或消失,患者缺少症状提醒而导致药物漏服。

\subsubsection{药物因素}

如药片太大,使患者吞咽困难;如药片太小,使一些患者拿、掰困难;如有些药物制剂带有不良气味或颜色,使患者尤其是儿童不易接受等。

\subsubsection{药物治疗方案因素}

复杂的给药方案,如药物种类多、用药次数频繁、用药量各不相同、用药时间严格、疗程过长、用药方式不便等,均可能增加患者的非依从性。

\subsubsection{社会和经济的影响}

由于受社会上某些不良宣传广告的影响,有的患者盲目听从虚假广告的误导,乱投医,擅自乱服偏方、秘方,不但没有治好疾病,反而导致严重不良后果,致使患者对疾病治疗失去信心。有的患者家庭经济条件较差,治疗费用过高,经济上难以承受而中断或放弃治疗,或擅自换用价格低的药品,从而造成疗效较差、不良反应较多,影响治疗效果。

\subsubsection{其他}

一些特殊职业者,如驾驶员、地质勘探人员、井下作业人员、建筑施工人员等,工作和生活的不规律造成了用药的低依从性。当某些患者用药受周围人员或家属不支持的影响较大时,不按医师处方用药的情况就会增加。如儿童服药是否依从,取决于家长。另外,一般来说,门诊患者的非依从性高于住院患者。

\subsection{提高患者依从性的措施}

患者产生不依从的原因很多,改善患者的依从性应针对原因改进工作,可从以下几个方面着手。

\subsubsection{加强对患者的用药指导}

向患者提供用药指导有助于患者正确认识药物,以达到正确使用药物、发挥药物应有疗效的目的,尤其是对一些安全范围较窄、过早停用产生严重后果或需要长期使用的治疗慢性疾病的药物。在对患者进行用药指导时,应根据患者的情况,采用其容易接受的方式来提供有关药物的信息;应以患者能理解的方式进行,如使用亲切的语言、保持温和友善的态度、表现出应有的同情心等,从而使患者感到宽慰。用药指导的主要内容包括五个方面。
\paragraph{药物的作用和用途}

对患者来说,由于不理解治疗的重要性而倾向于不依从是主要原因。特别是在慢性疾病治疗和预防中,或进行强迫性治疗时,患者更易不遵循医嘱。药师应告诉患者所服药物的名称和作用等,以消除患者的疑虑,使其认识到药物治疗的必要性和重要性。
\paragraph{药物的用法、用量及用药时间}

可在患者的药瓶或药盒上注明每日几次及每次的用量,也可以效仿国外药师通过设计外包装、说明书,以及信息活页等多种形式,传递给患者药物的用法、用量等重要信息。对于口服药,应交代最佳服用时间;对于外用药,应注明正确的用法;对于干粉吸入剂、气雾剂、鼻喷剂等,应向患者演示正确的使用、贮存方法等。
\paragraph{药物的不良反应}

有些药物在治疗的同时,不良反应也很多,降低了服药的依从性。有研究表明,药师通过清楚说明药物服用方法及药物的不良反应等举措可以改善患者的依从情况。
\paragraph{相互作用}

有些患者同时患有多种疾病,联用药物品种多,有些药物联用会产生有益的作用,有些药物联用则会加重病情。
\paragraph{注意事项}

说明用药的要求;如何贮藏药品及识别药品是否过期;用药期间的食物禁忌;是否需要复诊及何时复诊;复诊时需要向医师提供什么信息等。

\subsubsection{与患者建立良好的关系,赢得患者的信任}

医务人员要熟悉患者的心理,尊重患者的感受,理解患者。

\subsubsection{简化治疗方案}

治疗方案复杂是造成患者非依从的主要原因之一。因此,治疗方案应尽可能减少药品种类和用药次数,如减少一些非必需的药物,尽可能采用长效制剂或缓释制剂等。另外,药物的用法要简单、用量易掌握,方便患者的使用。