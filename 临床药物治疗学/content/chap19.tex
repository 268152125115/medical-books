\chapter{病毒感染性疾病的药物治疗}

\section{病毒性肝炎}

病毒性肝炎是由几种不同的嗜肝病毒(肝炎病毒)引起的,以肝脏炎症和坏死病变为主的一组全身感染性疾病。该病具有传染性较强、传播途径复杂、流行面广泛、发病率高等特点。目前已确定的病毒性肝炎共有5型,分别为甲、乙、丙、丁和戊型,其他如庚型肝炎病毒、输血传播病毒,目前尚未被确定为嗜肝病毒。我国是病毒性肝炎的高发区,本章主要介绍乙型病毒性肝炎的药物治疗。

\subsection{病因及发病机制}

病毒性肝炎是由几种不同的嗜肝病毒引起的肝脏损伤和炎症的传染性肝病。在已确定的5种肝炎病毒中,甲型和戊型肝炎通过粪口途径传播,起病急、病程短、能够自愈,不会转变为慢性肝炎;乙型、丙型和丁型肝炎主要通过输血、血制品、注射和母婴间传染,起病时症状不明显,可演变成慢性,并可发展为肝硬化和原发性肝癌。

乙型肝炎的发病机制:人体受到乙型肝炎病毒(HBV)感染后,可出现不同的结局,其机制尚不完全清楚,主要由病毒和宿主之间的相互作用所决定。由于HBV不会直接引起肝细胞损害,故目前认为乙型肝炎的发病主要与宿主的免疫应答有关。

\subsection{临床表现}

HBV感染的潜伏期为30~160d,平均为60~90d,临床类型呈多样化,可表现为急性肝炎、慢性肝炎、肝功能衰竭、淤胆型肝炎或HBV慢性携带等。95%的成人HBV感染可以最终痊愈,伴有血清HBsAg消失和抗HBs抗体的出现。大约30%的成人急性HBV感染者表现为黄疸型肝炎,其中0.1%~0.5%表现为爆发性肝炎。

急性肝炎主要症状有乏力、食欲减退、恶心、呕吐、厌油、腹胀、肝区痛、尿黄等。甲、戊型肝炎起病急,多数有发热;乙、丙、丁型肝炎起病相对较缓。慢性肝炎表现为反复出现乏力、食欲减退、厌油、肝区不适、尿黄等,部分患者无明显症状。重型肝炎有严重的消化道症状、极度乏力,同时可出现肝性脑病前驱症状。淤胆型肝炎起病类似急性黄疸型肝炎,但自觉症状常较轻,常有皮肤瘙痒。急性肝炎和淤胆型肝炎可有肝肿大症状,并有压痛、肝区叩痛,部分患者可有轻度脾大,有黄疸的患者有皮肤和巩膜黄染。

\subsection{实验室检查}

肝功能检查可出现血清胆红素、ALT、AST、血清碱性磷酸酶及γ谷氨酰转肽酶水平升高。重型肝炎时黄疸迅速加深,ALT反而下降,呈胆酶分离现象,提示大量肝细胞坏死。慢性活动性肝炎及肝硬化时血清白蛋白降低、球蛋白升高。肝性脑病患者血氨明显升高。血清学检测HBV标志物如下。

\subsubsection{HBsAg(表面抗原)和HBsAb(表面抗体)检测}

HBsAg在疾病早期出现,HBsAg阳性是HBV感染的主要标志。血清HBsAb的出现,是HBV感染恢复的标志。注射过乙肝疫苗者,也可出现血清HBsAb阳性,提示已经获得HBV的特异性免疫。

\subsubsection{HBcAg和HBcAb(核心抗体)检测}

在血清中一般不能检测出HBcAg。HBcAb阳性,提示感染过HBV,可能既往感染,也可能为现感染。其滴度高,表明HBV正在复制,有传染性;滴度低,表明既往感染过HBV。

\subsubsection{HBeAg(e抗原)和HBeAb(e抗体)检测}

HBeAg阳性,提示有HBV复制,在HBV感染的早期出现,传染性强。HBeAb阳性是既往感染HBV的标志。

\subsection{诊断}

根据流行病学资料、临床症状、体征和实验室检查等,很容易诊断出HBV感染。对诊断不明的患者应争取作肝组织学检查。

\subsection{治疗}

\subsubsection{治疗原则}

消除病原、保护肝细胞、消退黄疸、促进肝细胞再生及防治并发症。

\subsubsection{非药物治疗}

适当休息、合理饮食,注意心理调节。饮食方面,蛋白量应在病变恢复需要和肝功能可耐受之间。如发生低蛋白血症、水肿及腹水时,需要给予高蛋白饮食。肝性脑病患者早期须严格限制蛋白摄入。高糖和高维生素是肝病患者的常规饮食,脂肪量不必过分限制。

\subsubsection{药物治疗}

药物治疗原则:病毒性肝炎的治疗主要有护肝治疗和抗病毒治疗。抗病毒治疗是慢性乙肝和丙肝治疗的根本措施。减轻肝脏炎症,促使肝细胞修复和功能恢复,是治疗病毒性肝炎的重要措施,包括消炎、降酶、退黄、促使肝细胞再生等。常用药物如下。
\paragraph{干扰素}

干扰素是一类由单核细胞和淋巴细胞产生的细胞因子,包括IFN-α、IFN-β和IFN-γ,它们在同种细胞上具有广谱的抗病毒、影响细胞生长,以及分化、调节免疫功能等多种生物活性。聚乙二醇干扰素α-2a(Peginterferon
alfa-2a,PEG-IFNα-2a)是聚乙二醇与重组干扰素α-2a结合而成,其延缓了干扰素α的体内排泄过程,延长了半衰期,缩小体内血药峰谷浓度差,提高了临床疗效。PEG-IFN-α与利巴韦林联合应用是目前最有效的抗HCV治疗方案,其次是普通IFN-α与利巴韦林联合疗法。

干扰素具有抗病毒、抗细胞增殖、抗肿瘤、免疫调节和抗肝脏纤维化五大作用,是HBV、HCV和HDV感染的标准治疗药物。IFN-α因抗HBV疗效优于IFN-β和IFN-γ而被广泛采用。目前,IFN治疗的主要对象是慢性乙型肝炎和丙型肝炎患者,而且在乙肝患者中,有以下情况者疗效较好:治疗前血清ALT或AST有反复波动或酶活力持续升高、血清HBeAg的P/N值偏低、病程较短者;肝脏有活动性炎症病变,无重叠感染者;无HIV感染或使用免疫抑制剂治疗者。

干扰素高剂量时可引起发热、流感样症状,可能引起轻度骨髓抑制作用。应用IFN
3~6h后,患者会出现发热、寒战等症状,部分人还可有头痛、肌肉痛和全身倦怠感。70%的患者会出现乏力症状,40%的患者会产生食欲减退。目前,尚无有效药物对抗此类慢性症状,但这些症状可以通过适量运动、足量营养摄入、足量饮水和引导患者参与社会活动来缓解。过敏体质者、孕妇及哺乳期妇女慎用。

用法及用量:慢性乙型肝炎患者用IFN 500万IU皮下注射,每日1次,4周后改为每周3次,疗程3~6个月。慢性丙型肝炎患者用IFN
300万IU联合RBV每日800mg抗病毒治疗16周或24周。
\paragraph{核苷类似物}

用于治疗慢性乙型肝炎的有拉米夫定、阿德福韦、恩替卡韦和替比夫定。

(1)拉米夫定:是很强的反转录酶抑制剂,通过抑制HBV反转录酶和(或)竞争性地抑制HBV
DNA聚合酶而抑制HBV
DNA的复制,长期治疗可以减轻症状,降低肝纤维化和肝硬化的发生率。拉米夫定治疗儿童慢性乙型肝炎的疗效与成人相似,安全性良好。对乙型肝炎肝移植患者,移植前用拉米夫定,移植后拉米夫定与乙型肝炎免疫球蛋白联用,可明显降低肝移植后HBV再感染,并可减少乙型肝炎免疫球蛋白的用量。

拉米夫定抑制HBV复制的能力强,不良反应较少和较轻,每日只需1次给药,患者的依从性好。但拉米夫定耐药发生率高,5年耐药率高达70%,停药后可能加重病情,从而限制了其长期应用。不良反应轻微,表现为轻微头痛、一过性嗜睡、恶心、疲乏、肝区不适等,且发生率较低,患者可较快适应而耐受。

用法用量:口服,每次100mg,每日1次。对于HBeAg阳性患者,拉米夫定治疗至少1年,须间隔6个月连续检测2次,确认疗效巩固方可停药;对于HBeAg阴性患者,疗程至少1年,须间隔6个月连续检测3次,直至HBV
DNA检测阴性,并且ALT水平正常方可停药。

(2)阿德福韦:为嘌呤类衍生物,经细胞酶磷酸化,形成阿德福韦二磷酸盐发挥抗病毒作用。阿德福韦是5′-单磷酸脱氧阿糖腺苷的无环类似物,对拉米夫定耐药变异的代偿期和失代偿期肝硬化患者均有效。用较大剂量时有一定肾毒性,主要表现为血清肌酐水平升高和血磷水平下降,但每日10mg剂量对肾功能影响较小。每日10mg,治疗48~96周,有2%~3%的患者血清肌酐浓度较基线值上升44.2μmol/L(0.5mg/dL)以上。因此,对应用阿德福韦治疗者,应定期监测血清肌酐和血磷浓度。阿德福韦的适应证为肝功能代偿的成年慢性乙型肝炎患者,尤其适用于需长期用药或已发生拉米夫定耐药者。对于已发生拉米夫定耐药者,目前主张在拉米夫定治疗的基础上加用阿德福韦酯,以降低阿德福韦耐药率,并提高疗效。

(3)恩替卡韦:是环戊酰鸟苷类似物。作用于HBV
DNA复制的起始、反转录和DNA正链合成等3个环节,从而抑制HBV的复制。恩替卡韦对拉米夫定耐药株、阿德福韦耐药株具有较强的抑制作用。由于恩替卡韦对HBV聚合酶具有高度选择性,故其细胞毒性极低,不良反应较少。口服吸收良好,生物利用度高,有效血浆半衰期在20~24h,血浆蛋白结合率低。成人每次0.5mg,口服,每日1次,能有效抑制HBV
DNA的复制,疗效优于拉米夫定。

(4)替比夫定:是一种新型左旋核苷类药物,化学名为β-L-2'{-}脱氧胸苷,对HBV
DNA聚合酶具有特异性的抑制作用。口服吸收良好,不受进食影响。临床不良反应较少,少数患者会出现肌酸激酶升高,治疗过程中需检测肌酸激酶水平。临床用于慢性乙型肝炎的治疗。用法:每次600mg,口服,每日1次。

替比夫定在抗病毒及临床疗效方面优于拉米夫定,治疗1年耐药率为4.5%,低于拉米夫定组。不良反应发生率和特点与拉米夫定相似,对部分患者会引起肌酸激酶水平升高。

\subsection{预防}

\subsubsection{管理传染源}

措施包括对患者进行登记及统计,对患者及其家属进行消毒、隔离和预防的指导。患者应注意个人卫生,密切接触者应进行乙肝疫苗的免疫接种。

\subsubsection{切断传播途径}

加强对献血员的筛查,对血制品进行HBsAg检测,对各种医疗器械和用具实行严格消毒,提倡使用一次性注射器、检查和治疗用具,防止医源性传播。

\subsubsection{保护易感人群}

接种乙肝疫苗是预防HBV感染的最有效方法。接种对象主要是新生儿、婴儿及高危人群。接种疫苗后有抗体应答的保护效果一般至少可持续12年,因此,一般人群不需要进行抗HBs抗体监测或加强免疫,但对高危人群可进行抗HBs抗体监测,如抗HBs抗体水平<10mIU/mL,可给予加强免疫。

\section{获得性免疫缺陷综合征}

获得性免疫缺陷综合征(acquired immunodeficiency
syndrome,AIDS)简称艾滋病,是由人类免疫缺陷病毒(human immunodeficiency
virus,HIV)引起的一种严重的传染性疾病。自1981年在美国报告首例AIDS患者后,至今全球已有199个国家和地区报告HIV感染者或艾滋病患者。中国内地自1985年6月发现首例外籍艾滋病患者以来,大致经历了三个阶段:1985---1989年的散发期,1990年代前半期的局部流行期,1990年代后半期至今的快速增长期。现在我国艾滋病流行呈现四大特点:疫情上升速度有所减缓;性传播逐渐成为主要传播途径;疫情地区分布差异大;流行因素广泛存在。如不加以有效控制,感染人数将急剧增加。

\subsection{病因及发病机制}

HIV是RNA病毒,属反转录病毒科。HIV又分为HIV-1和HIV-2两个型,核酸序列的同源性为40%。HIV-2目前主要流行于西非,HIV-1则广泛分布于世界各地。在大部分国家中,人们所指的艾滋病病毒都是指HIV-1,常用HIV表示。HIV-1选择性侵犯CD4{+}
T淋巴细胞,也能感染B细胞、各种神经胶质细胞及骨髓干细胞,是引起艾滋病的主要病毒株。

HIV感染者是本病的传染源。无症状HIV感染者及艾滋病患者均具有传染性。本病的传播途径多种多样,但一般日常生活接触不会感染艾滋病,常见的传播途径主要是通过性接触、血源传播和母婴传播。其他少见的传播途径还有经破损的皮肤、牙刷、刮脸刀片、口腔科操作及应用HIV感染者的器官移植或人工授精等。

人群普遍易感,成人高危人群包括静脉注射吸毒者,同性恋、性滥交或卖淫嫖娼者,血友病或经常接受输血、血制品患者,器官移植者,非法采供血者等。感染者中男女性别差异已趋接近,发病年龄主要为40岁以下的青壮年。

HIV侵入人体后,直接侵犯人体免疫系统,攻击和杀伤的是人体免疫系统中最重要、最具有进攻性的T{4}
淋巴细胞,与其高亲和力结合,吸附于宿主细胞上,病毒包膜与细胞膜融合,病毒RNA进入细胞,并与细胞核的遗传物质DNA整合为一体,HIV随免疫细胞DNA复制而复制。病毒的繁殖和复制使免疫细胞遭到破坏和毁灭,并放出更多的病毒。新增殖病毒再感染更多的细胞。就这样,病毒一代代地复制、繁殖,免疫细胞不断死亡,使人体发生多种不可治愈的感染和肿瘤,最后导致被感染者的死亡,病死率极高。

\subsection{临床表现}

本病的临床无症状期长短不一,一般为2~10年。HIV感染人体后分为四期:急性感染期、潜伏期、艾滋病前期和典型艾滋病期。不是每个患者都会出现完整的四期表现,但每个疾病阶段的患者在临床上都可以见到。

\subsubsection{Ⅰ期(急性感染期)}

HIV感染可能是无症状或者仅引起短暂的非特异性的症状,如发热、皮疹、淋巴结肿大,还会出现乏力、出汗、恶心、呕吐、腹泻、咽炎等,有的还出现急性无菌性脑膜炎,表现为头痛、神经症状和脑膜刺激征。急性感染期症状常较轻微,容易被忽略。

\subsubsection{Ⅱ期(HIV无症状潜伏期)}

急性感染期后,临床上出现一个长短不等、相对健康、无症状的潜伏期。此期持续时间一般为6~8年,其时间长短与感染病毒的数量、型别、感染途径、机体免疫状况、营养条件及生活习惯等因素有关。感染者无任何症状,但病毒在持续繁殖,具有强烈的破坏作用和传染性。在无症状期,由于HIV在感染者体内不断复制,免疫系统受损,CD4{+}
T淋巴细胞计数逐渐下降。

HIV感染人体初期,血清中抗HIV抗体尚未产生,此时临床检测抗HIV抗体常呈阴性,称为窗口期。此期一般数周到3个月。

\subsubsection{Ⅲ期(艾滋病前期)}

此期开始出现与艾滋病有关的症状和体征,患者已具备了艾滋病的最基本特点,即细胞免疫缺陷,仅症状较轻而已。主要的临床表现有除腹股沟淋巴结外,全身其他部位两处或两处以上淋巴结肿大,常伴有疲乏、发热、全身不适和体重减轻等。患者经常出现各种特殊性或复发性的非致命感染,而反复感染会加速病情进展,进入典型的艾滋病期。

\subsubsection{Ⅳ期(艾滋病期)}

是艾滋病的最终阶段。本期主要表现有一般症状(发热、乏力、全身不适、盗汗、厌食、体重下降>10%)、严重的细胞免疫缺陷导致的各种致命性机会性感染,以及因免疫缺陷而继发恶性肿瘤。在此阶段,免疫功能全面崩溃,患者出现各种严重的综合病症,直至死亡。常见的机会性感染包括由卡氏肺囊虫、巨细胞病毒、隐孢子虫等引起的卡氏肺孢子虫肺炎、巨细胞病毒性视网膜炎及隐球菌脑膜炎;恶性肿瘤常为卡波西肉瘤和淋巴瘤等。

\subsection{实验室检查}

确诊艾滋病最重要的根据是患者的血样检测结果。

\subsubsection{HIV血清抗体检测}

大多数HIV感染者在3个月内血清抗体转阳,因此测定血清抗体是目前确定有无HIV感染最简便、快速而有效的方法。抗体检测主要有ELISA和免疫荧光试验。为防止假阳性,可做蛋白印迹法进一步确证;快速检测法,也称金标法,也是抗体检测的一种方法。

\subsubsection{病毒抗原检测}

由于抗体出现晚于抗原,因而不能早期诊断。如果在抗体出现之前的窗口期筛选献血员,就会出现假阴性,后果非常严重。因此,筛选献血员最好加测病毒抗原。在HIV感染早期尚未出现抗体时,血中就有p24抗原,可用ELISA检测该抗原存在。用PCR法检测HIV基因,具有快速、高效、敏感和特异等优点,目前该法已被应用于HIV感染早期诊断及艾滋病的研究中。

\subsubsection{病毒核酸检测}

HIV阳转之前的窗口期,还可以用反转录聚合酶链反应技术检测HIV
RNA。此法灵敏度更高,检测周期短,也有助于早期诊断、筛选献血员、药物疗效考核等。

\subsection{诊断}

HIV/AIDS的诊断需结合流行病学史(包括不安全性生活史、静脉注射毒品史、输入未经抗HIV抗体检测的血液或血制品、HIV抗体阳性者所生子女或职业暴露史等)、临床表现和实验室检查等进行综合分析,慎重做出诊断。

\subsection{治疗}

迄今尚无彻底清除HIV的药物,因此本病的治疗强调综合治疗,包括一般治疗、抗病毒治疗、恢复或改善免疫功能的治疗及机会性感染或恶性肿瘤的治疗。

\subsubsection{一般治疗}

对无症状HIV感染者,可保持正常的工作和生活,但需密切监测病情的变化。对艾滋病前期或艾滋病患者,应根据病情卧床休息,给予高能量、多维生素饮食,不能进食者,应静脉输液补充营养。

\subsubsection{抗病毒治疗}

抗病毒治疗是艾滋病治疗的关键。目前应用于临床的艾滋病治疗药物有6大类29种,包括核苷类反转录酶抑制剂(nucleoside
reverse transcriptase
inhibitors,NRTIs)、非核苷类反转录酶抑制剂(non-nucleoside reverse
transcriptase inhibitors,NNRTIs)、蛋白酶抑制剂(protease
inhibitors,PIs)、融合抑制剂(fusion
inhibitors,FIs)、病毒进入抑制剂(entry
inhibitors,EIs)和整合酶抑制剂(integrase inhibitors,IIs)。

单一药物治疗虽然有一定的抗病毒效果,但临床改善有限。1996年何大一博士开创的HAART(鸡尾酒疗法),目前已被证实是针对HIV感染最有效的治疗方法。为达到强大的抗病毒作用,该疗法要求最少联合使用3种抗病毒药物,将血浆中的HIV
RNA抑制在低水平或检测不出的水平。

治疗方案:推荐的方案包括2种NRTIs加一种NNRTIs,或两种NRTIs加一种PIs。药物选择要考虑每个患者的情况和药物特点。
\paragraph{NRTIs}

NRTIs有齐多夫定、拉米夫定、扎西他滨、司坦夫定等。齐多夫定是世界上第一个获得美国FDA批准生产的抗艾滋病药品,疗效确切,是“鸡尾酒”疗法最基本的组合成分。用于艾滋病或艾滋病相关综合征患者的治疗,如伴皮肤黏膜假丝酵母菌病、体重减轻、淋巴结肿胀、发热原因不明的HIV感染者、伴卡氏肺孢子虫肺炎的患者等。因易产生耐药性,临床多与其他抗HIV感染的药物联合应用。常见不良反应有骨髓抑制,用药期间要定期查血常规。比较罕见的严重不良反应有酸中毒、严重的肝肿大,严重者可导致死亡。在使用齐多夫定治疗期间如出现肝肿大、肝功能异常、酸中毒时要立即停药。阿司匹林、苯二氮䓬
类、西咪替丁、保泰松、吗啡、磺胺等均抑制AZT葡萄糖醛酸化,降低其清除率,可增加中毒危险,应避免联用。成人常用量每次口服200mg,每4h
1次。贫血患者可每次给药100mg。注射剂以5%葡萄糖液稀释后静脉滴注,切勿静脉推注。
\paragraph{NNRTIs}

NNRTIs有奈韦拉平、地拉韦啶、艾法韦伦等。其作用机制是直接与HIV的病毒反转录酶结合。与NRTIs相比,其作用特点是非竞争性抑制反转录酶活性,不需磷酸化,也不会整合到病毒脱氧核糖核酸中。

常用药物为奈韦拉平,血浆半衰期长达25~30h,口服生物利用度高,且不受进食影响。因其是肝细胞色素P450代谢酶(CYP3A,CYP2B)诱导剂,其他主要由CYP3A、CYP2B代谢的药物在与本药合用时,奈韦拉平可以降低这些药物的血浆浓度。因此,如果患者正在接受一个稳定剂量由CYP3A或CYP2B代谢的药物的治疗时,若开始合用本药,前者药物剂量需要调整。主要不良反应为皮疹、中毒性表皮溶解症,偶见药物性肝炎。治疗初始8周是关键阶段,需严密的监测。常用剂量为每次200mg,每日1次。2周后改为每次200mg,每日2次。
\paragraph{PIs}

PIs常用药物有沙奎那韦、利托那韦、茚地那韦等。可抑制蛋白酶的活性,其主要作用于艾滋病病毒复制的最后阶段,由于蛋白酶被抑制,从而阻止了病毒的成熟,防止了新的感染。茚地那韦用于成人HIV-1感染。可与抗反转录病毒复制剂联合应用治疗成人的HIV-1感染。单独应用治疗临床上不适宜用NRTIs或NNRTIs治疗的成年患者。常见不良反应是药物结晶引起的肾结石,多在开始治疗后数日发生,因此在服用茚地那韦期间饮水量不能低于1500mL,以防肾结石的形成。其他不良反应还有转氨酶水平升高、乏力、恶心、腹泻、眩晕、头痛、烦躁、药疹、味觉异常等。常用剂量为每次800mg,每日3次,肝功能不全患者或与酮康唑合用剂量应减至每次600mg。
\paragraph{FIs}

FIs有恩夫韦地等。HIV与宿主细胞融合是侵入宿主的第一步。FIs通过阻止病毒与宿主细胞的融合,阻止了病毒侵入宿主细胞的最初阶段。恩夫韦地为HIV-1跨膜融合蛋白GP41内高度保守序列衍生而来的一种合成肽类物质,可防止病毒融合及进入细胞内。本品可与病毒包膜糖蛋白的GP4l亚单位上的第一个7次重复序列相结合。以阻止病毒与细胞膜融合所必需的构象改变。常见不良反应为注射部位局部疼痛、结节、红斑、腹泻、呕吐,发热和胰腺炎等。成人常用剂量为每次90mg,每日2次,皮下注射。

\subsection{预防}

因迄今尚无彻底清除HIV的药物,故艾滋病的防治以预防为主。由于艾滋病主要通过性接触、血液或血制品及母婴传播,因此预防措施包括避免性接触感染HIV、防止注射途径的传播、加强血制品管理、切断母婴传播、加强宣教、加强艾滋病的监测等。