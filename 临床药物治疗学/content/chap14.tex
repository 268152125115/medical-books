\chapter{血液系统疾病的药物治疗}

\section{缺铁性贫血}

\subsection{定义}

缺铁性贫血是由于铁摄入不足或丢失过多,体内贮铁耗尽,影响血红素合成,导致缺铁性红细胞生成的小细胞低色素性贫血。

\subsection{病因及发病机制}

\subsubsection{病因}

常见的铁缺乏原因有摄入不足、需求过多和铁丢失过多。

(1)摄入不足:主要是饮食含铁量不足和吸收障碍。动物源食物比植物源食物含铁量高且易于吸收,铁缺乏者可选用高铁饮食。胃酸分泌不足、胃部手术、胃肠道功能紊乱(长期腹泻、慢性肠炎)等多种原因都会导致铁吸收障碍,造成缺铁性贫血。

(2)需求过多:婴幼儿生长迅速而铁贮备量少,乳汁含铁量低,若喂养不合理,易发生缺铁性贫血。青少年偏食、妊娠和哺乳期妇女铁需求量增加,若不补充高铁食物,也可能发生缺铁性贫血。

(3)丢失过多:慢性失血是本病最常见的病因。各种出血性疾病,如慢性胃肠道出血、消化性溃疡、肿瘤、月经过多、痔疮等均可导致缺铁性贫血。

正常情况下铁的消耗和补充保持动态平衡,如长期出血负铁平衡的情况,则可导致缺铁和血细胞生成障碍。缺铁是一个渐进的过程,一般分为三个阶段:缺铁(贮铁减少,血红蛋白正常)、缺铁性红细胞生成(铁代谢紊乱,无贫血)和缺铁性贫血。

\subsubsection{发病机制}

当体内缺铁严重时,粒细胞、血小板的生成也会受到影响。同时,细胞中含铁酶和铁依赖性酶的活性降低,可能影响人的精神、行为、免疫功能及儿童的生长、智力发育;缺铁也可能引起黏膜组织病变和外胚叶组织营养障碍。

\subsection{临床表现}

在体内铁储备尚未耗竭之前,临床上可无症状。当储备铁耗竭后,主要表现为皮肤黏膜苍白、乏力、活动后心悸、气促、眼花、耳鸣等。踝部可出现浮肿,儿童可有生长发育迟缓、注意力不集中、性格改变等症状,此外还可有某些特殊的神经系统症状如容易兴奋、激动、烦躁、头痛等。偶有上皮细胞组织异常所产生的症状,如舌痛或萎缩性口炎等。异食癖为缺铁性贫血的特殊表现。

\subsection{治疗}

\subsubsection{药物治疗原则}

缺铁性贫血首选口服制剂,餐后服用。服药时应注意,同服谷类、乳类、茶、含鞣酸饮料等会抑制铁剂的吸收;鱼、肉类、维生素C可加强铁剂的吸收。铁剂治疗在血红蛋白恢复正常后至少持续4~6个月,待铁蛋白正常后停药。若口服铁剂不能耐受或吸收障碍,可用右旋糖酐铁肌注,注意过敏反应。

\subsubsection{常用药物}

\textbf{铁剂:}

【适应证】 适用于预防或治疗各种原因引起的缺铁,包括儿童或婴儿期需铁量增加而食物中供应量不足、铁吸收障碍、妊娠中后期以及慢性失血等。

【用法和用量】 口服铁剂:硫酸亚铁,成人口服每次300mg,每日3次。缓释片每次450mg,每日2次,饭后服用。注射铁剂:右旋糖酐铁,深部肌肉注射,成人每次100~200mg,每1~3日1次。

【注意事项及不良反应】 口服铁剂有轻度胃肠反应,重者餐后服用,但对药物吸收有所影响。注射铁剂期间,不宜同时口服铁,以免发生毒性反应。口服糖浆铁制剂容易使牙齿变黑。避免婴儿肌肉注射铁剂。

\section{再生障碍性贫血}

\subsection{定义}

再生障碍性贫血(aplastic
anemia,AA,简称再障)是由多种原因引起的骨髓造血功能衰竭所致,外周血呈现全血细胞减少。

\subsection{病因及发病机制}

\subsubsection{病因}

(1)物理因素:主要指电离辐射,如γ射线、X射线等。

(2)化学因素:指药物及化学物质,如杀虫剂、染发剂、化疗药物等。

(3)生物因素:指病毒,如乙型肝炎病毒、微小病毒感染等。

\subsubsection{发病机制}

(1)造血干细胞缺陷:再障患者造血干细胞异常,集落形成能力显著降低,数量明显减少。

(2)造血微环境异常:再障患者的骨髓“脂肪化”,毛细血管坏死,骨髓基质细胞分泌的各类造血调控因子异常。

(3)免疫功能异常:T细胞功能异常亢进、T细胞直接杀伤和淋巴因子介导的造血干细胞过度凋亡引起的骨髓衰竭。

\subsection{临床表现}

\subsubsection{急性型再障}

起病急,病程短,进展迅速,早期突出的症状常为出血和感染。贫血进行性加重症状明显。皮肤瘀点、瘀斑,鼻出血、齿龈出血、消化道出血、月经过多等出血症状均较常见,而且不易控制,如出现颅内出血、严重感染如肺炎和败血症,病情险恶,预后差。

\subsubsection{慢性型再障}

起病缓,进展较慢,病程较长,以贫血为首发及主要表现。患者一般出现倦怠无力、劳累后气促、心悸、头晕及面色苍白等。出血症状轻微,多限于皮肤黏膜,且不严重,内脏出血较少见。感染发热症状较轻微,且以呼吸道感染为主,严重感染少。

\subsection{治疗}
\paragraph{药物治疗原则}

慢性或轻型再障以雄激素治疗为主,急性或重型再障应以免疫抑制剂为主。
\paragraph{常用药物}

\textbf{司坦唑醇:}属于17α-烷基雄激素类,为治疗慢性再障的首选药物。

【适应证】 可刺激造血功能,直接刺激骨髓干、祖细胞增殖分化,用于治疗再障。

【用法和用量】 成人口服,开始每次2mg,每日2或3次,女性酌减。效果明显后逐渐、缓慢减量,至维持量每日2mg。

【不良反应及注意事项】 服药初期偶见水肿,继续用药可自行消失。长期用药可能引起黄疸、肝坏疽、诱发肝癌。本品有雄性化作用,女性患者可能出现月经紊乱现象,停药后自行缓解。

\textbf{抗人T细胞免疫球蛋白:}

【适应证】 适用于年龄大于40岁或无合适供髓者的严重型再障。

【用法和用量】 静脉滴注,每日2mg/kg,以250~500mL氯化钠注射液稀释,连用5d。输液前2h给予氢化可的松,可改善耐受。

【不良反应及注意事项】 静脉滴注时见短暂高热、寒战,有时伴有关节痛、低血压、心率加快、呼吸困难。本品可诱导产生其他抗体,故接种减毒活疫苗者禁用。可致过敏反应,过敏体质者禁用。治疗期间定期监测血细胞计数,当发生严重的持续性血小板或白细胞计数减少,应停药。长期使用本品,患者免疫监视功能下降,注意癌变。

\section{巨幼细胞性贫血}

\subsection{定义}

巨幼细胞性贫血是指叶酸、维生素B{12}
缺乏,遗传性或药物等原因引起DNA合成障碍所致的一类贫血。

\subsection{病因及发病机制}

\subsubsection{病因}
\paragraph{叶酸缺乏}

(1)摄入不足:见于婴儿、儿童及妊娠期和哺乳期妇女,需要量可增加3~10倍。另外,还可见于慢性溶血、恶性肿瘤、甲状腺功能亢进、慢性乙醇中毒等患者。

(2)吸收障碍:某些小肠疾病如小肠炎症、小肠切除、空肠结肠瘘以及慢性腹泻等吸收不良综合征等均可影响叶酸的吸收。

(3)利用障碍:某些药物如苯妥英钠、口服避孕药、MTX等可影响叶酸代谢及利用。
\paragraph{维生素B{12} 缺乏}

(1)摄入不足:长期营养不良或偏食可减少维生素B{12}
的摄入;婴幼儿、妊娠及某些疾病叶酸需要量增加,可发生摄入相对不足。

(2)吸收障碍:肠道疾病、肠道寄生虫、某些药物(对氨基水杨酸、新霉素、二甲双胍、秋水仙碱、奥美拉唑等)、胃酸和胃蛋白酶缺乏、胰蛋白酶缺乏、先天性或疾病(恶性贫血、胃切除、胃黏膜萎缩)所致内因子缺乏,都会影响维生素B{12}
吸收。吸收障碍是维生素B{12} 缺乏最常见的原因。

(3)利用障碍:运输维生素B{12}
的运输蛋白缺乏、应用一氧化氮等均可影响维生素B{12} 的转运和利用。

\subsubsection{发病机制}

主要是体内缺乏叶酸或维生素B{12} 。叶酸和维生素B{12}
是细胞合成脱氧核糖核酸过程中的重要辅酶,缺乏可导致DNA合成障碍而RNA合成不受影响,细胞核质发育不平衡,质老核幼,呈巨幼样变。维生素B{12}
还参与神经组织的代谢,缺乏可造成神经髓鞘合成障碍,从而导致脱髓鞘病变,轴突变性,最后可导致神经元细胞死亡。神经系统可累及周围神经、脊髓后侧索及大脑。

\subsection{临床表现}

(1)贫血:表现为乏力、疲倦、头晕、耳鸣、活动后心悸、气促等一般慢性贫血的症状,贫血可呈进行性加重,部分患者可有轻度黄疸或便秘。

(2)消化系统症状:患者可有食欲减退、腹胀、腹泻或便秘等,部分患者可发生舌炎,表现为舌痛,舌乳头萎缩,在恶性贫血时尤为显著。此外,还可发生口角炎和口腔黏膜小溃疡。

(3)体重减轻及消瘦在叶酸缺乏的患者常见。

(4)神经系统症状:典型的表现为四肢远端发麻、深感觉障碍、共济失调等周围神经炎表现,另外还可出现亚急性或慢性脊髓后侧索联合变性,可表现为腱反射尤其是膝腱及跟腱反射早期亢进,以后减弱以致消失。

\subsection{治疗}
\paragraph{药物治疗原则}

对于叶酸缺乏性巨幼红细胞性贫血,血红蛋白浓度恢复正常即可,不需维持治疗;对于恶性贫血或胃全部切除的维生素B{12}
缺乏性巨幼细胞贫血者需终生维生素B{12} 维持。
\paragraph{常用药物}

\textbf{叶酸:}
水溶性B族维生素,是细胞生长和分裂的必需物质。在体内被还原成四氢叶酸,是DNA合成的重要辅酶,参与体内氨基酸、核酸合成,促进红细胞增殖和成熟。

【适应证】 用于各种巨幼红细胞贫血,尤其适于营养不良、婴幼儿、妊娠期叶酸需要量增加所致的巨幼红细胞性贫血。也用于再生障碍性贫血和白细胞减少症的辅助治疗。

【用法和用量】 口服叶酸,成人每次5~10mg,每日3次,共14d,维持量每日2.5~10mg。

【不良反应及注意事项】 不良反应较少,罕见过敏反应。大剂量时可引起黄色尿。长期应用叶酸者可能出现恶心、厌食、腹胀等。

\textbf{维生素B{12} :}
属水溶性B族维生素,为含钴的复合物,是核苷酸合成的重要辅酶,参与叶酸代谢、三羧酸循环等体内多种生化过程,参与维持有鞘神经纤维功能完整。

【适应证】 用于治疗恶性贫血,与叶酸合用治疗各种巨幼红细胞性贫血,抗叶酸药引起的贫血。

【用法和用量】 开始时,肌肉注射25~100μg,每日或隔日1次;2周后改为每周2次,每次50~100μg,至血象正常;维持量每月肌肉注射100μg。用于神经系统病变时,可酌情增量。

【不良反应及注意事项】 肌肉注射偶见皮疹、瘙痒、腹泻、低血钾、高尿酸血症,少见过敏性休克。痛风患者使用维生素B{12}
,可能诱发痛风,应慎用。心脏病患者应避免注射本品。缺乏维生素B{12}
同时又缺乏叶酸的患者,单独应用维生素B{12}
可能掩盖叶酸缺乏症状,应同时补充叶酸。

\section{白细胞减少症、中性粒细胞减少症和粒细胞缺乏症}

\subsection{定义}

成人外周血液中白细胞计数持续低于4.0×10{9} /L(≥10岁儿童低于4.5×10{9}
/L;<10岁儿童低于5.0×10{9}
/L),称为白细胞减少症。中性粒细胞绝对数低于2.0×10{9}
/L(≥10岁儿童低于1.8×10{9} /L;<10岁儿童低于1.5×10{9}
/L),称为中性粒细胞减少症。低于0.5×10{9}
/L,称为粒细胞缺乏症。本组疾病分为先天性和获得性两类。

\subsection{病因及发病机制}

\subsubsection{生成障碍}

(1)骨髓损伤:电离辐射和化学毒物等物理或化学因素直接损伤骨髓干细胞和祖细胞或骨髓造血环境可造成全血细胞减少。

(2)药物抑制:多数药物可抑制或干扰粒细胞的生成,影响细胞代谢,阻碍细胞分裂。大多数药物直接毒性作用造成粒细胞减少,与药物剂量相关,呈剂量依赖性,如β-内酰胺类抗生素;有的与剂量无关,如左旋咪唑、卡托普利等;某些药物也可通过免疫机制引起粒细胞减少。

(3)病毒或细菌感染:中性粒细胞减少可见于病毒感染、细菌感染和分枝杆菌感染等。病毒感染后中性粒细胞减少在儿童尤为常见。

(4)生成受抑或衰竭:白血病等血液系统恶性肿瘤或恶性实体瘤骨髓转移可抑制正常造血;再生障碍性贫血可由于骨髓功能衰竭造成全血细胞减少。

\subsubsection{细胞破坏或消耗过多}

(1)免疫相关性:药物诱发的免疫性粒细胞减少或自身免疫性粒细胞减少。

(2)非免疫性:病毒感染、败血症或脾功能亢进。

\subsubsection{分布异常}

\subsection{临床表现}

白细胞和中性粒细胞减少症大多起病缓慢,少数患者无明显症状。粒细胞缺乏症起病急,可突然畏寒、高热、头痛及乏力;出现严重感染。

\subsection{治疗}

感染既是粒细胞减少和缺乏的原因也是其结果,粒细胞缺乏患者极易发生危及生命的细菌和真菌感染,应根据病原体的培养结果有针对性地用药,并做到早期、广谱、联合和足量给药。同时需加强支持治疗,注意营养和补给各种维生素。

\subsubsection{白细胞减少症和中性粒细胞减少症药物治疗}
\paragraph{药物治疗原则}

对于长期病情稳定,无明显症状、无感染倾向的轻度减少,可不必依赖药物治疗,适当休息,补充营养,随访观察。对白细胞和中性粒细胞数有明显减低且伴有相应症状者,可选用维生素B{4}
、鲨肝醇、利可君等不同机制药物1或2种联合应用。如无明显的感染证据,一般不预防使用抗生素。用药期间应常检查粒细胞数。
\paragraph{常用药物}

\textbf{维生素B{4} :}

【适应证】 该药是腺嘌呤,是核酸的前体,可参与RNA和DNA的合成,促进白细胞增生。

【用法用量】 口服每次10~20mg,每日3次。

【注意事项】 注射时,需溶于2mL磷酸氢二钠缓冲液中,缓慢注射,不能与其他药物混合注射。此外,还应警惕其促进肿瘤发生的可能性。

\textbf{鲨肝醇:}

【适应证】 用于各种原因引起的粒细胞减少。在造血系统中含量较多,可能为体内的一种造血因子,促进粒细胞的增长。

【用法用量】 口服每次50mg,每日3次。

【不良反应】 治疗剂量偶见口干、肠鸣亢进。

\textbf{利可君:}

【适应证】 预防、治疗白细胞减少症及血小板减少症。

【用法用量】 口服每次20mg,每日3次。

\subsubsection{粒细胞缺乏症药物治疗}
\paragraph{药物治疗原则}

一旦确诊为粒细胞缺乏症,应作为急症进行处理。除必须进行隔离和其他的支持治疗外,应尽早使用相应药物。粒细胞缺乏症患者几乎在1~3d内均发生严重感染,根据感染部位和可能感染的菌种不同,在病原菌未明确前可选择1或2种抗生素联合治疗。此外,还应及早开始造血因子治疗,白细胞计数升高应首选G-CSF和GM-CSF。
\paragraph{常用药物}

\textbf{G-CSF和GM-CSF:}

【适应证】 用于治疗恶性肿瘤放、化疗所致的白细胞减少症,以及骨髓移植、再生障碍性贫血和某些存在白细胞计数低下的免疫缺陷性疾病的治疗。

【用法用量】 每日100~300μg皮下注射或静脉滴注。应用持续时间可根据白细胞恢复情况而定,一般应用1周左右,疗效差者可适当延长。

【不良反应】 应注意可能出现的发热、畏寒、肌肉酸痛等不良反应。

\textbf{地塞米松:}

【适应证】 适用于免疫因素造成的粒细胞缺乏症或危重病例。在应用足量抗生素类药物的同时,可短期应用。

【用法用量】 每日5~10mg静脉滴注,用药时间一般不超过1周。

【不良反应】 较大剂量易引起糖尿病、消化道溃疡和类库欣综合征症状,对下丘脑---垂体---肾上腺轴抑制作用较强,并发感染为主要的不良反应。