\chapter{消化系统疾病的药物治疗}

\section{消化性溃疡}

消化性溃疡(peptic
ulcer,PU)指胃肠道黏膜被胃酸和胃蛋白酶消化而发生的溃疡,好发于胃和十二指肠,也可发生在食管下段、小肠、胃肠吻合术后吻合口及异位的胃黏膜。溃疡的黏膜坏死缺损超过黏膜肌层,有别于糜烂。胃溃疡(gastric
ulcer,GU)和十二指肠溃疡(duodenal
ulcer,DU)是最常见的消化性溃疡,故一般所谓的消化性溃疡,是指GU与DU。

\subsection{流行病学}

男性患消化性溃疡病的人数多于女性,男女之比GU为(4.4~6.8)∶1,DU为(3.6~4.7)∶1。DU比GU多见,在胃癌高发区则GU多于DU。DU多见于青壮年,GU多见于中老年。DU发病高峰比GU早10年。本病具有显著的地理环境差异性,我国南方患病率高于北方,城市高于农村,这可能与饮食习惯、工作节奏有关。发作有季节性,秋冬和冬春之交是高发季节。

\subsection{病因与发病机制}

胃十二指肠黏膜除了经常接触高浓度胃酸外,还受到胃蛋白酶、微生物、胆盐、乙醇、药物和其他有害物质的侵袭。但在正常情况下,胃十二指肠黏膜能够抵御这些侵袭因素的损害作用,维持黏膜的完整性。这是因为胃十二指肠黏膜具有一系列防御和修复机制,包括黏液/碳酸氢盐屏障、黏膜屏障、丰富的黏膜血流、上皮细胞更新、前列腺素和表皮生长因子等。消化性溃疡的发生是由于对胃十二指肠黏膜有损害作用的侵袭因素与黏膜自身防御修复之间失去平衡的结果。

\subsubsection{胃酸和胃蛋白酶}

胃酸与胃蛋白酶的自身消化是形成消化性溃疡的主要原因。盐酸是胃液的主要成分,由壁细胞分泌。胃蛋白酶原由胃体和胃底部的主细胞分泌,胃蛋白酶原经盐酸激活转化成胃蛋白酶,pH值1~3时胃蛋白酶最活跃,能水解食物蛋白、胃黏液中糖蛋白,甚至自身组织蛋白,pH值>4时活性迅速下降。胃酸和胃蛋白酶水平升高均可引起消化性溃疡,但胃蛋白酶原激活依赖胃酸的存在,因此胃酸的存在是溃疡发生的决定性因素。

\subsubsection{幽门螺杆菌}

大多数研究已证实消化性溃疡与幽门螺旋杆菌({H} .pylori,{HP}
)有密切相关性。约70%GU及95%~100%DU均感染{Hp} 。{Hp}
感染导致消化性溃疡的发病机制尚未完全阐明,{Hp}
造成的胃炎和胃黏膜屏障的损害是促使消化性溃疡发生和难于愈合的重要因素。

\subsubsection{非甾体类抗炎药}

非甾体类抗炎药(non-steroidal anti-inflammatory
drugs,NSAIDs)近年来临床应用越来越广泛,是引起消化性溃疡另一个重要的因素,常见的药物有阿司匹林、吲哚美辛、舒林酸、吡罗昔康、乙酰氨基酚和保泰松等。大约15%~30%应用NSAID的患者会发生消化性溃疡,其中2%~4%的患者可能发生溃疡出血或穿孔。NSAID溃疡并发症的危险因素包括既往消化道并发症、年龄(>65岁)、同时应用抗凝剂、糖皮质激素和其他NSAIDs,以及慢性病特别是心血管疾病。

由于胃黏膜接触摄入NSAIDs时间较十二指肠黏膜长,故溃疡好发于胃窦部和幽门前区,GU发病率为10%~20%,DU发病率为2%~5%。溃疡形态多样,大小为0.2~3.0cm不等,呈多发、浅表性溃疡。

\subsubsection{其他危险因素}
\paragraph{其他药物}

如糖皮质激素、抗肿瘤药物和抗凝药的广泛使用也可诱发消化性溃疡病,亦是上消化道出血不可忽视的原因之一。尤其应重视目前已广泛使用的抗血小板药物,能增加消化道出血的风险性,如噻吩吡啶类药物氯吡格雷等,该类药物通过阻断血小板膜上的ADP受体发挥抗血小板作用。与阿司匹林不同,ADP受体拮抗剂并不直接损伤消化道黏膜,但可抑制血小板衍生的生长因子和血小板释放的血管内皮生长因子,从而阻碍新生血管生成和影响溃疡愈合,即阻碍已受损消化道黏膜的愈合。当阿司匹林与氯吡格雷联合应用时危险性更高。
\paragraph{吸烟}

烟草刺激胃酸分泌增加,抑制胰液和胆汁的分泌而减弱其在十二指肠内中和胃酸的能力;使幽门括约肌张力减低,胆汁反流,破坏胃黏膜屏障。
\paragraph{不良的饮食和生活习惯}

咖啡、浓茶、烈酒、辛辣食品等,以及偏食、饮食过快等不良饮食习惯,均易引起消化不良症状,但尚无确切资料证明可增加溃疡的发病率。高盐可损伤胃黏膜,增加GU发生的危险性。
\paragraph{胃十二指肠运动异常}

部分DU患者的胃排空比正常人快,使十二指肠球部的酸负荷量增大,黏膜易遭损伤。部分GU表现为胃排空延缓和十二指肠-胃反流,刺激胃窦部G细胞分泌胃泌素,增加胃酸的分泌。
\paragraph{心理因素}

长期精神紧张、焦虑或情绪波动者易罹患消化性溃疡。应激事件往往可引起应激性溃疡或促发消化性溃疡急性穿孔。心理因素可能通过迷走神经兴奋影响胃十二指肠分泌、运动及黏膜血流的调节。

\subsection{临床表现}

本病患者临床表现不一,多数表现为中上腹反复发作性节律性疼痛,少数患者无症状,或以出血、穿孔等并发症发生作为首发症状。近年来由于抗酸剂、抑酸剂的广泛应用,症状不典型患者日益增多。

\subsubsection{疼痛}

(1)部位:大多数患者以中上腹疼痛为主要症状。少部分患者无疼痛表现,特别是老年人溃疡、维持治疗中复发性溃疡和NSAIDs相关性溃疡。胃或十二指肠后壁溃疡,特别是穿透性溃疡可放射至背部。

(2)程度和性质:隐痛、钝痛、灼痛或饥饿样痛,持续性剧痛提示溃疡穿透或穿孔。

(3)节律性:DU疼痛好发于两餐之间,持续不减直至下餐进食或服制酸药物后缓解,即“空腹痛”。部分DU患者,由于夜间的胃酸较高,可发生半夜痛。GU疼痛常在餐后0.5h内发生,经1~2h后逐渐缓解,直至下餐进食后再复出现,即“餐后痛”。

(4)周期性:反复周期性发作是消化性溃疡的特征之一,尤以DU更为突出。上腹疼痛发作可持续几天、几周或更长,继以较长时间的缓解。以秋末至春初较冷的季节更为常见。有些患者经过反复发作进入慢性病程后,可失去疼痛的节律性和周期性特征。

(5)影响因素:疼痛常因精神刺激、过度疲劳、饮食不慎、药物影响和气候变化等因素诱发或加重。可因休息、进食、服制酸药、以手按压疼痛部位、呕吐等方法而减轻或缓解。

\subsubsection{其他症状}

可伴有非特异性症状,如唾液分泌增多、胃灼热、反胃、嗳酸、嗳气、恶心、呕吐等其他胃肠道症状。

\subsection{特殊类型的消化性溃疡}

\subsubsection{无症状型溃疡}

约15%~30%消化性溃疡者无明显症状。这类消化性溃疡可见于任何年龄,但以老年人尤为多见。NSAIDs溃疡占无症状性溃疡的30%~40%。

\subsubsection{幽门管溃疡}

幽门管位于胃远端,与十二指肠交接。与DU相似,幽门管溃疡常伴胃酸分泌过高,餐后可立即出现中上腹疼痛,程度较为剧烈而无节律性,抑酸疗效差。由于幽门管易痉挛和瘢痕形成,常引起梗阻而呕吐,或出现穿孔和出血。

\subsubsection{难治性溃疡}

通常指经正规治疗(DU治疗8周,GU治疗12周)后,仍有腹痛、呕吐和体重减轻等症状的消化性溃疡。可能为穿透性溃疡或特殊部位的溃疡,如球后和幽门管溃疡,应与其他疾病如胃泌素瘤、克罗恩病、消化道淀粉样变性、淋巴瘤等鉴别。

\subsubsection{应激性溃疡}

指在严重烧伤、颅脑外伤、脑肿瘤、严重外伤和大手术、严重的急性或慢性内科疾病(如脓毒血症)等应激情况下,在胃或十二指肠、食管产生的急性黏膜糜烂和溃疡。严重烧伤引起的急性应激性溃疡称为Curling溃疡;颅脑外伤、脑肿瘤或颅内神经外科手术引起的溃疡称为Cushing溃疡。

\subsection{诊断}

根据本病具有慢性病程、周期性发作和节律性中上腹疼痛等特点,可作初步诊断。内镜检查是确诊的手段,对于内镜检查阴性仍然怀疑者,可行X线钡餐检查,钡剂填充溃疡的凹陷部分所造成的龛影是诊断溃疡的直接征象。

\subsection{治疗}

一般治疗为避免过度紧张与劳累。溃疡活动期伴并发症时,需卧床休息。戒烟酒,避免食用咖啡、浓茶、辛辣等刺激性食物;不过饱,避免可诱发溃疡病的药物,如NSAIDs、肾上腺皮质激素、利舍平等。常用治疗药物如下。

\subsubsection{降低胃酸药物}
\paragraph{碱性制酸药}

碱性制酸药可中和胃酸,降低胃蛋白酶活性,缓解疼痛,促进溃疡愈合。此类药物曾是治疗溃疡主要药物之一,但因为愈合溃疡的疗效低,目前仅作为止痛的辅助药物,如碳酸氢钠、氢氧化铝、氢氧化镁及其复方制剂等。
\paragraph{H$_{2}$ 受体阻断药(H$_2$ RA)}

H$_2$ RA选择性竞争结合H$_{2}$ 受体,使胃酸分泌明显减少,促进溃疡愈合。H$_2$
RA价格低廉,DU治疗需要8周,GU治疗时间应当更长。常用药物有西咪替丁、雷尼替丁、法莫替丁。

(1)西咪替丁:可抑制基础胃酸分泌5h。常用剂量为每次0.4g,每日2次口服,餐后服用或睡前1次服0.8g。注意西咪替丁有轻度抗雄性激素作用,用药剂量较大时可引起男性乳房发育、女性溢乳、性欲减退、阳痿、精子计数减少等,停药后消失;可抑制细胞色素P450催化的氧化代谢途径,与其他药物合用时可降低这些药物的代谢,致其药理活性或毒性增强,这些药物包括苯二氮䓬
类药、华法林、苯妥英钠、普萘洛尔、美托洛尔、甲硝唑、茶碱、咖啡因、氨茶碱、维拉帕米(异搏定)、利多卡因、三环类抗抑郁药等;制酸药可能会影响西咪替丁的吸收,故若合用,两者应至少相隔1h服用;肾功能不全者剂量减量。

(2)雷尼替丁:抑制胃酸作用为西咪替丁的5倍,常用剂量为每次0.15g,每日2次口服,不建议用于儿童特别是8岁以下者。

(3)法莫替丁:抑制胃酸分泌效果比西咪替丁强40倍,作用可维持12h以上,常用剂量为每次20mg,每日2次口服。该药物不与肝脏细胞色素P450酶作用,故不影响茶碱、苯妥英、华法林及地西泮等药物的代谢。
\paragraph{质子泵抑制药(proton pump inhibitions,PPI)}

是一种苯丙咪唑硫氧化物,需酸性环境才能被激活,吸收入血后进入壁细胞,在分泌小管的酸间隙内质子化,转化为活性物质次磺酰胺,后者与质子泵管腔面上的2个巯乙胺共价结合,对ATP酶产生不可逆的抑制作用,从而阻断酸分泌的最后步骤,待新的ATP酶合成后,酸分泌才能恢复,明显减少任何通路引起的酸分泌,抑酸作用较H$_2$
RA更强且作用持久。消化性溃疡通常采用标准剂量的PPI,每日1次,早餐前0.5h服药。治疗十二指肠溃疡疗程4周,胃溃疡为6~8周,通常胃镜下溃疡愈合率均在90%以上。对于存在高危因素及巨大溃疡的患者建议适当延长疗程。对胃泌素瘤的治疗,通常应用双倍标准剂量的PPI,分为每日2次用药。常用的PPI主要有以下5种。

(1)奥美拉唑:第一代PPIs。奥美拉唑在肝脏中经细胞色素P450(CYP)酶系完全代谢,其中主要依赖特异的同功型CYP2C19。治疗溃疡的标准剂量为每日20mg。

(2)兰索拉唑:第一代PPIs。兰索拉唑主要通过肝药物代谢酶CYP2C19和CYP3A4进行代谢,生物利用度较奥美拉唑提高30%。治疗溃疡的标准剂量为每日30mg。

(3)泮托拉唑:第一代PPIs。泮托拉唑具有独特的代谢途径,可通过肝细胞内的细胞色素P450酶系的第Ⅰ系统与第Ⅱ系统进行代谢。当与其他通过P450酶系代谢的药物合用时,泮托拉唑可通过第Ⅱ酶系统进行代谢,从而不易发生药物代谢酶系的竞争性作用,减少体内药物间的相互作用。其生物利用度比奥美拉唑提高7倍。治疗溃疡的标准剂量为每日40mg。

(4)雷贝拉唑:第二代PPIs。在CYP2C19抑制剂中,雷贝拉唑最弱,故个体差异小,与其他药物的相互作用较少。治疗溃疡的标准剂量为每日10mg。

(5)埃索美拉唑:第二代PPIs。奥美拉唑的S异构体,完全经细胞色素P450酶系统代谢,比奥美拉唑的抑酸更强且持久,疗效较稳定,个体差异低于奥美拉唑、兰索拉唑。治疗溃疡的标准剂量为每日20mg。

\subsubsection{胃黏膜保护药}

对于老年人消化性溃疡、难治性溃疡、巨大溃疡、复发性溃疡建议在抗酸、抗{Hp}
治疗同时,应用胃黏膜保护剂。

(1)铋剂:在酸性环境下与溃疡面的黏蛋白形成螯合剂,覆盖于胃黏膜上发挥治疗作用,促进胃上皮细胞分泌黏液,抑制胃蛋白酶活性,促进前列腺素的分泌,对胃黏膜起保护作用。能干扰{Hp}
的代谢,可用于根除{Hp}
的联合治疗。有舌苔、牙齿黑染、黑便等不良反应,慢性肾功能不全者慎用,为避免铋在体内过量积聚引起脑病,不宜长期使用。常用者如胶体果胶铋,每次100~200mg,每日3或4次,餐前0.5h服用。当患者同时口服铋剂及抑酸剂时,铋剂会因为失去酸性环境而不能发挥有效作用,而抑酸剂也会因黏膜保护剂而影响其药效,故需分时服用。

(2)硫糖铝:在酸性胃液中,凝聚成糊状黏稠物,附着于黏膜表面,阻止胃酸、胃蛋白酶侵袭溃疡面,有利于黏膜上皮细胞的再生和阻止氢离子向黏膜内逆弥散,促进内源性前列腺素合成。每次1g,每日4次,饭前1h及睡前空腹嚼碎服用。不良反应主要为便秘,甲状腺功能亢进、佝偻病等低磷血症患者不宜长期服用,因为可能会造成体液中磷的缺乏。

(3)米索前列醇(Misoprostol):能抑制胃酸分泌,增加胃十二指肠黏膜黏液/碳酸氢盐分泌,增加黏膜血流量,加速黏膜修复。主要用于NSAIDs溃疡的预防。不良反应主要是腹泻、皮疹等,能引起子宫收缩,孕妇慎用。

(4)铝碳酸镁:是抗酸抗胆汁的胃黏膜保护剂,每次0.5~1g,每日3或4次,饭后1~2h嚼服。大剂量服用可导致软糊状便和大便次数增多。

(5)替普瑞酮:可促进胃黏膜中高分子糖蛋白与磷脂的合成,增加胃黏膜中前列腺素的合成,促进胃黏液分泌和黏膜上皮细胞的复制能力,提高黏膜的防御能力。每次50mg,每日3次,饭后服用。可能出现便秘、口渴、肝酶升高或胆固醇升高等不良反应。

(6)瑞巴派特:增加胃黏膜中前列腺素的含量,抑制中性粒细胞产生过氧化酶,清除活性氧对胃黏膜上皮的损伤,抑制{Hp}
黏附与化学趋化因子的产生而发挥黏膜保护作用。每次0.1g,早、晚及睡前口服。可能出现便秘等胃肠道症状、月经异常、肝肾功能异常、过敏等反应。

\subsubsection{胃肠动力药物}

当部分患者出现恶心、呕吐和腹胀等症状,提示有胃潴留、排空迟缓、胆汁反流或胃食管反流者,可予促进胃动力药物,如甲氧氯普胺、多潘立酮、莫沙必利、伊托比必利等。

(1)甲氧氯普胺(胃复安):多巴胺D{2}
受体拮抗剂,同时还具有5-羟色胺第4(5-HT{4}
)受体激动效应,作用于延髓催吐化学感受区(CTZ)中的多巴胺受体,具有中枢性镇吐作用;还可促进胃及上部肠段的运动。每次5~10mg,口服,每日3次。静脉制剂每次10~20mg,每日剂量不超过0.5mg/kg。可能会出现昏睡、烦躁不安、疲怠无力等反应;因为阻断下丘脑多巴胺受体,抑制催乳素抑制因子,促进泌乳素的分泌,故有一定的催乳作用,可能引起溢乳与男性乳房发育;大剂量长期应用可能因阻断多巴胺受体,使胆碱能受体相对亢进而导致锥体外系反应,如肌震颤、发音困难、共济失调等,肾功能衰竭者更易发生。禁用对象:对普鲁卡因过敏者;癫痫
患者;嗜铬细胞瘤患者;因行化疗和放疗而呕吐的乳癌患者;胃肠道出血、机械性肠梗阻或穿孔患者,可因用药使胃肠道的动力增加,病情加重。

(2)多潘立酮:为外周多巴胺受体阻滞剂,直接作用于胃肠壁,可增加食道下部括约肌张力,防止胃-食道反流,增强胃蠕动,促进胃排空,协调胃与十二指肠运动,不易透过血脑屏障。每次10mg,每日3次,饭前15~30min口服。抑制胃酸的分泌药物会降低多潘立酮的口服生物利用度,故不宜同时服用;由于垂体位于血脑屏障外,多潘立酮可能会引起催乳素水平升高;多潘立酮主要经肝脏CYP3A4酶代谢,不宜与其他可能会延长Q-T间期的CYP3A4酶强效抑制剂合用,如氟康唑、伏立康唑、克拉霉素、胺碘酮;不建议大剂量长时间应用,特别是对于肝功能中度或重度受损者,已经存在心电活动异常或心律异常者,以及出现上述情况风险较高的患者应该慎用。

(3)莫沙必利:选择性5-HT{4}
受体激动剂,通过兴奋胃肠道胆碱能中间神经元及肌间神经丛的5-HT{4}
受体,促进乙酰胆碱的释放,从而增强上消化道的运动。每次5mg,餐前口服,每日3次。可能出现腹泻、腹痛、口干、皮疹、倦怠、头晕、肝酶升高等症状,无锥体外系不良反应。

(4)伊托必利:具有拮抗多巴胺D{2}
受体和乙酰胆碱酯酶抑制活性,通过两者的协同作用发挥胃肠促动力作用,由于有拮抗多巴胺D{2}
受体活性的作用,尚有一定的抗呕吐作用。每次50mg,餐前口服,每日3次。可能会产生腹泻、腹痛、过敏、肝功能异常等不良反应。

\subsection{消化性溃疡病的抗Hp 治疗}

\subsubsection{根除Hp 适应证和推荐强度}

消化性溃疡是根除{Hp} 最重要的适应证,对消化性溃疡{Hp}
阳性者,无论是溃疡初发或复发,活动或静止,有无并发症都应行{Hp}
感染治疗已得到国际上的共识。根除{Hp}
可促进溃疡愈合,显著降低溃疡复发率和并发症发生率。

应注意依据根除适应证进行{Hp}
检测,不应任意扩大检测对象(表\ref{tab13-1})。治疗所有{Hp}
阳性者,但如无意治疗,就不要进行检测。

\begin{longtable}[]{@{}lll@{}}
    \caption{根除{Hp} 适应证和推荐强度}
    \label{tab13-1}                                                                                     \\
    \toprule
    \endhead
    {Hp} 阳性疾病                                                      & 强烈推荐 & 推荐\tabularnewline
    \midrule
    消化性溃疡(不论是否活动和有无并发症史)                           & √        & \tabularnewline
    胃MALT淋巴瘤                                                       & √        & \tabularnewline
    慢性胃炎伴消化不良                                                 &          & √\tabularnewline
    慢性胃炎伴胃黏膜萎缩、糜烂                                         &          & √\tabularnewline
    早期胃肿瘤已行内镜下切除或手术胃次全切除                           &          & √\tabularnewline
    长期服用PPI                                                        &          & √\tabularnewline
    胃癌家族史                                                         &          & √\tabularnewline
    计划长期服用NSAIDs(包括低剂量阿司匹林)                           &          & √\tabularnewline
    不明原因的缺铁性贫血                                               &          & √\tabularnewline
    特发性血小板计数减少性紫癜                                         &          & √\tabularnewline
    其他{Hp} 相关性疾病(如淋巴细胞性胃炎、增生性胃息肉、Ménétrier病) &          &
    √\tabularnewline
    个人要求治疗                                                       &          & √\tabularnewline
    \bottomrule
\end{longtable}

\subsubsection{Hp 感染的治疗}
\paragraph{根除{Hp} 抗菌药物的选择}

流行病学调查表明,推荐用于根除治疗的抗菌药物中,甲硝唑耐药率达60%~70%,克拉霉素达20%~38%,左氧氟沙星达30%~38%,耐药显著影响根除率,治疗失败后易产生耐药(原则上不可重复应用)。阿莫西林、呋喃唑酮(痢特灵)和四环素的耐药率仍很低(1%~5%),治疗失败后不易产生耐药(可重复应用)。在选择抗菌药物时应充分考虑药物的耐药特性。
\paragraph{三联疗法}

详见表\ref{tab13-2},PPI或铋剂+表中抗菌药物中选择两种。

\begin{table}
    \centering
    \caption{三联疗法的剂量和疗程}
    \label{tab13-2}
    \begin{tabular}{ccccc}
        \toprule
        PPI        & PPI剂量    & 抗菌药物   & 抗菌药物剂量  & 疗程   \\
        \midrule
        奥美拉唑   & 20 mg bid  & 克拉霉素   & 0.5 bid       &        \\
        兰索拉唑   & 30mg bid   & 阿莫西林   & 1.0 bid       &        \\
        泮托拉唑   & 40mg bid   & 甲硝唑     & 0.4 bid       &        \\
        雷贝拉唑   & 10 mg bid  & 替硝唑     & 0.5 bid       & 7~14d \\
        埃索美拉唑 & 20 mg bid  & 四环素     & 0.75 bid      &        \\
        铋剂       & 铋剂剂量   & 呋喃唑酮   & 0. 1 bid      &        \\
        枸橼酸铋钾 & 220 mg bid & 左氧氟沙星 & 0.5qd或0.2bid &        \\
        \bottomrule
    \end{tabular}
\end{table}

\paragraph{四联疗法}

PPI+铋剂+抗菌药物中的两种(剂量同三联方案)。疗程为10~14d。以下为2012年我国江西井冈山{Hp}
共识会议推荐的根除方案。

(1)抗菌药物4种组成方案:①阿莫西林+克拉霉素;②阿莫西林+左氧氟沙星;③阿莫西林+呋喃唑酮;④四环素+甲硝唑或呋喃唑酮。这4种抗菌药物组成方案中,3种治疗失败后易产生耐药的抗菌药物(甲硝唑、克拉霉素和左氧氟沙星)分别属于不同方案,仅不易耐药的阿莫西林、呋喃唑酮有重复。这些方案的优点:均有相对较高的根除率;任何一种方案治疗失败后,即使不行药敏试验,亦可再选择一种其他方案治疗。方案③和④疗效稳定、价廉,潜在的不良反应发生率可能稍高。

(2)青霉素过敏者推荐的抗菌药物组成方案:①克拉霉素+左氧氟沙星;②克拉霉素+呋喃唑酮;③四环素+甲硝唑或呋喃唑酮;④克拉霉素+甲硝唑。需注意的是,青霉素过敏者初次治疗失败后的抗菌药物选择余地小,应尽可能提高初次治疗根除率。
\paragraph{方案的选择}

在克拉霉素高耐药率(>15%)地区,一线方案首先推荐铋剂四联方案;在克拉霉素低耐药率地区,除推荐标准三联疗法外,亦推荐铋剂四联疗法作为一线方案。对铋剂有禁忌者或经证实{Hp}
耐药率仍较低的地区,可选用非铋剂方案,包括标准三联方案等。在我国多中心随机对照研究中,序贯疗法(前5d为PPI+阿莫西林,后5d为PPI+克拉霉素+甲硝唑,共10d)与标准三联疗法相比并未显示优势。伴同疗法(同时服用PPI+克拉霉素+阿莫西林+甲硝唑)我国缺乏相应资料不予推荐,且后者需同时服用3种抗菌药物,不仅有可能增加抗菌药物的不良反应,还使治疗失败后抗菌药物的选择余地减小。

在四联方案中,如初次治疗失败,可在剩余方案中再选择一种方案进行补救治疗。如果经过上述四联方案中的两种方案治疗失败后再次治疗时,再失败的可能性很大。在这种情况下,需再次评估根除治疗的风险-获益比。胃MALT淋巴瘤、有并发症史的消化性溃疡、有胃癌危险的胃炎(严重全胃炎、胃体为主胃炎或严重萎缩性胃炎等)或有胃癌家族史者,根除{Hp}
获益较大。
\paragraph{后续治疗}

对于{Hp} 阳性的消化性溃疡常规行{Hp}
根除治疗结束后,仍应继续使用PPI至疗程结束。再予2~4周DU或4~6周GU抑酸分泌治疗。

\subsection{难治性溃疡的治疗}

首先需排除{Hp}
感染、服用NSAIDs和胃泌素瘤的可能,排除其他病因如克罗恩病所致的良性溃疡及早期溃疡型恶性肿瘤。难治性溃疡去除病因后(如根除{Hp}
感染、停服NSAIDs等),使用PPI或加倍剂量后大多数溃疡均可愈合。如果药物治疗失败者宜考虑手术。

\section{胃食管反流病}

胃食管反流病(gastroesophageal reflux
disease,GERD)系指胃内容物反流入食管,引起不适症状和(或)并发症的一种疾病。GERD可分为非糜烂性反流病(non-erosive
reflux disease,NERD)、糜烂性食管炎(erosive
esophagitis,EE)和Barrett食管(barrett's
esophagus,BE)三种类型。在GERD的三种类型中,NERD约占70%;EE可合并食管狭窄、溃疡和消化道出血;BE有可能发展为食管腺癌。

\subsection{流行病学}

GERD的发病率有明显的地理差异。西方国家较为常见,但亚洲的发病率正在逐年上升。

\subsection{病因与发病机制}

GERD的主要发病机制是抗反流防御机制减弱和反流物对食管黏膜攻击作用的结果。

\subsubsection{食管抗反流防御机制减弱}

抗反流防御机制包括抗反流屏障、食管对反流物的清除及黏膜对反流攻击作用的抵抗力。
\paragraph{抗反流屏障}

是指在食管和胃交接的解剖结构,包括食管下括约肌(lower esophageal
sphincter,LES)、膈肌脚、膈食管韧带、食管与胃底间的锐角(His角)等,上述各部分的结构和功能上的缺陷均可造成胃食管反流,其中最主要的是LES的功能状态。

LES是指食管末端约3~4cm长的环形肌束。正常人静息时LES压为10~30mmHg,为一高压带,防止胃内容物反流入食管。LES部位的结构受到破坏时可使LES压下降,如贲门失弛缓症手术后易并发反流性食管炎。某些因素可导致LES压降低,如某些激素(如胆囊收缩素、胰高糖素、血管活性肠肽等)、食物(如高脂肪、巧克力等)、药物(如钙拮抗药、地西泮)等。腹内压增高(如妊娠、腹水、呕吐、负重劳动等)及胃内压增高(如胃扩张、胃排空延迟等)均可引起LES压相对降低而导致GERD流。

一过性LES松弛(transient LES
relaxation,TLESR)是近年研究发现引起GERD的一个重要因素。正常情况下当吞咽时,LES即松弛,食物得以进入胃内。TLESR是指非吞咽情况下LES自发性松弛,其松弛时间明显长于吞咽时LES松弛的时间。TLESR既是正常人生理性GERD的主要原因,也是LES静息压正常的胃食管反流病患者的主要发病机制。
\paragraph{食管清除作用}

正常情况下,一旦发生GERD,大部分反流物通过1或2次食管自发和继发性蠕动性收缩将食管内容物排入胃内,即容量清除,是食管廓清的主要方式,剩余的则由唾液缓慢地中和。故食管蠕动和唾液产生的异常也参与发病机制。
\paragraph{食管黏膜屏障}

反流物进入食管后,食管还可以凭借食管上皮表面黏液、不移动水层和表面\ce{HCO3-}
、复层鳞状上皮等构成的上皮屏障,以及黏膜下丰富的血液供应构成的上皮屏障,发挥其抗反流物对食管黏膜损伤的作用。因此,任何导致食管黏膜屏障作用下降的因素(长期吸烟、饮酒以及抑郁等),将使食管黏膜不能抵御反流物的损害。

\subsubsection{反流物对食管黏膜的攻击作用}

在食管抗反流防御机制下降的基础上,胃酸与胃蛋白酶是反流物中损害食管黏膜的主要成分,胆汁反流时,非结合胆盐和胰酶也参与损害食管黏膜。

\subsubsection{裂孔疝和GERD}

裂孔疝是部分胃经膈食管裂孔进入胸腔的疾病,可引起GERD并降低食管对酸的清除,导致GERD病。不少GERD患者伴有裂孔疝,但裂孔疝并不都合并GERD,两者之间的病因关系还不明确。

\subsection{临床表现}

GERD的临床表现可分为典型症状、非典型症状和消化道外症状。典型症状有胃灼热、反流;非典型症状为胸痛、上腹部疼痛和恶心、反胃等;消化道外症状包括口腔、咽喉部、肺及其他部位(如脑、心)的一些症状。

\subsubsection{胸骨后烧灼感或疼痛}

本病的主要症状,多在餐后1h左右发生;半卧位、前屈位或剧烈运动可诱发,而过热、过酸食物则可使之加重,口服制酸剂后症状多可消失。但胃酸缺乏者烧灼感主要由胆汁反流所致,故服用制酸剂效果不显著。烧灼感的严重程度与病变的轻重不平行。严重食管炎尤其是瘢痕形成者可无或仅有轻微烧灼感。但注意烧心和(或)反流症状并不是GERD所特有,烧心症状也可见于消化性溃疡、功能性烧心、嗜酸性细胞性食管炎等,少数食管癌或胃癌患者也可有烧心和(或)反流症状。

\subsubsection{胃、食管反流}

每于餐后、躯体前屈或夜间卧床睡觉时,有酸性液体或食物从胃、食管反流至咽部或口腔,此症状多在胸骨后烧灼感或烧灼痛发生前出现。

\subsubsection{咽下困难}

初期常可因食管炎引起继发性食管痉挛而出现间歇性咽下困难。后期由于食管瘢痕形成狭窄,烧灼感或烧灼痛减轻,而出现永久性咽下困难,进食固体食物时可在剑突处引起堵塞感或疼痛。

\subsubsection{消化道外症状}

反流的胃液尚可侵蚀咽部、声带和气管而引起慢性咽炎、慢性声带炎和气管炎,临床上称之Delahunty综合征。胃液反流及胃内容物吸入呼吸道尚可致吸入性肺炎。近年来研究表明GERD与部分反复发作的哮喘、咳嗽、声音嘶哑、夜间睡眠障碍、咽炎、耳痛、龈炎、癔球症、牙釉质腐蚀等有关。婴儿LES尚未发育,易发生GERD并引起呼吸系统疾病甚至营养、发育不良。

\subsection{诊断}

反复胃灼热是GERD的特征性症状。胃灼热频率、程度及时间与内镜检查病变的轻重无关。对于伴有典型反流综合征又缺乏报警症状(吞咽困难、吞咽痛、出血、体重减轻或贫血)的患者,可行PPI诊断性治疗。PPI试验:服用标准剂量PPI,例如奥美拉唑20mg每日2次,疗程1~2周,服药后若症状消失或明显改善则为PPI试验阳性,支持GERD的诊断;若停药后症状复发,复治再次取得阳性效果,则比监测食管pH值更可信。反之,如症状未见改善,并非是选择抗反流手术的指征,而是需鉴别其他可引起胃灼热的疾病。PPI试验已被证实是GERD诊断简便、无创、敏感的方法,缺点是特异性较低。

对于PPI治疗无效或具有报警症状以及症状长期持续易造成BE危险的患者应该做进一步的检查。若内镜发现食管下段有明显黏膜破损及病理支持GERD的炎症表现,则EE诊断明确。对NERD的诊断,有限的资料显示大多数NERD在其演进过程中并不发展为EE。患者以胃灼热症状为主诉时,如能排除可能引起胃灼热症状的其他疾病,且内镜检查未见食管黏膜破损,可做出NERD的诊断。

\subsection{治疗}

\subsubsection{调整生活方式}

(1)体位:是减少反流的有效方法,如餐后保持直立,避免过度负重,不穿紧身衣,抬高床头等。

(2)饮食:肥胖者应减肥。睡前2~3h勿进食,以减少夜间食物刺激的胃酸分泌。饮食宜少量、高蛋白、低脂肪和高纤维素,限制咖啡因、酸辣食品、巧克力等。

(3)药物:许多药物能降低LES的压力,使抗反流屏障失效,如黄体酮、茶碱、PGE{1}
、PGE{2}
、抗胆碱药、β受体激动药、α受体阻滞药、多巴胺、地西泮和钙拮抗药等。

\subsubsection{药物治疗}

(1)质子泵抑制药(PPI):能持久抑制基础与刺激后胃酸分泌,是治疗GERD最有效的药物。目前临床应用的有奥美拉唑、兰索拉唑、泮托拉唑、雷贝拉唑、埃索美拉唑等,药物剂量分别为每次20、30、40、10、20mg,每日1或2次口服。PPI治疗仅部分有效的患者,增加至双倍剂量或更换不同的PPI可进一步缓解症状。PPI常规或双倍剂量治疗8周后,多数患者症状完全缓解,甚至愈合。但由于PPI只是减轻反流物刺激,患者LES张力未能得到根本改善,不能防止反流,故停药后约80%的病例在6个月内复发。所以推荐在愈合治疗后继续维持治疗1个月。若停药后仍有复发,建议在再次取得缓解后给予按需维持治疗。按需维持治疗即在PPI中任选一种,当有症状出现时及时用药控制症状,可节省患者的治疗费用。

(2)H$_2$
受体拮抗药:可缓解烧心症状,治疗GERD的疗效显著低于PPI,可用于NERD患者的维持治疗。有夜间反流客观证据者,如需要可在PPI日间治疗的基础上睡前加服H$_2$
RA,但有可能在服用1个月后出现快速耐药,疗效会有所降低。夜间酸突破的定义是PPI每日2次饭前服用,夜间(22∶00---06∶00)胃内pH值<4.0的连续时间>60min。

(3)抗酸药和黏膜保护药:近来较常用的有铝碳酸镁,尤其适用于非酸反流相关GERD患者。

(4)促动力药:如多潘立酮、莫沙必利等。

(5)用药个体化:不同患者用药要个体化。可根据临床分级,轻度GERD及反流性食管炎(RE)可单独选用PPI或促动力药;中度GERD及RE宜采用PPI和促动力药联用;重度GERD宜加大PPI口服剂量,或PPI与促动力药联用。对久治不愈或反复发作伴有明显焦虑或抑郁者,应加用抗抑郁或抗焦虑治疗。5-羟色胺再摄取抑制药(如氟西丁)或5-羟色胺及去甲肾上腺素再摄取抑制药(如文拉法辛)可用于伴有抑郁或焦虑症状的GERD患者的治疗。根除{Hp}
是否增加GERD发生危险性的问题尚有争议,我国第四次全国幽门螺杆菌感染处理共识报告支持对长期需要PPI治疗的患者根除{Hp}
。

\subsubsection{GERD的内镜治疗}

美国食品药品监督管理局批准两种新的内镜手术治疗GERD,即Stretta和EndoCinc法,初步结果提示两者可使GERD患者对药物治疗的依赖性降低30%~50%,但长期安全性及有效性仍然有待随访。对于并发食管狭窄的患者,应当首选扩张治疗;在进行内镜手术之前也宜先行扩张治疗。

\section{消化道出血}

消化道出血(gastrointestinal
bleeding)根据出血部位分为上消化道出血和下消化道出血。上消化道出血是指Treitz韧带以上的食管、胃、十二指肠和胰胆等病变引起的出血,包括胃空肠吻合术后的空肠上段病变。Treitz韧带以下的肠道出血称为下消化道出血。根据失血量与速度将消化道出血分为慢性隐性出血、慢性显性出血和急性出血。根据出血病因可分为非静脉曲张性与静脉曲张性出血两类。短时间内消化道大量出血称急性大出血,常伴有急性周围循环障碍,病死率约占10%。不明原因消化道出血占消化道出血的3%~5%,指常规消化内镜检查(包括上消化道内镜、结肠镜)和X线小肠钡剂检查(口服钡剂或钡剂灌肠造影)不能明确病因的持续或反复发作的出血。

\subsection{部位与病因}

\subsubsection{上消化道出血的病因}

临床上最常见的出血病因是消化性溃疡、食管胃底静脉曲张破裂、急性糜烂出血性胃炎和胃癌,这些病因约占上消化道出血的80%~90%。

\subsubsection{下消化道出血病因}

(1)大肠癌和大肠息肉:最常见。

(2)肠道炎症性疾病:细菌性感染、寄生虫感染、非特异性肠炎、抗生素相关性肠炎、缺血性肠炎等。

(3)血管病变:作为下消化道出血病因的比例在上升,如毛细血管扩张症、血管畸形、肠血管瘤及血管畸形。

(4)肠壁结构异常:如肠套叠等。

(5)肛管疾病:痔疮、肛裂等。

\subsection{临床表现}

消化道出血的临床表现取决于出血病变的性质、部位、失血量与速度,与患者的年龄、心肾功能等全身情况也有关。

\subsubsection{呕血、黑便和便血}

呕血与黑便是上消化道出血特征性表现。出血部位在幽门以上者常伴有呕血;幽门以下出血,若出血量大、速度快,亦可因为血反流入胃腔引起恶心呕吐而表现为呕血。呕血多棕褐色呈咖啡渣样,若出血速度快而出血量多,未与胃酸充分混合即呕出,则为鲜红色或血块。若少量出血则表现为黑便、柏油样便(黏稠发亮)或粪便隐血试验阳性。出血速度过快,在肠道停留时间短,则解暗红色血便。下消化道出血一般为血便或暗红色大便,一般不伴呕血。右半结肠出血时,粪便颜色为暗红色;左半结肠及直肠出血,粪便颜色为鲜红色。在空回肠及右半结肠病变引起小量渗血时,也可有黑便。

\subsubsection{失血性周围循环衰竭}

消化道出血因循环血容量迅速减少可致急性周围循环衰竭,多见于短时期内出血超过1000mL者。临床上可出现头昏、乏力、心悸、出冷汗、黑矇或晕厥、皮肤湿冷;严重者呈休克状态。

\subsubsection{贫血与血象变化}

慢性消化道出血在常规体检中发现小细胞低色素性贫血。急性大出血后早期因有周围血管收缩与细胞重新分布等生理调节,血红蛋白、红细胞和血细胞压积的数值可无变化,一般需经3~4h以上才出现贫血。此后,大量组织液渗入血管内以补充失去的血浆容量,血红蛋白和红细胞因稀释而降低。出血后24~72h,血红蛋白可稀释到最大限度。失血会刺激骨髓代偿性增生,外周血网织红细胞增多。

\subsubsection{氮质血症}

在大量消化道出血后,血液蛋白分解产物在肠道内被吸收,以致血中氮质升高,称肠源性氮质血症。一般出血后1或2d达高峰,出血停止后3或4d恢复。

\subsubsection{发热}

大量出血后,多数患者在24h内出现低热,持续数日至1周。发热的原因可能由于血容量减少、贫血、血分解蛋白的吸收等因素导致体温调节中枢的功能障碍。分析发热原因时要注意寻找其他因素,例如有无并发肺炎等。

\subsection{诊断}

\subsubsection{确定消化道出血}

必须排除消化道以外的出血因素。首先应与口、鼻、咽部出血区别;也需与呼吸道和心脏疾病导致的咯血相鉴别。此外,口服动物血液、骨炭、铋剂和某些中药也可引起粪便发黑,应注意鉴别。

\subsubsection{消化道出血的部位}

呕血和黑便多提示上消化道出血,血便大多来自下消化道。上消化道大出血可表现为暗红色血便,如不伴呕血,常难以与下消化道出血鉴别。而慢性下消化道出血也可表现为黑便,常难以判别出血部位,应在病情稳定后急诊内镜检查。

\subsubsection{出血严重程度的估计和周围循环状态的判断}

成人每日出血量达5~10mL时,粪隐血试验可呈现阳性;每日出血量达50~100mL时,可出现黑便;胃内积血量达250mL时,可引起呕血。1次出血量<400mL时,因轻度血容量减少可由组织液及脾脏贮血所补充,多不引起全身症状。出血量>400mL时,可出现头昏、心悸、乏力等症状。短时间内出血量>1000mL,可出现休克表现。

对于上消化道出血量的估计,主要应动态观察周围循环状态,特别是血压、心率。如果患者由平卧者位改为坐位血压下降幅度(>15mmHg)、心率加快(>10次/min),提示血容量明显不足,是紧急输血的指征。患者血红细胞计数、血红蛋白及血细胞比容测定,也可作为估计失血程度的参考。休克指数(心率/收缩压)也是判断失血量的重要指标,轻度出血为0.5,中度出血为1.0,重度出血为1.5以上。

\subsubsection{出血是否停止的判断}

有下列临床表现应认为有继续出血或再出血,需及时处理:①反复呕血,黑便次数增多,粪便稀薄,伴有肠鸣音亢进。②周围循环衰竭的表现经积极补液输血后未见明显改善,或虽有好转而又恶化。③红细胞计数、血红蛋白测定与血细胞比容持续下降,网织红细胞计数持续增高。④补液与尿量足够的情况下,血尿素氮再次增高。⑤管抽出物有较多新鲜血。

\subsubsection{出血病因诊断}

消化性溃疡患者多有慢性、周期性、节律性上腹疼痛或不适史。服用非甾体类抗炎药或肾上腺皮质激素类药物或处于严重应激状态者,其出血可能为急性胃黏膜病变。有慢性肝炎、酗酒、血吸虫等病史,伴有肝病、门脉高压表现者,以食管胃底静脉曲张破裂出血为最大可能。应当指出的是,肝硬化患者出现上消化道出血,有部分患者出血可来自于消化性溃疡、急性糜烂出血性胃炎、门脉高压性胃病。45岁以上慢性持续性粪便隐血试验阳性,伴有缺铁性贫血、持续性上腹痛、厌食、消瘦者,应警惕胃癌的可能性。50岁以上原因不明的肠梗阻及便血者,应考虑结肠肿瘤。60岁以上有冠心病、心房颤动病史的腹痛及便血者,缺血性肠病可能性大。

\subsection{治疗}

\subsubsection{一般治疗}

卧床休息,严密监测患者生命体征,必要时行中心静脉压测定。观察呕血及黑便情况。定期复查血红蛋白浓度、红细胞计数、血细胞比容与血尿素氮。意识障碍、排尿困难及所有休克患者留置尿管,记录每小时尿量。意识障碍的患者要将头偏向一侧,避免呕血误吸。

\subsubsection{补充血容量}

立即查血型与配血,尽快建立有效的静脉输液通道补充血容量,改善周围循环,防止微循环障碍引起脏器功能障碍,酌情输血,对于严重出血的患者,应当开放两条甚至以上的通畅静脉通路。在配血同时可先用葡萄糖酐或其他血浆代用品。紧急输血指征:改变体位出现血压下降、心率增快、晕厥;失血性休克;Hb<70g/L,血细胞比容<25%。应避免输血、输液量过多而引起急性肺水肿;对肝硬化门静脉高压患者,门静脉压力的增加会诱发再出血;肝硬化患者宜用新鲜血。需要注意的是,不宜单独输血而不输晶体液或胶体液,因患者急性失血后血液浓缩,此时单独输血并不能有效改善微循环缺血、缺氧状态。在积极补液的前提下如果患者的血压仍然不能提升到正常水平,为了保证重要脏器的血液灌注,可以适当地选用血管活性药物(如多巴胺),以改善重要脏器的血液灌注。

\subsubsection{急性上消化道出血的止血治疗}

(1)PPI:抑酸药物能提高胃内pH值,既可促进血小板聚集和纤维蛋白凝块的形成,避免血凝块过早溶解,有利于止血和预防再出血,又可治疗消化性溃疡。大出血者推荐大剂量PPI治疗。如奥美拉唑80mg静脉推注后,以8mg/h输注持续72h;埃索美拉唑80mg静脉推注后,以8mg/h的速度持续静脉泵入(滴注);泮托拉唑的使用方法为每次40mg,每日1或2次,静脉滴注。

(2)H$_2$
RA:止血效果较PPI差,注射剂有法莫替丁、雷尼替丁等。法莫替丁的使用方法为20mg缓慢静脉推注或滴注,每日2次。

(3)止血药物:疗效尚未证实,不推荐作为一线药物使用,对没有凝血功能障碍的患者,应避免滥用此类药物。对有凝血功能障碍者,可静脉注射维生素K{1}
。

(4)洗胃:对插入胃管者可用冰冻去甲肾上腺素溶液(去甲肾上腺素8mg,加入冰生理盐水100~200mL)。

(5)生长抑素及其类似物。生长抑素是由14个氨基酸组成的环状活性多肽,能够减少内脏血流、降低门静脉压力、抑制胃酸和胃蛋白酶分泌、抑制胃肠道及胰腺肽类激素分泌等,是肝硬化急性食道胃底静脉曲张出血的首选药物之一,也被用于急性非静脉曲张出血的治疗。生长抑素静脉注射后在1min内起效,15min内即可达峰浓度,半衰期为3min左右,有利于早期迅速控制急性上消化道出血。使用方法:首剂量250μg缓慢推注作为负荷剂量,继以250μg/h静脉泵入(或滴注),疗程5d。因其半衰期极短,滴注过程中不能中断,若中断超过5min,应重新注射首剂。对于高危患者(Child-Pugh
B、C级或红色征阳性等),选择高剂量(500μg/h)生长抑素持续静脉泵入(滴注),在改善患者内脏血流动力学、控制出血和提高存活率方面均优于常规剂量。难以控制的急性上消化道出血,可根据病情重复250μg冲击剂量快速静脉滴注,最多可达3次。少数患者用药后产生恶心、眩晕、脸红等反应,当滴注速度高于50μg/min时,患者会出现恶心和呕吐现象。由于生长抑素抑制胰岛素及胰高血糖素的分泌,在治疗初期会引起短暂的血糖水平下降。

生长抑素类似物保留了生长抑素的多数效应,也可作为急性静脉曲张出血的常用治疗药物,其在非静脉曲张出血方面的治疗作用尚待进一步研究证实。奥曲肽是人工合成的8肽生长抑素类似物。皮下注射后吸收迅速而完全,30min血浆浓度可达到高峰,消除半衰期为100min,常用剂量每次0.1mg,每8h皮下注射1次。静脉注射后其消除呈双相性,半衰期分别为10min和90min。使用方法:首剂静脉滴注100μg,继以25~50μg/h持续静脉泵入(滴注),必要时剂量加倍。最常见的不良反应为腹泻、腹痛、胃肠胀气以及注射部位局部疼痛或刺激。

(6)血管升压素及其类似物。垂体后叶素:对平滑肌有强烈收缩作用,尤对血管及子宫作用更强。剂量为0.2IU/min静脉持续滴注,可逐渐增加剂量至0.4IU/min。该药可致腹痛、血压升高、心律失常、心绞痛等不良反应,严重者甚至可发生心肌梗死,故对患有肾炎、心肌炎、血管硬化、骨盆过窄等患者不宜应用,对老年患者应同时使用硝酸甘油,以减少该药的不良反应。

特利加压素是合成的血管升压素类似物,是一种前体药物,本身无活性,在体内经氨基肽酶作用,脱去其N末端的3个甘氨酰残基后,缓慢“释放”出有活性的赖氨酸加压素,主要作用是收缩内脏血管平滑肌,减少内脏血流量(如减少肠系膜、脾、子宫等的血流),从而减少门静脉血流、降低门静脉压,同时也可作用于食道和子宫等平滑肌。与血管加压素相比,它作用持久,不引起危险性并发症,包括促纤维蛋白溶解以及心血管系统方面的严重并发症。

(7)抗菌药物:主要用于肝硬化急性静脉曲张破裂出血者,无论有无腹水,短期内预防性使用抗菌药物有助于减少感染,提高存活率。

\subsubsection{气囊压迫术}

使用三腔二囊管对胃底和食管下段作气囊填塞,常用于药物止血失败者,可有效控制出血,但患者痛苦大,复发率高,并发症多,如吸入性肺炎、窒息、食管黏膜坏死、心律失常等,严重者可致死亡。目前已很少单独应用,仅作为过渡性疗法,以获得内镜或介入手术止血的时机。

\subsubsection{内镜治疗}

经过抗休克和药物治疗血流动力学稳定者应立即行急诊内镜检查,以明确上消化道出血原因及部位。经内镜直视下局部喷洒5%孟氏液(碱式硫酸铁溶液)、8%去甲肾上腺素液、凝血酶。也可在出血病灶注射1%乙氧硬化醇、1∶10000肾上腺素或血凝酶。内镜直视下应用高频点灼、激光、热探头、微波、止血夹等。对于门脉高压者,如果仅有食管静脉曲张,还在活动性出血者,可予以内镜下注射硬化剂止血,止血成功率为90%。如果在做内镜检查时,食管中下段曲张的静脉已无活动性出血,可用皮圈进行套扎。胃底静脉出血,宜注射组织黏合剂。

\subsubsection{介入治疗(选择性血管造影及栓塞治疗)}

急性大出血无法控制的患者应当及早考虑行介入治疗。临床推荐等待介入治疗期间可采用药物止血,如持续静脉滴注生长抑素+PPI控制出血,提高介入治疗成功率,降低再出血发生率。选择性胃左动脉、胃十二指肠动脉、脾动脉或胰十二指肠动脉血管造影,针对造影剂外溢或病变部位经血管导管滴注血管加压素或去甲肾上腺素,使小动脉和毛细血管收缩,进而使出血停止。无效者可用明胶海绵栓塞。

经颈静脉肝内门体静脉支架分流术主要适用于出血保守治疗(药物、内镜治疗等)效果不佳、外科手术后再发静脉曲张破裂出血或终末期肝病等待肝移植术期间静脉曲张破裂出血等待处理。经颈静脉肝内门体静脉支架分流术对急诊静脉曲张破裂出血的即刻止血成功率高,但远期(≥1年)疗效不满意。

\subsubsection{外科手术}

药物、内镜及介入治疗仍不能止血、持续出血将危及患者生命时,需不失时机进行手术。而对于食管胃底静脉曲张破裂者,若患者肝脏储备功能为Child-pugh
A级可行断流术,但并发症多,病死率较高。

\subsubsection{下消化道大量出血的处理}

基本措施是输血、输液、纠正血容量不足引起的休克。再针对下消化道出血的定位及病因诊断做出相应治疗。

对于炎症及免疫性病变(如重型溃疡性结肠炎、Crohn病、过敏性紫癜等),应通过抗炎达到止血的目的。

(1)糖皮质激素:大出血时,应予琥珀酸氢化可的松每日300~400mg或甲强龙每日40~60mg静脉滴注。病情缓解后可改口服泼尼松每日20~60mg。

(2)生长抑素或奥曲肽:大出血时使用方法同前,少量慢性出血,可皮下注射奥曲0.1mg,每日1~3次。

(3)5-氨基水杨酸类:适用于少量慢性出血。

肠息肉多在内镜下切除,痔疮可通过局部药物治疗、注射硬化剂及结扎疗法止血。

对于血管畸形如有条件可在内镜下止血治疗。各种病因引起的动脉性出血,若内镜下不能止血,可行肠系膜上、下动脉血管介入栓塞治疗。对于弥漫出血、血管造影检查无明显异常征象者或无法选择性插管的消化道出血患者,可经导管动脉内注入止血药物,使小动脉收缩,血流量减少,达到止血目的。

经内科保守治疗仍出血不止,危及生命,无论出血病变是否确诊,均是紧急手术的指征。

对弥漫性血管扩张病变所致的出血,内镜下治疗或手术治疗有困难,或治疗后仍反复出血,可考虑雌激素和孕激素联合治疗。

\section{炎症性肠病}

炎症性肠病(inflammatory bowel
disease,IBD)是一种病因不清的慢性非特异性肠道炎症性疾病,包括溃疡性结肠炎(ulcerative
colitis,UC)和克罗恩病(crohn's
disease,CD)。溃疡性结肠炎是结肠黏膜层和黏膜下层连续性炎症,通常先累及直肠,逐渐向全结肠蔓延。克罗恩病为可累及全消化道的肉芽肿性炎症,非连续性,最常累及部位为末端回肠、结肠和肛周。

\subsection{流行病学}

IBD在西方国家较为常见,目前我国发病率也逐年上升,UC和CD的发病率分别为11.6/10万和1.4/10万。据我国IBD协作组调查,我国IBD发病率男性高于女性。青春后期或成年初期是IBD主要的发病年龄段。

\subsection{病因与发病机制}

\subsubsection{环境因素}

近年来,IBD的发病率在社会经济较发达的地区持续升高,如北美、北欧,继之西欧、日本、南美等。移民学研究提示,南亚裔发病率低,但移居至英国后IBD发病率增高,表明环境因素起着重要作用。流行病学研究提出不少与IBD相关的环境因素。目前确切的是吸烟对UC者起保护作用,被动吸烟者中发病率也明显降低,而吸烟却促进CD者恶化。

\subsubsection{遗传因素}

IBD一级亲属中发病率是普通人群的30~100倍。目前认为IBD不仅是多基因疾病,也是一种遗传异质性疾病,患者在一定环境因素作用下由于遗传易感性而发病。

\subsubsection{感染因素}

至今尚未发现直接特异性微生物感染与IBD的确切关系。肠道感染可能是疾病的一种诱发因素,特别是菌群的改变可能通过抗原刺激、肠上皮细胞受损、黏膜通透性增加,引起肠黏膜持续性炎症。肠道黏膜免疫反应异常激活是导致IBD肠道炎症持续性发生、发展和转归的直接因素。

目前,对于IBD的病因尚未完全明确。认为其发病机制可能为:环境因素作用于遗传易感者,在肠腔内菌丛或食物等抗原参与下,启动了肠道的免疫系统,引起肠道免疫炎症反应过度亢进且持续发展。UC和CD是同一疾病的不同亚型,均为免疫调节紊乱引起肠黏膜难以自限的炎症反应,由于致病因素和参与免疫的炎症因子不同,最终导致不同的组织损伤。

\subsection{临床表现}

一般起病缓慢,少数急骤。病情轻重不一。易反复发作,发作的诱因有精神刺激、过度疲劳、饮食失调、继发感染等。

\subsubsection{腹部症状}

(1)腹泻:血性腹泻是UC最主要的症状,粪中含血、脓和黏液。轻者每日2~4次,严重者可达10~30次,呈血水样。CD腹泻为常见症状,多数每日大便2~6次,糊状或水样,一般无脓血或黏液;与UC相比,便血量少,鲜红色少。

(2)腹痛:UC常局限于左下腹或下腹部,阵发性痉挛性绞痛,疼痛后多有便意,排便后疼痛暂时缓解。绝大多数CD均有腹痛,性质多为隐痛、阵发性加重或反复发作。部位以右下腹多见,与末端回肠病变有关,其次为脐周或全腹痛。餐后腹痛与胃肠反射有关。少数首发症状以急腹症手术,发现为克罗恩病肠梗阻或肠穿孔。

(3)里急后重:因直肠炎症刺激所致。

(4)腹块:部分CD可出现腹块,以右下腹和脐周多见,多由于肠粘连、肠壁和肠系膜增厚、肠系膜淋巴结肿大所致,内瘘形成以及腹内脓肿等均可引起腹块。

(5)肛门症状:CD偶有肛门内隐痛,可伴肛旁周围脓肿、肛瘘管形成。

\subsubsection{全身症状}

(1)贫血:常有轻度贫血,疾病急性暴发时因大量出血致严重贫血。

(2)发热:急性重症病例常伴有发热及全身毒血症状。1/3CD者有中度热或低热,间歇出现,为活动性肠道炎症及组织破坏后毒素吸收引起。

(3)营养不良:肠道吸收障碍和消耗过多,常引起患者消瘦、贫血、低蛋白血症等表现。年幼时患病者伴有生长发育迟缓的表现。

(4)肠外表现:包括口腔、眼部、皮肤、肝胆、骨关节、泌尿、血液系统都可出现相关病变。

\subsection{诊断}

主要手段包括病史采集、体格检查、实验室检查、影像学、内窥镜检查和组织细胞学特征。

\subsubsection{UC诊断标准}

若有典型临床表现为疑诊UC患者,应安排进一步检查;根据临床表现和结肠镜或钡剂灌肠检查中一项,可为拟诊者,若有病理学特征性改变,可以确诊;初发病例、临床表现和结肠镜改变均不典型,应列为“疑诊”随访;对结肠镜检查发现的轻度直肠、乙状结肠炎症不能等同于UC,需认真检查病因,观察病情变化。诊断包括疾病类型、病情程度、活动性、病变范围、并发症和肠外表现,以便选择治疗方案、用药途径和评估预后。
\paragraph{临床类型}

分为初发型、慢性复发型、慢性持续型和暴发型。
\paragraph{临床病情程度}

UC病情分为活动期与缓解期,活动期按照严重程度分为轻度、中度、重度。

(1)轻度:最常见,起病缓慢,排便次数增加不多,粪便可成形,血、脓和黏液较少,腹痛程度较轻,全身症状和体征少。

(2)中度:介于轻度和重度之间。但可在任何时候发展为重度。

(3)重度:起病急骤,有显著腹泻、便血、贫血、发热、心动过速、厌食和体重减轻,甚至发生失水和虚脱等毒血症状。常有严重腹痛、全腹压痛,发展成中毒性巨结肠。血白细胞计数增多,ESR加速,低白蛋白血症。改良的Truelove和Witts严重程度分型标准临床较实用,如表13-3所示\footnote{ESR表示红细胞沉降率(erythrocyte sedimentation rate)。}。

\begin{table}
    \centering
    \caption{改良的Truelove和Witts严重程度分型标准}
    \label{tab13-3}
    \begin{tabular}{ccccccc}
        \toprule
        严重程度分型 & 排便/(次/d) & 便血   & 脉搏/(次/min) & 体温/°C & 血红蛋白     & ESR  \\
        \midrule
        轻度         & <4          & 轻或无 & 正常          & 正常    & 正常         & < 20 \\
        重度         & ≥6          & 重     & > 90          & >37. 8  & < 75\%正常值 & >30  \\
        \bottomrule
    \end{tabular}
\end{table}

\paragraph{病变范围}

分为直肠炎、直肠乙状结肠炎、左半结肠炎、广泛性结肠炎以及全结肠炎。
\paragraph{肠外表现}

(1)皮肤黏膜:表现为口腔溃疡、结节性红斑、多型红斑、坏疽性脓皮病等。

(2)眼损害:虹膜炎、巩膜炎、葡萄膜炎等。

(3)骨关节损害:可有一过性游走性关节痛,骨质疏松和骨化。偶尔有强直性脊椎炎。

(4)肝胆系统:脂肪肝、慢性活动性肝炎、胆管周围炎、硬化性胆管炎等。

(5)血液系统:可有贫血、血液高凝致血栓栓塞。

(6)泌尿系统:盆腔瘘管、肾盂肾炎和肾石病。
\paragraph{疗效标准}

结合临床症状与内镜检查。

(1)临床疗效评定:

① 缓解:临床症状消失,结肠镜复查见黏膜大致正常或无活动性炎性反应。

② 有效:临床症状基本消失,结肠镜复查见黏膜轻度炎性反应。

③ 无效:临床症状、结肠镜复查均无改善。

(2)与糖皮质激素治疗相关的特定疗效评价:

①
激素无效:经泼尼松或相当于泼尼松每日0.75mg/kg治疗超过4周,疾病仍处于活动期。

②
激素依赖:虽能保持缓解,但激素治疗3个月后,泼尼松仍不能减量至每日10mg;在停用激素3个月内复发。

\subsubsection{CD诊断标准}

有典型临床表现为疑诊CD,若符合结肠镜或影像学检查中一项,可为拟诊;若有非干酪样性肉芽肿、裂隙状溃疡和瘘管及肛门部病变特征性改变之一,可以确诊;初发病例、临床表现和结肠镜改变均不典型,应列为“疑诊”随访。
\paragraph{临床类型}

分为狭窄型、穿透型和炎症型(非狭窄型和非穿透型),各型间有交叉或互相转化。
\paragraph{临床病情程度}

可分为缓解期、轻度、中度、重度。

(1)轻度:无全身症状、腹部压痛、包块与梗阻者为轻度。

(2)重度:有腹痛、腹泻及全身症状和并发症者。

(3)中度:介于两者间。

临床上用简化克罗恩病活动指数(CDAI)计算法评估疾病活动的严重程度,如表\ref{tab13-4}所示。

\begin{longtable}[]{@{}ll@{}}
    \caption{CDAI计算法评估疾病活动的严重程度}
    \label{tab13-4}\\
    \toprule
    \endhead
    项目        & 评分\footnote{≤4分为缓解期;5~8分为中度活动期;≥9分为重度活动期。}\tabularnewline
    \midrule
    一般情况    &
    良好(0分);稍差(1分);差(2分);不良(3分);极差(4分)\tabularnewline
    腹痛        & 无(0分);轻(1分);中(2分);重(3分)\tabularnewline
    腹块        & 无(0分);可疑(1分);确定(2分);伴触痛(3分)\tabularnewline
    腹泻        & 稀便每日1次记1分\tabularnewline
    伴随疾病\footnote{伴随疾病包括关节痛、虹膜炎、结节性红斑、坏疽性脓皮病、阿弗他溃疡、裂沟、瘘管、脓肿。} & 每种症状记1分\tabularnewline
    \bottomrule
\end{longtable}

\paragraph{病变范围}

分为小肠型、结肠型、回结肠型,此外消化道其他部位也可累及,如食管、十二指肠等。
\paragraph{肠外表现}

常见关节痛(炎)、口疱疹性溃疡、结节性红斑、坏疽性脓皮病、炎症性眼病、慢性活动性肝炎、脂肪肝、胆石症、硬化性胆管炎和胆管周围炎、肾结石、血栓性静脉炎、强直性脊椎炎、淀粉样变性、骨质疏松和杵状指等。

\subsection{治疗}

\subsubsection{一般治疗}

慢性疾病常伴有营养不良,主张高糖、高蛋白、低脂饮食,少渣饮食能减少排便次数。适当补充叶酸、维生素和微量元素,全肠外营养适用于重症患者及中毒性巨结肠、肠瘘、短肠综合征等并发症者。必要时予以输血。戒烟在CD患者中有益于疾病控制。应用止泻剂(洛哌丁胺)可减轻肠道蠕动,缓解便意窘迫。但严重结肠炎时,止泻剂、解痉剂、阿片制剂、NSAID等需禁忌,有诱发中毒性巨结肠的可能。对于中毒症状明显者可考虑静脉用广谱抗菌药物。因疾病反复发作,迁延终生,患者常见抑郁和焦虑情绪,需予心理问题的防治。

\subsubsection{治疗常用药物}
\paragraph{氨基水杨酸制剂}

5-氨基水杨酸(5-Aminosalicyliacid,5-ASA)是治疗的有效成分,几乎不被吸收,其作用机制是通过对肠黏膜局部花生四烯酸代谢多个环节进行调节,抑制前列腺素、白三烯的合成,清除氧自由基,抑制免疫反应。由于5-ASA在胃酸内多被分解失效,因此常通过下述给药系统进入肠道,发挥其药理作用。

1)柳氮磺吡啶(SASP)

5-ASA通过偶氮键连接于磺胺吡啶,使之能通过胃,进入肠道。在结肠释放,SASP的偶氮键被细菌打断,5-ASA得以释放,发挥其抗炎作用,是治疗轻、中度或经糖皮质激素治疗已有缓解的重度UC常用药物。服用SASP者需补充叶酸,多饮水,保持高尿流量,以防结晶尿的发生。不良反应主要分为两类,一类是剂量相关的不良反应,如恶心、呕吐、食欲减退、头痛、可逆性男性不育等,餐后服药可减轻消化道反应;另一类不良反应属于过敏,有皮疹、粒细胞减少、自身免疫性溶血、再生障碍性贫血等,因此服药期间应定期复查血象,一旦出现此类不良反应,应改用其他药物。对磺胺及水杨酸盐过敏者、肠梗阻或泌尿系梗阻患者、急性间歇性卟啉症患者、2岁以下小儿应避免使用。

2)5-ASA前体药

主要在结肠释放。

(1)巴柳氮:巴柳氮钠是一种前体药物,是5-ASA与P氨基苯甲酰丙氨酸偶氮化合物,口服后以原药到达结肠,在结肠细菌的作用下释放出5-ASA和4-氨基苯甲酰-β-丙氨酸。每日4~6g,分次服用。

(2)奥沙拉秦:是两分子5-ASA的偶氮化合物,到达结肠部位后其偶氮键在细菌作用下断裂,分解为两分子5-氨基水杨酸并作用于结肠炎症黏膜。每日2~4g,分次进餐时口服。

3)5-ASA

(1)美沙拉嗪(结构特点:甲基丙烯酸酯控释pH值依赖):释放特点是pH值依赖,药物释放部位在回肠末端和结肠。

(2)美沙拉嗪(结构特点:乙基纤维素半透膜控释时间依赖):释放特点是纤维素膜控释时间依赖,药物释放部位在远段空肠、回肠、结肠。美沙拉嗪一般活动期病变使用剂量为每日3~4g,维持期予每日2g,分3或4次口服。以5-ASA含量计,SASP、巴柳氮、奧沙拉嗪1g分别相当于美沙拉嗪0.4、0.36和1g。

(3)美沙拉嗪栓:每次0.5~1g,每日1或2次是治疗UC直肠炎的有效措施。
\paragraph{糖皮质激素}

通过抑制T细胞激活及细胞因子分泌发挥抗炎作用。经过多年循证医学已证明其无维持缓解作用,由于存在较多不良反应,限制了其长期应用。适用于IBD急性活动且对足量5-ASA无反应尤其是病变广泛者,治疗CD时可在初期即开始使用糖皮质激素。起始剂量需足量,否则疗效降低。常用剂量:泼尼松每日0.75~1mg/kg,达到症状缓解后开始逐渐缓慢减量至停药,若快速减量会导致早期复发,完全缓解后可每周减5mg。宜同时补充钙剂与维生素D。重症患者可先予大剂量静脉滴注,如甲泼尼龙每日40~60mg,或氢化可的松每日300~400mg,剂量再大不会增加疗效,但剂量不足会降低疗效。

局部用激素如氢化可的松琥珀酸钠盐(禁用酒石酸制剂)每晚100~200mg;布地奈德泡沫剂每次2mg、每日1或2次,适用于病变局限在直肠者,该药激素的全身不良反应少。

对急性重度活动性UC,静脉用糖皮质激素为首选治疗。在静脉用足量糖皮质激素治疗大约5d仍然无效,应转换治疗方案。转换治疗方案,一是转换药物的所谓“拯救”治疗,依然无效才手术治疗;二是立即手术治疗。转换治疗方案的选择取决于病情、内外科沟通和医患沟通。环孢素是“拯救”治疗的主要药物。
\paragraph{免疫调节剂}

通过阻断淋巴细胞增殖、活化或效应机制而发挥作用。适用于激素依赖或无效及激素诱导缓解后的维持治疗,能有效维持撤离激素的临床缓解或在维持症状缓解下减少激素用量。应用AZA和6-巯基嘌呤(Mercaptopurine,6-MP)对CD活动期及维持缓解期均有效,对UC也有一定疗效。

(1)AZA:欧美治疗剂量为每日1.5~2.5mg/kg,有认为亚裔人种剂量偏低,如每日1mg/kg,但尚未达成共识。AZA存在量效关系,剂量不足会影响疗效,剂量太大不良反应严重又不能接受,因此,在AZA治疗过程中应根据疗效和不良反应进行剂量调整。使用AZA维持撤离激素缓解有效的患者,疗程一般不少于4年,如继续使用,其获益与风险应与患者商讨。需要严密监测AZA的不良反应,不良反应以服药3个月内常见,又尤以1个月内最常见。但是骨髓抑制可迟发,甚至有发生在1年及以上者。用药期间应全程监测定期随诊。第1个月内每周复查1次全血细胞,第2、3个月内每2周复查1次全血细胞,之后每月复查全血细胞,半年后全血细胞检查间隔时间可视情况适当延长,但不能停止;前3个月每月复查肝功能,之后视情况复查。欧美推荐在使用AZA前检查硫嘌呤甲基转移酶基因型,对基因突变者避免使用或减量严密监测下使用,硫嘌呤甲基转移酶基因型检查预测骨髓抑制的特异度很高,但敏感度低(尤其在汉族人群),有一定局限性。

(2)6-MP:欧美剂量为每日0.75~1.5mg/kg。由于该类药物需3~4个月才能达到稳态血药浓度,不能单独用于诱导缓解,治疗时可与激素联用,待免疫调节剂起效后,激素再逐渐减量。

临床上,UC的治疗时常会将氨基水杨酸制剂与硫嘌呤类药物合用,但氨基水杨酸制剂会增加硫嘌呤类药物骨髓抑制的不良反应,应特别注意。

AZA不能耐受者可试换用6-MP。硫嘌呤类药物无效或不能耐受者,可考虑换用MTX。

环孢素(Cyclosporin,CsA)起效迅速,多小于1周,每日2~4mg/kg静脉滴注,因不良反应大,适于短期治疗严重UC且激素治疗无效患者,可缓解症状,避免急诊手术,使用期间需定期监测血药浓度,严密监测不良反应。有效者,临床症状缓解后可改为口服CsA治疗(4~6mg/kg)一段时间(不超过6个月),逐步过渡到硫嘌呤类药物维持治疗。不能耐受者改为MTX。
\paragraph{生物制剂}

(1)英夫利昔单抗(Infliximab,IFX):是目前治疗IBD应用时间较长的生物制剂,能使大部分IBD患者(包括儿童)得到长期维持缓解、组织愈合的作用。IFX是一种人-鼠嵌合型单克隆抗体肿瘤坏死因子(TNF-α)抑制剂,主要适用于CD者,经传统治疗即激素及免疫抑制剂治疗无效或不能耐受者;合并瘘管经传统治疗(抗生素、免疫抑制剂和外科引流)无效者,激素抵抗的顽固性重度UC者的治疗药物。静脉推荐注射5mg/kg,在第0、2、6周作为诱导缓解,随后每隔8周给予相同剂量作长期维持治疗。规律用药的缓解率优于间断给药,当治疗反应欠佳时,剂量可增至10mg/kg,或者缩短给药间期。单次使用IFX
5mg/kg的有效率可达58%,对瘘管者使用IFX
3次后,55%的CD者瘘管可愈合。在使用IFX前正在接受激素治疗时应继续原来治疗,在取得临床完全缓解后将激素逐步减量至停用。对原先已使用免疫抑制剂无效者无必要继续合用免疫抑制剂。对IFX治疗前未接受过免疫抑制剂治疗者,IFX与AZA合用可提高撤离激素缓解率及黏膜愈合率,但两药长期合用可增加机会感染和淋巴瘤的发生风险,尤其应避免年轻患者长期两药合用。目前尚无足够资料提出何时可以停用IFX。对IFX维持治疗达1年,保持撤离激素缓解伴黏膜愈合及CRP正常者,可以考虑停用IFX继续以免疫抑制剂维持治疗。对停用IFX后复发者,再次使用IFX可能仍然有效。

(2)阿达木单抗(Adalimumab):是一种全人重组人IgG{1}
抗TNF单抗,赛妥珠单抗(Certolizumab
Pegol)是人源化抗TNF单抗Fab段,两者分别通过每2周和每4周皮下注射给药。若对一种抗TNF药物无反应或不耐受,仍可尝试另一种抗TNF药物。
\paragraph{抗生素类}

抗生素常用于CD并发症的治疗,即肛周病变、瘘管、炎性包块及肠道狭窄时细菌过度增长等。甲硝唑和环丙沙星是最常用于CD的一线治疗抗生素,部分患者症状可缓解,但停药后会复发。尚无数据显示抗生素对UC有效,仅用于暴发性结肠炎。使用抗生素将增加艰难梭状芽孢杆菌相关疾病的风险。

\subsubsection{治疗原则和方案选择}

治疗前,首先应对病情进行综合评估,包括病变累积范围、部位,病程的长短、疾病严重程度以及患者的全身情况,给予个体化、综合化的治疗。原则上应尽早控制疾病的症状,维持缓解,促进黏膜愈合,防止复发,防治并发症和掌握手术治疗时机(见表\ref{tab13-5})。

\begin{longtable}{cp{4cm}p{4cm}p{4cm}}
    \caption{UC和CD的治疗原则和方案}
    \label{tab13-5}\\
        \toprule
               & 远端UC                                  & 广泛性UC                          & CD                                    \\
        \midrule
        轻度   & 直肠或口服5-ASA、直肠GCS                & 局部十口服5-ASA                   &
        无并发症时仅可给予5-ASA、肛周病变时予抗生素、回肠和(或)右半结肠病变时给予GCS                                                 \\
        中度   & 直肠或口服5-ASA、直肠GCS                & 局部+口服5-ASA                    & 口服GCS、AZA/6- MP、MTX抗TNF          \\
        重度   & 直肠或口服5-ASA、口服或静脉GCS、直肠GCS & 静脉GCS、静脉CsA或静脉IFX         & 口服或静脉GCS、皮下或肌肉MTX、静脉IFX \\
        顽固性 & 口服或静脉GCS+ AZA/6-MP                 & 口服或静脉GCS+ AZA/6-MP或IFX或CsA & 静脉IFX                               \\
        静止期 & 直肠或口服5-ASA、口服AZA/6- MP          & 口服5-ASA、口服AZA/6- MP          & AZA/6-MP或MTX                         \\
        肛瘘   & -                                       & -                                 & 口服抗生素、AZA/6-MP、静脉IFX         \\
        \bottomrule
\end{longtable}

UC首次发病时治疗效果较好,此后病情长期缓解和持续者各占10%,余者病情缓解与反复间歇发作常交替。而CD以慢性渐进型多见,部分自行缓解,常有反复,大多数患者经治疗后,可获得某种程度的缓解。急性重症病例常有严重毒血症和并发症,预后较差。

对所有患者一般均推荐长期或终生维持缓解。5-ASA每日1~2g常用于缓解期UC的维持治疗,对CD则作用有限;当5-ASA治疗无效时,免疫抑制剂可用于UC和CD维持缓解。CD术后给予5-ASA或6-MP/AZA口服,以减轻复发的频率及严重程度。GCS不用于维持疗法。

对病变局限在直肠或直肠乙状结肠者,强调局部用药(病变局限在直肠用栓剂、局限在直肠乙状结肠用灌肠剂),口服与局部用药联合应用疗效更佳。

