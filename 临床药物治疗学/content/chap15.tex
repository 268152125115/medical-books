\chapter{泌尿系统疾病的药物治疗}

\section{慢性肾功能不全}

\subsection{定义和流行病学资料}

慢性肾脏疾病(chronic kidney
disease,CKD)是指各种原发病或继发性CKD患者进行性肾功能损害所出现的一系列症状或代谢紊乱的临床综合征。CKD一词最早出现在2001年K/QODI的《慢性肾脏病贫血治疗指南》,2002年,在美国肾脏病基金会组织撰写的肾脏病/透析临床实践指南(KDOQI)中,首次正式提出了CKD的定义及分期。2004---2006年,经过KDIGO的再次修改及确认,CKD取代了慢性肾衰竭(CRF)、慢性肾损伤(CKI)等名称,成为对各种原因所致CKD(病程3个月以上)的统称。

据国际肾脏病学会统计,本症自然人群年发病率约为(98~198)/10{6}
,其中,经济发达国家发病率较高,约为(400~900)/10{6}
。我国近年的统计资料显示,CKD的年发病率为约2%~3%,尿毒症的年发病率为约(100~130)/10{6}
,且有逐年增多的趋势。

\subsection{病因}

近年来,CKD的病因构成有所变化,在西方国家继发性因素已占主要原因,其中糖尿病和高血压是CKD的两大首位因素,约占50%。然而,我国仍以慢性肾小球肾炎为主,但继发性因素引起的CKD逐年增多,依次为高血压、糖尿病和狼疮性肾炎。近年来,乙肝相关性肾炎导致的CKD也为国内外学者所关注。

\subsection{发病机制}

慢性肾功能不全的进展除各种病因性肾脏疾病特异性病理生理改变外,尚存在因大量肾单位丧失引起的一系列病理生理进展的共同机制。即使病因性肾脏疾病已经停止活动,但这种共同的损伤机制仍会待续进行,导致肾单位的进一步减少,最终发展为终末期肾衰竭,直至尿毒症。以下为慢性肾脏病进展的共同机制。

\subsubsection{肾小球血流动力学改变}

各种病因引起的肾单位减少,导致健存肾单位代偿性肥大。肾毛细血管内静水压和肾小球血流量增加,单个肾单位的肾小球滤过率增加,形成肾小球高灌注、高压力和高滤过。由此引起的肾小球内血流动力学变化,可进一步损伤、活化内皮细胞、系膜细胞,产生释放血管活性介质、细胞因子和生长因子,从而加重肾单位肥大和肾小球内血流动力学变化,形成恶性循环,最终导致肾小球硬化。

\subsubsection{尿蛋白加重肾脏损伤作用}

蛋白尿不仅使机体营养物质丧失,更重要的是大量蛋白从肾小球滤出后引起的一系列不良反应:

(1)肾小管上皮细胞重吸收的蛋白过多,导致细胞溶酶体破裂,释放溶酶体酶和补体。

(2)肾小管细胞摄取白蛋白、脂肪酸过多合成和释放上皮源性有化学趋化作用的脂质,引起炎症细胞浸润,释放细胞因子。

(3)滤过的大量蛋白与远端肾小管产生的Tamm-Horsfall蛋白相互反应阻塞肾小管。

(4)尿液中补体成分增加,在肾小管原位产生补体C5b-9膜攻击复合物,并可激活近曲小管刷状缘的补体替代途径。

(5)肾小管蛋白质代谢产氨增多,随之产生活化的氨基化补体C{3} 。

(6)尿中转铁蛋白释放铁离子,产生游离OH{-} 。

(7)刺激肾小管上皮细胞分泌内皮素,导致间质缺氧,产生致纤维化因子。

蛋白尿通过上述反应引起肾小管间质进一步损害及纤维化。此外,在肾功能完全正常的情况下,饮食中蛋白质负荷可使肾小球滤过率增加20%~30%。饮食中蛋白质负荷可加重肾小球高滤过状态,促进肾小球硬化;可增加尿蛋白排泄而加重尿蛋白的损伤作用。

\subsubsection{肾素-血管紧张素-醛固酮系统作用}

肾脏富含肾素-血管紧张素-醛固酮系统成分。血管紧张素Ⅱ升高可上调转化生长因子、肿瘤坏死因子、血管细胞黏附分子、核转录因子、血小板源性生长因子、成纤维细胞生长因子、胰岛素样生长因子等生长因子的表达,促进氧化应激反应,刺激内皮细胞纤溶酶抑制因子的释放,从而促进细胞增殖、细胞外基质积聚和组织纤维化。

\subsubsection{血压升高}

CKD患者常常因水钠潴留、RAAS激活以及交感神经系统兴奋而合并高血压。血压升高可增加肾小球内毛细血管压力,促进肾小球硬化;此外,长期高血压引起的肾血管病变,导致的肾缺血性损伤也可加快组织的纤维化进程。因此,高血压是导致CKD进展、肾功能恶化的重要因素之一。

\subsubsection{脂质代谢紊乱}

慢性肾脏病患者常合并脂质代谢紊乱。研究发现,巨噬细胞、系膜细胞和肾小管细胞可以产生反应性氧自由基而氧化脂蛋白,氧化的低密度脂蛋白可以刺激炎症和致纤维化细胞因子的表达和诱导细胞凋亡,而且氧化修饰的脂蛋白又可以产生反应性氧自由基,进一步氧化修饰脂蛋白,最终引起巨噬细胞大量侵入、细胞凋亡及细胞外基质积聚,导致组织损伤。

\subsubsection{肾小管间质损伤}

肾小管间质炎症、缺血及大量尿蛋白均可以损伤肾小管间质。而肾小管间质损伤引起:

(1)肾小管萎缩引起“无小管”肾小球,导致肾小球萎缩。

(2)肾小管周围毛细血管床减少引起肾小球毛细血管内压升高,导致肾小球硬化。

(3)浸润的炎症细胞和肾小管上皮细胞分泌的细胞生长因子加剧肾组织炎症和纤维化。

(4)肾小管上皮细胞在各种细胞、生长因子刺激下发生转分化,分泌细胞外基质而促进肾组织纤维化。

(5)引发肾小管重吸收、分泌和排泄障碍,导致球-管失衡、肾小球滤过率降低。

\subsection{临床表现和分期}

\subsubsection{肾功能不全分期}

按照改善全球肾脏病预后组织(Kidney Disease:Improving Global
Outcomes,KDIGO)的分期标准,可分为5期:①1期,肾功能正常,肾小球滤过率≥90mL/min;②2期,肾功能轻度下降,肾小球滤过率在60~89mL/min;③3期,肾功能中度下降,肾小球滤过率在30~59mL/min,其中又分为3a和3b期;④4期,肾功能重度下降,肾小球滤过率在15~29mL/min;⑤5期,肾衰竭,肾小球滤过率<15mL/min。

\subsubsection{临床表现}
\paragraph{消化系统表现}

食欲减退和晨起恶心、呕吐是尿毒症常见的早期表现。尿毒症晚期,因唾液中的尿素被分解成为氨,而呼出气体中有尿味和金属味。晚期患者胃肠道的任何部位都可出现黏膜糜烂、溃疡,而发生胃肠道出血、消化性溃疡多见的原因可能与胃液酸性变化、{Hp}
感染、胃泌素分泌过多等因素有关,但目前尚不十分清楚。
\paragraph{心血管系统表现}

较常见的心血管病变主要有高血压和左心室肥厚、心衰、尿毒症性心肌病、心包积液、心包炎、血管钙化和动脉粥样硬化等。近年发现,由于高磷血症、钙分布异常和“血管保护性蛋白”(如胎球蛋白A)缺乏而引起的血管钙化,在心血管病变中亦起着重要作用。
\paragraph{呼吸系统表现}

体液过多或酸中毒时均可出现气短、气促,严重酸中毒可致呼吸深长。体液过多、心功能不全可引起肺水肿或胸腔积液。由尿毒症毒素诱发的肺泡毛细血管渗透性增加、肺充血可引起“尿毒症肺水肿”,此时肺部X线检查可出现“蝴蝶翼”征,及时利尿或透析,上述症状可迅速改善。
\paragraph{血液系统表现}

CRF患者血液系统异常主要表现为肾性贫血和出血倾向。大多数患者一般均有轻、中度贫血,其原因主要由于红细胞生成素缺乏,故称为肾性贫血;如同时伴有缺铁、营养不良、出血等因素,可加重贫血程度。晚期CRF患者有出血倾向,如皮下或黏膜出血点、瘀斑、胃肠道出血、脑出血等。
\paragraph{骨骼病变}

肾性骨营养不良(即肾性骨病)相当常见,包括纤维囊性骨炎(高周转性骨病)、骨生成不良、骨软化症(低周转性骨病)及骨质疏松症。在透析前患者中骨骼X线发现异常者约35%,但出现骨痛、行走不便和自发性骨折相当少见。而骨活体组织检查约90%可发现异常,故早期诊断要靠骨活检。
\paragraph{水、电解质代谢紊乱}

慢性肾衰时,酸碱平衡失调和各种电解质代谢紊乱相当常见。在这类紊乱中,以代谢性酸中毒和水钠平衡紊乱最为常见。

(1)代谢性酸中毒:轻度慢性酸中毒时,多数患者症状较少,但如动脉血\ce{HCO3-}
<15mmol/L,则可出现明显食欲不振、呕吐、虚弱无力、呼吸深长等。

(2)水钠代谢紊乱:主要表现为水钠潴留,或低血容量和低钠血症。易出现血压升高、左心功能不全和脑水肿。低血容量主要表现为低血压和脱水。

(3)钾代谢紊乱:当GFR降至20~25mL/min或更低时,肾脏排钾能力逐渐下降,此时易于出现高钾血症;尤其当钾摄入过多、酸中毒、感染、创伤、消化道出血等情况发生时,更易出现高钾血症。

(4)钙磷代谢紊乱:主要表现为磷过多和钙缺乏。钙缺乏主要与钙摄入不足、活性维生素D缺乏、高磷血症、代谢性酸中毒等多种因素有关,明显钙缺乏时可出现低钙血症。

\subsection{诊断标准}

CKD诊断标准:有下列情况之一即可诊为CKD。

(1)肾脏损伤(有结构或功能异常)时间≥3个月,有以下一项或多项表现,而不论其GFR是否下降:①血或尿液组分异常;②影像学检查结果异常;③肾活检异常。

(2)GFR<60mL/(min·1.73m{2} )时间≥3个月,而不论有无上述肾损伤表现。

\subsection{治疗原则}

\subsubsection{早期预防及延缓进展}

如何对CRF患者进行早期预防,并延缓CRF患者的病情进展,已成为各国十分关注的一个问题。所谓早期预防又称一级预防,是相对已有的肾脏疾患或可能引起肾功能损害的疾患(如糖尿病、高血压等)进行及时有效的治疗,防止CKD的发生。所谓二级预防是指对已有轻、中度CKD的患者及时进行治疗,延缓CKD的进展,防止尿毒症的发生。特别是我国这样一个人口众多的发展中国家,透析与移植治疗目前尚未普及,更应加强CKD的早期预防和延缓病程进展,重视非透析治疗的发展、改进和推广。

\subsubsection{原发病和诱因治疗}

对于初次诊断的CKD患者,必须积极重视原发病的诊断,对慢性肾炎、狼疮性肾炎、紫癜性肾炎、糖尿病肾病等,都需要保持长期治疗,同时,也应积极寻找CKD的各种诱发因素,合理纠正这些诱因有可能会使病变减轻或趋于稳定。

\subsubsection{并发症的治疗}

积极治疗CKD导致各种并发症,包括心血管病变、肾性贫血、矿物质及骨代谢异常等。

\subsection{药物治疗}

\subsubsection{降压治疗}

良好的血压控制不仅可以延缓CRF的进展,而且可以减少心脑血管并发症的发生,降低患者病死率。降压治疗是慢性肾功能不全一体化治疗的重要组成部分。根据2012年KDIGO指南推荐血压控制目标为:尿蛋白>1g/d者,血压<125mmHg/75mmHg;尿蛋白<1g/d者,血压<130mmHg/80mmHg。
\paragraph{ACEI和ARB}

ACEI和ARB是目前临床治疗肾性高血压的基石,也是目前循证依据最多的具有肾脏和心血管保护作用的药物。2010年中国高血压防治指南指出,ACEI和ARB既有降压、又有降低蛋白尿的作用。因此,对高血压伴CKD患者,尤其有蛋白尿患者,应作为首选。

ACEI和ARB的作用:①减少血管紧张素Ⅱ合成或抑制其生物学效应;②降低交感神经的兴奋性及去甲肾上腺素的释放;③抑制激肽酶对缓激肽的降解,增加前列腺素的合成,从而具有良好的降压疗效。

更重要的是ACEI和ARB具有良好的肾脏保护作用。①改善肾血流动力学,降低肾小球内压,减少蛋白尿。②抑制系膜细胞增殖,减少细胞外基质沉积,延缓肾小球硬化。③维持肾脏调节水钠平衡的功能。④增加胰岛素敏感性,改善CRF患者的胰岛素抵抗现象和糖代谢异常。⑤改善脂代谢。

常用的ACEI类药物:贝那普利(Benazepril)每日10~20mg,1次口服;依那普利(Enalapril),每日5~20mg,1次口服;雷米普利(Ramipril),每日1.25~5mg,1次口服;福辛普利(Fusinopril)每日5~20mg,1次口服。其中贝那普利和福辛普利为肝肾双通道排泄药物。ACEI的主要不良反应:咳嗽、皮疹、味觉异常及粒细胞减少;在严重肾衰竭时可引起高钾血症并加重贫血;在低容量血症、肾动脉狭窄时会导致急性肾衰竭。

常用的ARB类药物:氯沙坦(Iosartan)每日25~100mg,1次口服;缬沙坦(Valsartan)每日80~160mg,1次口服;厄贝沙坦(Irbesartan)每日150~300mg,1次口服。ARB的不良反应与ACEI相似,但不引起咳嗽。

使用ACEI和ARB都应严密监测肾功能变化,如果肌酐升高超过基线值30%以上,应减量;超过50%,应停药。严重肾衰竭患者应慎用,妊娠、高血钾症、双侧肾动脉狭窄患者慎用。
\paragraph{钙通道阻滞剂(CCB)}

CCB通过抑制细胞膜钙通道而抑制血管平滑肌收缩,减少外周血管阻力,降低血压;对盐敏感型及低血浆肾素活性型高血压也有良好效果,不影响重要脏器的供血,不影响糖脂质及尿酸的代谢,并可改善心肌组织重塑,延迟动脉粥样硬化形成。

在肾保护作用方面:①增加肾脏血流量,但不明显增加肾小球的高滤过与毛细血管内压;②抑制系膜细胞增殖,减少细胞外基质产生;③调整系膜的大分子物质转运;④减少自由基的产生;⑤改善入球小动脉的血管重塑;⑥减少组织钙化。近年研究结果显示,非二氢吡啶类的钙通道阻滞剂(地尔硫卓、维拉帕米)可改善肾小球内毛细血管内压,也具有降低尿蛋白作用。

常用的CCB类药物:长效硝苯地平(Nifedipine)每日30~60mg,1次口服;氨氯地平(Amlodipine)每日2.5~10mg,1次口服;非洛地平(Felodipine)每日2.5~5mg,1次口服;拉西地平(Lacidipine)每日2.5~5mg,1或2次口服。CCB主要副作用为头痛、面色潮红及心悸,少数患者可出现血管神经性水肿。
\paragraph{联合药物治疗}

慢性肾功能不全时常常需要2种以上降压药物联合应用才能达到降压目标。ACEI或ARB与CCB联合应用是临床上常用组合,具有增强药物疗效,减少不良反应的效果;如仍未达到降压目标,可在此基础上加用利尿剂与α、β受体阻滞剂。但利尿剂与β受体阻滞剂影响糖、脂质代谢,并发糖尿病的患者应慎用。

对肾功能显著受损如血肌酐(Cr)>265.2μmol/L的患者、肾小球滤过率<30mL/(min·1.73m{2}
),或有大量尿蛋白的患者宜首选CCB;对终末期肾病未透析患者一般不用RAAS及噻嗪类利尿剂;而用CCB或袢利尿剂。

\subsubsection{肾性贫血治疗}

肾性贫血是CKD的重要临床表现,是CKD患者合并心血管并发症的独立危险因素,有效治疗肾性贫血是CKD一体化治疗的重要组成部分。
\paragraph{重组人促红细胞生成素(rHuEPO)}

rHuEPO是临床上治疗肾性贫血的主要药物,在我国临床应用已经10余年,是一种糖蛋白激素,相对分子质量约3400。

(1)rHuEPO治疗肾性贫血的靶目标值:Hb水平应不低于110g/L(Hct>33%),目标值应在开始治疗后4个月内达到,但不推荐Hb维持在130g/L以上。

靶目标值应依据患者年龄、种族、性别、生理需求以及是否合并其他疾病情况进行个体化调整:①伴有缺血性心脏病、充血性心力衰竭等心血管疾病的患者不推荐Hb>120g/L;②糖尿病的患者,特别是并发外周血管病变的患者,需在监测下谨慎增加Hb水平至120g/L;③合并慢性缺氧性肺疾病患者推荐维持较高的Hb水平。

(2)rHuEPO的使用时机和给药途径:若间隔2周或者以上连续2次Hb检测值均低于110g/L,并除外铁缺乏等其他贫血病因,应开始实施rHuEPO治疗。rHuEPO治疗肾性贫血,静脉给药和皮下给药同样有效。但皮下注射的药效动力学表现优于静脉注射,并可以延长有效药物浓度在体内的维持时间,节省治疗费用。皮下注射较静脉注射疼痛感增加。

(3)rHuEPO的初始使用剂量:皮下给药剂量为每千克体重100~120IU,每周2或3次。静脉给药剂量为每千克体重120~150IU,每周3次。①初始剂量选择要考虑患者的贫血程度和导致贫血的原因,对于Hb<70g/L的患者,应适当增加初始剂量;②对于非透析患者或残存肾功能较好的透析患者,可适当减少初始剂量;③对于血压偏高、伴有严重心血管事件、糖尿病的患者,应尽可能地从小剂量开始使用rHuEPO。

(4)rHuEPO的剂量调整。①rHuEPO治疗期间应定期检测Hb水平:诱导治疗阶段应每2~4周检测1次Hb水平;维持治疗阶段应每1~2月检测1次Hb水平。②应根据患者Hb增长速率调整rHuEPO剂量:初始治疗Hb增长速度应控制在每月10~20g/L范围内稳定提高,4个月达到Hb靶目标值。如每月Hb增长速度<10g/L,除外其他贫血原因,应增加rHuEPO使用剂量25%;如每月Hb增长速度>20g/L,应减少rHuEPO使用剂量25%~50%,但不得停用。③维持治疗阶段EPO的使用剂量约为诱导治疗期的2/3。若维持治疗期Hb浓度每月改变>10g/L,应酌情增加或减少EPO剂量25%。
\paragraph{铁剂}

对于尚未进入透析的CKD患者和腹膜透析患者,由于血液丢失机会相对要少,可先给予口服铁剂纠正铁缺乏,但对临床效果不佳者,应将口服铁剂改为静脉铁剂。蔗糖铁是最安全的静脉补铁形式,其次是葡萄糖酸铁,而静脉注射右旋糖酐铁有引起严重急性反应的危险。治疗方案如下。

(1)一般情况下可先选择口服铁剂:硫酸亚铁(ferrous
sulfate)每日0.3g,分3次口服;富马酸亚铁(ferrous
fumarate)每日0.2~0.4g,分3次口服;葡萄糖酸亚铁(ferrous
gluconate)每日0.3~0.6g,分3次口服。

(2)如患者口服铁剂无效或合并胃肠道疾病、不易口服铁剂时,可选用静脉补铁;若患者转铁蛋白饱和度(TSAT)<20%和(或)血清铁蛋白<100ng/mL,需每周静脉补铁100~125mg,连续8~10周;若患者转铁蛋白饱和度≥20%,血清铁蛋白水平≥100ng/mL,则每周1次静脉补铁25~125mg;若补铁后患者TSAT≥50%和(或)血清铁蛋白≥800ng/mL,应中止静脉补铁3个月。停药3个月后若复查TSAT≤50%,血清铁蛋白水平≤800ng/mL,恢复静脉补铁,但用量减去原量的1/3~1/2。

\subsubsection{肾性骨病治疗}

肾性骨病又称肾骨营养不良,是一类影响CKD患者骨骼系统的病理生理状态的总称。肾性骨病在CKD的早期就开始,当肾小球滤过率(GFR)<50%时,已有一半患者出现肾性骨病,当进入CKD终末期时,几乎所有的患者出现肾性骨病。肾性骨病根据骨转化状态可分为高转化型骨病、低转化型骨病和混合型骨病。高转化型骨病又称纤维囊性骨炎,是继发性甲状旁腺功能亢进(甲旁亢)的主要表现,成骨细胞和破骨细胞异常活跃,在骨小梁表面形成骨吸收陷窝或囊腔是其主要病理特征。低转化型骨病根据病理又分为骨软化症和无动力性骨病,前者以新形成的类骨质矿化缺陷为特征,与铝沉积密切相关,后者则表现为骨的低转化状态包括骨质减少和骨质疏松。临床上许多CKD患者具有高转化和低转化两种表现,称为混合型。
\paragraph{高转化性骨病的治疗}

1)控制血磷

控制血磷水平的措施主要包括饮食控制、透析和磷结合剂的应用。当ESRD患者血磷>1.78mmol/L时,其通过食物摄入的磷量控制在800~1000mg/d。除了饮食控制和透析之外,许多CKD患者的高磷血症往往控制不佳,还需要通过口服磷结合剂来限制磷的吸收。95%的ESRD患者在疾病的晚期需要用磷结合剂,目的是将血磷水平控制至正常范围,K/DOQI指南推荐将透析前的血磷水平控制在1.13~1.78mmol/L。

(1)含铝的磷结合剂:20世纪70年代,含铝的复合物如氢氧化铝、碳酸铝凝胶等由于良好的降磷效果曾一度被推荐为ESRD患者高磷血症的标准治疗方案用药。但是,大量铝的吸收导致患者血浆和组织中铝的含量增高,当铝在中枢神经系统积聚,将引起透析相关脑病综合征;在骨骼中沉积,则影响骨基质矿化和成骨细胞活性,进而引起骨软化和贫血。鉴于铝对中枢神经系统和骨骼的巨大毒性,目前含铝的磷结合剂已不在ESRD患者高磷血症的治疗中应用,或者仅在其他药物治疗无效时短期应用。

(2)含钙的磷结合剂:20世纪90年代,含钙的磷结合剂开始广泛应用于临床。含钙的磷结合剂包括碳酸钙和醋酸钙两种,是目前国内继发性甲旁亢的首选治疗药物。碳酸钙的用量为1~6g/d,于餐中使用,其降磷作用和胃pH值相关,通过使用质子泵抑制剂调整胃pH值可影响碳酸钙的降磷作用。另一个含钙的磷结合剂是醋酸钙,其溶解度高,用量小,在获得与碳酸钙相同降磷效果时,其钙的摄入量仅为碳酸钙的一半,而且它的溶解度不依赖于胃的pH值水平。因此,其高钙血症的发生率也较低,远期并发症如血管钙化的风险也较低。除服用量大、患者的顺应性较差外,含钙磷结合剂的主要并发症是高钙血症以及由此引起的血管钙化和软组织钙化,患者的病死率上升。此外,由于活性维生素D{3}
和含钙磷结合剂的联合应用可引起PTH的过度抑制和钙的吸收,进而引起低转化型骨病和高钙血症,这也是含钙磷结合剂存在的重要问题。为防止高钙血症,由含钙的磷结合剂提供的钙离子含量不能超过1.5g,包含饮食的总钙摄入量应<每日2g。

(3)新型的磷结合剂:由于含铝和含钙磷结合剂的应用存在很多的并发症,因此新型的不含铝、钙的磷结合剂的研制显得至关重要。

最早出现的新型磷结合剂为盐酸司维拉姆(sevelamer)。它是含阳离子的聚烯丙胺,不含钙和铝,不增加钙负荷,口服后不被胃肠道吸收,可结合肠道中的磷,减少磷的吸收,从而降低血磷。该药已经广泛应用,临床试验结果发现它能有效降低ESRD透析患者的血磷水平,与传统的含铝、钙的磷结合剂不同的是,它不会发生铝的相关毒性和高钙血症。高钙血症发生率低,可允许患者使用更大剂量的活性维生素D{3}
,更好地控制继发性甲旁亢。此外,司维拉姆还有显著降低血清低密度脂蛋白浓度和升高高密度脂蛋白浓度以及降低PTH等作用。

2)调整血钙

CRF患者经常并发低钙血症,但钙制剂尤其是含钙的磷结合剂广泛应用以及活性维生素D的补充,常常引起钙负荷过度。而血钙增多、钙磷乘积增加可引起转移性钙化,长期高血钙抑制PTH细胞增生与分泌,导致低转化型骨病的发生。因此,应定期监测、维持血钙于正常范围以内。

对于高钙血症患者,限制钙的摄入,磷结合剂中离子钙的剂量不要超过1500mg,总离子钙的摄入(包括药物及饮食)不要超过2000mg。仍有高钙血症,则不要使用含钙的磷结合剂,并减少活性维生素D的剂量。

3)控制继发性甲状旁腺功能亢进

CKD
3~5期患者,首先评估患者是否存在高磷血症、低钙血症和维生素D缺乏,如有上述症状则减少饮食中磷的摄入,使用磷结合剂、钙剂和(或)天然维生素D,纠正可调节因素后如果PTH仍日益增高并持续高于正常值上限,则建议使用骨化三醇或维生素D类似物治疗。

CKD
5期患者,若PTH水平升高或持续上升,建议使用骨化三醇或维生素D类似物,或拟钙剂治疗,或将拟钙剂与骨化三醇或维生素D类似物联用。

骨化三醇的使用:轻中度甲旁亢,采用小剂量每日疗法(每日0.25~0.5μg),中重度甲旁亢,采用大剂量间歇疗法,PTH在300~500pg/mL,推荐剂量为1~2μg,biw;PTH在500~1000pg/mL,推荐剂量为2~4μg,biw;PTH>1000pg/mL,推荐剂量4~6μg,biw。其不良反应主要为高血钙综合征或钙中毒、高磷血症、软组织钙化。
\paragraph{低转化性骨病的治疗}

低转化性骨病(无动力型骨病)的治疗主要以预防为主,包括预防与治疗铝中毒;合理使用活性维生素D,避免过分抑制PTH分泌;合理使用钙剂,避免高血钙;严格掌握甲状旁腺手术适应证,全切后要加前臂甲状旁腺种植。

\section{肾小球肾炎}

肾小球肾炎不是单一的疾病,而是由多种病因和多种发病机制引起的,病理类型各异、临床表现又常有重叠的一组疾病。如果疾病起始于肾小球或病因不清者,则为原发性肾小球疾病;如果肾小球疾病是全身系统性疾病的一部分则为继发性肾小球疾病。原发性肾小球疾病,根据肾脏组织病理,又可分为多种类型;而不同病因或发病机制的肾小球疾病,又可表现为同一种病理类型。不同类型的肾小球疾病,临床表现及愈后各不相同。肾小球肾炎可以表现为炎症损害为主的肾炎综合征,以及蛋白尿为主要表现的肾病综合征。迄今为止,许多类型的肾小球肾炎缺乏特效的治疗方法,主要针对水肿、高血压、高脂血症、血栓形成及其他相应的并发症进行支持治疗。部分类型的肾小球肾炎应积极进行免疫治疗。

\subsection{发病机制}

肾小球疾病的发病机制目前尚未完全清楚,多数学者认为免疫反应介导的炎症损伤在其发病机制中发挥重要作用;许多非免疫因素如高血压、蛋白尿、凝血异常、代谢紊乱以及遗传和血管性疾病也在肾小球疾病的慢性化进程中起重要作用。

\subsection{临床表现}

肾小球疾病患者可能会表现为一系列的症状和体征,如蛋白尿、血尿、水肿、高血压、肾功能损害等。它们通常又被分为两类症候群,肾炎综合征及肾病综合征。肾炎综合征反映肾小球炎症,通常导致血尿、白细胞尿、细胞管型及颗粒管型等。相反,肾病综合征反映非炎性的肾小球结构损伤,尿中较少有细胞及细胞管型,肾病综合征初期肾功能无明显变化。

红细胞通过开放的基膜表现为血尿,红细胞管型高度提示肾小球肾炎或血管炎,变形红细胞高度提示肾小球炎症。

蛋白尿提示基膜的分子和(或)电荷屏障受损,正常人24h尿蛋白很少超过150mg。

高血压在肾小球疾病患者中常见,与盐的潴留以及容量增多有关。血管紧张素等血管收缩物质的增多则是慢性肾小球疾病患者高血压的主要原因。

(1)肾炎综合征:主要表现为蛋白尿、血尿、水肿、高血压。血尿是肾炎综合征的典型表现,变形红细胞,特别是棘红细胞是肾性血尿敏感而又特异的指标,脓细胞及管型尿也非常常见。蛋白尿程度差别很大,严重肾炎患者可能存在肾功能受损。

(2)肾病综合征:主要表现为每日尿蛋白大于3.5g/1.73m{2}
、低蛋白血症、水肿,高脂血症。许多患者存在高凝状态。肾病综合征可能是原发性肾小球疾病的结果,也可能与系统性疾病,如糖尿病、红斑狼疮、淀粉样变等密切相关。

\subsection{诊断}

对于可疑的肾小球疾病患者,应详细询问病史以明确是否存在可能的系统病因;通过用药史、职业、环境接触史询问,有利于发现肾毒性药物的暴露。并通过体格检查及实验室评估发现是否存在导致肾病的系统疾病。此外,患者的年龄、性别和种族背景也有助于肾小球疾病分型的判断。

实验室检查如尿液检查有助于区分肾炎综合征及肾病综合征。肾小球滤过率(GFR)可以用来评估肾小球的受损程度,疾病早期,GFR多正常。

尽管肾小球疾病的病因可以通过临床表现和实验室结果评估,很多时候仍需通过肾活检以明确诊断。

\subsection{治疗}

肾小球肾炎患者的治疗包括针对肾小球疾病的特殊药物治疗以及针对高血压、水肿控制、疾病进展预防等的支持治疗。对于肾病综合征患者,还应特别关注大量蛋白尿导致的并发症的治疗,如低白蛋白血症、高脂血症和血栓栓塞的防治。由于大量蛋白尿会导致肾功能加速恶化,因此减少蛋白尿非常重要。

免疫抑制剂单独或联合使用,通常用于改善肾小球肾炎的异常免疫过程。糖皮质激素除了免疫抑制作用,还有抗感染作用。细胞毒性药物,如环磷酰胺、苯丁酸氮芥、AZA、环孢霉素、霉酚酸酯也通常用于治疗肾小球疾病。

由于肾炎的发病与许多免疫因子有关,血浆置换清除这些因子有助于疾病的治疗。

\subsubsection{一般治疗}

表现为肾病综合征的患者应控制饮食,每日钠摄入量应控制在50~100mEq,蛋白质的摄入量为0.8~1g/kg,胆固醇摄入少于200mg。每日总脂肪摄入量应少于每日热卡摄入的30%。钠盐限制不仅有利于控制水肿,对于控制高血压和蛋白尿也非常重要。限制蛋白不仅有助于减少蛋白尿,也有助于延缓肾脏疾病的进展。此外,患者应该戒烟,因为吸烟可以剂量依赖性的增加终末期肾功能衰竭(ERSD)发生的风险。
\paragraph{水肿}

肾性水肿患者的治疗包括限盐、卧床休息、使用护腿弹力长袜和利尿剂。考虑到严格的限盐难以实现,长时间卧床休息易导致血栓栓塞,通常需要使用袢利尿剂,如呋塞米。但由于大量蛋白尿的存在,利尿剂效果较差,经常需要使用大剂量的利尿剂。如中度水肿患者通常需要每日使用呋塞米160~480mg。在某些情况下,可以联合使用噻嗪类利尿剂以增强利尿效果,或使用袢利尿剂持续输注。严重的水肿患者,还可以静脉输注白蛋白扩容以增加利尿剂效果。然而,白蛋白可能导致充血性心力衰竭,值得注意。
\paragraph{高血压}

良好的控制血压,对于延缓肾脏疾病进展以及降低心血管风险非常重要。慢性肾脏病患者的血压目标值为低于130mmHg/80mmHg。ACEI、ARB可以延缓糖尿病肾病以及非糖尿病肾病患者的肾功能恶化进展。非二氢吡啶类钙拮抗剂(如地尔硫卓、维拉帕米)可以减少蛋白尿、保护肾功能,可以作为联合用药使用。相反,二氢吡啶类钙拮抗剂(如硝苯地平等)降血压作用明显,但没有降蛋白尿效用。
\paragraph{蛋白尿}

饮食蛋白限制有助于延缓肾功能恶化。适度的蛋白控制(每日0.8g/kg)适用于中度肾功能减退患者。同时,减少蛋白质摄入也减少了磷和钾的摄入量。然而,蛋白控制必须考虑机体对蛋白的需求,避免发生营养不良。对于重度肾功能不全的非透析患者[GFR<25mL(min·1.73m{2}
)],蛋白摄入量应减少至每日0.6g/kg。

蛋白尿是肾功能下降和心血管疾病的一个独立危险因素。减少蛋白尿可以延缓肾功能恶化以及进展至ERSD的时间。最近的研究表明,ACEI不仅可以减少滤过分数、降低球内压,还可以直接影响足细胞,减少尿蛋白以及肾小球疤痕形成。此外,ACEI也可以降低血管紧张素Ⅱ诱导的肾脏细胞增殖作用。因此,其对蛋白尿有益的影响远远超出了其抗高血压作用。尽管如此,联合使用ACEI及ARB并不推荐。

非甾体类抗炎药(NSAIDs)可能通过抑制前列腺素E{2}
,降低球内压和肾小球滤过率,修复基膜屏障,减少蛋白尿。吲哚美辛和甲氯酚酸是研究最多的两个NSAIDs,其降蛋白作用与ACEIs相当,与后者联用,可取得更强的降蛋白疗效。然而,由于NSAIDs的潜在肾脏毒性,特别是对于肾功能欠佳者,长期使用此类药物并不推荐。
\paragraph{高脂血症}

肾病综合征患者的异常脂质代谢会增加动脉粥样硬化和冠心病的风险。因此,对于肾病综合征患者的持续血脂异常,特别是对于HDL正常或偏低,而VLDL以及LDL水平较高的患者应给予积极治疗。具有冠心病、吸烟、高血压等高危因素的患者,应特别注意。

单纯低脂饮食是不够的。通常应使用HMG-CoA抑制剂,即他汀类药物。他汀类药物除了降脂作用,还可以通过抑制细胞增殖及基膜聚集、抗感染作用和免疫抑制作用等其他机制发挥肾脏保护作用。最近的临床研究表明,他汀类药物可以减少蛋白尿和延迟肾功能恶化。
\paragraph{抗凝治疗}

肾病综合征患者经常出现肾静脉血栓、肺栓塞以及其他严重栓塞事件,尤其多见于膜性肾病患者。对于血栓高危患者,如长期卧床、静脉使用大剂量激素等患者应接受预防性抗凝治疗,以华法林为首选。

\subsubsection{各种肾小球肾炎治疗}
\paragraph{微小病变(MCD)}

微小病变肾病是一组以单纯性肾病综合征为表现的疾病。光镜下肾小球基本正常,可有轻度系膜增生,近端肾小管上皮细胞可见脂肪变性,故又被称为“类脂性肾病”。电镜下肾小球的特征性表现为弥漫性足突融合,肾小球内一般无电子致密物沉积。免疫荧光阴性(或C{3}
和IgM低强度染色)。

临床上无论是儿童MCD还是成人MCD都有一定比例的自发缓解率。但是鉴于未治疗时大量的蛋白尿、严重的低蛋白血症、水肿和脂肪代谢紊乱可以加快肾小球硬化,容易导致感染、血栓和栓塞、急性肾衰竭、蛋白质和脂肪代谢异常,增高成人患者发生冠心病、心肌梗死事件的风险。所以,MCD的诊断明确后,即应给予治疗,以尽早、尽快地使肾病综合征得以缓解。

(1)支持治疗。由于激素治疗疗效佳,蛋白尿及高脂血症多可在短时间内缓解,因此,在微小病变型的肾病综合征的初始治疗中,不建议使用他汀类药物治疗高脂血症;血压正常的患者也无须使用ACEI或者ARBs减少尿蛋白。如有指征,出现急性肾损伤的微小病变患者可采用肾脏替代治疗,并联合糖皮质激素,糖皮质激素用量同MCD的初始治疗。

(2)初始治疗。微小病变患者的初始治疗使用激素有效率最高。激素可以使大约90%的儿童患者蛋白尿缓解,10年肾存活率>95%,考虑到激素的极佳疗效,以及儿童肾病综合征的流行病学特点(MCD比例>90%),儿童肾病综合征患者可以不行肾穿刺而直接使用激素治疗。初始剂量每日60mg/m{2}
或2mg/kg,最大剂量不超过每日60mg,足量激素使用4~6周后,减量至单次口服泼尼松40mg/m{2}
或1.5mg/kg的剂量隔日给药(最大剂量为40mg隔日口服),在以后的2~5月内逐渐减量。

表现为肾病综合征成人MCD患者的初始治疗也推荐使用激素;以泼尼松计算,推荐日剂量为1mg/kg(最大80mg)或隔日剂量2mg/kg(最大120mg);如果获得完全缓解,又能耐受,足量激素维持至少4周,如未达完全缓解则最多维持16周;在获得缓解的患者,建议激素在6个月内缓慢减量。

在激素相对禁忌或不能耐受大剂量激素的患者(如未控制的糖尿病、精神病、严重骨质疏松等),建议口服环磷酰胺(CTX)或钙调神经蛋白抑制剂(CNIs)。环磷酰胺建议口服每日2~2.5mg/kg,使用8周;但是环磷酰胺有一定生殖毒性和致肿瘤作用,应注意累计剂量。推荐CNIs(环孢素每日3~5mg/kg或他克莫司每日0.05~0.1mg/kg分次服),疗程1~2年。

(3)复发治疗。超过一半的MCD肾病患者会出现复发,三分之一的患者会频繁复发或激素依赖。对于非经常复发的微小病变,建议使用糖皮质激素,起始剂量及疗程参照初始治疗。对于频繁复发(FR)/激素依赖(SD)的MCD患者,建议口服环磷酰胺;对于尽管使用了环磷酰胺仍复发的FR/SD
MCD患者,或需要保全生育功能的患者,建议使用CNIs,剂量疗程参见初始治疗;对不能耐受激素、环磷酰胺和CNIs者,也可使用麦考酚酸酯每次500~1000mg,每日2次,疗程1~2年。

(4)激素抵抗者。MCD多对激素敏感,如患者对于皮质激素抵抗,应重新评估以寻找引起肾病综合征的其他原因。
\paragraph{特发性局灶节段性肾小球硬化(FSGS)}

FSGS的病变特征是部分(局灶)肾小球和(或)肾小球部分毛细血管襻(节段)发生硬化性改变;早期就可以出现明显的肾小管-间质病变。蛋白尿、肾病综合征是其突出的临床表现。本病对各种治疗的反应均较差,疾病呈慢性进展性过程,最终发生CRF。

(1)支持治疗:FSGS的支持治疗方案参照蛋白尿治疗方案。RAS阻断剂是减少蛋白尿常规用药。然而,在肾病综合征患者应推迟使用,以观察起始激素治疗的疗效;在严重肾病综合征患者,更应延迟使用,因为此时由于低灌注、急性肾小管坏死引起急性肾损伤的概率明显升高。

(2)初始治疗:仅在有肾病综合征表现的特发性FSGS的治疗中应用皮质激素及免疫抑制剂。激素剂量、疗程、减量等参照MCD治疗。建议对大剂量皮质激素治疗存在相对禁忌证或不能耐受的患者(如未控制的糖尿病、精神病、严重的骨质疏松症)采用钙调磷酸酶抑制剂作为一线治疗。

(3)复发:FSGS肾病综合征复发的治疗同成人微小病变复发的治疗推荐。

(4)皮质激素抵抗者:对皮质激素抵抗的局灶节段性肾小球硬化,建议环孢素用量为每日3~5mg/kg,分次服用,目标浓度125~175ng/mL。由于环孢素起效慢,停药后容易复发。一般要求至少观察4~6月;如有部分或者完全缓解,建议继续环孢素治疗至少12个月后缓慢减量。他克莫司可以作为环孢素的一种替换方案,剂量每日0.1~0.2mg/kg,目标浓度5~10ng/mL。对于皮质激素抵抗且不能耐受环孢素的局灶节段性肾小球硬化患者,建议采用麦考酚酸乙酯和大剂量地塞米松的联合治疗。

\section{肾移植的免疫抑制治疗}

随着肾移植手术的日益成熟以及新型免疫抑制剂在临床的广泛应用,肾移植患者人肾长期存活率明显提高,肾移植已成为终末期肾病的最佳治疗方法。但是由于异体器官的植入,免疫排斥反应难以避免,肾移植患者需要终生使用免疫抑制剂。如何更好地使用免疫抑制剂,提高疗效,防治排异反应,减少不良反应发生,成为肾移植受者长期人肾存活的最重要的保障。

\subsection{排异反应发生机制}

同种异体肾移植后,受者出现的肾脏排斥反应主要与以下因素有关。

\subsubsection{T细胞介导的排异反应}

T细胞被认为是同种异体移植排异反应的主要介导者,因此目前抗排异治疗主要是针对T细胞。T细胞介导的排异反应可分为抗原呈递、T细胞识别、激活及增殖四个时期,相互间形成串联反应,最终产生细胞毒性T淋巴细胞,造成移植肾损伤。

\subsubsection{抗体介导的排异反应}

在抗体介导的血管反应中,T细胞也参与了B细胞的激活,使B细胞进一步增殖和分化成为分泌抗体的浆细胞,少量抗体结合于血管壁即可诱发抗体介导的移植物排异反应。

\subsection{临床表现}

根据排异反应发生机制、病理、时间和临床表现的不同,通常可分为超急性、加速性、急性和慢性排异反应。

\subsubsection{超急性排异反应(hyperacute rejection,HAR)}

HAR发生的主要原因是肾移植术前受体体内存在针对供体的抗体。其病理表现为肾内大量中性粒细胞弥漫浸润,肾小球毛细血管和微小动脉血栓形成,随后发生广泛肾皮质坏死,最终供肾动脉、静脉内均有血栓形成。HAR一般发生在移植肾血管开放后即刻或48h内,根据术后突发血尿、少尿或无尿,移植肾彩超显示皮质血流无灌注伴有明显肿胀,肾活检显示典型改变者可明确诊断。对于HAR,目前尚无有效的治疗,一旦确诊应行移植肾切除术,术前可通过监测受体群体反应性抗体水平和供受体淋巴毒试验进行预防。

\subsubsection{加速性排异反应(accelerated rejection,ACR)}

通常发生在移植术后24h~7d内,发病机制仍未完全清楚,可能与受体体内预存针对供体的抗体有关。病理上以肾小球和间质小动脉的血管病变为主,表现为淋巴细胞浸润血管内膜,血栓形成,重者可发生血管壁纤维素样坏死、间质出血和肾皮质坏死,免疫组织化学可发现肾小管周围毛细血管C4d沉积。临床表现为发热、高血压、血尿或少尿,移植肾肿胀、质硬、压痛明显,肾功能快速恶化并丧失。ACR总体治疗效果较差,目前临床上常用的治疗方法有:①尽早使用抗淋巴细胞球蛋白(anti
lymphocyte
globulin,ALG)或抗CD3单克隆抗体等;②大剂量丙种球蛋白;③血浆置换去除抗体;④治疗无效且患者情况允许可尽早切除移植肾,恢复透析状态,以避免其他并发症发生。

\subsubsection{急性排异反应(acute rejection,AR)}

最常见的排异反应可以发生在移植后的任何时候,更多见于肾移植后的前3个月内。然而,移植后远期甚至10年以上的肾移植受者也可能发生急性排异反应。一般而言,发生越早程度越重。大部分AR属于急性细胞性排异,有时体液因素也会参与。临床主要表现为尿量减少、体重增加、轻中度发热、血压上升,可伴有移植肾肿胀、血肌酐上升、移植肾彩超阻力系数升高等;病理穿刺提示间质和肾小管单核细胞浸润(小管炎),亦可见单核细胞在血管内膜浸润(血管内膜炎),伴有间质水肿等。对于AR的治疗关键在于尽早诊断,此时肾活检尤为必要,一旦诊断应及时治疗。治疗方法:①甲基泼尼松龙冲击治疗。②单克隆或多克隆抗体,适用于激素冲击效果差的患者。③对于有体液因素参与的排异反应可同时进行血浆置换去除抗体;也可联合大剂量丙种球蛋白中和抗体。④注意预防强化治疗的并发症,包括多/单克隆抗体可能产生的过敏反应以及强化治疗后容易发生的感染等并发症。

\subsubsection{慢性排异反应(chronic rejection,CR)}

CR一般发生在移植术后3~6个月之后,是影响移植肾长期存活的主要因素。病因包括免疫因素和非免疫因素,如供受体HLA匹配不佳、免疫抑制剂不足、供肾缺血再灌注损伤、急性排异程度和次数、病毒感染、高血压、高脂血症等。临床表现为蛋白尿、高血压、移植肾功能逐渐减退以及贫血等,主要通过移植肾活检诊断,病理表现为间质广泛纤维化,肾小管萎缩,肾小球基膜增厚硬化并逐渐透明样变,最终肾小球硬化;同时伴有小动脉内膜增厚,狭窄直至闭塞。目前对于CR无特别有效的治疗方法,处理原则为保存残存肾功能,减慢肾功能损害的进展速度,同时对症处理高血压、高脂血症,使用ACEI或ARB制剂等。此外可以根据移植肾的病理情况,如果考虑慢性排异为主,可适当增加免疫抑制剂;而对于C4d阳性,诊断抗体介导的排异患者可考虑血浆置换和使用丙种球蛋白。

\subsection{防治原则}

\subsubsection{一般治疗原则}

同种异体肾移植系指不同基因型的同种肾移植,受者移植后出现排异反应几乎是不可避免的。因此,预防、及时发现和治疗排异反应是移植肾长期存活的关键。应嘱患者定期随访肾功能、尿量、肝功能、免疫抑制剂血药浓度等;定期评估肾功能和免疫抑制剂的不良反应等;教育患者熟悉常见排异反应的表现,如有尿量减少、发热、移植肾区胀痛等情况,应及时就诊。

\subsubsection{药物治疗原则}
\paragraph{长期用药}

移植免疫耐受技术目前在肾移植尚未建立,因而所有肾移植受者都需要长期服用免疫抑制剂预防移植肾排异反应。移植免疫的基础理论清楚表明,人体内长期存在的记忆T细胞尚不能被清除,现阶段肾移植受者不具备肝移植受者免疫耐受的机制。对肾移植受者而言,耐受是暂时的(肾移植同时行供者骨髓移植可能例外)。尽管临床有时可见到“依从性不佳”的肾移植受者“自动”停用了所有免疫抑制剂,并在一定时期内没有发生排异反应,但这种情况并不持久和安全。在一次感染或严重创伤后,患者的移植排异机制可能被激活,耐受终将被打破。这已被国内外许多临床病例所证实。因此,目前肾移植受者长期维持免疫抑制治疗是必须的,只是维持剂量与方案有所差别。
\paragraph{联合用药}

为取得长期良好的免疫抑制效果,应选择“优化”的免疫抑制剂组合方案。以降低急性和慢性移植肾排异反应;减少免疫抑制剂的不良反应;减少肾移植后感染、肿瘤的发生率;促进肾移植受者和移植器官的长期存活;取得有良好的成本-效益比。
\paragraph{时间化用药}

肾移植受者在移植后不同阶段要解决的主要问题并不相同,不同时间段免疫抑制剂应用的重点因而也并不一致。早期重点为降低急性排异反应,后期则为减少慢性移植肾失功,减少药物相关的不良事件发生,促进移植肾及移植受者的长期生存。因而,应根据移植后不同阶段的重点问题调整免疫抑制剂的治疗方案,逐渐减少直至维持剂量。
\paragraph{个体化用药}

肾移植受者的免疫状态、遗传背景、基因多态性、与供者免疫及非免疫因素的相互作用等不尽相同,因而应当力求实现免疫抑制剂的个体化使用。针对不同肾移植个体选用最适合的免疫抑制剂组合和剂量,使免疫抑制剂达到疗效最优化,不良反应最小化。“个体化原则”是目前及未来器官移植领域免疫抑制剂临床应用的趋势。

\subsection{药物治疗}

\subsubsection{常用的免疫抑制剂}

常用免疫抑制剂的种类包括①糖皮质激素:包括泼尼松、泼尼松龙、甲泼尼龙等;②钙调素抑制剂(caleneurin
inhibitor,CNI):环孢素、他克莫司等;③抗增殖药:包括AZA、吗替麦考酚酯(mycophenolate
mofetil,MMF)、咪唑立宾等;④生物制剂:常用的有ALG、抗胸腺细胞球蛋白(anti
thymocyte globulin,ATG)、单克隆抗体等;⑤其他:西罗莫司等。
\paragraph{糖皮质激素}

激素是临床上最早也是最常用的免疫抑制剂,可通过减弱增殖的T细胞对特异性抗原及同种异体抗原的作用,而达到抑制炎症反应及移植物免疫反应的结果。泼尼松在肝内转化为泼尼松龙后生效,主要在肝内代谢,由肾脏排泄,经胆汁及粪便的排泄量极微。通常,手术当日及术后3d静脉滴注甲泼尼龙500~1000mg作为冲击治疗。术后第4日起改为口服泼尼松(龙),自每日60~80mg始,每日10mg逐日递减。减至每日10~20mg维持,3~6个月逐渐减至维持量每日7.5~15mg。发生急性排异反应时可使用大剂量甲泼尼龙500~1000mg静脉滴注冲击治疗。皮质类固醇的不良反应主要有药物性库欣综合征、感染、高血压、糖尿病、白内障及无菌性骨坏死等。
\paragraph{CNI}

环孢素A(Cyclosporine
A,CsA)是目前肾移植患者主要使用的强效免疫抑制剂之一。口服后由小肠吸收,服药后2~4h(平均2.8h)血浓度达峰值。在肝内由肝细胞内质网及细胞色素P450微粒体酶系统代谢,代谢产物有20种,大部分经胆道排泄,仅6%由尿中排泄,生物半衰期为14~27h。环孢素对T淋巴细胞亚群具有高度特异性抑制作用,作用于细胞周期G{1}
早期阶段;另外,环孢素对于B淋巴细胞也有一定的影响。CsA最显著的不良反应为肝、肾毒性,其他还包括高血压、多毛、痤疮、齿龈增生等。

他克莫司(FK506)系从放线菌{Streptomycestsknbaenisis}
酵解产物中提取的一种大环内酯类抗生素,具有很强的免疫抑制作用,其强度约为CsA的50~100倍。口服吸收快,主要吸收部位在小肠。血药峰浓度出现在口服后0.5~3h,半衰期3.5~40.5h,平均8.7h,主要经肝脏P4503A细胞色素系统代谢,经胆汁和尿排泄。主要通过抑制细胞内钙和钙调蛋白依赖性的丝氨酸/苏氨酸磷酸酶神经钙蛋白的活化,阻断IL-2基因转录,抑制细胞活化。FK506常见不良反应有糖尿病、神经系统副作用(包括震颤、失眠等)、肾毒性和胃肠道反应。

CNI的个体差异大,因此用药剂量应根据血药浓度调整,制定个体化的给药方案。然而,目标浓度的确定对于移植患者非常重要。然而由于研究的欠缺,目前缺乏公认的推荐目标浓度。浓度过高可以增加不良反应,浓度过低则排异率明显增加,因此应根据移植后时间,联合用药情况、患者一般情况等制定个体化的给药方案。一般认为他克莫司目标血药谷浓度(C{0}
)可以参考生产厂家推荐的10(5~15)ng/mL;移植初期环孢素目标谷浓度(C{0}
)为200(150~300)ng/mL,峰浓度(C{2} )为1400~1800ng/mL,移植后期C{2}
目标值下调为800~1200ng/mL。当然,也有很多研究推荐较低的他克莫司及环孢素浓度,如两种谷浓度分别为5(3~7)ng/mL及75(50~100)ng/mL。
\paragraph{抗增殖药}

AZA属咪唑类6-巯基嘌呤衍生物。通过竞争性地反馈抑制嘌呤合成酶,阻止次黄嘌呤核苷酸转变为AMP和GMP,从而抑制嘌呤核苷酸的合成,并且干扰RNA的合成及代谢。AZA的口服剂量为术后3日内3mg/kg,之后递减,维持剂量为每日1~2mg/kg。不良反应主要有骨髓抑制,可引起白细胞、血小板计数减少;此外可导致肝功能损害,大剂量时有胃肠道和口腔的溃疡、脱发等。

MMF口服吸收后,迅速、完全地被转换为具有生物活性的MPA,平均口服生物利用度近94%,MPA在肝脏中被代谢成葡萄糖苷MPA,通过肾脏排泄,MPA半衰期近18h。MPA是单磷酸次黄嘌呤脱氢酶(hypoxanthine
dehydrogenase
monophosphate,IMPDH)可逆、非竞争性抑制剂,抑制鸟嘌呤核苷酸的经典合成途径,淋巴细胞增殖被阻断在细胞周期S期,从而发挥对淋巴细胞的免疫抑制效应。MMF常作为AZA的替代用药与CSA或他克莫司、皮质类固醇联合应用,剂量为每次0.5~1.0g,每日2次口服。主要不良反应是胃肠道反应、造血系统毒性等。MPA钠作用机制、免疫抑制作用、感染及胃肠道不良反应等均与MMF相似。
\paragraph{生物制剂}

(1)单克隆抗体。目前临床使用的单抗包括已批准的抗CD3{+}
T淋巴细胞单抗(OKT3)、巴利昔单抗(basiliximab)和达利珠单抗(daclizumab,抗Tac单抗),以及未正式批准在器官移植领域使用的利妥昔单抗(抗CD20单抗,rituximab),抗CD52单抗、抗CD154单抗和抗细胞毒T淋巴细胞相关抗原4(cytotoxic
T lymphocyte antigen
4,CTLA-4)单抗等。其中,OKT3是经典的淋巴细胞清除剂,只针对T淋巴细胞表面特定膜分子抗原CD3抗原决定簇的特异性抗体,但由于其会导致严重的细胞因子释放综合征,很少用于诱导疗法。巴利昔单抗与抗Tac单抗均为IL-2受体阻断剂,能够特异性地与活化的T淋巴细胞上的IL-2R的亚单位CD25相结合,从而抑制IL-2的免疫效应而不会影响其他T淋巴细胞。抗IL-2受体单克隆抗体的这种模式与CNI通过抑制钙调磷酸酶的活性阻断IL-2的活性有协同作用。

IL-2受体阻断剂因其安全、不良反应少被广泛用于各种类型患者的诱导疗法,尤其在患有其他系统伴随疾病的患者。KDIGO指南和中华医学会器官移植学分会制定的抗体类应用指南中,均推荐IL-2受体阻断剂作为抗体类诱导疗法的首选药。澳大利亚2007年公布了抗Tac单抗的应用指南中还推荐抗Tac单抗作为常规免疫抑制剂联合配伍的组成部分。

(2)多克隆抗体。ALG或ATG进入体内后,在肝脏调理素和补体(C{1} -C{4}
)的参与下,对T淋巴细胞具有直接细胞毒作用,使淋巴细胞溶解被网状内皮系统或循环的单核细胞吞噬或清除。一般用于肾移植围手术期诱导治疗以及皮质类固醇耐受的难治性排异反应。使用剂量为每日5mg/kg,静脉滴注,每日1次,使用7~10d,不良反应包括注射后出现高热、寒战、肌痛、荨麻疹,应预先注射地塞米松或甲基泼尼松龙,防止高热和过敏反应的发生。
\paragraph{其他}

西罗莫司,又称雷帕霉素,属于大环内酯类抗生素,结构与他克莫司相似,但其免疫抑制机制与他克莫司不同。它通过不同的细胞因子受体阻断信号转导,阻断T淋巴细胞及其他细胞由G{1}
期至S期的进程,与他克莫司相比,西罗莫司可阻断T淋巴细胞和B淋巴细胞的钙依赖性和非钙依赖性的信号转导通路。此药肾脏及神经毒性较小,主要不良反应有高血脂、肝损、骨髓移植、胃肠道反应、痤疮、皮疹、伤口愈合延迟等。

\subsubsection{治疗方案的选择}

临床使用的免疫抑制剂常常需要联合使用,以提高治疗效果,同时可以减少不良反应。目前肾移植术后较为常用的组合为:抗T淋巴细胞抗体诱导+三联治疗;抗白细胞介素(interlukin,IL)2受体单克隆抗体诱导+三联治疗;三联治疗。其中三联治疗的常用组合为以下三类药物同时使用,每类药物选择一种。①泼尼松、泼尼松龙、甲泼尼龙等。②环孢素、他克莫司、西罗莫司。③MMF、霉酚酸钠、AZA、咪唑立宾等。当然,目前也有不使用激素的免疫抑制方案,移植后期也可以根据疗效、不良反应等选择二联免疫抑制治疗。
\paragraph{诱导期治疗}

器官移植后的免疫抑制剂药物疗法包括早期的诱导治疗和后期的维持治疗。早期诱导治疗的常规标准方法是应用大剂量的皮质激素和高剂量的钙调神经磷酸酶抑制剂(CsA或FK506)与抗代谢药(AZA或MMF)。而生物蛋白制剂应用,即抗体药物作为器官移植后早期实施免疫抑制覆盖治疗的方法,称为抗体诱导治疗。多年的临床应用证明,抗体诱导治疗可减少30%~40%肾移植后早期急性排异反应。

为预防急性排异反应,KDIGO指南推荐在肾移植术前或术中即开始联合应用免疫抑制药物,并将使用生物制剂进行诱导治疗纳入到肾移植受者初始的免疫抑制方案中。推荐IL-2受体拮抗剂作为诱导治疗的一线用药,而对于有高排异风险的肾移植受者,建议使用抗淋巴细胞制剂而不是IL-2受体拮抗剂。
\paragraph{维持期治疗}

根据移植术后时间维持治疗可分为初始维持阶段与长期维持阶段。

(1)初始维持阶段:KDIGO指南推荐,在初始维持阶段应联合使用免疫抑制剂,包括钙调磷酸酶抑制剂和抗增殖药物,包含或不包含糖皮质激素。建议将他克莫司作为CNI一线用药,将麦考酚酯作为抗增殖药物的一线用药。建议在肾移植术前或术中就开始使用他克莫司或CsA,以预防急性排异反应的发生,而不是等到移植肾功能开始恢复才使用。对于低排异风险的患者和接受过诱导治疗的患者,建议移植术后1周内可停止使用糖皮质激素。

KDIGO指南建议早期撤除激素的初衷在于长期用药将产生严重的不良反应,例如糖尿病、高血压、高脂血症,这些都是导致心血管疾病的重要诱因,而心血管疾病又是导致肾移植受者死亡的重要原因。

(2)长期维持阶段:KDIGO指南指出:在长期维持阶段,如未发生急性排异反应,建议移植术后2~4个月内采用最低维持剂量的免疫抑制药物。CNI应持续应用,而不是停药。如果移植后1周仍在使用泼尼松,建议继续使用而不是停药。低剂量他克莫司联合MMF加激素已成为新近移植受者首选的长期免疫抑制维持治疗方案。该方案具有相对较少的肾毒性和更强的免疫抑制作用,已在全球范围内得到逐步推广。联合用药的目的是最大限度抑制排异反应,尽可能减少药品不良反应。由于环孢素和他克莫司具有相似的免疫抑制作用,但药品的不良反应不同,因此可根据受者的不同情况在两者之间相互转换。从肾移植术后远期抗排异反应治疗方案中撤除CNI的观点在移植界有很大争议,停用环孢素应全面权衡利弊。基于停药后可能会增加急慢性排异的发生率,目前国内对激素的撤停普遍持谨慎态度,一般在无严重并发症的情况下不主张完全停用激素,而倾向于小剂量维持。如要使用西罗莫司,推荐在移植肾功能完全恢复、手术伤口愈合之后使用。
\paragraph{急性排异反应治疗}

在怀疑出现急性排异反应时,推荐在治疗急性排异前进行活检,除非活检会明显延迟治疗。针对临床、亚临床以及临界型急性排异均应给予治疗。一般推荐糖皮质激素作为急性细胞性排异的初始用药。使得对于激素治疗效果不佳的急性细胞性排异和复发的急性细胞性排异,建议使用淋巴细胞消减性抗体或者抗T细胞抗体OKT3。此外,对于发生急性排异的受者,对未使用麦考酚酯者建议加用麦考酚酯,对正在使用AZA者建议换用麦考酚酯。