\chapter{绪论}

\section{概述}

\subsection{药物学和临床药物治疗学的发展简史}

药物(drug)是指用于预防、治疗、诊断人的疾病,有目的地调节人的生理功能并规定有适应证或者功能主治、用法和用量的物质。药物并非现代社会所特有,事实上,远古时代人们为了生存,从生活经验中得知某些天然物质可以治疗疾病与伤痛,这就是药物的起源。公元1世纪前后,我国的《神农本草经》及埃及的《埃伯斯医药籍》等这些巨著的出现,表明药物的发展已经开始从简单的经验逐步上升到理论系统的高度。此后,许多科学家开始从天然的动植物中提炼具有活性成分的物质,如意大利生理学家F.Fontana(1720---1805)通过动物实验对千余种药物进行了毒性测试,推断天然药物都有其活性成分。德国化学家F.W.Serturner(1783---1841)首先从罂粟中分离提纯吗啡,并用犬证明其具镇痛作用,1823年又在金鸡纳树皮中发现了奎宁。诸如此类的发现,使得药物的发展越来越迅速,药物的品种开始极大地丰富。第二次世界大战结束后,出现了许多前所未有的药理新领域及新药,如抗生素、抗癌药、抗精神病药、抗高血压药、抗组胺药和抗肾上腺素药等。药物的发展已经逐步形成一门学科,即药物学。

药物的快速发展,给药物治疗学带来极大的益处,药物治疗也随着药物的发展,逐步从经验治疗过渡至实验研究,到现在人们已开始深入研究,给临床治疗带来很大帮助。随着人们的重视,逐步开始形成一门独立的学科,即临床药物治疗学。

临床药物治疗学(clinical
pharmacotherapeutics)是通过应用药物治疗的手段,针对疾病的发生、发展情况和患者的生理病理状况,制定和实施合理的药物治疗方案,达到消除疾病或控制症状,从而减轻或解除患者痛苦的一门学科。它是临床治疗中一个重要的组成部分。

虽然药物学的发展给药物治疗学带来极大帮助,但是直到20世纪70年代末才真正独立形成一门学科进入教学系统,至此,临床药物治疗学开始蓬勃发展起来。1980年,美国为其在读药学博士(Pharm.D)开设药物治疗学课程;世界著名的“{Pharmacotherapy}
”杂志于1981年在美国创刊;世界卫生组织于1982年成立了一个基本药物应用专家委员会,对临床合理应用基本药物提出了原则性的指导意见;1980年8月,国际药理联合会和英国药理学会在伦敦联合召开了第一届国际临床药理与治疗学会议,以后每隔3~4年召开一次;如第七届临床药理与治疗学会议于2000年7月在意大利召开,第八届于2004年在澳大利亚召开,第九届于2008年在加拿大召开。1996年中国创刊了《中国临床药理学与治疗学》杂志,表明我国也开始逐步重视并开展临床药物治疗学这门学科。

\subsection{临床药物治疗学的目的和方法}

\subsubsection{临床药物治疗学的目的}

临床药物治疗学是运用药学相关学科(如药理学、临床药理学、生物药剂学等)基础知识,针对疾病的病因和临床发展过程,结合患者的病理、生理、心理和遗传特征,研究疾病临床治疗实践中药物合理应用的策略。其目的在于对患特定疾病的特定患者,制定和实施合理的个体化药物治疗方案,以获得最佳的治疗效果并承受最低的治疗风险。

\subsubsection{临床药物治疗学的原则和方法}

临床药物治疗学的核心是合理用药(rational drug
use)。随着现代科学技术的发展和药物品种的增加,合理用药需要更科学、更完整的含义。合理用药即以当代药物和疾病的系统知识和理论为基础,安全、有效、经济、适当地使用药物。

有效性是指药物的治疗效果必须确切;安全性是指药物在正常剂量下不会造成严重危害,是个相对的概念;适当性是指将适当的药物以适当的剂量,在适当的时间,经适当的途径,给适当的患者,使用适当的疗程,达到适当的治疗目标。

合理用药的判断标准包括①按药物的临床用药适应证选用药物,药物的药理作用能针对疾病的病因和病理生理改变;②所选用的药物对患者具备有效、安全、经济和适当四个方面的要素;③在明确遗传多态性与药物反应多态性的基础上,采用个体化给药方案,确定临床用药剂量、用法、疗程,药物调剂配伍恰当;④患者应无禁忌证,所用治疗药物对患者引发不良反应的可能性最低或易于控制、纠正;⑤患者对临床所用的药物具有良好的依从性。

药物治疗,应以“安全、经济、有效”为中心,始终贯彻预防为主、防治结合的原则,但有时病情复杂多变,有时进展迅速。因此,在这一原则下做好药物治疗工作,应对千变万化的情况,医师除了有高度的责任心外,还需要做到以下几点。

(1)以患者为中心,标本兼治。“本”是疾病的病因,是本质。如果患者有充足的时间,病情发展缓慢,医师则应全力寻找病因,力求根治疾病。“标”是疾病的症状,是表象。但这并不说明治标不重要,有些病情起病急骤,危及生命,应先迅速缓解症状,控制症状,才能赢得时间来寻找病因。如有可能,应“标”“本”同治。药师应根据病情轻重缓急,透过现象看本质,抓住矛盾的主体,又要随时注意矛盾的转化。

(2)始终贯彻个体化原则。由于患者年龄、性别、体重、生理状况、环境因素、病情程度、病变范围、病程阶段、肝肾功能、并发症、既往治疗的反应,以及对药物的吸收、代谢、排泄率的不同,治疗方案也应有所不同。如患者肾功能不全时,通过肾脏代谢排泄的药物剂量应及时调整,以避免肾功能的进一步恶化。

(3)树立发展的观点。患者的病情随时变化,现代医学对一些疾病的观点也在变化,甚至一些药物的适应证也在变化,也有一些药物在上市后因发现了严重不良反应而退出市场,药师在这些千变万化的医疗信息中,如何进行选择后树立发展的观点,这对药物治疗的未来和对临床药师的未来和发展,都有着至关重要的地位。只有树立了发展的观念,对各种医疗信息及时了解,才能在临床医学团队中占有一席之地而不被淘汰。

\subsection{临床药物治疗学与其他相关学科的关系}

\subsubsection{药理学与临床药物治疗学的关系}

药理学与临床药物治疗学都是研究药物与人体相互作用的科学,但各有侧重。药理学侧重于药物作用的理论研究;而临床药物治疗学侧重于研究药物的应用问题,着重研究在疾病防治中选择药物和用药方法以及制定药物治疗方案等实际问题。药理学是临床药物治疗学的理论基础,临床药物治疗学是药理学理论在临床的实际应用。

\subsubsection{内科学与临床药物治疗学的关系}

内科学是研究疾病的临床表现、诊断、鉴别诊断和治疗原则的一门综合学科。医师既要重视对疾病临床表现(如症状、体征、物理和生化检查的改变、疾病的分类或分型等)的诊断和分析,又要合理选择治疗手段(介入或手术治疗、物理治疗、药物治疗)。因此,对千变万化的疾病和千差万别的个体,如何正确地选择和使用药物,往往力不从心,从而出现许多不合理的用药现象。在发达国家医疗机构的医学团队中,疾病的药物治疗由临床医师和药师共同负责,医师更关注诊断和分析疾病,药师更关注合理用药。我国多数医疗机构没有设置临床药师的岗位,即使在设有临床药师的医院,无论是体制因素还是知识储备,临床药师都还不能做到与医师共同负责对患者的药物治疗,他们还需要进一步学习和发展。

\subsection{临床药物治疗学的发展和未来}

\subsubsection{循证医学和临床药物治疗学}

循证医学(evidence based
medicine,EBM)是指有意识地、明确地、审慎地利用现有最好的研究证据制定关于个体患者的诊治方案。EBM的核心思想是在临床医疗实践中,尽管有些经验可能是正确的,但对患者的诊治决策都应依赖于客观的科学证据,而不是某些个人的主观经验。EBM为合理药物治疗提供了更加科学的证据,为评价疾病治疗的效果提供了可靠依据,但是其结论来自临床药物治疗学的研究和实践。EBM是寻求、应用证据的医学,它更强调的是一种医学研究和疾病治疗的唯物思想。循证医学应用到临床药物治疗学中,就是尽可能应用对药物疗效和不良反应评价最佳的证据,制定对患者的用药方案。

\subsubsection{药物基因组学与临床药物治疗学}

药物基因组学(pharmacogenomics)是根据生物多样性理论,利用基因组学的技术与方法,研究药物作用相关基因,阐述药物作用相关基因的多态性造成的药动学和药效学的变化,从而在理论上指导新药研制中的优化药物设计和临床用药时的药物个体差异预测。药物基因组学可以帮助人们确定不同个体对药物的不同反应,如哪种药物对哪个患者有效,哪种药物对哪个患者无效,哪种药物对哪个患者有害,从而指导临床用药,避免用药的盲目性,减少药物不良反应,提供个体化给药方案。有些药物在使用中出现特异性反应,经诊断明确属于遗传变异,则需避免使用能引起不良反应的同种药物,或调整剂量。如黄种人的慢乙酰化发生率约为10%~20%,故在应用异烟肼时注意调整剂量或加服维生素B{6}
,防止多发性神经炎;甲基结合酶缺乏者应用头孢菌素易引起低凝血酶原症。药物基因组学的迅猛发展,必定给未来的临床合理用药带来质的飞跃。

\section{临床药学和临床药师}

\subsection{临床药学}

临床药学是以患者为对象,研究药物及其剂型与病体相互作用和应用规律的综合性学科,旨在用客观科学的指标来研究具体患者的合理用药。其核心问题是最大限度地发挥药物的临床疗效,确保患者的用药合理与安全。

临床药学的工作是面向患者,以患者利益为中心。其特点在于它的临床实践性,药师在工作中始终和患者在一起,了解患者的生理、病理条件,根据患者复杂多变病情的防治需要,运用药剂学、药理学与药物治疗学等专业知识,密切结合临床患者的状况,针对性地给患者合理选药、正确用药,并监测用药过程,准确判断其疗效与不良反应,从而摸索用药规律,确保患者用药的安全性、有效性和经济性。

国外将临床药学定义为一个与合理用药有关的实践性学科领域。在这个领域里,临床药师提供的服务是有利于优化治疗、促进健康、预防疾病的药学服务。作为一个学科,临床药学的宗旨是致力于改善患者的健康状况与生命质量。

综合国内外的认识,临床药学工作就是药师要利用药学专业知识、技术、方法和药师特有的思维,针对医师、护士、患者在药物治疗各个环节中存在的问题提供具体的药学技术服务与帮助。

\subsection{临床药师}

临床药师是走进病房、来到患者床边为患者提供药学服务的药师。在国外,临床药师已有专门的职称系列,而获得临床药师的职称并不容易,要有相应的教育背景与专门的培训,拥有深厚的可以改善患者的健康状况与生命质量的药物治疗知识,拥有确保最佳治疗效果的药物治疗经验与判断能力。

临床药师为患者提供服务的场所是在病区,具有涉及生化、药学、社会行为学与临床医学等相关学科的知识。为了达到理想的治疗目标,临床药师在工作中要综合运用相关专业知识、急救知识、法律法规、伦理学、社会学、经济学等循证治疗原则与指南。因此,临床药师对患者的药物治疗负有直接与间接的(作为顾问或者与其他医务工作者合作)责任。

在医疗系统中,临床药师是药物治疗的专家,可常规提供药物治疗评估服务,并可为患者及医务工作者提供合理用药的建议。临床药师是一个可为安全、有效、适当、经济的药物治疗提供科学的、有效的信息与建议的主要资源。

\subsubsection{临床药师的职责}

在2007年12月26日卫生部出台的临床药师制系列文件中,对临床药师的职责进行了如下阐述:临床药师是临床医疗治疗团队成员之一,应与临床医师一样,坚持通过临床实践发挥药学专业技术人员在药物治疗过程中的作用,在临床用药实践中发现、解决、预防潜在的或实际存在的用药问题,促进药物合理使用。

临床药师的主要工作职责有以下7个方面。

(1)深入临床科室了解药物应用动态,对药物临床应用提出改进意见。

(2)参与查房和会诊,参加危重患者的救治和病案讨论,对药物治疗提出建议。

(3)进行治疗药物监测,设计个体化给药方案。

(4)指导护士做好药品请领、保管和正确使用工作。

(5)协助临床医师做好新药上市后的临床观察,收集、整理、分析、反馈药物的安全信息。

(6)提供有关药物咨询服务,宣传合理用药知识。

(7)结合临床用药,开展药物评价和药物利用研究。

\subsubsection{国外临床药师的发展}

不同国家临床药师的开展程度不同,发展较好的国家为美国和英国。美国的临床药学工作始于20世纪60年代,到了70年代开始评价药师参与临床服务的效果;1990年提出药学监护的概念,并在1993年的国际药学会议上正式得到肯定;1997年美国临床药学院建立了有药师参与的合作药物治疗管理制度;2001---2003年,75%的州立法确认临床药师制,现在已有临床药学专业的专家。以下是国外临床药师发展的关键历程。
\paragraph{临床药学专家准则}

美国Veterans
Administration于1985年颁布临床药学专家准则。其中教育背景是其主要因素之一。多年来,美国的药学博士(Pharm.D)教育已输送了无数名合格的临床药师。国外的经验说明,临床药学专业的研究生培养是造就临床药师的重要途径。
\paragraph{确立药师为新医疗团队中的成员}

2004年在美国新奥尔良举行的第64届国际药学联合会(International
Pharmaceutical
Federation,FIP)上强调了药师在医疗体系中的角色与作用,确立患者与药师为新医疗团队中的核心。

FIP很早就明确宣布对临床药学工作的支持。为了提高医疗保健的效率,药师应该保持与医务人员之间密切合作,使药师能更多地在药物治疗过程中发挥作用。药师的技能在确保提供可靠的后勤供给以及在治疗过程中给患者提供药物技术服务方面,至关重要。
\paragraph{给予药师处方权}

在第64届FIP大会上,专家的报告中提及越来越多国家的卫生当局凭着对药师的信任,授予药师处方权,以便他们能够随访特定的患者,并为慢性病患者的治疗提供再配药服务。因此,药师的作用大大超越了传统的发药的角色。
\paragraph{药师开始有医疗文书}

在一些国家,所有患者都有一个记录其保健与医疗数据的“个人医疗档案”。药师可以看到该档案的治疗部分,并把用药方面的数据添加进去。
\paragraph{药历与临床药师的成长}

建立药历是临床药师成长的关键环节之一。不同国家药历的格式有所不同,美国推行的是SOAP药历。

S(subjective):主观性资料,包括患者的主诉、病史、药物过敏史、药品不良反应史和既往用药史等。

O(objective):客观性资料,包括患者的生命体征、各种临床生化检验结果、影像学检查结果、血液和尿液检测结果、粪便培养结果以及血药浓度检测结果等。

A(assessment):临床诊断以及对药物治疗过程的分析与评价。

P(plan):治疗方案,包括选择具体的药品名称、给药剂量、给药途径、给药时间间隔、疗程以及用药指导的相关建议。

从美国药历的建立看,药师着重临床化,只有走到病床边,亲自与患者交流,才能了解患者的真正需求,为患者提供更为有效的服务。

\subsection{药学服务}

药学服务(pharmaceutical
care,PC)是药师应用药学专业知识向公众(含医务人员、患者及其家属)提供直接的、负责任的、与药物使用有关的服务(包括药物选择、使用的知识和信息),以提高药物治疗的安全性、有效性与经济性,实现改善与提高人类生活质量的理想目标。

药学服务的概念最初是由Mikeal在1975年提出,1990年美国的Hepler CD和Strand
LM在《美国医院药学杂志》上对药学服务作了较全面的论述。1993年,美国医院药师协会对药学服务的统一定义是:“药师的使命是提供药学服务,药学服务是提供直接的、负责的与药物治疗有关的服务,目的是获得改善患者生活质量的确定结果”。这些结果包括治愈疾病,消除或减轻患者的症状,阻止或延缓疾病进程,预防疾病或症状的发生。

药学服务的目的是提高接受药物治疗患者的生活质量,这就要求药师的工作要从以药品为中心转变为以患者为中心、药师不仅要提供安全有效的药物,还应提供安全有效的药物治疗,要在患者用药前、用药过程中和用药后提供全程化的药学服务。为了提供这种负责的药学服务,就要求药师不但要掌握药学的基本知识、熟悉基础医学和临床医学的知识,并且要将这些知识转变成为患者制定个体化要素的治疗方案和对患者合理用药的指导,而临床药物治疗学则是为医学服务的理论和方法。

