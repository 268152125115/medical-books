\chapter{精神分裂症及其他精神病性障碍}

\section{精神分裂症}

\subsection{概念}

精神分裂症(schizophrenia)是一种常见的病因尚未完全阐明的精神疾病。多在青壮年起病,临床上以基本个性改变,思维、情感、行为的分裂,精神活动与环境的不协调为主要特征。本病患者一般无意识和智能方面的障碍,病程多迁延并呈进行性发展,部分患者可最终出现衰退和精神残疾。

\subsection{流行病学}

精神分裂症的发病年龄一般在15~45岁,多见于青壮年。男、女两性间发病没有明显的差异。根据世界卫生组织(WHO)1992年公布的资料,该病时点患病率为1‰~11‰;估计全球精神分裂症的终身患病率大概为3.8‰~8.4‰。我国于1982年进行的12地区精神疾病流行病学调查结果,精神分裂症的终生患病率为5.69‰。1994年进行的12年随访,上升为6.55‰,城市患病率高于农村,前者为7.11‰,后者为4.26‰。

\subsection{病因和发病机制}

\subsubsection{遗传}

家系调查、孪生子和寄养子调查资料以及分子遗传学研究证明,遗传因素在精神分裂症的发生中起一定作用。

1.家系调查 国内外不同地区的家系调查,发现精神分裂症患者近亲中的患病率,比一般居民高数倍。与患者血缘关系越近,患病率愈高。最早的家系调查是由Ernst
Rubin(1916)在慕尼黑进行的,他发现病人的兄弟姐妹中,该病的患病率最高。Kallmann(1938)调查了1087名精神分裂症先证者家属中的发病率,发现不但是兄弟姐妹,而且子女父母的患病率也较高。各级亲属中发病概率为4.3%~16.4%,其中子女16.4%,同胞11.5%~14.3%,父母9.2%~10.3%。上海(1958)对1198例精神分裂症患者的54576名家属成员进行调查,发现父母及同胞的精神分裂症患病率最高。患者亲属中精神病的患病率比一般人口高6.2倍。

2.双生子研究 在医学遗传学中,孪生子法是一种有效的方法。单卵孪生子(MZ)所获得的遗传信息几乎完全相同,而双卵孪生子(DZ)所获得的遗传信息并不完全相同。最早的孪生子研究是Luxenberger(1928)在慕尼黑进行的,他发现在19对MZ中,同病率为58%;而在13对DZ中,竟无一对同病。国内方惠泰(1982)报道50对精神分裂症孪生子研究结果,MZ同病率为46.4%,DZ同病率为18.2%。尽管上述资料的具体数字有差别,但均发现MZ同病率比DZ同病率高3~6倍。

3.寄养子研究 寄养子研究是为了区别遗传因素与环境因素的作用。家系调查和孪生子研究支持遗传因素的作用。但环境因素的作用尚不能完全排除,发病率较高可能是由于家庭成员异常精神行为的影响。寄养子研究可进一步澄清。为排除本病的发生是受家庭环境的影响,Heston(1966)将本病患者母亲的47名子女自幼寄养出去,由健康父母抚养,与50名双亲健康者的子女作对照。至成年后,实验组有5人患精神分裂症(患病率为10.6%,校正年龄后为16%),22人有病态人格;对照组无精神分裂症病人,9人有病态人格,差别有显著性。

4.分子遗传学研究 基因组扫描及候选基因筛查为目前主要的研究手段。研究发现具有独立重复的阳性区域主要集中在6p、22q和8p,候选基因的阳性结果主要涉及5-HT\textsubscript{2A}
受体、多巴胺D\textsubscript{3} 受体(DRD\textsubscript{3}
)及神经营养因子3(neurotropin-3, NT-3)等。

\subsubsection{生化}

1.多巴胺(DA)能假说 多巴胺活性过度假说是最被广泛接受的精神分裂症病因假说,主要源于精神药理方面的研究。拟精神病药物苯丙胺可以产生类精神分裂症的症状,并且可以使分裂症患者的病情恶化;抗精神病药物的药理作用是通过阻滞DA受体的功能而发挥治疗作用。近年来关于在纹状体处DA释放的正电子发射断层显像(PET)研究表明,精神分裂症患者这些区域的细胞外的DA浓度较正常人群为高。有的研究者认为精神分裂症患者DA受体敏感性增高,会导致精神分裂症发病,但这一研究目前只有间接的证据支持。

2.5-羟色胺假说 5-羟色胺(5-HT)神经源于中脑被盖核和中缝核,这两个核投射到皮质、纹状体、海马和其他边缘系统区域。5-HT是中枢神经系统中最丰富的神经递质,在大脑功能的许多方面起作用,尤其是控制觉醒水平和睡眠-觉醒周期。与心境抑郁、自杀行为也有密切关系。5-HT受体亚型至少有15种,其中5-HT\textsubscript{2}
受体与精神分裂症关系密切,如致幻剂麦角酸二乙酰胺(LSD)和仙人掌毒碱是吲哚类物质,对5-HT\textsubscript{2}
起激动作用,能引起精神病性症状,从而提出了分裂症的5-HT假说。5-HT\textsubscript{2}
拮抗剂利坦色林(ritanserin)能改善分裂症的阴性症状和情感症状,但不能改善阳性症状。

3.谷氨酸假说 有许多研究显示精神分裂症患者脑脊液中谷氨酸功能降低,而谷氨酸能活性的降低是由于谷氨酸受体N-甲基-d-天门冬氨酸(NMDA)受体含量下降所致。另外非竞争性NMDA受体措抗剂苯环己哌啶(phencyclidine,
PCP)在正常人群及精神分裂症病人中均可引起阳性症状、阴性症状及认知损害症状。抗精神病药物可阻断苯环己哌啶的某些临床作用,所以可改善阳性症状、阴性症状及认知损害症状。

\subsubsection{脑结构和脑影像学的异常}

精神分裂症患者尸检证明脑内存在异常,包括边缘系统和颞叶结构萎缩。杏仁核、海马、海马旁回等,海马不同区域均有不同程度的体积减小,但这些异常均不是精神分裂症患者的特征性改变,也并不是所有的精神分裂症病人都存在这样的改变。

计算机断层扫描(CT)和核磁共振(MRI)研究表明,精神分裂症较为一致的发现是脑室增大,脑沟回增宽。但这些变化与病程长短及是否接受治疗无关。精神分裂症的功能磁共振成像(functional
magnetic resonance imaging,
fMRI)研究自20世纪90年代以来飞速发展。静息状态fMRI研究的初步报道显示,患者存在脑区功能连接异常和区域一致性下降。任务状态下fMRI研究大部分显示,在幻听(尤其是言语性幻听)的精神分裂症患者可见双侧或左侧与听觉相关脑区激活(包括颞横回、颞中回和颞上回皮层以及右侧颞中回)。但也有相反的结果。

\subsubsection{心理、社会因素}

不少学者发现,精神分裂症可在各种各样的精神因素影响下诱发,诸如恋爱失败,婚姻破裂,学习、工作受挫等。调查资料表明,精神分裂症发病前有精神诱因者占44%~77%。但多数学者认为,精神因素对精神分裂症发病的作用,是建立在个体心理承受能力的基础之上。在实际生活中,我们常常看到,有些人在生活中遇到极大不幸,生活的道路极为坎坷,但他们并未患精神病;反之,有的人在微不足道的、几乎是很多人都可能遇到的一些挫折面前,表现情绪沮丧,以至精神失常。

调查显示,精神分裂症患者的生活事件明显多于一般人群,但生活事件是发病的原因还是结果还不能确定。美国的调查发现,生活贫困、经济条件低劣和居住在贫民区的最低社会阶层的人群,精神分裂症的患病率较高。我国的调查也得出类似的结果,即经济水平低、无职业的人群中,精神分裂症的患病率明显高于经济水平高的职业人群的患病率。

\subsubsection{病前个性}

许多学者注意到,精神分裂症患者中的50%~60%在得病前具有某种特殊的个性特征,其表现为孤僻、内向性格、怕羞、多疑、敏感、思考问题缺乏逻辑性、好想入非非等。在精神病学中,有的学者把这种个性特征称为“分裂性人格”。根据这一现象,一般认为精神分裂症的发病与病前个性特征有一定关系。

\subsection{临床表现}

\subsubsection{早期症状}

认识精神分裂症的早期症状是十分重要的,可以早发现及早治疗。急性起病者病前很难发现或者根本就不存在早期症状。大部分患者是在无明显诱因下缓慢起病,仔细观察分析一般都能发现有一些早期精神症状。Hafner(1992)对德国232例首次发病的精神分裂症病人在其入院后3~5周症状有所缓解后进行症状评定及知情人提供资料,以测定回顾性的发病和早期症状,发现非特异性前驱症状在精神病性症状出现之前已有数年之久;大多数(73%)很长时期的前驱症状是非特异性的或阴性症状,出现率在10%以上的早期症状为:不安,抑郁,焦虑,思考和注意力集中困难,担心,缺乏自信,无力、迟钝,完成工作困难,社会退缩、不信任,社会退缩、交往减少。1999年,Hafner统计德国、美国、加拿大8位作者对初次入院精神分裂症病人早期症状出现时间,发现其在2.1~5年之间。

\subsubsection{临床症状}

1.思维联想障碍 思维联想过程缺乏连贯性和逻辑性,是本病具有特征性的症状。其特点是病人在意识清楚的情况下,思维联想散漫或分裂,缺乏具体性和现实性。交谈时可表现对问题的回答不切题,对事物叙述不中肯,使人感到不易理解,称为“思维松弛”。

有时病人可在无外界原因影响下,思维突然中断,或涌现出大量思维并伴有明显不自主感,称强制性思维。有些病人用一些很普通的词或动作,表示某些特殊的,除病人自己以外别人无法理解的意义,称病理象征性思维,或将两个或几个完全无关的词拼凑起来,赋予特殊意义,称语词新作。

2.情感障碍 情感淡漠,情感反应与思维内容以及外界刺激不配合,是精神分裂症的重要特征。最早涉及的是较细腻的情感,如对同事欠关心、同情,对亲人欠体贴等。以后,病人对周围事物的情感反应变得迟钝,对生活和学习的兴趣减少。随着疾病的发展,病人的情感日益淡漠,最后病人可丧失与周围环境的任何情感联系。

在情感淡漠的同时,可出现情感反应与环境的不协调,与思维内容不配合。病人可为琐事而勃然暴怒,或含笑叙述自己不幸的遭遇,后者称为情感倒错。

3.意志行为障碍 病人的活动减少,缺乏主动性,行为变得孤僻、被动、退缩,即意志活动减退。病人对社交、工作和学习缺乏要求,表现为不主动与人交往,无故旷课或旷工等。严重时行为极端被动,对生活的基本要求亦如此。病人不注意清洁卫生,长期不洗澡,不梳头,生活懒散,终日无所事事,呆坐或卧床。部分病人的行为与环境完全不配合,吃一些不能吃的东西(如肥皂、污水),伤害自己的身体等,称意向倒错。

4.其他常见症状

(1)感知觉障碍:幻觉见于半数以上的病人,其特点是内容荒谬,脱离现实。最常见的是幻听,主要是言语性幻听。具有特征性的是听见两个或几个声音在谈论病人,彼此争吵,或以第三人称评论病人(评论性幻听);幻听也可以是命令性的。有时声音重复病人的思想,病人想什么、幻听就重复什么(思维鸣响)。

此外,精神分裂症还可以出现视幻觉,幻视的形象往往很逼真,颜色、大小、形状清晰可见,内容多单调离奇。幻嗅、幻触、幻味较少见。

(2)妄想:妄想是精神分裂症最常见的症状之一。在部分病例中,妄想可非常突出。精神分裂症妄想的主要特点是:①内容离奇,逻辑荒谬。②妄想所涉及范围人的一举一动是针对他的。③病人对妄想的内容多不愿主动暴露,并往往企图隐蔽它。有的病人坚信有外力在控制、干扰和支配他的思想和行为(被控制感),甚至认为有某些特殊仪器、电波、电子计算机在操纵或控制他(影响妄想)。有时则坚信自己的内心体验,所想的事已尽人皆知(被洞悉感),影响妄想和被洞悉感是精神分裂症的特征症状。

原发性妄想在本病出现的频率并不高,但在诊断上有重要意义,也是本病的特征性症状。这种妄想发生突然,完全不能用病人当时的处境和心理背景来解释。例如一病人从外地回来,一下火车突然感到环境变了,看到周围的人态度也变了,谈话中都在议论与他有关的事等。继发性妄想常发生于幻觉基础之上。

\subsection{临床类型}

\subsubsection{偏执型}

该型以妄想为主,常伴有幻觉,以幻听较多见,是最为常见的精神分裂症类型。起病年龄较其他各型为晚,以青壮年和中年为主。病初表现为敏感多疑,逐渐发展成妄想,并有不断泛化趋势,妄想内容日益脱离现实、荒谬离奇。妄想种类常见的有关系妄想、被害妄想、被培养妄想和非血统妄想。有时可伴有幻觉和感知觉综合障碍。情感和行为常受幻觉和妄想支配,表现多疑、恐惧,甚至出现自伤及伤人行为。此型病程发展较其他类型缓慢,精神衰退现象较不明显,预后相对较好。

【\textbf{病例}
】 {患者感到过去的女友买通了自己父母,对自己进行监视,家中有窃听器、摄像头,街上有特务在跟踪,包括修自行车的、皮匠都在监视他(被害妄想),这些特务分成两种人,一种人是过去的女友派来害他的,另一种人是中央派来保护他的,为什么要保护他呢?因为他是大学教师、有学历、有理论、有培养前途,中央组织部正在有心培养他(被培养妄想)。他和父亲回到老家,乡亲们对他们热情款待,他感悟到自己不是现在父母生的,而是中央某高级领导人的儿子,是哪个领导人的儿子,他也不清楚,但从乡亲们的脸上看,他们心里都清楚,只是自己不清楚(非血统妄想)。}

\subsubsection{青春型}

本型也较为多见。多发病于青春期,起病较急,病情发展较快。主要症状是思维内容离奇,难以理解,思维破裂,情感喜怒无常,行为幼稚、愚蠢、零乱。以“三乱”、“两症”和一过性为特征。

“三乱”是指思维破裂、情感倒错和意向倒错。思维破裂是指在意识清晰的前提下说话句与句之间缺乏联系;情感倒错是指情感体验与思维内容相反(倒)或不一致(错),所谓情感不协调,通常指的是后者;意向倒错是指与本能有关的活动(即意向,包括食欲、性欲和无条件防御反射)与常人相反(倒)或不一致(错)。“两症”是指行为怪异症状和色情症状,行为怪异症状是指患者突然做出一些古怪动作,让人无法理解,如突然钻到床下,学狗叫;色情症状则表现为当着异性的面脱裤子,触摸异性,纠缠异性。“一过性”是指这种患者的幻觉妄想是一过性的。

【\textbf{病例}
】 {男性,18岁,冲动毁物,说话前言不搭后语1个月而住院。检查时答不切题,问他年龄,答“百分之百,张国荣没死,因为眼睛里长眼睛,眼睛看他,看蔡亦银,他是我老婆,你不知道我多喜欢她(思维破裂)”。问到是否有人对他不好时,答“他要害我,比我高,比我白,比我净,我好渴哎,我心里一切的一切都是为中国人着想,不为美国人着想,美国人要想让我们的国土变小,变得像蚂蚁一样小(一过性妄想)”。患者的母亲在一旁急得流泪,患者却轻松地嘻嘻直笑(情感倒错)。他用手指抠肛门,然后送进嘴里吮,喝痰盂里的痰(意向倒错),一只脚套着痰盂走路(怪异行为),当众脱裤子,玩弄生殖器(色情症状)。诊断为精神分裂症青春型。}

\subsubsection{紧张型}

大多数起病于青壮年时期,起病较急,病程多呈发作性,主要表现为紧张性木僵和紧张性兴奋,两者交替出现,或单独发生。患者可出现紧张性木僵、蜡样屈曲、刻板言语和动作等紧张症状。有时会从木僵状态突然转变为难以遏制的兴奋状态,这时行为冲动,常有毁物伤人,一般数小时后可缓解,或回复进入木僵状态。此型有可能自动缓解,治疗效果较其他型好。

\subsubsection{单纯型}

本型较为少见,常在青少年时期起病,起病隐袭,缓慢发展,病程至少2年。表现为孤僻、被动、活动减少等情形日益加重,并日益脱离现实生活。临床症状主要为逐渐发展的人格衰退。一般无幻觉和妄想,以阴性症状为主。此型患者由于早期症状不典型,常不被人注意,往往经过数年的病情发展到较严重时才被发现。该型自动缓解者少,治疗效果和预后较差。

【\textbf{病例}
】 {某男,16岁,上小学时就好着急,受不得委屈,对学习不感兴趣。2年前表现更突出,觉得上学无乐趣,高兴就去上课,不高兴就称病不去。上课不听讲,作业也不做,遇有难度的问题干脆放弃,连看也不看一眼,学习成绩由前10名降至倒数第10名。与同学关系疏远,“别人不主动找我,我也没必要找人,井水不犯河水”。常在家睡懒觉,叫他起床就发脾气。平时听不得家里人劝他读书的话题,一听就生气,踢桌椅,拿东西就砸,用头撞墙,“我没用,是你们的负担”。2个月前干脆不上学了,成天抱着一本《故事会》或《少年文艺》来回翻,根本看不进去。诊断为精神分裂症单纯型。}

此外,由于在临床上无法归类于上述哪一个类型,临床上将之定为未定型。另外,根据临床残留症状、病期、社会功能状态等,可以分为精神分裂症残留型、衰退型,精神分裂症后抑郁等类型。还有根据阳性症状和阴性症状进行分型,Ⅰ型精神分裂症以阳性症状为特征,主要表现幻觉、妄想,明显的思维形式上障碍等。Ⅱ型精神分裂症以阴性症状为主,主要表现思维贫乏、情感淡漠、意志减退等。前者对抗精神病药物反应良好,后者相对较差。

\subsection{认知症状}

1.注意障碍 注意是让意识聚焦于刺激而不是其他(注意的选择性),并且维持一定的时间长度,以允许信息在更高的层面上进行加工(注意的维持性)。精神分裂症可有明显的注意障碍,如一位17岁男患者治疗后说:“我好像是变了一个人,以前不在意妈妈做事的细节,视而不见,只觉得妈妈永远穿那几件衣服,一点新鲜感也没有。她一讲话我就嫌烦,啰里啰嗦。现在我开始注意到妈妈经常在换衣服,有生活情调,妈妈讲的话,我能将心比心了”。

精神分裂症的注意选择性障碍,监视说话意图的能力减退,易出现思维散漫;监视情感表达的能力减退,故易出现情感不协调。精神分裂症患者的注意选择性下降,分不清重要和不重要的信息(如把注意投向街上无关的人),注意的维持性下降,妨碍信息在更高层面上进行加工,以致不能良好地判断和推理,易出现妄想(如认为街上的人都在议论自己)。精神分裂症患者有的注意障碍是素质性标志,即使病情缓解时,注意障碍仍存在,这可能是缓解后易复发的基础。但有的注意障碍是状态指标,经治疗可以缓解。在人际交流中,注意听讲是人际交流的良好开端;相反,不注意听讲会使人感到缺乏兴趣和没有礼貌。精神分裂症患者有注意障碍,在人际交流中不注意听讲,从而妨碍了社交能力。

2.操作性记忆障碍 操作性记忆是指刺激数秒钟后短时间内使用该信息的能力,如看过电话号码后立即拨号。精神分裂症患者的操作性记忆能力降至正常人的4个标准差以下,难以记住刚才说过的话和意图,易出现思维散漫,甚至思维破裂;不易记住新朋友的名字,故难以发展友谊;不易将事情做得完整,如洗完衣服总有1~2件衣服忘了晾,洗衣机插头也忘了拔。

3.长期记忆障碍 长期记忆是从既往经历的事件中获得信息的能力。精神分裂症患者迅速从提示性回忆中获益比健康人少,故精神分裂症复发时又再次相信曾被否定的妄想内容是真实的。长期记忆损害可能记不起与朋友共同经历的事件,故难以维持友谊。长期记忆损害不易记住工作程序,给工作带来不便;不能记住生活琐事,给生活带来不便。

4.实施功能 是使用抽象思维去解决问题的能力,如解数学题。精神分裂症的实施功能差,从而促进了阴性症状(如逻辑减退、情感迟钝、意志减退和社交退缩)和自知力缺乏(对自己精神状态的认识缺乏)的发生。而自知力缺乏导致服药不合作,增加自伤和伤人危险性,最终导致预后差。实施功能减退导致不善于计划自己的工作日,不知该优先做什么,不能使用已知技术解决问题,因此降低职业成功的可能性。

5.认知与康复训练 在康复训练中,警觉性、操作性记忆、长期记忆和实施功能差的精神分裂症患者疗效最差,因为其学习技术能力最差。

\subsection{预后}

精神分裂症患者约有1/4的人痊愈、1/4的人轻度缺损、1/4的人明显缺损、1/4的人痴呆。痊愈是指精神症状消失,自知力恢复,完全或部分恢复工作。精神分裂症患者后来仅有10%的人可参加竞争性全日制工作(如大学教师),20%的可参加非竞争性的工作。轻度缺损是指尽管尚有精神症状,多少也影响学习、生活和工作,但无大碍。如一患者恢复期残留一症状,即怕上课时在黑板上写字手抖,尽管时轻时重,但教学任务均能完成,还受到好评。明显缺损是指精神症状明显影响学习、生活和工作。如某患者病后学习成绩差,感到同学歧视他(可能确实如此),因而不肯上学,长期在家,无动力、无亲情、无打算。痴呆是一种分裂性痴呆,这种患者感到脑中空洞,想不出东西(思维贫乏),对亲人不关心,冷漠(情感淡漠),自己卫生不能自理,不洗澡,不理发,不换衣,甚至自己的经期卫生也不能自理(意志缺乏),非但不能工作,反而需要人照顾。

\subsection{影响预后的因素}

影响预后的因素很多,但没有一种是特异性的,将这些因素综合起来,往往能大体判定预后。

1.一般情况

(1)性别、年龄:男性、发病年龄小者预后差,反之则好。发病年龄越小,预后相对越差。

(2)婚姻:独身、分居、守寡和离婚者预后差,已婚者预后好。

(3)经济和环境:经济地位高的患者预后好,住在非工业城市较工业城市的预后好,住在发展中国家比发达国家的预后好。

2.家族史和个人史

(1)家族史:家族有精神分裂症且预后差者,患者的预后也差;家族有精神分裂症且预后好者,患者的预后也好;家族有情感性障碍者,患者病程又呈间歇发作者预后好。

(2)产后并发症:有严重产后并发症的患者预后比无严重产后并发症的差。

(3)社会支持:社会对精神病的态度影响预后。家庭照顾好、与家属关系好的患者预后好,反之,家属高情感表达(经常批评责骂、显示敌意以及情绪激动等)的患者预后差。社会隔离者预后最差。

(4)病前能力和人格:病前工作能力强、人格健全者,因为社会适应性好,故预后好;相反,病前工作能力差,呈分裂性人格、偏执性人格、反社会人格者,因为社会适应性差,故预后差。

3.其他

(1)诱发因素:起病时无明显的精神因素或躯体因素作为诱因者,预后较差。

(2)疾病经过:缓慢起病、病程长、持续病程、多次复发、无精神因素者预后差,反之则好。病情自发缓解,每次缓解均无残留症状者预后好。

(3)阳性、阴性和认知症状:阳性症状预后好;相反,原发性阴性症状和认知障碍明显的预后差。

(4)其他症状:出现一级症状者预后差(一级症状包括争论性幻听、评论性幻听、思维鸣响、思维被扩散、思维被撤走、思维阻塞、思维插入、躯体被动体验、情感被动体验、冲动被动体验及妄想知觉);有假性幻觉、嗅幻觉或生殖器触幻觉的预后差。情感活跃(包括躁狂、抑郁、内疚、焦虑、紧张)、心身症状和意识障碍者预后好。

(5)诊断类型:单纯型预后最差,青春型次之;紧张型较好,偏执型最好。

4.检查和治疗的反应

(1)CT检查结果:脑室正常者预后好,脑室扩大者预后差。

(2)治疗机会:从未治疗过,愿意接受治疗,治疗及时合理,缓解后能坚持服药者预后好;反之则差。

\subsection{诊断}

中国精神疾病障碍分类与诊断标准(第3版)(CCMD-3)的精神分裂症诊断标准为:

1.症状标准 至少有下列2项,并非继发于意识障碍、智能障碍、情感高涨或低落。单纯型分裂症另规定。

(1)反复出现的言语性幻听。

(2)明显的思维松弛、思维破裂、言语不连贯,思维贫乏或思维内容贫乏。

(3)思维被插入、被撤走、被播散、思维中断,或强制性思维。

(4)被动、被控制,或被洞悉体验。

(5)原发性妄想(包括妄想知觉、妄想心境),或其他荒谬的妄想。

(6)思维逻辑倒错、病理性象征性思维,或语词新作。

(7)情感倒错或明显的情感淡漠。

(8)紧张综合征,怪异行为,或愚蠢行为。

(9)明显的意志减退或缺乏。

2.严重标准 自知力障碍,并有社会功能严重受损,或无法进行有效交谈。

3.病程标准

(1)符合症状标准和严重标准至少已持续1个月。单纯型病程至少持续2年。

(2)若同时符合分裂症和情感性精神障碍的症状标准,当情感症状减轻到不能满足情感性精神障碍症状标准时,分裂症状需继续满足分裂症的症状标准至少2周以上,方可诊断为分裂症。

4.排除标准 排除器质性精神障碍、精神活性物质和非成瘾物质所致精神障碍。尚未缓解的分裂症患者,若又罹患器质性精神障碍、精神活性物质和非成瘾物质所致精神障碍,应并列诊断。

\subsection{鉴别诊断}

1.神经衰弱 精神分裂症患者自知力不完整,情感反应不强烈,治疗要求不迫切;相反,神经衰弱患者自知力完整,情感反应强烈,治疗要求迫切。精神分裂症患者可有思维离奇、情感迟钝和兴趣减退,而神经衰弱患者则没有这些症状。

2.抑郁症 精神分裂症紧张型木僵患者不管医生尽多大努力,均无情感反应,患者表现情感淡漠,有时还伴有违拗。相反,抑郁性木僵患者在医生的耐心询问下,回答简短但切题,即使不能回答,其眼神和表情仍可与医生进行情感交流。

精神分裂症的自杀特征是计划不周,隐蔽性差,成功率低。如一女患者耳中听到声音骂她是“痴子”、“二五”,心里不能接受,将绳子挂在窗子上上吊,一只脚悬空,当感到颈子勒得难过时,另一只脚踮在床上。这是一种冲动性自杀,故计划不周;站在病床上自杀,隐蔽性差;上吊时还留一只脚踮在床上,成功率低。

相反,抑郁症患者的自杀特征是计划性强,隐蔽性好,成功率高。如一男性患者住院时抑郁严重,4周后“突然好转”,称想回家看看,家属也坚决要求假出院。傍晚时一到家,家属忙着烧菜,患者进厕所即插门,遂从5楼跳下,当场死亡,其实他在住院期间抑郁并未好转,当他打定主意自杀时,暗自庆幸自己找到出路,反而显出轻松愉快的神态,有计划地骗过医生和家属(计划性强),在所有人都察觉不到的情况下(隐蔽性好),以坚决和突然的方式自杀(成功率高)。

3.躁狂症 精神分裂症患者可有兴奋躁动,但不伴有情感高涨,情感变化与思维内容和环境不相配合,动作较单调刻板。而躁狂症患者情感活跃、生动、有感染力,情感变化与思维内容和环境相配合,洞察反应较敏捷。

4.反应性精神障碍 由精神因素促发的精神分裂症,在疾病早期思维和情感障碍均可带有浓厚的反应色彩,但随着病情的发展,其妄想越来越荒谬,离精神因素越来越远,并缺乏相应的情感反应,不主动暴露其内心体验。而反应性精神障碍即使有妄想,其内容也不荒谬,且紧紧围绕精神因素,情感反应强烈而鲜明,能主动叙述其内心体验,以求得支持和同情。

\subsection{治疗}

\subsubsection{药物治疗}

20世纪50年代,第一个抗精神病药物氯丙嗪问世,以后又有多种抗精神病药物被用来治疗精神分裂症。目前常用抗精神病药物分为传统的抗精神病药物和非典型抗精神病药物两大类。前者包括氯丙嗪、奋乃静、氟奋乃静、三氟拉嗪、氟哌啶醇、舒必利、泰尔登等,以及长效药物如五氟利多、氟奋乃静葵酸酯、氟哌啶醇葵酸酯、哌普嗪棕榈酸酯;后者包括氯氮平、利培酮、奥氮平、喹硫平、齐拉西酮、阿立哌唑以及长效利培酮注射液。

1.急性期治疗

(1)肌内注射:对兴奋躁动或拒绝服药的患者,可肌内注射氟哌啶醇10mg,每日2次。病情改善后用口服药物治疗。肌内注射因引起局部疼痛、硬块和无菌性脓肿,故使用常不超过3天。

(2)治疗量:以疗效好和不良反应小为准。一般成人剂量氯丙嗪为每日300~600mg,奋乃静每日20~60mg,氟哌啶醇每日12~20mg,氯氮平每日300~400mg,利培酮每日2~6mg,奥氮平每日5~20mg,喹硫平每日300~800mg,齐拉西酮每日120~160mg,阿立哌唑每日15~30mg。

(3)选药法:典型和非典型抗精神病药物对阳性症状均有效,而对阴性症状非典型的抗精神病药物优于典型的抗精神病药物。非典型药物目前临床使用相对较多,因为它的不良反应较小,对患者工作、生活、学习影响较小,故已为患者所接受。虽然其价格相对较贵,但患者的生活质量明显提高了,而且可以恢复他们原有的社会功能。典型抗精神病药物,因相对不良反应较大,临床使用量渐下降。因它的价格比较低,故仍有使用的市场。氯氮平为非典型药,疗效较好,但由于不良反应较大,作为首发患者,不要作为首选治疗药。在国外本品是限制使用的药物。治疗以单一药物为最佳。

(4)加药和服法:口服药一般采用渐增法,药物从小剂量开始(例如氯丙嗪每日100mg,分2次服用)。住院时,以2周左右时间逐渐增至治疗量(例如氯丙嗪增至每日300~600mg,分2次服用);门诊则需要更长时间(如4周以上)增至治疗量。如剂量不大,可睡前顿服;如剂量较大,应分次服用。

(5)起效时间和换药:一般来说,兴奋躁动在1周内起效,幻觉妄想在4~8周起效。如既无疗效也无锥体外系反应,可能是剂量不足。急性病例使用治疗量4~8周无效即考虑换药,慢性病例使用治疗量3~6个月无效才考虑换药。

(6)继续治疗:精神分裂症药物治疗应系统而规范,强调早期、足量、足疗程的“全病程治疗”。目前对急性症状控制后,原剂量维持多长时间仍有不同的看法,大多数人的观点是至少维持2~6个月,然后考虑减量,至维持剂量。

2.维持治疗

(1)维持时间:精神分裂症患者缓解后,不维持抗精神病药的复发率比维持的至少高2倍。首发精神分裂症患者停药1年的复发率为41%~57%,而复发精神分裂症患者则为74%~80%。有人推荐前者至少维持治疗1~2年,后者至少维持治疗5年乃至终生。

(2)维持剂量:在治疗剂量范围内,维持剂量与复发率呈负相关。维持剂量越高,复发率越低,但不良反应越多;相反,维持剂量越低,复发率越高,但不良反应越少。因此,只要病情稳定,应将药物减至最低有效量。所谓最低有效量,就是能控制病情,但不良反应小得基本上看不出来。维持治疗的剂量应个体化,一般为急性期治疗剂量的1/2~2/3。

\subsubsection{物理治疗}

物理治疗主要是目前开展的无抽搐电休克或电休克治疗,对精神分裂症的急性症状,如冲动、伤人、拒食、自杀、紧张性木僵等有较好的疗效。而且采用无抽搐电休克治疗方法已明显减少了原治疗的不良反应,消除患者与家人的紧张恐惧心理,治疗的适应证已明显放宽。其主要的不良反应有短期的记忆障碍,少数患者可以出现头痛和牙齿松动等,治疗是比较安全的。

\subsubsection{心理与康复治疗}

心理治疗是精神科常用的治疗方法,对精神分裂症也适用,特别在患者的恢复期,通过心理治疗,不但可以改善患者的精神症状,同时让患者提高对疾病的认识能力,增加治疗依从性,帮助患者增加与人和社会接触交流的技巧,提高他们的自信心,面对可能遇到的各种困难和机遇。康复对恢复期患者是非常重要的,鼓励他们多参加社会活动和相关的训练活动及康复活动,可改善他们的日常生活能力和人际关系。详见康复章节。

\subsection{难治性精神分裂症的治疗}

难治性精神分裂症系指诊断正确,过去5年内用过两类结构不同的3种抗精神病药,足量、足程治疗(氯丙嗪等价剂量每日600mg以上,至少治疗8周)仍无效者。

\subsubsection{难治性阳性症状的治疗}

1.抗精神病药

(1)足量:如服药起初6周毫无改善,应测定血药浓度,浓度低可能说明患者依从性差,或为快代谢型,或是药物吸收不完全。这时可改用长效注射制剂。血药浓度过高非但引起不良反应,而且疗效也下降,减药反而会有改善症状。

(2)足程:抗精神病药显效前6周较快,以后逐渐变慢,提示治疗前6~8周只要稍有改善,继续治疗3~6个月时病情可获进一步改善。

(3)加药:Kimon等曾把口服氟奋乃静每日20mg无效的患者随机分成3组,一组将氟奋乃静加至每日80mg,另一组换成氟哌啶醇,第三组继续以氟奋乃静每日20mg治疗,结果发现3组疗效无显著差别。

(4)联合治疗:当典型抗精神病药中的高效价药和低效价药联用时,有的改善明显,有的改善不明显。当两种不典型抗精神病药联用时,部分症状可得到改善。一些医生提倡用利培酮加氯氮平治疗,但一般不作为一线治疗。

(5)改用氯氮平:Kane等报道,氯氮平治疗6周后,30%的难治性患者改善,而氯丙嗪仅有4%的改善。如果延长疗程,其改善率还会增加。Meltzer(1992年)发现,对住院的难治性精神分裂症患者,用氯氮平治疗前6周内有30%的显效,3个月时又有20%的显效。4~6个月有10%~20%的显效。

(6)改用新型抗精神病药:对不能耐受氯氮平的患者可选用新型抗精神病药,但新型抗精神病药的效果也是参差不齐,喹硫平对阳性、阴性和情感症状与氟哌啶醇等效;奥氮平每日15mg对阴性症状的疗效比氟哌啶醇(每日10~20mg)好,每日10~15mg对情感症状疗效较好,但对阳性症状无更多优势;利培酮每日6mg(为美国人剂量,中国人为每日2~4mg)对阳性、阴性和抑郁症状均优于氟哌啶醇(每日20mg)。它们的疗效都不能与氯氮平相比,证据是氯氮平有效的患者换成利培酮或奥氮平时,4周复发率均高于80%。

2.非抗精神病药

(1)联合碳酸锂:一些研究提示,抗精神病药加碳酸锂对有情感症状的难治性患者有效,也能缓解精神分裂症的核心症状,另一些研究未能证实这一点。

(2)联合抗抽搐药:卡马西平对有脑电图异常或明显兴奋冲动的患者有效,但因能降低抗精神病药血药浓度,故应慎用。因本品能降低白细胞,故不可与氯氮平联用。丙戊酸钠对难治性精神分裂症的疗效虽有报道,但未得到重复。

(3)联合苯二氮䓬
类药物:加用苯二氮䓬
类药物可改善其激越及阳性症状。但只有中度改善,持续时间亦短,且易造成药物滥用或停药后症状反跳。

(4)联合电休克:许多个案和小样本结果提示,抗精神病药合并电休克治疗对部分难治性患者有一定疗效,对阳性或情感症状疗效较好,但对慢性难治性患者的效果则不肯定。

\subsubsection{难治性阴性症状的治疗}

应将原发性阴性症状与继发性阴性症状(锥体外系症状、心绪不良或封闭环境所致的阴性症状)区别开来。当未经治疗的猜疑引起情绪退缩时,阴性症状似乎很突出,但用典型抗精神病药即有效;锥体外系症状引起的阴性症状可减药或改服锥体外系症状少的药物或加用苯海索治疗有效;继发于抑郁和焦虑的阴性症状用抗精神病药加抗抑郁药或抗焦虑药有效;继发于环境剥夺的阴性症状用心理社会治疗有效。

1.抗5-HT\textsubscript{2A}
受体药物 ①氯氮平:Kane等报道,氯氮平改善精神分裂症患者的简明精神病量表阴性症状因子分比氯丙嗪更明显。但该组的阳性症状分也很高,有可能是通过改善阳性症状而改善继发性阴性症状。故氯氮平对原发性阴性症状是否有效尚有待研究。②利培酮:北美两项研究指出,利培酮比氟哌啶醇能显著改善阴性症状。但目前不能肯定该药是否对原发性阴性症状有效。

2.抗抑郁药 ①丙米嗪:Sinis等给精神分裂症或分裂情感性患者开始服抗精神病药,在排除药物所致静坐不能后,随机加用丙米嗪或安慰剂,第1周每日50mg,第2周每日100mg,第3周每日150~200mg。结果发现,丙米嗪比安慰剂显著改善阴性症状。②麦普替林:针对32例慢性精神分裂症患者的阴性症状,Yamagami等用常规抗精神病药联合麦普替林治疗,治疗2周后发现,自发性运动减少和缺乏活力改善70%以上,情感退缩和不合作改善50%以上。③氟西汀:Goff等给9例精神分裂症或分裂情感性患者使用氟西汀20mg,治疗第6周时难治性精神分裂症患者显效,其中阴性症状改善23%。

\textbf{(孙 静)}

\section{偏执性精神障碍}

偏执性精神障碍(paranoid mental
disorders)是指一组以系统性妄想为主要症状而病因未明的精神障碍,若有幻觉则历时短暂且不突出。在不涉及妄想的情况下,不表现明显的精神异常。本病的妄想常具有系统化的倾向。病程进展缓慢,一般不出现人格与智能衰退,并有一定的工作和社会适应能力。

在既往的疾病分类中,偏执性精神障碍分为偏执狂(paranoia)与偏执状态(paranoid
state)两种并列的疾病。1994年的CCMD-2-R、2001年的CCMD-3以及ICD-10把这两个疾病合并为一个诊断,使用同一编码。

\subsection{病因}

本病病因不明,多于30~40岁缓慢起病。患者病前大多具有特殊的性格缺陷,表现为主观、固执、敏感多疑,对他人怀有戒心,自我中心,自命不凡,遇事专断、不能冷静面对现实等。当遇到某种心理社会因素时不能妥善应付,而将事实加以曲解,进而形成妄想。生活环境的改变如拘役、移民、受到不公正待遇、被隔绝等容易诱发本病。

\subsection{临床表现}

本病起病缓慢,多不为周围人所察觉,逐步发展成一种或一整套相互关联妄想,内容可为被害、嫉妒、钟情、夸大、疑病等。妄想多持久,有时持续终生,很少出现幻觉,也不出现精神分裂症的典型症状,如被控制感、思维被广播等。妄想不泛化,内容不十分荒谬,结构层次与条理分明,推理过程有一定的逻辑性。多与患者的亲身经历与处境等密切相关,并根据环境变化,赋予一些新的解释。

被害妄想较多见。患者认为人身受到迫害,受到跟踪与监视,名誉受到玷污,个人权利受到了侵犯。被害常与诉讼相伴随,一旦诉讼失败,患者多不断地扩大自己的对立面,甚至认为某个部门乃至全社会的人都在迫害他,他会不惜一切代价、不择手段地反复上告,不达目的决不罢手。

嫉妒妄想多见于男性。患者无端怀疑配偶的忠贞,千方百计地质问、检查、跟踪配偶,常常偷偷检查配偶的信件和提包搜集证据。有时患者会在妄想支配下产生伤害行为。

钟情妄想多见于未婚中年女性,她所认定的爱人多具有较高的社会地位、名声,虽然对方已有配偶,也不认识患者,但患者仍坚信其以暗示的方式表达情意,即使遭到当面拒绝,也深信对方是爱自己的,只不过不敢公开恋情而已。

夸大妄想,患者声称自己才华出众,定会有重大发明与创造,也可能成为国内乃至国际首富,或者具有先知先觉的能力。

\subsection{诊断与鉴别诊断}

偏执性精神障碍是以内容相对固定的系统性妄想为主要临床相,妄想内容与患者本人经历与处境密切相关,具有一定的现实性,有些内容不经深入了解,难辨真伪。只要不涉及妄想内容,患者的情感、行为、言语与态度等均正常。社会功能受损,病程持续3个月以上,自知力丧失,并排除相关疾病即可诊断本病(CCMD-3)。

本病主要与偏执型分裂症鉴别。精神分裂症以原发性妄想为主,内容不固定、不系统,并且荒诞离奇,有泛化趋势,常见幻觉。随着病程进展晚期往往导致精神衰退。

\subsection{治疗与预后}

本病患者往往拒绝治疗,也难以随访。当出现兴奋、激越造成社会性危害时,可选用低剂量抗精神病药物治疗,必要时应用注射剂,镇静情绪与缓解妄想。

心理治疗可试用,但难以建立有效的医患关系。由于患者依从性极差,药物与心理治疗均难以进行,总体疗效差。

绝大部分患者病程呈持续性,终生不愈,直至年老体衰,妄想与相应的情绪行为可有所缓解。极少部分病例经治疗可以缓解。

\section{急性短暂性精神病}

急性短暂性精神病(acute and transient
psychosis)指一组起病急骤,以精神病性症状为主的短暂精神障碍。它们具有以下共同特点:①起病急;②以精神病性症状为主,至少具有以下一项:片断的或多种妄想,片断的或多种幻觉,言语紊乱,行为紊乱或紧张症;③病程短暂,一般在数小时至1个月;④预后良好,多数能够完全缓解或基本缓解。

本组疾病包括分裂样精神病、旅途性精神病、妄想阵发(CCMD-3)。

1.分裂样精神病 符合精神分裂症的症状学标准、严重程度标准、排除标准,但是病程不超过1个月。

2.旅途性精神病 指在旅行途中,由于有综合性应激因素而急性起病的精神障碍。患者一般文化水平较低,具有不良的个性,如性格内向等。患者在起病前,多有明显的精神应激,在长途旅行中,由于车船内过分拥挤,造成过度疲劳、慢性缺氧、缺乏睡眠、水分与营养缺乏,患者又可能由于背井离乡,到陌生的地方谋生,对未来充满担心、忧虑,或者携带财物,担心被抢被骗等,是其发病的主要原因。临床主要表现为轻度意识障碍,片断的妄想、幻觉,如认为周围的人想谋财害命,或者认为周围的人跟踪迫害自己,因而情绪紧张、恐惧,往往伴有行为紊乱,如伤害周围的人或者企图砸坏车窗逃跑等。本病病程短暂,停止旅行、经过适当处理与充分休息之后可完全缓解,不遗留后遗症。

3.妄想阵发 又称急性妄想发作,是一种以生动多样的妄想性体验为特征的发作性精神病。在我国,此病名初见于CCMD-2。本病多见于青壮年,不发生于儿童,罕见于50岁以上者。本病多无发病诱因,常突然急性起病,以突然产生多种结构松散、变幻不定的妄想为主,如被害、夸大、嫉妒、影响妄想、被控制感及神秘体验,妄想常混合存在。在妄想背景上,患者常沉溺于丰富、生动的幻觉之中。情感变化多端,随着妄想的变化起伏不定,可有情感高涨、抑郁、焦虑、激越等。意识具有双重性,患者一方面表现为定向良好,与人接触正常且能与环境相适应;另一方面,又表现为恍惚,有一种梦样感觉。行为异常,常有大喊大叫,多与妄想及情感变化相关联。

本病病程短暂,最长不超过3个月。预后良好,多数完全恢复正常,少数有复发倾向。

本病的诊断需排除器质性、心因性精神障碍、分裂样精神病及分裂情感性精神病。治疗可选用小剂量不良反应小的抗精神病药物如利培酮、奥氮平等,疗程结束后根据具体情况给予短期维持治疗。另外,合并心理治疗可提高效果,对预防复发亦有益处。