\chapter{脑器质性精神障碍}

\section{概  述}

脑器质性精神障碍是指由脑部病理或病理生理学改变所致的一类精神障碍,并以此与功能性精神障碍相区别。物质滥用和精神发育迟滞虽然符合上述定义,但常规并不包含在此类障碍中。

脑器质性精神障碍的原发病因与精神症状之间并不存在特异性的依存关系,不同的病因可引起相同的精神症状,相同的病因也可引起不同的精神症状。脑器质性精神障碍主要包括两类综合征:第一类综合征以认知功能或意识障碍为主,如痴呆、谵妄等;第二类综合征的临床表现与功能性精神障碍相似,如精神病性综合征、抑郁综合征、焦虑综合征等。

诊断脑器质性精神障碍可根据下列情况:①有引起精神障碍的脑部疾病、脑损伤或脑功能不全的证据;②脑部病变与精神症状发作有时间上的关系;③精神障碍可因原发性脑部疾病的变化而发生相应的变化;④精神症状不是由其他病因引起(如明显的家族遗传史或应激等诱发因素)。

以下介绍临床上常见的几种器质性综合征。

\subsection{谵妄}

谵妄(delirium)是一组表现为急性、广泛性的认知障碍,尤以意识障碍为主要特征。因其急性起病、病程短暂、病程发展迅速,故又称急性脑病综合征(acute
brain
syndrome)。导致谵妄的原因很多,包括感染、代谢及内分泌紊乱、电解质紊乱、颅内损伤、手术后状态、多种疾病的晚期和药物等。心理社会应激如亲人死亡或迁移到陌生的环境等对谵妄发生具有诱发作用。在社区患者中谵妄较少见,在综合医院的住院患者中,谵妄的发生率是10%~30%,但在术后外科患者中则有50%会出现谵妄。值得注意的是,许多疾病的终末期会伴发谵妄,如癌症患者中25%~40%会出现谵妄,晚期癌症患者中则上升至85%。

谵妄通常急性起病,症状变化大,通常持续数小时至数天。有些患者发病前可表现有前驱症状,如坐立不安、焦虑、激越行为、注意涣散和睡眠障碍等,前驱期持续1~3天。

谵妄的特征包括:意识障碍、注意力不集中以及对周围环境与事物的清晰度降低等。意识障碍有明显的昼夜节律变化,表现为昼轻夜重。定向障碍包括时间和地点定向障碍,严重者可出现人物定向障碍。记忆障碍以即刻记忆和近记忆障碍最明显。睡眠、觉醒周期不规律,可表现为白天嗜睡而晚上活跃。感知障碍尤其常见,包括感觉过敏、错觉和幻觉。患者对声光特别敏感。错觉和幻觉则以视错觉和视幻觉较常见,患者可因错觉和幻觉产生继发性的片断妄想、冲动行为。患者好转后对谵妄时的表现会部分遗忘或遗忘。

目前在国外,谵妄的诊断主要采用CAM(confusion assessment
method)的标准,即:①急性精神状态的改变和波动的病程;②注意力不能集中;③思维紊乱;④意识状态的改变。其中第①、②点是必备的。根据谵妄典型的临床症状可做出诊断。还可根据病史、体格检查及实验室检查明确谵妄的病因,如躯体疾病、电解质紊乱、感染、酒精或其他物质依赖等。

对于谵妄的治疗,主要包括病因治疗、支持治疗和对症治疗。病因治疗是指针对原发脑部器质性疾病的治疗,是最重要的。支持治疗一般包括维持水、电解质平衡,适当补充营养。对症治疗是指针对患者的精神症状给予精神药物治疗。为避免药物加深意识障碍,应尽量小剂量、短期治疗。

\subsection{痴呆}

痴呆(dementia)是指较严重的、持续的认知障碍。临床上以缓慢出现的智能减退为主要特征,有不同程度的人格改变,但没有意识障碍。因本病起病缓慢,病程较长,故又称为慢性脑病综合征(chronic
brain syndrome)。

引起痴呆的病因很多,如中枢神经系统变性疾病、颅内占位性疾病、颅内感染、代谢障碍和内分泌障碍、血管性疾病、脑外伤、中毒和缺氧等。如能及时发现这些病因,及时治疗,预后相对较好。由内分泌障碍、维生素缺乏及神经梅毒等所致的痴呆患者中,10%~15%的人可以好转或痊愈。

痴呆发生多缓慢隐匿。记忆减退是常见症状。早期出现近记忆力受损,表现为当天发生的事不能回忆,刚刚做过的事或说过的话不能记住等。随后远记忆力也受损,使日常生活受到影响。患者常伴有语言障碍,在疾病的初期患者的语言表达仍属正常,但随着病情的发展,会逐渐出现语言功能障碍,不能讲完整的语句,找词困难,命名障碍,出现错语症,交谈能力减退,阅读理解受损,但朗读可相对保留,最后完全失语。患者可出现人格改变,通常表现兴趣减少、主动性差、情感淡漠、社会性退缩,但也可表现为脱抑制行为,如冲动、幼稚行为等。情绪症状包括抑郁、焦躁不安、兴奋和欣快等。部分患者出现片断妄想、幻觉状态等。患者的社会功能受损,对自己熟悉的工作不能完成;晚期生活不能自理,运动功能逐渐丧失,甚至穿衣、洗澡、进食以及大小便均需他人协助。

目前主要根据临床表现、有关量表评定,作出痴呆的诊断,然后对病史、病程的特点、体格检查及神经系统检查、辅助检查的资料进行综合分析,确定是何种原因引起的痴呆。

痴呆治疗的目标是提高患者的生活质量,减轻患者给家庭带来的负担。由于患者的临床症状涉及认知缺损、行为紊乱和情绪改变等多个方面,因此,对于痴呆患者的治疗,应遵循个体化和多方位的原则。主要包括社会心理治疗、针对认知功能的药物治疗和针对伴发的心理行为症状的治疗三个方面(详见有关章节)。此外,痴呆常常是一个进展性的过程,在每一治疗阶段,医生需密切关注日后可能出现的症状,同时帮助患者及其家人对这些可能出现的症状有所了解。

\subsection{遗忘综合征}

遗忘综合征(amnestic syndrome)又称柯萨可夫综合征(Korsakov's
syndrome),是以记忆障碍为主要临床表现,无意识障碍,无其他认知功能损害为特征的一种器质性精神障碍。

滥用酒精导致维生素B\textsubscript{1}
缺乏是遗忘综合征的最常见的病因,其他如严重缺氧或窒息经抢救复苏后、一氧化碳中毒、海马区梗死性病变、脑炎、第三脑室肿瘤、颅脑外伤等也可导致遗忘综合征。

遗忘综合征的主要临床表现是严重的记忆障碍,特别是近记忆障碍,患者学习新事物很困难,记不住新近发生的事情,但患者的注意力和即刻回忆正常。错构和虚构在遗忘综合征患者中也较常见。患者的其他认知功能和技能则相对保持完好。

遗忘综合征的治疗主要是对因治疗,如维生素B\textsubscript{1}
缺乏所致者,要及时补充大量维生素B\textsubscript{1}
,血管性疾病所致者需治疗原发的血管性疾病等。

\subsection{其他}

脑器质性精神障碍还有与功能性精神障碍类似的表现,如幻觉妄想、焦虑抑郁情绪、类躁狂状态、睡眠障碍、人格改变等。但亦有其特点,如幻觉妄想,往往其内容表现片段、多变、不系统等等。


\section{常见脑器质性精神障碍}

\subsection{阿尔茨海默病}

阿尔茨海默病(Alzheimer's disease,
AD)是一组病因未明的原发性退行性脑变性疾病。常起病于老年前期或老年期,潜隐发病,病程缓慢进展,临床上以不可逆智能损害为主要表现。病理改变以大脑皮质弥漫性萎缩、沟回增宽,脑室扩大,神经元大量减少,并可见老年斑(senile
plaques, SP)、神经原纤维缠结(neurofibrillary tangles,
NFT)等病变,胆碱乙酰化酶及乙酰胆碱含量显著减少。65岁以上的老年人群中,痴呆的患病率为4%~7%。患病率随着年龄增加而增加,80岁以上的患病率达20%以上。尸脑研究表明,50%~70%的痴呆为AD,女性多于男性。AD的发病危险因素包括:高龄、丧偶、低教育、独居、经济窘迫和生活颠沛、痴呆或先天愚型家族史、脑外伤史、抑郁症史等。

\subsubsection{临床表现}

1.记忆障碍 患者多为隐匿起病,早期易被患者及家人忽略,主要表现为逐渐发生的记忆障碍,当天发生的事不能记忆,刚刚做过的事或说过的话不记得,熟悉的人名记不起来,忘记约会,忘记贵重物品放何处,词汇减少。早期出现经常性遗忘,主要表现近记忆力受损,随后远记忆力也受损,使日常生活受到影响。

2.认知障碍 是AD特征性的临床表现,掌握新知识、熟练运用及社交能力下降,并随时间的推移而逐渐加重。渐渐出现语言功能障碍,不能讲完整的语句,口语量减少,找词困难,命名障碍,出现错语症,交谈能力减退,阅读理解受损,但朗读可相对保留,最后完全失语;计算力障碍常表现算错账,付错钱,最后连最简单的计算也不能;严重时出现视空定向力障碍,穿外套时手伸不进袖子,迷路或不认家门,不能画最简单的几何图形;不会使用最常用的物品如筷子、汤匙等,但仍可保留运动的肌力和协调。

3.精神症状 包括抑郁、情感淡漠或失控、焦躁不安、兴奋和欣快等,主动性减少,注意力涣散,白天自言自语或大声说话,恐惧害怕单独留在家里;部分患者出现片段妄想、幻觉状态和攻击倾向等,有的怀疑配偶有外遇,怀疑子女偷他的钱物,行为怪异,把不值钱的东西当作财宝藏匿起来;可忽略进食或贪食;多数患者有失眠或夜间谵妄。

AD可分为三型:

(1)老年前期型:发病年龄小于65岁;临床表现有颞叶、顶叶或额叶受损的证据,除记忆损害外,可较早产生失语(遗忘性或感觉性)、失写、失读、失算,或失用等症状;发病较急,呈进行性发展。

(2)老年型:发病在65岁以后;临床表现以记忆损害为主的全面智能损害;潜隐起病,呈非常缓慢的进行性发展。

(3)非典型或混合型:临床表现不典型,如65岁以后起病却具有老年前期型临床特征或同时符合脑血管病所致痴呆的诊断标准,但又难以作出并列诊断者。

AD为慢性进行性病程,总病程一般为5~10年。通常可将病程分为三期,但各期间可存在重叠与交叉,并无截然界限。

第一期(早期):一般持续1~3年,以近记忆障碍、学习新知识能力下、视空间定向障碍、缺乏主动性为主要表现。生活自理或部分自理。

第二期(中期):病程继续发展,智能与人格改变日益明显,出现皮质受损症状,如失语、失用和失认,也可出现幻觉和妄想。神经系统可有肌张力增高等锥体外系统症状。生活部分自理或不能自理。

第三期(后期):呈明显痴呆状态,生活完全不能自理。有明显肌强直、震颤和强握、摸索及吸吮反射,大小便失禁,可出现癫痫
样发作。

总的预后不良,部分患者病程进展较快,最终常因营养不良、肺炎等并发症或衰竭死亡。

\subsubsection{诊断与鉴别诊断}

由于AD病因未明,临床诊断仍以病史和症状为主,辅以精神、智力和神经系统检查,确诊的金标准为病理诊断(包括活检与尸检)。AD的临床诊断可根据以下几点:①老年期或老年前期发生的进行性认知障碍。②以记忆尤其是近记忆障碍、学习新知识能力下降为首发症状,继而出现智力减退、定向障碍和人格改变。③体检和神经系统检查未能发现肿瘤、外伤和脑血管病的证据。④血液、脑脊髓液、EEG及脑影像学检查不能揭示特殊病因。⑤无物质依赖或其他精神病史。

鉴别诊断应考虑以下疾病:

1.轻度认知障碍(mild cognitive impairment,
MCI) 一般仅有记忆力障碍,无其他认知功能障碍,如老年性健忘与遗忘。健忘是启动回忆困难,通过提示可使回忆得到改善;而遗忘是记忆过程受损,提示不能改善。

2.血管性痴呆(VD) AD与VD的鉴别要点见表\ref{tab5-1}。

\begin{table}[ht]
    \caption{AD与VD的鉴别}
    \label{tab5-1}
    \centering
    \begin{longtable}{lp{5cm}p{7cm}}
    \toprule
    & AD & VD \\
    \midrule
    起病 & 隐渐 & 较急,常有高血压史\\
    病程 & 进行性缓慢进展 & 波动或阶梯恶化\\
    早期症状 & 近记忆障碍 & 头昏、头痛、注意力不集中、工作效率下降\\
    精神症状 & 全面痴呆 & 以记忆障碍为主的局限性痴呆\\
    & 判断力、自知力丧失 & 判断力、自知力较好\\
    & 有人格改变 & 人格改变不明显 \\
    & 淡漠或欣快 & 情感脆弱或易怒 \\
    神经系统 & 早期多无局限性体征 & 局限症状和体征,如病理反射、偏瘫\\
    CT & 弥漫性脑皮质萎缩 & 多发梗塞,腔隙和软化灶和脑萎缩\\
    Hachinski评分 & <4 & >7\\
    \bottomrule
    \end{longtable}
  \end{table}

3.抑郁症 抑郁症患者可表现出痴呆症候群,但其起病有明确的时间,情绪低在前而后才觉得记忆力下降,主诉较多,强调自己有病,回答问题常答“不知道”,临床测验并不显示记忆功能缺陷,通过治疗,“痴呆”可痊愈。而老年性痴呆患者的起病往往说不清具体日期,患者否认有病,对周围事物漠不关心,智力测验尽管其努力想做好和答好,但成绩普遍较差,且痴呆进行加重;治疗效果不明显。故据上所述,则两者不难鉴别。

\subsubsection{治疗}

1.治疗原则 目前尚无特效治疗方法,主要为对症治疗。本病病因不明,目前尚无特效疗法,对轻症患者重点应加强心理支持与行为指导,使患者尽可能长期保持生活自理及人际交往能力。鼓励患者参加适当活动和锻炼,并辅以物理治疗、康复治疗、作业治疗、记忆和思维训练。重症患者应加强护理,注意营养、预防感染。

2.治疗方案

(1)促进脑部代谢药物:脑血流减少和糖代谢减退是AD重要的病理改变,使用扩血管药物增加脑血流及脑细胞代谢药可能改善症状或延缓疾病进展。常用银杏叶提取物制剂、脑复康和都可喜,以及钙离子拮抗剂尼莫地平等。

(2)改善认知功能药物:目前常用乙酰胆碱酯酶(AChE)抑制剂,抑制ACh降解并提高活性,改善神经递质传递功能。①多奈派齐(donepezil):是第二个被美国批准治疗AD的AChE抑制药,选择性与AChE结合,副作用明显减少,可每日用药一次,每天5~10mg,肝脏毒副作用低;②石杉碱甲(Huperzine
A):是我国从中草药千层塔中提取的AChE抑制剂,且对AChE有选择性,可改善认知功能,每天50~400μg,副作用小。

(3)神经保护性治疗:①抗氧化剂:维生素E和单胺氧化酶抑制剂丙炔苯丙胺可延缓AD进展,但仍有待研究;②非甾体类抗炎药:有可能防止和延缓AD发生。

(4)精神症状的治疗:①如患者有焦虑、激越、失眠症状,可考虑用短效或中效苯二氮䓬
类药,如阿普唑仑和劳拉西泮等,剂量应小且不宜长期应用。应注意过度镇静、嗜睡、言语不清、共济失调和步态不稳等副作用。②20%~50%的AD患者有抑郁症状,有时抑郁症状可能较轻且历时短暂。首先应予劝导,心理治疗、社会支持、环境改善即可缓解,必要时可使用抗抑郁药治疗。选择性5-羟色胺回收抑制剂(SSRIs),如帕罗西汀、舍曲林等,因抗胆碱能和心血管副作用较小,临床上使用较多。③部分患者会出现行为紊乱、激越、攻击性和幻觉与妄想,可给予小剂量新型抗精神病药如利培酮、奥氮平和喹硫平等治疗。

\subsection{血管性痴呆}

血管性痴呆(vascular dementia,
VD)是指由于脑血管疾病引起的,以痴呆为主要临床表现的脑功能衰退性疾病。VD是老年期痴呆的第二位原因,占老年期痴呆的20%,仅次于AD。多发生于60岁以上老人,男性多于女性。临床表现形式常与脑血管病损部位、大小及次数有关。主要分为两大类,一是痴呆症状,二是血管病脑损害的局灶性症状。VD起病急缓不一,多伴有高血压病,常在一次或多次卒中发作后起病,部分患者没有明显的卒中发作,表现为脑动脉硬化的早期表现。VD多呈阶梯式发展,可多次叠加,直至出现全面痴呆。

VD的临床表现主要包括:早期症状、局限性神经系统的症状及体征以及痴呆症状。

\subsubsection{早期症状}

见于VD缓慢起病者。潜伏期较长,一般不易被早期发现,可表现为脑衰弱综合征和轻度认知障碍(mild
cognitive impairment, MCI)。

1.脑衰弱综合征 可发生在脑动脉硬化的无症状期,持续时间可长达数年之久。表现为轻微的头晕、头痛,脑力劳动易疲劳,注意力不易集中,思维迟钝,工作效率降低,睡眠质量下降,失眠多梦。近期记忆力减退,常引起继发性焦虑。情感障碍为典型症状,表现为情绪不稳定,情感脆弱,严重时情感失禁,控制不住情感反应,无明显原因或为小事易伤感、易激惹、易怒,常为克制不住情感而感到苦恼。早期人格保持良好,一般理解、判断可保持,自知力存在。神经系统检查可能仅有眼底视网膜动脉硬化的征象。

2.MCI 并未达到痴呆的严重程度,主要表现为:近记忆损害,不能集中注意力,专注于某一项工作的能力下降;对新事物、新情况的理解和反应能力降低,解决问题的能力下降,参与社会的主动性下降;语言运用能力下降,表达及理解语言的能力下降,找不出合适的词汇表达自己的思想,或以许多较详细的叙述代替专门词汇。总体认知功能和日常生活能力正常。随着脑侧支循环的建立,脑动脉硬化症状的好转,MCI症状也可随之好转。

\subsubsection{局限性神经系统症状及体征}

VD急性或亚急性起病者常常为关键部位或大面积的病损引起。由于脑血管受损的部位不同,可出现不同的症状和体征。位于左大脑半球皮质的病变,可能有失语、失用、失读、失写、失算等症状;位于右大脑半球的皮质病变,可能有视空间障碍;位于皮质下神经核团及传导束的病变,可能出现相应的运动、感觉及锥体外系症状,患者也可出现强哭、强笑、假性球麻痹症状;前动脉闭塞累及额叶时,出现淡漠、木僵、自言自语;后动脉缺血颞、枕叶损害时可有幻觉、妄想、偏盲;丘脑损害有时症状较复杂,可有遗忘、淡漠、轻瘫、共济失调、多动等。

\subsubsection{痴呆}

痴呆的早期,核心症状是记忆障碍,以近记忆障碍为主,晚期出现远记忆障碍。与AD不同的是:虽然出现记忆障碍,但在相当长的时间内,自知力保持良好,知道自己易忘事情,常准备有备忘录。早期的另一症状是病理性赘述,说话啰嗦;有的患者提笔忘字,或有流利型失语现象。此期患者的日常生活自理能力、理解力、判断力以及人际交往和处理事物的礼仪、习惯均保持良好状态,人格保持良好,故称“网眼样痴呆”。

在痴呆的发展过程中,认知障碍加重,记忆力、定向力、智能障碍严重。患者在行为及人格方面也逐渐发生相应的改变,如变得自私、吝啬、收集废物、无目的徘徊;生活逐渐变得不能自理,不知饥饱、不知冷暖、外出走失、大小便不能自理,不认识亲人,达到全面痴呆。

\subsubsection{诊断标准}

1.必须符合痴呆的一般标准。

2.高级认知功能损害分布不均,某些受影响,某些相对保存。也许记忆受损极显著,而思维、推理和信息处理只受轻微影响。

3.至少表现出下列一种局灶性脑损伤的临床证据:①单侧肢体痉挛性力弱;②单侧腱反射亢进;③深反射亢进;④延髓性球麻痹。

4.根据病史检查或化验,有证据表明存在明显的脑血管病,并且有理由相信此病与痴呆的发生有病因学的联系。

\subsubsection{治疗}

VD是由脑血管病变引起的痴呆,治疗主要针对两个方面,一是对脑血管病的防治,二是改善脑血液循环,改善脑功能。

1.预防性治疗

(1)控制血糖、血脂,控制血压,尤其是收缩压控制在135~150mmHg。

(2)抗血小板聚集:研究发现,阿司匹林可降低脑卒中的发病率。脑血管病患者可给予口服肠溶阿司匹林50~100mg,每晚口服1次,或用其他抗血小板聚集药物。

2.改善脑微循环,增加脑血流量,提高氧利用率,可在一定程度上改善认知功能。

(1)麦角碱制剂:如喜得镇1mg每日3次,口服。

(2)钙离子拮抗药:如尼莫地平20~40mg每日3次,口服。

3.胆碱酯酶抑制药 研究发现VD患者也存在胆碱能系统的损害,且临床试验发现胆碱酯酶抑制剂能改善VD患者的认知功能。目前用于临床的药物有多奈哌齐、重酒石酸卡巴拉汀、石杉碱甲、加兰他敏等。

4.脑代谢促进药 可提高细胞对葡萄糖、氨基酸的利用率,增强记忆力,改善认知功能,如脑复康、脑复新、胞二磷胆碱等。

5.抗精神病药 精神症状明显时,可小剂量使用喹硫平、奋乃静等,症状控制后即可停药。

6.康复治疗 针对患者具体情况给予肢体功能、语言功能及认知功能方面的康复训练。

\subsection{颅脑外伤所致的精神障碍}

颅脑外伤所致的精神障碍系指颅脑遭受直接或间接外伤后,在脑组织损伤的基础上产生的各种精神障碍。精神障碍可在外伤后立即出现,也可在外伤后较长一段时间后出现。急性期精神障碍多系脑弥漫性损伤所致。慢性期精神障碍则与大脑神经细胞坏死、胶质细胞增生、瘢痕形成、囊肿等病变有关。脑组织受损伤越重,产生精神障碍的机会越大。除了器质性因素外,个体素质、社会心理因素在精神障碍的发生发展中也起一定作用。据统计,颅脑外伤后的存活者中,出现各种类型及程度的精神障碍者超过1/4。

\subsubsection{临床表现}

颅脑外伤引起的精神障碍可分为两大类,即脑外伤急性期的精神障碍及脑外伤慢性期的精神障碍。后者有的是从急性期延续保持下来的精神异常,有的是在急性期过去之后经过一段时间,才逐渐发展起来的。

1.急性期精神障碍

(1)脑震荡:脑震荡是头部外伤引起的急性脑功能障碍,是脑外伤的最轻形式。临床主要表现为意识障碍及近事遗忘。脑震荡患者的意识障碍较轻,持续时间较短,昏迷不超过半小时。在头部受到外力打击后有短暂的意识完全丧失,伴有面色苍白、瞳孔散大、对光反应及角膜反射迟钝或消失、脉搏细弱、呼吸缓慢、血压降低;然后再经过意识模糊阶段而逐渐醒转与恢复。意识恢复之后,患者对受伤前后的经历遗忘,其中对受伤前一段时间的经历遗忘称为逆行性遗忘,对受伤当时及稍后经历的遗忘称为顺行性遗忘。意识障碍及对外伤前后经历的遗忘是诊断为脑震荡的必备条件。脑震荡之后,患者会出现头痛、头昏、眩晕、恶心或呕吐、对声光线刺激敏感、情绪不稳、易疲劳、注意涣散、记忆力减退、自主神经功能失调、失眠、多梦等症状。神经系统检查一般没有阳性体征。这些症状通常在1~2周内逐渐消退。若迁延不愈或间隔一段时期后再发生则称脑震荡后综合征。

(2)谵妄:发生于较重的脑外伤,患者先有昏迷,在转成清醒的过程中,可能发生紧张恐惧、言语错误、失定向,并可伴发多种精神病性症状,如恐怖性幻视和错觉及片断的妄想、精神运动性兴奋等。持续数小时至数天不等,当患者意识恢复后,常不能回忆受伤前后的经过。有的患者虽然意识不清,但相对比较安静,显得精神萎靡、淡漠无欲、少言少动,继而意识完全恢复清醒。少数情况下,意识障碍可持续数月之久,然后转为痴呆状态。谵妄在亚急性硬膜下血肿的患者更常见。脑外伤谵妄的原因可能比较复杂,包括水、电解质和酸碱平衡紊乱,感染,药物的不良反应,缺血,缺氧等因素。

2.慢性精神障碍

(1)智能障碍:严重的脑外伤可引起智力受损,出现遗忘综合征,甚至痴呆。严重程度与脑外伤后遗忘(PTA)时间的长短有关。对于闭合性脑外伤的患者,如PTA时间在24小时以内,智力多能完全恢复;若PTA超过24小时,情况便不容乐观。年长者和优势半球受伤者发生智能障碍的机会较大。

(2)人格改变:患者的人格改变多伴有智能障碍,一般表现为情绪不稳、焦虑、抑郁、易激惹,甚至阵发暴怒,也可变得孤僻、冷漠、自我中心、丧失进取心等。如仅损害额叶,可出现如行为放纵等症状,但智力正常。人格改变也可以是患者对脑外伤及其后果的心理反应的表现。

(3)脑外伤后精神病性症状:部分头部外伤的患者经过一段时间后会出现精神病性症状,如精神分裂样症状与情感症状等。脑外伤可直接导致精神症状,也可对有精神病素质者起到诱因作用。另外,脑外伤及其后遗症对患者社会、心理的影响,也与精神病性症状的发生、发展有关。当然,有些患者的精神病和脑外伤并无直接关系。一般而言,脑外伤和精神症状出现相隔愈久,两者直接因果关系的几率便愈低。

(4)脑震荡后综合征(post concussional
syndrome):这是各种脑外伤后最普遍的慢性后遗症,主要表现为头痛、眩晕、注意力不集中、记忆减退、对声光敏感、疲乏、情绪不稳及失眠等。器质性与非器质性因素都可导致此综合征。虽然可能有器质性改变,但多数情况下躯体及实验室检查并无异常发现。该综合征与社会、心理因素有很大关系,如索赔等。

(5)癫痫
:约5%的闭合性脑外伤患者出现继发性癫痫
,约半数复杂性颅骨骨折患者可出现继发性癫痫
,开放性脑外伤患者中高达30%~50%。继发性癫痫
多出现在伤后1~3个月,不少患者出现在伤后5年或10年以上。外伤性癫痫
一般是由外伤后粘连、瘢痕及局限性萎缩所引起。闭合性脑外伤以精神运动性发作较多,开放性脑外伤则以局限性发作及大发作较多。

\subsubsection{治疗}

脑外伤急性阶段的治疗主要由神经外科处理。危险期过后,应积极治疗精神症状。对于幻觉、妄想、精神运动性兴奋等症状可给予苯二氮䓬
类药物或抗精神病药物口服或注射。对于外伤后神经症患者应避免不必要的身体检查和反复病史采集。支持性心理治疗、认知行为治疗配合适当的药物治疗(如抗抑郁药、抗焦虑药)都是可行的治疗方法。智能障碍患者应首先进行神经心理测试,再根据具体情况制订出康复训练计划。对人格改变的患者可尝试行为治疗,并帮助患者家属及同事正确认识及接纳患者的行为,尝试让他们参与治疗计划。如症状迁延不愈,应弄清是否存在社会心理因素。

\subsection{颅内感染所致的精神障碍}

虽然颅内感染的患者大多就诊于神经内科,精神科医师仍会遇到这类问题。颅内感染可分别位于蛛网膜下隙(脑膜炎)、脑实质(脑炎)或局限于脑或脑膜并形成包围区域(脑脓肿),但实际上损害很少呈局限性。

\subsubsection{病毒性脑炎}

以单纯疱疹病毒性脑炎最常见,一般发病无季节性与区域性,故常为散发性病毒性脑炎。

本病多为急性或亚急性起病。部分患者病前有上呼吸道感染或肠道感染史。急性起病者常有头痛,可伴脑膜刺激征,部分病例可有轻度或中度发热。精神症状可以是首发症状,也可是主要临床表现。精神运动性抑制症状较多见,表现为言语减少或缄默不语、情感淡漠、迟钝、呆板,甚至不饮不食呈木僵状态;也可表现为精神运动性兴奋,如躁动、言语增多、行为紊乱、欣快、无故哭泣或痴笑等;可有幻视、幻听、各种妄想等;记忆、计算、理解能力减退相当常见。多数患者在早期有意识障碍,表现为嗜睡、精神萎靡、神志恍惚、定向障碍、大小便失禁,甚至昏迷或呈去皮质状态。癫痫
发作相当常见,以全身性发作最多,有的以癫痫
持续状态为首发表现。可出现肢体上运动神经元性瘫痪、舞蹈样动作、扭转性斜颈、震颤等各种不随意运动。脑神经损害并不少见,如眼球运动障碍、面肌瘫痪、吞咽困难、舌下神经麻痹等。自主神经症状以多汗常见,伴有面部潮红、呼吸增快等。其他如瞳孔异常、视乳头水肿、眼球震颤、共济失调和感觉障碍均可见。

实验室检查可见血白细胞总数增高、脑脊液检查压力增高,白细胞和(或)蛋白质轻度增高,糖、氯化物正常。血和脑脊液IgG可增高。脑电图检查大多呈弥漫性改变或在弥漫性改变的基础上出现局灶性改变,且随临床症状好转恢复正常,对诊断本病有重要价值。本组疾病一般预后较好。重型病例的死亡率为22.4%~60%。一部分存活者遗留轻重不等的神经损害体征或高级神经活动障碍。本病复发率约为10%。抗病毒治疗如无环鸟苷(acyclovir)能有效降低脑炎患者(如单纯疱疹病毒性脑炎)的死亡率,但必须在患病初期使用。另外,积极的对症治疗(如降温、脱水)合并激素治疗和支持疗法(如补充液体、加强护理等)十分重要。

\subsubsection{脑膜炎}

1.化脓性脑膜炎 常见病原菌有脑膜炎双球菌、肺炎双球菌、链球菌、葡萄球菌、流感杆菌和大肠杆菌等。本病起病急,可表现为头痛、发热、呕吐、怕光、易激惹、癫痫
发作等。精神症状以急性脑器质性综合征为主,患者可有倦怠,表现为意识障碍,如嗜睡、昏睡甚至昏迷,可伴有幻觉、精神运动性兴奋等。颈部强直及克氏征(Kernig's
sign)阳性是诊断的重要依据。治疗以抗生素为主,配合对症治疗和支持疗法。

2.结核性脑膜炎 由结核杆菌侵入脑膜引起。在前驱期,精神症状主要表现为类神经症样表现,患者萎靡不振、易激惹、睡眠不稳等。在儿童中,可表现为以往安静的变得烦躁、无端尖叫、易哭,而活泼的儿童变得懒言少动和精神呆滞等。成年人以头痛多见,对声光刺激敏感,易激惹。由于隐匿起病,有时发热较轻微及颈部强直不明显,较易误诊。病情严重时可出现幻觉、妄想等精神病性症状及焦虑、抑郁等情感症状,晚期患者可出现记忆障碍,但大多可在接受治疗后复原。残留的精神症状包括认知障碍与人格改变。治疗主要是早期、联合、适量、规律、全程使用抗结核药物,同时对症予以抗精神病药物,症状控制后及时停药。

\subsubsection{脑脓肿}

本病主要由葡萄球菌、链球菌、肺炎双球菌或大肠杆菌等引起。可经血液或由头部感染灶直接蔓延入脑。

典型症状包括头痛、呕吐和谵妄。脓肿较大者可有颅内高压症状。部分脓肿可潜伏数月才出现病症。此期间患者常仅感到头痛、疲倦、食欲差、体重下降、便秘,偶有发冷、抑郁和易激惹。此外,不同部位的脓肿会有不同的症状,如额叶脓肿会表现为记忆障碍和人格改变,颞叶脓肿可造成言语障碍等。

脑脊液检查虽然对诊断有帮助,但由于颅内压较高,腰穿有一定风险,最好进行CT或MRI检查。

治疗以抗生素控制感染、消除颅内高压、治疗原发病灶为主,有时需考虑穿刺抽脓和行脓肿切除术。现代治疗能降低患者死亡率,但70%的患者康复后会出现癫痫
发作,所以病愈后应继续服用抗癫痫
药至少5年。

\subsubsection{克雅病}

克雅病(Creutzfeldt-Jacob disease,
CJD)是由朊病毒(prions)感染所致的中枢神经系统变性疾病,潜伏期长,表现为持续进展性痴呆,患者多在1~2年内死亡。病最早由Creutzfeldt(1920年)和Jacob(1921年)首先报道,因而定名为克雅病。CJD的年发病率约1/100万,以色列、利比亚人发病率最高,为一般人群的10~20倍。我国自20世纪80年代以来,已有数十例CJD病例报道。85%以上为散发性,病因和传染途径不十分清楚。

本病多见于50~70岁的中老年人,呈亚急性起病。前驱症状类似感冒,疲倦乏力,继而出现注意力不集中、精神涣散、烦躁、易激惹等精神行为异常;有些患者头疼头晕、肢体乏力行走不稳。可有视力减退、复视、视物变形乃至视幻觉。病情迅速进展,出现不自主动作、动作缓慢、肌强直、震颤、手足徐动和舞蹈症状,多数患者出现肌阵挛抽动,可被声光刺激诱发而反复发作,少数患者可有失神、癫痫
样抽搐发作。很快出现认知障碍、记忆力丧失、痴呆、定向障碍,甚至出现偏执妄想、虚构,少数兴奋、躁动。痴呆呈持续性进行性加重,最终卧床缄默不语。神经检查体征:可有复视、眼震、凝视、麻痹、轮替试验指鼻试验不能,步态蹒跚,共济失调,多数患者出现肌张力高、强直、震颤、锥体外束征和肢体无力的锥体束征等,最终呈木僵和醒状昏迷状态。

脑脊液可有非特异性改变。脑电图检查对经典型克雅病具有重要辅助价值,其特殊性改变:早期仅见散在α波减少,相继出现α波减少消失,出现Q波和δ波,病情进展则α波消失,出现棘波慢波综合和高幅三相波,最终脑电图背景电静息和周期性同步发放(PSD)。

本病预后不良,多数在发病后3~12个月死亡,绝大多数在2年内死亡。目前尚无特殊治疗方法,但仍需给予良好的护理和有效的对症治疗。抽搐、肌阵挛和兴奋躁动应给予积极治疗为宜。

\subsection{颅内肿瘤所致精神障碍}

颅内肿瘤所致精神障碍(mental disorders due to brain
tumor)是由于颅内肿瘤侵犯脑实质,压迫邻近的脑组织或脑血管,造成脑实质破坏或颅内压增高所致的精神障碍。有20%~40%的颅内肿瘤患者出现精神症状。

\subsubsection{病因和发病机制}

原发性颅内肿瘤的原因尚不清楚。有些与遗传因素有关,有些与物理、化学等环境因素有关,少数可能由局部受损引起。在胚胎发育过程中,有些细胞或组织可停止生长,残留于颅内,以后可发展形成颅内肿瘤,称为先天性颅内肿瘤,只占颅内肿瘤的一小部分。其虽然具有胚胎组织残留的特点,但这些组织的增殖仍可能是由于其他因素影响的结果。

\subsubsection{临床表现}

1.精神症状 肿瘤的性质、部位、生长速度、有无颅内高压及患者的个性特征等因素均可影响精神症状的产生与表现。在颅内肿瘤早期可表现为易激惹、易怒、焦虑、抑郁等神经症症状。生长缓慢的颅内肿瘤患者易出现人格改变、痴呆。生长较快的颅内肿瘤患者,易出现意识障碍。意识障碍可呈波动性,与颅内压改变有关。第三脑室带蒂的肿瘤常因堵塞导水管致颅内压急剧升高、意识障碍迅速恶化。但在改变体位时,导水管恢复通畅,颅内压下降,意识障碍减轻,甚至完全清醒。轻度意识障碍时,患者可表现为思维迟缓、贫乏或不连贯,行为紊乱。颅内肿瘤损害丘脑、乳头体等结构可出现遗忘综合征。枕叶肿瘤可出现原始性视幻,觉,颞叶肿瘤可出现复杂的幻视、幻听、幻嗅和幻味,顶叶肿瘤可产生幻触和运动性幻觉。不同脑区的肿瘤也可产生相同的幻觉。总之,在颅内肿瘤的整个病程中都可出现类似神经症、癔症、情感障碍、精神分裂症和偏执性精神病的表现。

2.局限性症状 大脑是一个有机的整体,精神活动往往是整体性的。某些脑区如额叶、颞叶、顶叶等受损时可各有一些特殊的表现,但大脑某一脑区的损害由于常影响邻近脑区,可引起与邻近脑区有关的症状。因此,脑部局限性损害产生的精神障碍并不一定具有特殊的形式。然而,特定脑区损害所产生的精神症状对于临床定位诊断具有一定的参考价值。

(1)额叶肿瘤:额叶肿瘤特别是前额叶肿瘤早期很少有明显的神经病学定位体征。但50%~80%的患者在肿瘤初期即可出现精神症状,双侧额前叶病变者可出现无欲-运动不能-意志缺乏综合征(apathetic-akinetic-abulic
syndrome),表现情感淡漠,兴趣缺乏,不注意仪表整洁,迟钝,漫不经心,想象力和思维能力减退,缺乏主动性,记忆力和智力受损,行动迟缓,表情迷惘、呆滞。优势半球额叶肿瘤引起智能损害者较非优势半球额叶肿瘤为多。情感障碍可有易激惹、抑郁、欣快或淡漠等。前额叶和额叶眶面的肿瘤,在早期可出现人格障碍,行为变得放纵、笨拙、稚气和戏谑,自制力缺乏等。神经系统症状可有尿失禁和步态障碍,可伴癫痫
大发作。左侧额下回后部肿瘤(Broca区)时发生运动性失语,强握反射和摸索征亦为常见的症状。

(2)颞叶肿瘤:约一半颞叶肿瘤患者会出现颞叶癫痫
。局限于颞叶的肿瘤可有两种形式的精神紊乱,包括沟回发作和发作间期的行为与情绪改变。前者多以幻味、幻嗅开始,继而出现梦样状态、不真实感、视物显大或显小,空间和时间感知障碍等。后者表现病态人格或原有的人格特征突出化和偏执倾向、易激惹、欣快、焦虑。常有暴发性情绪和暴力行为。多数颞叶受损患者可伴有智力缺损,小部分患者可出现类精神分裂症样症状,如幻觉、妄想等。

(3)顶叶肿瘤:顶叶肿瘤引起精神症状者远较额叶、颞叶肿瘤少。顶叶肿瘤在早期就出现神经病学症状与体征,因此一般不易误诊为功能性精神障碍。顶叶肿瘤损害左半球角回一带时的特征是Gerstman综合征,表现为手指失认、失算、失写和左右不分等,并可出现对自体和事物(如衣服)的左右都不能分辨,表现穿衣踌躇和困难,称为穿衣失用症(dressing
apraxia)。非优势半球侧的顶叶肿瘤则常见到视觉空间障碍和复杂的体像障碍,如单侧忽略现象和疾病缺失感亦即偏瘫侧失认症。情感障碍以焦虑、抑郁多见,亦可出现幻触和肢体运动性幻觉,在感觉缺失和感觉异常的基础上可产生非真实感、人格解体等。

(4)枕叶肿瘤:枕叶肿瘤较少产生精神症状,肿瘤早期即可出现头痛、同侧视野缺损、象限盲、偏盲或色觉失认等。枕叶肿瘤时产生癫痫
放电,可出现原始性视幻觉。当出现复杂视幻觉时,常提示肿瘤侵及顶颞区,较早出现颅内压增高,表现出相应的神经精神症状。

(5)间脑肿瘤:间脑深部中线结构和第3脑室肿瘤的典型症状是出现遗忘-虚构综合征。第3脑室肿瘤易引起脑脊液循环的慢性阻塞,大脑皮质萎缩,出现进行性痴呆。嗜睡为间脑肿瘤的特征性症状之一,常呈发作性,几乎无法抗拒。发作性意识障碍在肿瘤早期神经系统损害的体征出现之前就可出现。意识障碍发作时出现肌肉过度强直。间脑后部和中脑上部肿瘤可出现运动不能性缄默症,患者意识清楚,沉默不语,卧床不动,双眼能注视检查者,对于痛觉刺激有反射性回缩动作。此外,患者可出现人格改变和阵发性周期性精神障碍,情绪波动性大,时而情绪低落,时而高涨。

(6)胼胝体肿瘤:胼胝体肿瘤较多引起精神障碍,胼胝体咀部肿瘤有92%出现精神症状,压部为89%,中部为57%。常见的精神症状为智能障碍与情绪障碍,而且症状在肿瘤生长初期即可出现。特别是胼胝体前部的肿瘤在未出现神经系统体征、颅内压增高之前即已发生明显的精神衰退。

(7)垂体肿瘤:垂体肿瘤时的精神症状可由于垂体本身的损害、继发性内分泌障碍和垂体肿瘤扩展到蝶鞍区以外等因素所引起,因此,垂体肿瘤时的精神症状是多种多样的。

(8)天幕下肿瘤:天幕下肿瘤比天幕上肿瘤较少产生精神障碍。可表现全面智能障碍,其程度跟颅内压成正比,也可出现与颅内压增高无关的发作性缄默、意识模糊直至意识丧失。有时产生情绪障碍、人格改变及幻觉、妄想等精神病性症状。

\subsubsection{诊断}

精神症状本身一般对脑肿瘤无诊断或定位价值。临床诊断以局灶性神经体征或局灶性癫痫
及颅内压增高征象为主要依据。除细致的精神检查外,应详细询问病史,认真进行神经系统检查,以免忽略可能存在的神经系统体征。对无原因头痛、部分性癫痫
、成年后首次发生癫痫
、伴有阳性神经系统体征的全身性癫痫
、颅内压增高症、认知功能进行性减退、某个特定脑功能(例如:言语、空间定向)的进行性损害、颅内某个特定解剖部位的局限神经损害、各种神经内分泌紊乱,脑神经麻痹或进行性视力减退,婴幼儿反复发作呕吐及头围增大,肿瘤患者突然出现神经症状等均应考虑除外颅内肿瘤。

临床疑有脑瘤时可选择以下方法做进一步检查:头颅CT、MRI、腰椎穿刺、生化测定基础血清泌乳素水平、血清和脑脊液中的甲胎蛋白(AFP)和人绒毛膜促性腺激素(HCG)等和开颅或立体导向技术下进行肿瘤活检,有助于明确诊断。

\subsubsection{治疗}

确诊颅内肿瘤应及时进行手术治疗。对于颅内压升高的患者应及时控制颅内压。对于不适宜手术的患者,可通过放疗或化疗抑制肿瘤的生长和扩散。不论肿瘤的类型或预后如何,医生均应给患者和其家属高度关怀,对焦虑、抑郁、易激惹、木僵等症状,应给以适当的抗精神药物,遵循最低有效剂量原则,且不宜久服。兴奋躁动常为颅内压增高即将出现昏迷的前兆,在使用镇静药物的同时,还应以脱水剂降低颅内压。

\subsection{梅毒所致精神障碍}

20世纪初期,梅毒所致精神障碍很普遍。随着抗生素的应用,梅毒发病率显著下降。自20世纪末期以来,梅毒再次流行,且常与HIV合并感染。由于梅毒的神经精神症状多样化、无特异性,因此很难根据临床症状作出正确的诊断。

\subsubsection{临床表现}

一期梅毒常表现为局部溃疡,可伴有焦虑、紧张、沮丧等情绪反应,无严重的精神症状。在初次感染后6~24周,进入二期梅毒,中枢神经系统可能受累,常见有疲乏、厌食和体重减轻,伴有多个器官系统感染的症状,可出现梅毒性脑膜炎,表现为头痛、颈项强直、恶心、呕吐和局灶性神经系统体征。

通常在首次感染后5年内出现三期梅毒的临床表现,包括良性梅毒瘤、心血管和神经梅毒。约10%未经治疗的患者可出现神经性梅毒。除脑膜刺激征外,还可表现为淡漠、易激惹和情绪不稳及人格改变、记忆和注意障碍等。无症状性神经梅毒是指缺乏临床表现,但脑脊液检查阳性的梅毒患者。在初次感染后4~7年,可发生典型的亚急性脑膜血管性梅毒。其临床表现比脑膜梅毒更严重,常伴有妄想、易激惹、人格改变和认知功能缺损等精神症状,随病情进一步恶化,可发展为痴呆。脊髓痨(tabes
dorsalis)通常发生在初次感染梅毒后20~25年,最具特征性的神经系统症状是脊髓后部脱髓鞘和脊髓背侧根部的萎缩,有的可出现性功能障碍、尿失禁、剧痛、全身闪电样疼痛和躯干运动失调等。60%的患者可出现阿罗瞳孔(Argyll-Robertson's
pupils),即瞳孔对光反射迟钝或消失而调节辐辏反射存在。以上描述的任何精神症状可与神经系统的综合征同时出现。

麻痹性痴呆(general paralysis of the
insane),通常在感染后15~20年出现。典型病程常表现为隐匿起病,初时出现类似神经衰弱的症状,如头晕、头痛、睡眠障碍、易兴奋、易疲劳、易激惹、注意力不集中和记忆力减退等,此期称为麻痹前类神经衰弱期。随着病情进展,可出现个性和智力改变,自私、缺乏责任感,记忆力逐渐减退,可出现夸大妄想和嫉妒妄想,内容荒谬、怪诞、愚蠢可笑,情绪多不稳定,易激惹或情感脆弱和强制性哭笑。晚期痴呆日益加重,即使很简单的问题也不能回答,言语零碎、片段,对家人不能辨认,情感淡漠,本能活动相对亢进。

\subsubsection{诊断}

根据冶游史、早期梅毒感染史、年龄为30~50岁、神经系统体征、精神症状,尤其是人格改变和智能障碍,结合血清学检查如荧光梅毒螺旋体抗体吸附试验(FTA-ABS
test)和梅毒螺旋体停动试验(TPI
test)阳性,诊断便可成立。血清和脑脊液试验阴性者,则不支持神经梅毒的诊断。本病应注意与酒精中毒性精神病、精神分裂症、情感障碍、神经症等鉴别。详细的体格检查以及血清学检查有助于与上述疾病区别。

\subsubsection{治疗}

驱梅毒治疗首选青霉素。普鲁卡因青霉素G能够消除绝大多数麻痹性痴呆病例脑部的梅毒感染。总剂量为600万~1200万U,每日80万U肌内注射,2~3周内完成,总量亦可达1800万~2400万U。第1疗程青霉素治疗应住院施行,因为有5%~10%的患者有发生赫氏反应(Herxheimer
reaction)的危险。可在治疗前3天服用泼尼松5mg,每日3次,以预防这种反应发生。无论何时发生赫氏反应,都应立即停用青霉素,并口服泼尼松。当反应消失后,可再度使用青霉素,亦可用红霉素代替,每日2g,连服30天为1个疗程。第1个疗程结束2~6个月后,应复查脑脊液。如果细胞数和蛋白含量仍不正常,应立即进行第2个疗程。如果脑脊液已恢复正常,即使症状没有好转,也无需再驱梅毒治疗,因为症状好转要有个过程。在治疗后6~12个月需复查脑脊液,然后每年检查1次,至少连续5年。复查脑脊液主要是观察细胞数目变化,如果细胞数超过5×10\textsuperscript{6}
/L是再治疗的指征。即使治疗成功,FTA-ABS试验可继续保持阳性多年,故FTA-ABS试验不适用于病情监测。

对激惹、兴奋、幻觉、妄想可用抗精神病药物;对抑郁症状可用抗抑郁药;对癫痫发作应使用抗癫痫
药物对症处理。出现神经系统症状可酌情采用针灸、理疗、按摩和功能训练,并给予必要的生活照料。

\subsection{癫痫所致精神障碍}

癫痫 所致精神障碍(mental disorders
due to
epilepsy)是指一组反复发作的脑异常放电导致的精神障碍。由于累及的部位和病理生理改变不同,导致的精神症状各异。本病可分为发作性精神障碍和持续性精神障碍两类。前者为一定时间内的感觉、知觉、记忆、思维等障碍,心境恶劣,精神运动性发作,或短暂精神分裂症样发作,发作具有突然性、短暂性及反复发作的特点;后者为分裂症样障碍、人格改变或智能损害等。癫痫
见于各个年龄组,病因不一。癫痫
的患病率为0.5%~2%。约10%的人在一生中可能曾有过1次癫痫
发作。

\subsubsection{病因和发病机制}

癫痫
发作可分为原发性(特发性)癫痫
和继发性(症状性)癫痫
。症状性癫痫
是脑部疾病或多种全身性疾病的临床表现。原发性癫痫
是指目前原因不明确的一类癫痫
。癫痫
的病因很多,概括起来可分为遗传、感染、中毒(重金属、酒精等)、脑肿瘤、脑外伤、脑血管病、脑变性疾病、代谢障碍(钙、镁、钠等代谢异常)、药物(利舍平、氯氮平、氯丙嗪等)等几类。

\subsubsection{临床表现}

1.发作前精神障碍 表现为先兆或前驱症状。先兆是一种部分发作,在癫痫
发作前出现,通常只有数秒,很少超过一分钟。不同部位的发作会有不同的表现,但同一患者每次发作前的先兆往往相同。

前驱症状发生在癫痫
发作前数小时至数天,尤以儿童较多见。患者表现为易激惹、紧张、失眠、坐立不安,甚至极度抑郁,5%的颞叶癫痫
患者可出现幻嗅先兆,症状通常随着癫痫
发作而终止。

2.发作时精神障碍

(1)自动症(epileptic
automatisms):自动症是指发作时或发作刚结束时出现的意识混浊状态。此时患者仍可维持一定的姿势和肌张力,在无意识中完成简单或复杂的动作和行为。

自动症主要与颞叶自发性电活动有关,有时额叶、扣带回皮质等处放电也可产生自动症。80%患者的自动症为时少于5分钟,少数可长达1小时。

自动症发作前常有先兆,如头晕、流涎、咀嚼动作、躯体感觉异常和陌生感等。发作时突然变得目瞪口呆,意识模糊,无意识地重复动作如咀嚼、咂嘴等,偶可完成较复杂的技术性工作。事后患者对这段时间发生的事情完全遗忘。

(2)神游症(fugue):比自动症少见,历时可达数小时、数天甚至数周。意识障碍程度较轻,异常行为较复杂,对周围环境有一定感知能力,亦能做出相应的反应。患者表现为无目的地外出漫游。患者可出远门,亦能从事协调的活动,如购物、简单交谈。发作后遗忘或回忆困难。

(3)朦胧状态(twilight
state):发作突然,通常持续1至数小时,有时可长达1周以上。患者表现为意识障碍,伴有情感和感知觉障碍,如恐惧、愤怒等,也可表现情感淡漠、思维及动作迟缓等。

3.发作后精神障碍 癫痫
发作后可出现意识障碍,定向障碍,自动症、朦胧状态,或产生短暂的偏执、幻觉等症状,通常持续数分钟至数小时不等。

4.发作间精神障碍 癫痫
患者在多年发作后,可出现慢性癫痫
性分裂样精神病(chronic epileptic schizophreniform
psychosis),表现为在意识清晰时出现幻听、思维联想障碍、思维云集和被害妄想等类似精神分裂症偏执型的症状群。约50%的颞叶癫痫
患者可出现人格改变,表现为固执、自我中心、病理性赘述、病理性激情和病理性心境恶劣等。少数癫痫
患者会出现记忆减退、注意困难和判断能力下降,并伴有行为障碍。临床上也可见到以焦虑、抑郁为主的情感症状等。值得注意的是,癫痫
患者的自杀率是常人的4~5倍,因此应注意预防患者自杀。

\subsubsection{诊断}

本病的诊断主要根据既往癫痫
发作史以及发作性的精神病症状群,故详细询问病史非常重要,患者可能因意识障碍不能提供详细的发作情况,故应尽可能向知情者了解发作的特点及伴随情况,特别要注意有无局限性发作的表现。90%的癫痫
患者有脑电异常。脑电结果必须结合临床及其他检查进行综合分析。脑电图正常并不能完全排除癫痫
。CT和MRI能探测到脑结构或形态的损害,功能性脑影像技术如SPECT和PET可反映脑局部血流及代谢异常,对癫痫
的定位诊断有帮助。

\subsubsection{治疗}

治疗癫痫
的一般原则是:尽可能单一用药,定期进行血药浓度监测。

在抗癫痫
治疗的基础上,根据精神障碍的特点选用精神药物。精神运动性发作首选卡马西平控制发作。卡马西平对点燃效应引起的边缘系统电活动有选择性抑制作用,能有效控制发作。对发作性的行为症状如冲动、攻击等可用苯二氮䓬
类药物治疗。值得注意的是,许多抗精神病药物(如氯丙嗪、氯氮平等)及三环、四环类抗抑郁药可降低抽搐阈值,增加癫痫
发作的危险。对发作间歇期出现的精神分裂样症状宜用致癫痫
作用弱的抗精神病药如硫利达嗪(甲硫达嗪)、丁酰苯类等。癫痫
伴发的抑郁障碍,可选用致癫痫
作用弱的5-羟色胺再摄取抑制剂等治疗,但往往没有被及时诊断。对复杂部分性发作,特别是颞叶癫痫
伴精神病性症状时,可用电休克治疗,人工诱发的大发作可使精神症状解除。

\subsection{HIV/AIDS所致精神障碍}

获得性免疫缺陷综合征(acquired immune deficiency syndrome,
AIDS)是由人类免疫缺陷病毒(HIV)感染所致,1981年在美国首次发现和确认。HIV能直接侵犯中枢神经系统,主要杀死人体的$\text{CD}^+_4$
T细胞,使机体对危害生命的机会性感染的易感性增加,病人可患罕见的细胞免疫缺陷病。HIV具有亲神经性,可直接侵犯中枢神经系统,导致HIV脑病,神经病理学改变可有神经元减少、多核巨细胞、小胶质细胞结、弥散性星形细胞增生、白质空泡形成及脱髓鞘等。本病主要是基底核和皮层下白质受累,而大脑皮层灰质影响较少。

HIV感染者易出现各种不同的精神障碍,可分为原发性或继发性。原发性并发症是由于HIV直接侵犯中枢神经系统或HIV破坏免疫系统所致;继发性并发症是由机会性感染、肿瘤、HIV感染导致的脑血管疾病和药物治疗的副作用等引起。患者的心理、社会因素亦可影响精神症状的发生、发展。

\subsubsection{临床表现}

1.轻度认知功能障碍 患者早期表现乏力、倦怠、丧失兴趣、性欲减退,以后逐渐发展为近记忆障碍、注意力集中困难、反应迟缓和轻度认知功能缺陷,但日常生活功能并无严重损害。

2.HIV相关痴呆(HIV associated dementia,
HAD) 痴呆是AIDS常见的临床表现,约占70%。通常HAD涵盖从轻微认知-运动障碍到艾滋病痴呆复合征(AIDS-dementia
complex,
ADC)的两极。ADC多出现于疾病终末期,起病潜隐,早期表现注意力不集中、记忆力下降等,逐渐出现认知、行为和运动功能的衰退,直至全面性痴呆、二便失禁、缄默和瘫痪等。可出现幻觉、妄想等精神病性症状。HIV感染伴发痴呆是预后差的标志,50%~75%的患者在出现痴呆的6个月内死亡。

3.谵妄 在整个AIDS病程中都有可能出现谵妄,终末期患者中谵妄发生率可达30%~40%,病因包括脑部HIV感染、治疗艾滋病药物副作用、继发性感染(尤其是卡氏肺囊虫肺炎)、水电解质紊乱等。

4.其他 患者可因患有此疾病而出现心因性反应,表现为焦虑、抑郁及睡眠障碍或创伤后应激障碍,严重者可出现自杀行为,也可能出现躁狂样和类分裂样症状。

\subsubsection{诊断依据}

1.主要根据病史,如静脉吸毒史、冶游史、男同性恋者,结合体格检查和病毒免疫学检查中的阳性发现。

2.意识障碍,尤以谵妄状态更为多见。

3.精神症状多继发于HIV感染之后,随着病情的变化而起伏。

\subsubsection{治疗}

1.积极治疗原发病 可采用“鸡尾酒疗法”(高效抗逆转录病毒治疗,HAART),如混合使用齐多夫定、拉米夫定和依非韦伦这三种药物抗病毒。伴有PCP者可使用复方磺胺甲噁唑。

2.对精神症状的对症治疗 可根据具体症状有针对性地使用小剂量抗精神病药物,如利醅酮,抗抑郁剂如SSRIs类以及改善睡眠的苯二氮䓬
类药物等。


