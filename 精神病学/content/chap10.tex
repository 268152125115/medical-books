\chapter{心 境 障 碍}

心境障碍又称情感性精神障碍(mood disorders, affective
disorder),是以显著而持久的情感或心境改变为主要特征的一组疾病。通常以情感高涨或低落为主要的或原发的症状,伴有相应的认知和行为改变,可有精神病症状,如幻觉、妄想或紧张综合征。间歇期精神状态基本正常,有复发倾向,预后一般较好,部分可有残留症状或转为慢性。

根据中国精神障碍分类方案与诊断标准第3版(CCMD-3),心境障碍包括躁狂发作、双相障碍、抑郁症和持续性心境障碍等类型。双相障碍具有躁狂和抑郁交替发作的临床特征,既往称躁狂抑郁性精神病双相型。躁狂症或抑郁症是指仅有躁狂或抑郁发作,临床上单纯躁狂发作少见。

\subsection{流行病学}

由于疾病概念、诊断标准、流行病学调查方法的不同,国内外患病率报道差异较大。国内在20世纪50、70和80年代做了大规模的流行学调查研究工作,1982年采用国际上较先进的科学方法,统一的诊断标准,同一工具、方法,调查结果心境障碍时点患病率为0.037%,终生患病率0.076%。1993年在上述的12个地区中的7个地区进行了第二次调查,心境障碍时点患病率为0.052%,终生患病率为0.083%,有增高趋势。近日中国疾病预防控制中心精神卫生中心提供的数据显示,近10年来,我国心境障碍(或情感障碍)的患病率在1.38‰~8.6‰,高于1993年的调查结果。

欧美报道为3.6%~25%,新西兰报道为0.24%。美国(2006年)双相心境障碍I型患病率为1.6%(男性1.6%,女性1.7%)。好发年龄在20~30岁。

\subsection{病因及发病机制}

本病的病因尚不清楚。可能与遗传、生化、生理、内分泌及心理社会等因素有关。

\subsubsection{遗传与心理社会因素}

1.遗传因素

(1)家系研究:心境障碍有明显的家族聚集性,双相情感障碍先证者的一级亲属心境障碍的患病率为一般人群的10~30倍。血缘关系越近,患病率越高,有早期遗传倾向(anticipation),即发病年龄逐代提早,疾病严重程度逐代增加。父母一方患病其子女患病率为30%~35%,双方患病其子女患病率为70%~75%。Chang(2000年)报道,约90%的双相障碍儿童伴有注意缺陷与多动障碍(ADHD)。

(2)双生子与寄养子研究:研究提示,单卵双生子(MZ)的同病率是65%,双卵双生子(DZ)约14%。有研究发现,患有心境障碍的寄养子,其亲生父母患病率为31%,而其养父母只有12%,这些研究说明遗传因素起着重要的作用。

但是,基因遗传的机制尚不清楚。有作者报道双相障碍的易感基因位于11p15.5,还有研究报道双相障碍与X染色体上遗传标记连锁,这些报道均未得到众多学者的重复和证实。

(3)分子遗传学研究:心境障碍的疾病基因或易感基因尚在研究中。分子遗传学研究涉及多条染色体和基因。

2.心理社会因素 一般认为,人的高级神经活动类型(气质)和认知模式是抑郁症发病的基础,负性社会生活事件是抑郁症发病的诱因,而社会支持系统是影响抑郁症发生发展及预后的一个重要因素。

(1)心理学因素:研究表明,抑制型气质及内向人格具有抑郁症的易感性。美国临床心理学家贝克(Beck)提出认知模型,他认为抑郁症患者的认知模型包括两个层次,即浅层的负性自动想法(negative
automatic thoughts)和深层的功能失调性假设(underlying dysfunctional
assumptions
schemas)。徐俊冕等曾对我国抑郁症患者的认知特征进行了研究,发现抑郁症患者比正常人及精神分裂症患者有更多的负性想法和功能失调性态度。抑郁越严重,负性自动想法出现越频繁;随着抑郁缓解,自动想法减少至正常。说明贝克的两层次认知模式在抑郁症的发生发展中起着重要作用。

(2)负性社会生活事件:负性社会生活事件对于抑郁症的发病起到诱因作用,有研究表明,负性生活事件(丧偶、离婚、婚姻不和谐、失业、严重躯体疾病、家庭成员去世)发生的6个月内,抑郁症发病危险系数增加6倍。Kessler(1997)对抑郁症的负性生活事件进行研究,结果显示,负性生活事件越多,性质越严重,抑郁症发病率就越高,抑郁症状也就越严重。

(3)社会支持系统:当遭遇负性生活事件后,如果能及时接受社会支持系统的物质精神安慰,可以减轻或消除负性生活事件对其的影响,削弱或消除其生理、心理反应,避免抑郁症的发生。即使抑郁症已经发生,良好的社会支持系统仍会对抑郁症的康复起到促进作用,并能够防止复发。

\subsubsection{神经生化因素}

心境障碍的发生不能归因于单一因素,但是,脑内神经递质的功能特别是去甲肾上腺素、5-羟色胺以及多巴胺等的功能障碍与心境障碍的发生有密切关系。

1.去甲肾上腺素能(NE)假说 NE能学说是由Schildkraut于1965年首先提出,他认为抑郁症是由于脑内NE浓度降低所造成的,其依据为:使用耗竭NE的药物利血平,可使NE耗竭,发生抑郁症;抑郁症时,尿中MHPG(脑内NE的主要产物)下降,说明NE合成、释放减少;而长期使用能增加NE的药物,如三环抗抑郁药、单胺氧化酶抑制剂可使尿中MHPG升高,NE生成增加,从而改善抑郁症状。Sulser(1975)又进一步提出,突触后β受体超敏,反馈使突触前α\textsubscript{2}
受体超敏,是导致NE生成释放减少,突触间隙NE数量下降的重要机理。

2.5-羟色胺(5-HT)假说 由Coppen于1965年首先提出,认为抑郁症是由于中枢神经系统5-羟色胺释放减少,突触间隙的含量下降而引起的。部分三环类抗抑郁药(TCA\textsubscript{s}
)、5-羟色胺再摄取抑制剂(SSRI\textsubscript{s}
)可阻滞5-HT的回吸收,有抗抑郁作用。单胺氧化酶抑制剂(MAIO)能抑制5-HT的降解,具有抗抑郁作用。利舍平(利血平)可耗竭5-HT,导致抑郁;选择性5-HT耗竭剂(对氯苯丙胺)可逆转三环类抗抑郁药单胺氧化酶抑制剂的抗抑郁作用,可导致抑郁。随着对5-HT系统研究的深入,发现5-HT受体不同亚型其作用有别,如新一代抗抑郁药5-HT\textsubscript{2}
拮抗剂奈法唑酮(nefazodone)和5-HT\textsubscript{1}
激动剂伊沙匹隆(ipsapirone)可能对5-HT受体系统有另外的作用。

3.多巴胺(DA)假说 研究发现,抑郁症患者脑内多巴胺功能减退,而躁狂症增高。降低多巴胺浓度的药物(如利血平)和多巴胺活性降低的疾病(如帕金森病)均可导致抑郁;多巴胺受体激动剂(溴隐亭、普拉克索等)有抗抑郁作用。新型的抗抑郁药安非他酮(bupropin)主要是阻断多巴胺再摄取。

4.γ-氨基丁酸(GABA)假说 GABA是中枢神经系统主要的抑制性神经递质,临床研究发现多种抗癫痫
药如卡马西平、丙戊酸钠具有抗躁狂和抗抑郁作用。它们的药理作用与脑内GABA含量的调控有关。有研究发现,双相障碍患者血浆和脑脊液中GABA水平降低。

\subsubsection{脑结构与脑功能的异常}

1.脑结构研究 CT研究发现心境障碍患者脑室较正常对照组扩大。脑室扩大的发生率为12.5%~42%。

2.功能性影像学检查 PET、SPECT研究发现,心境障碍患者前额叶皮层、杏仁核、海马、纹状体等相应脑区功能异常。易于发作患者fMRI研究在前额叶、边缘皮层、扣带回、下丘脑、海马等脑区激活有差异。

\subsubsection{神经内分泌功能异常}

1.下丘脑-垂体-肾上腺轴(HPA) 抑郁患者血浆皮质醇水平增高和尿皮质醇及代谢产物17-羟皮质类固醇排出增多,可能为下丘脑中NE能神经元抑制下丘脑分泌促皮质激素释放因子(CRF),从而控制与调节血中皮质醇水平。此外,研究发现抑郁症患者不仅血浆皮质醇水平增高,而且分泌的昼夜节律也有改变,无晚间自发性皮质醇分泌抑制。约40%的患者在晚上11时服用地塞米松1mg后,次日下午4时及晚上11时测定血浆皮质醇高于138nmol/L,即地塞米松不能抑制皮质醇分泌。

2.下丘脑-垂体-甲状腺轴(HPT) 甲状腺功能亢进可发生躁狂,甲状腺功能减退可出现抑郁,表明甲状腺功能状况与心境障碍关系密切。有研究报道,重性抑郁患者促甲状腺激素(TSH)对甲状腺激素(TRH)反应迟缓。

3.脑电生理研究 躁狂与抑郁患者都有睡眠障碍。多导睡眠图(polysomnogram)发现,抑郁症睡眠潜伏期延长,总睡眠时间减少,觉醒增多,早醒,睡眠效率下降,δ睡眠减少和睡眠时相转换增多。其特征性改变是快眼运动(REM)睡眠潜伏期缩短,只有40~50分钟(正常成人平均为70~90分钟),因此认为具有特征性生物学标志;另一特征性改变是REM密度增加,其敏感性为35%~95%,特异性为62%~100%。

\subsection{临床表现}

\subsubsection{躁狂发作}

躁狂发作的典型症状是情感高涨、思维奔逸和动作行为增多。

1.情绪高涨 病人主观体验特别喜悦,自我感觉良好,兴高采烈,洋洋自得,讲话时眉飞色舞,喜笑颜开,精神涣散,甚至感到天空格外晴朗,周围事物的色彩格外精彩,自己也感到格外地快乐和幸福。轻者以愉快欢乐、热情为主,颇具“感染力”,常博得周围人的共鸣;有时以愤怒、易激惹、敌意等为临床特征,他们可为一些小事暴跳如雷,在极度激惹时可产生破坏及攻击行为,但很快转怒为喜或赔礼道歉。

2.思维奔逸 主要表现为思维联想过程明显增快,言语增多,语速加快,自觉思维非常敏捷,话多滔滔不绝,或难以打断患者的话题。患者联想迅速,一个概念接着一个概念,典型的联想加快,出现“音联”、“意联”,严重时出现言语跳跃,类似破裂样思维。患者在言谈过程中随着环境的变化而转换话题,称为“随境转移”。患者自我评价过高,表现为高傲自大,目空一切,自命不凡,盛气凌人。

3.活动增多 患者精力旺盛,活动明显增多,时刻忙碌不停,但做事却有头无尾,有始无终;好管闲事,对自己的行为缺乏正确的判断,随心所欲,不顾后果;任意挥霍钱财,十分慷慨,随意将物品送给他人,社交活动增多,随便请客,行为轻浮,好接近异性。有时手舞足蹈,如演戏一般。对各项活动均感兴趣,主动积极参与。病情严重时,自我控制能力下降,举止粗鲁,甚至有冲动毁物行为。

4.注意转移 患者注意力易受周围环境影响,但不能持久,所以不断变换话题和活动内容。

5.严重病人可出现精神病性症状,可出现夸大观念或妄想,自我评价过高,认为自己是最能干的人、最有财富的人等,能力过人,目空一切,自命不凡,自我炫耀,趾高气扬,摆阔气;但其内容多与心境一致,一般持续时间不长。

6.躯体症状 由于患者自我感觉良好,故很少有躯体不适主诉。常表现为面色红润,两眼有神,体格检查可发现瞳孔轻度扩大,心率加快。如果患者极度兴奋,体力过度消耗,可引起脱水,体重减轻。患者食欲增加,性欲亢进,睡眠需要普遍减少。

7.其他 不同亚型及年龄其症状表现亦有差异,如轻躁狂,症状一般较轻,虽具有临床症状的特点,但尚能进行交谈,生活可以自理。躁狂发作极为严重者,则可有轻度意识障碍和严重的精神运动性兴奋,行为紊乱无目的指向,伴有冲动行为,可有短暂、片段的幻觉和妄想,思维的不连贯等,管理困难,有时出现躯体消耗性衰竭,称为谵妄性躁狂。老年躁狂症与成人相似,但活动多不明显,情感易激惹,以夸大为主。儿童躁狂症与成人不同,以行为障碍为突出,情感表达显得单调,而行为表现活动增多,要求多,无故捣乱,逃学,常具有攻击破坏行为。症状大多不够典型。

【\textbf{典型病例}
】 {张某,男性,27岁,已婚,工人。因话多、自吹、不眠3周由家人及同事送到医院急诊。3周来患者睡眠时间明显较少,整夜忙着打扫住处,购买了电脑、音响设备,而患者并不会使用这些电器,自吹已有3个女性与他发生了性关系,挥霍、用钱大方,近期经常饮酒。既往无饮酒习惯,无特殊药物使用史。精神检查:神志清楚,戴着橘黄色的帽子,两只袜子不匹配,兴高采烈,说话的声音高、语速快,话多,难以被打断,称自己是“伟人”,必定能成功,对被送到医院很生气,认为他们是“嫉妒他能与异性成功交往”而送他来医院。未引出幻觉,无自杀念头。}

\subsubsection{抑郁发作}

抑郁发作是最常见的发作形式。几乎所有的心境障碍的患者在病程中出现抑郁发作。临床上是以情感低落、思维迟缓、意志活动减退和躯体症状为主,严重者可出现幻觉、妄想等精神病性症状。

1.情绪低落 情绪症状是抑郁症的最显著、最普遍的症状。抑郁症病人的情绪症状主要包括两个方面:抑郁心情和兴趣的消失。主要表现为显著的情绪低落,郁郁寡欢,愁眉苦脸,长吁短叹,悲伤、焦虑、易怒。抑郁症病人的生活中,似乎充满了无助和绝望。典型的病例,抑郁情绪常有晨重夜轻的波动,抑郁症状在早晨最明显,患者往往觉得几乎没有力量从床上起来,随着一天的推移,情绪会慢慢好转一些,晚上的心情相对最好。另外一个情绪症状是兴趣的消失:抑郁症患者往往体会不到生活的乐趣,过去感兴趣的事物,喜欢参加的活动,现在一点也引不起他们的兴趣。部分患者可伴有焦虑、激越症状,特别是更年期和老年抑郁症患者更明显。时常表现“心烦”、“烦闷”。有的患者在诊视时强作笑颜,虽然内心抑郁,而表情上加以掩饰,应引起重视。

2.思维迟缓和消极 患者思维联想的速度缓慢,自觉“脑子好像是生了锈”、“脑子好像是涂了一层糨糊一样开动不了”,表现为主动语言减少,语速减慢,声音低沉,思考问题困难,工作和学习能力下降。抑郁症患者对自己的评价总是消极的,自罪、自责,夸大自己的缺点,缩小自己的优点,表现为认知上的不合逻辑性和不切实际性。这种消极的思维,为他眼中的自己和未来都蒙上了一层厚厚的灰色。一旦有挫折发生,抑郁症患者就会把全部责任归咎于他们自己。极度抑郁的患者认为他们应该为自己的“罪恶”而受到惩罚,主要是感到无兴趣,无价值,无望,无助,无用感,总把事物看成暗淡的,自知思考困难,工作能力下降,反省过去感到内疚、自责、自罪,有时夸大自己的“罪孽”,形成罪恶妄想。

3.意志活动减退 患者意志活动明显受到抑制。临床表现为行动缓慢,生活被动,不想做事,感到全身乏力,明明知道自己应该做什么,但缺乏动力,显得疏懒、力不从心;不想与周围人交往、接触,或整日卧床,不想去上班,不愿外出;兴致缺乏,对原本感兴趣的活动和业余爱好也不再感兴趣;严重患者吃、喝、个人卫生都不顾,甚至发展为不语、不动、不食,可达木僵状态,称为“抑郁性木僵”,但仔细检查仍可流露出痛苦抑郁的情绪体验。少有自发的活动,讲话语音低沉,思维活动过程缓慢,严重者可以呆坐不语或卧床不起、不食,陷于木僵状态。

严重抑郁发作的患者常伴有消极自杀的观念或行为。病人由于消极悲观的情绪以及自责自罪的思想而萌发绝望的念头,认为“自己活着是多余的,是累赘”,认为“结束自己的生命是一种解脱”等,这是抑郁症最危险的症状,应提高警惕。抑郁症患者的自杀已成为第3位的死亡原因,自杀行为不仅会发生在疾病严重阶段,也常发生在抑郁的缓解期,特别伴有明显焦虑的病人,应特别引起警惕。男性的自杀率比女性高4倍。

4.躯体或生物学症状 很常见。常有食欲减退、体重减轻、睡眠障碍、性功能低下和心境昼夜波动等生物学症状。典型的睡眠障碍是早醒,比平时早2~3小时,醒后不复入睡,陷入悲哀气氛中。有的病人表现为入睡困难,睡眠不深,少数患者表现为食欲增强,体重增加。

5.其他 抑郁发作时可出现焦虑、人格解体、现实解体及强迫症状。老年抑郁症患者除有突出的焦虑烦躁情绪外,还可表现为易激惹、精神运动性迟滞和躯体不适,较年轻患者更为明显。因思维联想明显迟滞以及记忆力减退,可出现明显的认知功能损害症状,类似痴呆表现,如记忆力、计算力、理解和判断能力下降,称之为“抑郁性假性痴呆”。产后抑郁症也叫产后忧郁症,是妇女在生产孩子之后由于生理和心理因素造成的抑郁症,症状有紧张、疑虑、内疚、恐惧等,极少数严重的会有绝望、离家出走、伤害孩子或自杀的想法和行动。

【\textbf{典型病例}
】 {程某,女性,37岁。因自觉生活难以应付、眠差、食欲减退一个月就诊。一个月前丈夫抛弃她和孩子而去,从此以后患者出现睡眠障碍,每晚仅睡3~4小时,凌晨3~4点醒后不能再入睡,对生活没有信心,不知这样的生活怎样活下去,觉得自己能力很差,对任何东西都没有兴趣,食欲下降,体重减少了15斤。精神检查:神志清楚,衣冠整洁,情绪沮丧,烦躁不安,近2周有时听到有声音说她“不好”,经常有自杀念头但没有自杀行为,没有自杀是因为孩子还需要她照顾。未引出幻觉,定向力正常。}

\subsubsection{双相障碍}

双相障碍(bipolar
disorder)临床特点是反复(至少2次)出现心境的明显改变,有时表现为心境高涨、精力充沛和活动增多,有时表现为心境低落、精力减退和活动减少。发作间隙期通常能完全缓解。

躁狂症状和抑郁症状可在一次发作中同时出现,患者既有躁狂又有抑郁的表现。如患者活动明显增多,讲话滔滔不绝,同时又有消极抑郁的情绪。躁狂症状和抑郁症状也可快速转换,因日而异,甚至因时而异。混合发作时,临床上躁狂症状和抑郁症状均不典型,易误诊,一般持续时间较短,多数转入躁狂相或抑郁相。

\subsubsection{环性心境障碍}

环性心境障碍(cyclothymia)是一种轻性的双相障碍,反复出现心境高涨与低落,但程度较轻,均不符合躁狂或抑郁的诊断标准。躁狂发作时表现为愉悦、活跃和积极,而转为抑郁时不再乐观自信,变得沉闷痛苦,随后进入心境相对正常的间歇期。一般新景象对正常的间隙期可长达数月,其主要特征是持续性心境不稳定。这种心境的波动与生活应激无明显关系,与患者的人格特征有密切关系,既往称之为“环性人格”。

\subsubsection{心境恶劣障碍}

心境恶劣障碍(dysthymic
disorder)指一种以持久的心境低落为主的轻度抑郁,从不出现躁狂,常伴有焦虑、躯体不适和睡眠障碍。患者有求治要求,但无明显的精神运动性抑制或精神病性症状,生活不受影响。许多心境恶劣状态始于儿童时期,而且普遍认为是一种带有抑郁素质的人格障碍。另外一些人认为心境恶劣其实是焦虑障碍。患者抑郁常持续2年以上,其间无长时间的完全缓解,如有缓解,一般不超过2个月。CCMD-2-R称之为“抑郁性神经症”。

\subsubsection{儿童、少年抑郁}

儿童和少年抑郁症表现学校恐怖,逃学,出走,哭闹,违拗,发脾气,厌倦不安,孤独退缩,对周围不感兴趣等,有的也以行为障碍突出。20%~30%的成年双向障碍患者初发年龄在20岁以前。儿童或青少年期表现有抑郁的往往隐藏着成年后发生双相障碍的可能。

\subsection{病程与预后}

\subsubsection{病程}

大多数为急性或亚急性起病,好发季节为春末夏初。躁狂发作年龄在30岁左右,抑郁发作年龄较晚。躁狂发作的自然病程一般持续数周到6个月,平均为3个月左右。而抑郁发作持续时间通常6个月至1年,平均为9个月。有人认为反复发作性躁狂,每次发作持续时间几乎相等。躁狂和抑郁的发作没有固定的顺序,可连续多次躁狂发作后有一次抑郁发作。也可能反过来,或躁狂和抑郁交替发作,但很少有混合发作发展成躁狂发作。发作间歇期症状可完全缓解,也有20%~30%的双相Ⅰ型和15%的双相Ⅱ型患者持续存在情绪不稳。间歇期的长短不一,可从数月到数年。随着年龄增长和发作次数的增加,正常间歇期有逐渐缩短的趋势。

\subsubsection{预后}

虽然双相障碍可有自限性,但如果不加治疗,复发几乎是不可避免的。未经治疗者中,50%的患者能够在首次发作后的第一年内自发缓解,其余的在以后的岁月里缓解的不到1/3,终生复发率达90%以上,约有15%的患者自杀死亡,10%转为慢性状态。在应用锂盐治疗双相障碍以前,患者一生平均有9次发作。抑郁症发作持续时间相对躁狂要长,平均病程为6~8个月。一般认为发作次数越多,病情越严重。伴有精神病性症状者病程持续时间较长。有人对心境障碍随访20年,发现单相抑郁平均发作次数为4~6次,双相抑郁为7~9次。预后可能与遗传、人格特点、躯体疾病、社会支持、治疗充分与否等因素有关。

\subsection{诊断与鉴别诊断}

心境障碍的诊断应根据病史、临床症状、病程及躯体、神经系统检查和化验检查,常规进行甲状腺功能的检查,以排除躯体疾病、药源性因素引起的继发性心境障碍。由于制定了各种检查工具和诊断标准,使临床检查、诊断标准化、计量化,结果更具有客观性、可比性和可重复性。

常用的量表有Bech-Refaelsen躁狂量表、Hamilton抑郁量表和Zung抑郁自评量表。各量表的项目包括主要和常见症状。每个症状按其严重度和出现频率分成若干等级。评分结果可以反映疾病性质和严重程度,如Hamilton抑郁量表17项,总分少于6分者,一般认为无抑郁症状,17~24分可以肯定为抑郁,大于25分为严重抑郁。

\subsubsection{诊断要点}

心境障碍诊断标准很多,当今世界上影响较大的两大诊断系统,为国际通用的《国际疾病分类手册第10版》(ICD-10)和美国《精神障碍诊断统计手册第4版》(DSM-Ⅳ)。中国精神障碍分类与诊断标准第3版(CCMD-3)于2001年中华精神科学会通过,并在全国范围内执行。

1.临床诊断特征 躁狂发作以显著而持久的情感高涨为主要表现,伴有思维奔逸、活动增多、夸大观念或夸大妄想、睡眠需要减少、性欲亢进、食欲增加等。抑郁发作以显著而持久的情绪低落为主要表现,伴有相应的思维迟缓、兴趣缺乏、快感缺失、意志活动减少、精神运动性迟缓或激越、自责自罪、继发性幻觉、自杀观念或行为、早醒、食欲减退、体重下降、性欲减退、抑郁心境的晨重晚轻的节律改变。根据患者症状特征的严重程度不同、持续时间长短分为不同的临床亚型。

2.病程特征 多数为发作性病程,发作间隙期精神状态可恢复至病前水平。既往有类似的发作,或病程中出现躁狂抑郁的交替发作,对诊断均有帮助。

3.躯体和神经系统检查以及实验室检查一般无阳性发现,影像学检查、神经电生理检查、量表评估可供参考。家族中特别是一级亲属有较高的同类疾病的阳性家族史。

\subsubsection{鉴别诊断}

1.继发性情感性精神障碍 脑器质性疾病、躯体疾病、某些药物和精神活性物质等均可引起继发性心境障碍,与原发性心境障碍的鉴别要点:①前者有明确的器质性疾病或有服用某种药物或使用精神活性物质史,体格检查有阳性体征,实验室及其他辅助检查有相应指标的改变。②前者可出现意识障碍、遗忘综合征及智能障碍,后者除谵妄性躁狂发作外,无意识障碍、记忆障碍及智能障碍。③器质性和药源性心境障碍的症状随原发疾病的病情消长而波动,原发疾病好转,或在有关药物停用后,情感症状相应好转或消失。④器质性疾病所致躁狂发作,其心境高涨的症状不明显,而表现为易激惹、焦虑和紧张,如甲状腺功能亢进;或表现为欣快、易激惹、情绪不稳,如脑动脉硬化,均与躁狂症有别。⑤前者既往无心境障碍的发作史,而后者可有类似的发作史。

2.精神分裂症 精神分裂症的早期常出现精神运动性兴奋,或出现抑郁症状,或在精神分裂症恢复期出现抑郁,类似于躁狂或抑郁发作,其鉴别要点为:①精神分裂症表现的精神运动兴奋或抑郁症状,其情感症状并非是原发症状,而是以思维障碍和情感淡漠为原发症状;心境障碍以心境高涨或低落为原发症状。②精神分裂症患者的思维、情感和意志行为等精神活动是不协调的,常表现言语零乱、思维不连贯、情感不协调,行为怪异;急性躁狂发作可表现为易激惹,精神病性症状,亦可出现不协调的精神运动性兴奋,但是情感症状的背景中出现,若患者过去有类似的发作而缓解良好,或用情绪稳定剂治疗有效,应考虑诊断为躁狂发作。③精神分裂症的病程多数发作进展或持续进展,缓解期常有残留精神症状或人格的缺损;而心境障碍是间歇发作性病程,间歇期基本正常。④病前性格、家族遗传史、预后和药物治疗的反应等均可有助于鉴别。

3.心因性精神障碍 心因性精神障碍中创伤后应激障碍常伴有抑郁,鉴别要点是:①前者常在严重的、灾难性的、对生命有威胁的创伤性事件,如被强奸、地震、被虐待后出现,以焦虑、痛苦、易激惹为主,情绪波动性大,无晨重夕轻的节律改变。②前者精神运动性迟缓不明显,睡眠障碍多为入睡困难,有与创伤有关的恶梦、梦魇,特别是睡梦中尖叫;而抑郁症有明显的精神运动性迟缓,睡眠障碍多为早醒。③前者常反复体验到创伤事件,有反复闯入性回忆,易惊。

\subsection{治疗}

\subsubsection{双相障碍的治疗}

双相障碍几乎终生循环反复发作,其发作的频率较抑郁障碍高,主要的治疗方法是心境稳定剂。也有主张在使用心境稳定剂的基础上联用抗抑郁药物,一旦抑郁症状缓解,可继续予心境稳定剂维持治疗,同时逐渐减少、停止抗抑郁药,避免转为躁狂。

1.常用的心境稳定剂 心境稳定剂是指对躁狂或抑郁发作具有治疗和预防复发作用,且不会引起躁狂与抑郁转相,或导致发作变得频繁的药物。不同的心境稳定剂对不同类型的作用不一致,如果单一的药物无效可加用第二种心境稳定剂。心境稳定剂长期维持治疗认为可防止发生新的发作并减少可能导致的更严重的发作。

(1)锂盐(lithium
carbonate):锂盐是最常用的心境稳定剂,最常用的药物是碳酸锂。国内自20世纪70年代广泛用于临床以来,主要用于躁狂症的急性发作及缓解期的维持治疗,有效率达80%,对躁狂的复发也有预防作用。

一般采用渐加法,始剂量常为每日0.5g,分2~3次口服,可在5~7天加至治疗量。急性躁狂的治疗量一般为每日900~2000mg,维持量每日500~1200mg。老人及儿童剂量应适当减少。治疗前肝、肾功能检查,甲状腺功能及其他生化检查是必要的。因锂盐的治疗指数低,治疗量与中毒量接近,因而需动态进行血锂浓度监测,每周1次,1个月后4~6周测定1次。如患者出现食欲不佳、发热、腹泻等怀疑有中毒可能时应随时检测。急性期治疗的有效血锂浓度为0.8~1.2mmol/L,维持治疗血锂浓度为0.5~0.8mmol/L。1.4mmol/L为有效治疗浓度的上限。

治疗急性躁狂发作时,在锂盐起效以前,为了控制患者的高度兴奋症状以防止患者衰竭,可合并抗精神病药或电抽搐治疗。当兴奋症状被控制后,逐渐减少、停止抗精神病药,继续使用锂盐,控制症状预防复发。

锂盐的作用机理、不良反应及处理参阅躯体治疗章节。

(2)抗癫痫
药(anticonvulsant):抗癫痫
药用于难治性的双相障碍、混合性发作或锂盐治疗无效或副作用明显难以耐受者。此类药物主要有酰胺米嗪(卡马西平)和丙戊酸盐。

卡马西平的剂量每日300~1200mg,分2~3次口服,治疗剂量血药浓度为6~12μg/ml;预防剂量是每日300~600mg,血药浓度为6μg/ml。

治疗初期常见的不良反应有眩晕、头痛、嗜睡、共济失调,可有皮疹,偶有白细胞减少、血小板减少、再生障碍性贫血等。严重可致剥脱性皮炎。超大剂量可导致精神错乱、谵妄,甚至昏迷,此时采用洗胃、服用活性炭和支持治疗。突然停药可致癫痫
大发作,所以必要时逐渐减量再停用。

丙戊酸盐:有丙戊酸钠、丙戊酸镁。丙戊酸钠也采用渐加法给药,开始每日400~600mg,分次服用,每隔2~3天增加200mg;治疗剂量每日800~1800mg,治疗血药浓度为50~100μg/ml。常见不良反应有厌食、恶心、腹泻等,偶见震颤、复视、运动失调,白细胞、血小板减少偶有发生。治疗期间定期检查血常规和肝功能,一旦出现异常立即停用。

如果疗效不好,可考虑换用或加用拉莫三嗪、托吡酯、加巴喷丁。

(3)抗精神病药:最近有研究显示,不典型抗精神病药物氯氮平、利培酮、奥氮平、喹硫平等,其长期效果如何尚待进一步研究。

2.电抽搐治疗和改良的电抽搐治疗 电抽搐治疗和改良的电抽搐治疗对急性重症躁狂发作的极度兴奋躁动、对锂盐治疗无效或不能耐受的患者有一定治疗作用。起效迅速,可单独应用或合并药物治疗,一般隔日一次,4~10次为1个疗程。合并药物治疗的患者应适当减少药物剂量。

\subsubsection{抑郁症的治疗}

抑郁症的治疗主要有:药物治疗、电休克治疗(ECT)、经颅磁刺激治疗(TMS)、迷走神经刺激治疗(VNS)、睡眠剥夺治疗、光照治疗、精神外科手术治疗以及心理治疗等。

1.药物治疗 抗抑郁药主要作用于5-羟色胺(5-HT)和去甲肾上腺素(NE)等神经递质。主要有单胺氧化酶抑制剂(MAOI)、三环类抗抑郁药(TCA)、四环类抗抑郁药、选择性5-羟色胺再摄取抑制剂(SSRI)、去甲肾上腺素再摄取抑制剂(NRI)、5-HT/NE再摄取抑制剂(SNRI)、NE和特异性5-HT能抗抑郁药(NaSSA)等。

(1)选择性5-羟色胺再摄取抑制剂(SSRI):共有5种,即氟西汀(Fluoxetine)、舍曲林(Sertraline)、帕罗西汀(Paroxetine)、西酞普兰(Citalopram)及氟伏沙明(Fluvoxamine)。SSRI的出现是抑郁症药物治疗的里程碑,其选择性地作用于5-HT系统,与TCA疗效相当,但不良反应明显减少,服用方便,治疗依从性好。已被作为各型抑郁症的一线治疗。氟西汀及帕罗西汀通常最小有效剂量为20mg/d,舍曲林及氟伏沙明为50mg/d,西酞普兰为40mg/d。

SSRI不良反应较少而轻微,尤其是抗胆碱能及心脏的不良反应少。常见的不良反应有恶心、呕吐、厌食、便秘、口干、震颤、失眠、焦虑等。在SSRI的代谢中,CYP酶系起重要作用,对合并其他躯体疾病的患者治疗时注意与药物的相互作用。

(2)三环类及四环类抗抑郁药(TCAs):TCAs主要包括丙米嗪、阿米替林、多塞平、氯米帕明等。曾认为是一类经典而有效的抗抑郁药。在新型抗抑郁药未应用于临床之前,TCAs常被作为一线抗抑郁药。该类药物引起的不良反应涉及面广、程度重、过量时易中毒致死,而且患者对药物的耐受性及依从性差。四环类抗抑郁药有马普替林、米安舍林。马普替林开始剂量为每次25mg,每日2次,最大剂量为每日100~225mg,老人剂量宜小。米安舍林剂量为每日30~60mg。不良反应类似三环类药物,但程度较轻。

(3)5-HT/NE再摄取抑制剂(SNRIs):代表药物为文拉法辛缓释剂(venlafaxine-XR,怡诺思)。初始剂量每日50~75mg,分3次服用,最大剂量可达每日200mg。本品不良反应较少。

(4)米氮平(mirtazapine,瑞美隆):属于NaSSAs,在对NE和5-HT的调节方面不同于其他抗抑郁药,它不阻断神经递质的再摄取,而阻断突触前NE能神经元末梢的肾上腺素自身受体和对突触前5-HT能神经元末梢有抑制作用的α受体,可同时增加NE和5-HT的释放,但不阻断神经递质的再摄取。治疗起始剂量应为15mg/d,逐渐加大剂量至获最佳疗效。有效剂量通常为15~45mg。

(5)MAOIs:MAOIs也是较早应用于临床的一类抗抑郁药,由于这类药物不良反应较多,临床应用受到限制。近几年研制出选择性A-MAOI,药物不良反应明显轻于无选择性的MAOIs,代表药物为吗氯贝胺。该药达峰时间为1~2小时,消除半衰期为1~2小时。MAOIs不能与SSRIs同时应用,两药的使用间隔时间至少为2周。

(6)盐酸安非他酮:安非他酮是去甲肾上腺素、5-羟色胺、多巴胺再摄取的弱抑制剂,对单胺氧化酶没有抑制作用。用药时从小剂量开始,起始剂量为一次75mg(1片),一日2次;服用3天后,逐渐增大剂量到一次75mg(1片),一日3次;以后逐渐增加至每日300mg的维持剂量。3日内剂量增加不超过一日100mg。最大剂量不超过一次150mg(2片),一日3次,用药间隔不得少于6小时。常见口干、失眠、头晕、头痛/偏头痛、易怒、恶心/呕吐、便秘、水肿、皮疹、尿频等不良反应。偶见肝功能异常、胃炎、幻觉、食欲减退和体重改变等。贫血、共济失调等罕见。

2.电抽搐治疗及改良的电抽搐治疗 对于有严重的消极自杀言行或抑郁性木僵的患者,电抽搐治疗应是首选的治疗;对使用抗抑郁药物无效的患者也可采用电抽搐治疗。6~10次为1个疗程。电抽搐治疗后仍需用药物维持治疗。

3.心理治疗 心理治疗主要有认知治疗、行为治疗、动力性心理治疗(dynamic
psychotherapy)、人际关系治疗(interpersonal
therapy),还有家庭治疗(family therapy)、集体治疗(group
therapy)和支持性心理治疗以及一些疗法的整合等。

4.难治性抑郁症的治疗 目前对难治性抑郁症的界定标准尚不统一,较多的学者认为,经过至少2种作用机制不同的抗抑郁药足够疗程和剂量的治疗仍无足够的疗效,可定为难治性抑郁症。还有学者认为,联用其他抗抑郁药后仍疗效欠佳者,才可定为难治性抑郁症。治疗建议有:①增加抗抑郁药的剂量,至最大治疗量的上限,在加药过程中应注意药物的不良反应;②抗抑郁药物合并增效,抗抑郁药与锂盐或甲状腺素或丁螺环酮或与非典型抗精神病药合用;③两种不同类型或不同药理机制的抗抑郁药的联用;④抗抑郁药合并电抽搐治疗,或采取生物-心理-社会综合干预措施。

\subsection{预防复发}

若第一次抑郁发作且经过药物治疗临床缓解的患者,药物的维持治疗时间多数学者认为需6月~1年;若为第二次发作,主张维持治疗3~5年;若为第三次发作,应长期维持治疗。维持治疗的药物剂量认为应与治疗剂量相同,已有学者认为可稍低于治疗剂量。

双相障碍的复发率明显高于单相抑郁障碍,锂盐对情感障碍复发有较好的预防作用,其有效维持治疗血锂浓度应在0.8~1.0mEq/L。每隔半年进行一次甲状腺功能检查,因为锂盐具有导致甲状腺功能低下的倾向。最近研究显示,丙戊酸钠对双相情感障碍的维持治疗效果与锂盐相当,而不良反应较少。

心理治疗和社会支持系统对预防心境障碍的复发也有非常重要的作用,应尽可能解除患者的心理负担和压力,帮助患者解决生活和工作中的实际困难,提高患者应对能力,并积极为其创造良好的环境,以防复发。