\chapter{应激相关障碍}

\section{概  述}

应激(stress)是指由于外在的社会心理因素刺激下所引起的人体的生理、行为及主观反应。临床常引用应激源(stressor)或应激性生活事件(stressful
life
event),简称为生活事件来表示各种超过一定阈值的刺激。当个体遭遇生活事件所引起的心理、生理和行为改变称为应激反应。

\subsection{应激源}

凡是引起应激反应的各种因素都可以成为应激源,包括物理性、化学性、生物性和心理社会性的等。归纳起来可以分为躯体性应激源和心理性应激源。作为心理社会应激源,根据个体参加社会活动的范围和人际关系可分为:

\subsubsection{恋爱婚姻与家庭内部问题}

恋爱中的重大精神刺激包括恋爱中争吵、失恋和遭抛弃;婚姻中夫妻双方长期分居、婚外情、夫妻发生纠纷、离婚、配偶患病、死亡等。家庭内部问题主要有子女不听话,学习成绩不理想,升学、就业困难等。此外,家庭中有需要长期照顾的老人或患者,生活困难或有财产纠纷等均是家庭长期慢性应激的来源。

\subsubsection{学校与职业场所问题}

青少年学业负担过重,家长对学生学习成绩要求过高,考试与升学失败,与老师、同学关系紧张。工作中竞争压力过大,与同事或上下级人际关系紧张,工作中出现差错事故受到批评处分,面临失业、下岗、退休等都是职业场所常见的应激源。

\subsubsection{社会生活变化与个人特殊遭遇}

社会生活变化可包括自然灾害与人为灾害,如地震、洪水、海啸、车祸、空难、战争、经济萧条等。个人特殊遭遇可构成明显的应激源,如意外致残,突患严重的躯体疾病。经济破产、事业失败、被强奸、遗弃、虐待、绑架等。

\subsection{应激程度评估}

对心理社会应激程度的客观评定,主要根据心理应激强度或造成精神紧张的程度进行推断。目前评价不同生活事件造成的精神紧张的程度,常用“社会再适应评定量表”(social
readjustment rating scale,
SRRS)进行评估。将每一生活事件引起生活变化的程度或社会再适应需做出努力的大小称为生活变故单位(life
change unit,
LCU),用来反映心理应激的程度。根据生活事件量表评分,一年总分超过300分者易患精神疾病。

\subsection{刺激与反应}

一般来说,刺激与反应之间呈正相关曲线,即弱刺激引起弱反应,强刺激引起强反应。而当刺激强度增大到某种程度时,反应强度达到了最高值。超过这个界限的刺激强度称为超强刺激,此时随着刺激强度增加,机体出现的反应强度不再增加,甚至反而下降。因而当人类遭遇应激事件时,随着损失增大,情感反应愈加明显,但极大的损失却会引起情感麻木状态,而不是更强烈的情感反应。应激反应的强度既和外界刺激的强度有关,又与机体内部情况和其他外界环境因素相关,即与人们的个性、既往生活经验、机体状态、社会支持等多种因素有关。当机体处于疲劳、饥饿、感染、消耗、孕期、分娩期或因手术、外伤、药物依赖而削弱的机能状态下,其对精神刺激的耐受性降低。既往生活经验影响当前的应激能力,既往经历过类似应激源并曾有良好适应的人,对该应激源有良好的耐受性。而既往经历的应激源并曾有适应不良现象或应付失败的情况时,可能会对应激源无法耐受。此外,个性在心理应激过程中也起到很重要的作用。

\subsubsection{心理应激过程}

机体对外界各种刺激,首先对其性质进行辨认、评估,将其划分为有利的、无关的和有害的刺激,然后做出相应的情绪与行为反应。对有利的刺激出现阳性情绪和趋向行为。对有害刺激出现阴性情绪和回避行为,或采用相应的心理应付机制。

\subsubsection{生理应激过程}

面临应激源,机体可产生生理、生化、内分泌、代谢、免疫等一系列的应激反应。

1.应激反应 首先是大脑皮质,特别是前额叶(认知脑)的参与,随后还有边缘系统(情绪脑)和视丘下部的参与。它一方面和自主神经系统高级调节中枢联系;另一方面通过神经体液途径来调节垂体等内分泌腺体的活动。

2.神经内分泌 发生改变,应激反应可产生交感神经和副交感神经兴奋。下丘脑释放促肾上腺皮质激素释放因子(CRF),作用于垂体前叶促使肾上腺皮质激素(ACTH)释放。ACTH作用于肾上腺皮质,刺激糖皮质激素的合成与释放。

3.应激 可影响免疫功能。人类研究发现,应激状态下,可出现体液免疫因子IL-2含量降低、IL-6增高、IgM增高等变化。有研究表明,失业、分居、离婚、丧偶等也可引起体内免疫功能降低。

\section{应激相关障碍}

应激相关障碍(stress related
disorders)指一组主要由心理、社会(环境)因素引起异常心理反应,导致的精神障碍,也称反应性精神障碍(reactive
mental
disorder)。由于本病病程短暂,有的可自行缓解。据国内12个地区精神病流行病学调查,反应性精神病的总患病率为0.068%(1982年)。本病发病年龄以青壮年为多,也可见于少年和老年,性别无明显差别。本病的共同特点是:①心理社会应激应是引起本组精神障碍的直接原因,起主导作用。②临床主要表现与精神刺激因素密切相关。③病因消除或改变环境后,精神症状相继消失。④预后良好,无人格方面的缺陷等。

\subsection{病因}

1.急剧或持久的精神创伤或生活事件,即心理社会应激(psychosocial
stress)。这些应激源大致可为以下几类:

(1)灾难性事件:如严重的自然灾害,山洪暴发、地震、火灾、风暴;特大车祸或意外事故等严重威胁生命财产的重大灾难。

(2)生活事件:主要是家庭环境因素的改变等一些较严重的不良应激因素,如亲人突然亡故、家庭角色的变化、离退休、失业、破产、身患绝症、被强暴等重大精神打击。

(3)长期压抑紧张的心理状态:如家庭成员间不和睦、人际关系紧张、工作不顺利、事业挫折或工作紧张无法胜任等,造成长期的心理压抑和紧张。

(4)环境的突然变化:如被拘禁、隔离、移民或旅途中等。

2.个体易感素质 临床发现,并非所有受到严重应激刺激的人都出现精神症状,这表明了个体的易感素质起了重要的作用。这主要表现在个体对于应激源的认知评价以及个体对事物的体验和采取的应付方式上。同时,个体对精神刺激的耐受性和感受性等,这些均与个体的个性特点、神经类型、价值观、伦理道德观等有一定的关系。

3.躯体状况 尤其是躯体疾病、妊娠、分娩、术后及过度疲劳等情况下,可以使大脑代偿功能减弱,这样对精神刺激的耐受性降低,易出现精神障碍。

在分析病因时,应关注以下几个方面:①应激源的强度:一般剧烈而严重的精神刺激容易产生应激反应;②刺激持续时间;③个人对应激源的态度,包括对应激源的分析、判断、处理方法及应付能力等;④对精神刺激的感受性和耐受性;⑤精神刺激与个人切身利益的关系;⑥社会支持系统的作用等。

\subsection{精神应激的心身反应特点}

个体面对精神应激时的心理生理反应包括三个方面:心理行为反应、躯体反应和运用心理应付机制来减轻应激的心理反应。

1.应激的心理行为反应 焦虑、恐惧、抑郁是心理行为反应的核心症状。面对压力、危险或无力应付的事件时,个体常表现为焦虑和恐惧;而面对丧失(如失去亲人、财产、健康、爱情、自尊等)时,则常表现为抑郁。焦虑和抑郁常常并存,因为精神应激事件(如亲人的突然死亡)往往使个体既觉得无力应付又有丧失感。严重的应激障碍者还可能出现行为的紊乱,如精神运动性兴奋或精神运动性抑制。少数患者可能出现心因性妄想或轻度意识障碍,但持续时间很短。

2.应激的躯体反应 躯体反应主要表现为植物神经系统紊乱、躯体功能下降,出现各种躯体不适的症状,如心悸、血压升高、出汗、口干、肌肉紧张或抖动等。而面对分离或丧失性事件时,躯体表现为无力、躯体功能的下降和活动的减少。

3.心理应付方式和防御机制 个体运用心理应付方式和心理防御机制来缓解过强的应激体验,努力维持正常的心理和生理功能。心理应付方式的运用是有意识的活动,与个体的经验有关;心理防御机制则是无意识的精神过程,往往被个体不自觉地运用来缓解心理应激体验。

(1)心理应付方式:积极的心理应付方式包括求助、积极解决问题,适应和面对现实,发泄或倾诉负性情感,回避痛苦情境,接受或推卸责任等;不良的心理应付方式包括过度使用烟、酒、镇静药等成瘾物质,攻击行为,自伤或自杀等。这些心理应付方法都可以缓解短暂的应激反应,但不良的应付方法对于持久的应激反应不但无效,反而损害健康。例如患了严重的躯体疾病后,在心理上持久地回避接受这一事实,就不利于及时的求医和获得有效治疗。因此,一个人不但要有良好的应付方式,而且还要能根据情境的变换及时地调整应付策略。

(2)心理防御机制:心理防御机制首先是由弗洛伊德(S.
Freud)提出的,后来被其后继者进一步完善。心理防御机制有很多形式,在应激状态下常见的有压抑、否认、投射等,较少应用的有合理化、升华、认同等。心理防御机制是一个无意识的过程,个体应用这些防御机制时常常是不由自主的,且与理智的思考无关。

\subsection{发病机制}

在医学模式从单纯的生物医学模式向生物-心理-社会医学模式的转变过程中,对应激的研究有了很大的发展。研究发现面临应激源、处于应激状态下的机体,在体内可发生一系列的生理、生化、内分泌及免疫机制的变化。

1.个体的易感性在相同的应激源作用下,只有部分人表现出精神障碍,可以推断其发病与个体的易感性和应对能力有关。个体的心理脆弱,即使应激源的强度不大,也可能引起相关的精神障碍。但对于何种人格特征的人易患该病尚难定论。

2.生物学机制大量实验与观察证实,机体在应激状态时,可通过中枢神经系统、神经生化系统、神经内分泌系统、免疫系统等相互作用,影响机体内环境平衡,导致各器官功能障碍、组织结构的变化和精神障碍等。

实验证明,当人体处于恐惧、紧张和愤怒状态时,整个交感神经系统被激活,引起交感神经系统的功能亢进,出现心率加快、血压升高、胃肠功能紊乱、头痛、呼吸加快加深、尿频、血糖升高等现象,对身体各器官产生影响。

近年来在生物病因学方面的研究多集中在创伤后应激障碍。有研究报道,创伤后应激障碍的患者肾上腺素和去甲肾上腺素分泌增加;与有创伤性经历但未患病的对照组相比,创伤后应激障碍患者的基础心率和血压都高于对照组;一些与创伤有关的线索,如声音、图片或有关的想象,均能引起患者更大的生理反应。

研究发现,创伤后应激障碍的患者与正常人及有创伤性经历但未患病的人对照相比,存在皮质醇水平的低下和糖皮质激素受体的增加。当给予小剂量的地塞米松后,创伤后应激障碍患者会出现过度抑制。这种下丘脑-垂体-肾上腺素轴的异常与原发性抑郁障碍患者不同。

神经影像学发现,老兵和幼年时期有性虐待经历的妇女海马体积减小。海马功能的紊乱可能导致对刺激的过度反应,与记忆的缺失也有关。PET研究发现,当患者在回想创伤性事件时,中颞部的血供减少,而大脑中颞部通过抑制杏仁核的功能,在消除恐惧方面起着重要作用。

\subsection{分类和诊断原则}

由于对应激相关障碍的概念理解不一,因此,应激相关障碍的归属及分类也存在很大差别。在我国的CCMD-3中,应激相关障碍归类于第5大类,即癔症、应激相关障碍、神经症有关的精神障碍中。这一大类主要介绍了神经症的分类和诊断标准以及应激相关障碍的分类和诊断标准。

国际疾病分类ICD-10系统将急性起病(在2周内),同时出现精神病性症状或具有相应急性应激的患者都归类于急性而短暂的精神病性障碍(F23)一类中。亚型中包含精神分裂症状反应、偏执性反应、心因性偏执性精神病、反应性精神病,并要求在编码的第5位上指明急性精神病性障碍是否伴有急性应激,而且在其诊断要点中明确病程标准,如超过规定的病程标准,应更改相应的诊断。

ICD-10将有严重的应激性生活事件或有持续的不愉快心理社会因素产生急性应激或持续性创伤性障碍,结合临床特点和病程,则归类于严重应激反应及适应障碍(F43)。其认为严重应激性生活事件可产生急性应激反应,而持续性的创伤可导致适应障碍。而且强调,急性应激反应和适应障碍的临床表现形式有其特异性,否则就不能归类于此类中。这里所谓的临床表现形式特异性主要是指有没有精神病性症状,如有精神病性症状则应归类于F23中的急性而短暂的精神病性障碍,没有精神病性症状则归类于F43中。

美国《精神障碍的诊断统计手册》第4版(DSM-Ⅳ)分类中,则明显不同。其分类主要根据临床表现,以某些特殊症状进行分类。同时与ICD-10相似,有明确病程标准和诊断标准要点。

应激相关障碍的诊断原则是:事件的发生与精神症状出现必需有时间上的密切联系;精神刺激必须具有一定的强度;产生精神障碍必须与相关事件有联系一定的意义,如为了逃避或防御或为了满足愿望等;精神症状内容必须反映精神刺激特点;精神症状随精神刺激的消除而消退。

\section{应激相关障碍类型}

\subsection{急性应激障碍}

\subsubsection{概述}

急性应激障碍(acute stress disorders)又称为急性应激反应(acute stress
reaction),是指以急剧、严重的精神打击作为直接原因,患者在受刺激后立即(1小时之内)发病,表现为强烈恐惧体验的精神运动性兴奋,行为有一定的盲目性,或者为精神运动性抑制,甚至木僵。如果应激源消除,症状往往历时短暂,预后良好,缓解完全。

有关急性应激障碍的流行病学研究很少。仅有个别研究指出:严重交通事故后的发生率为13%~14%,暴力伤害后的发生率为19%,集体性大屠杀后的幸存者中发生率为83%。

\subsubsection{临床表现}

1.反应性朦胧状态(reactive twilight
state) 患者主要表现为定向障碍,对周围环境不能清楚感知,注意力狭窄。患者处在受精神刺激的情感体验中,表现出紧张、恐怖,难以进行交谈,有自发言语,缺乏条理,语句凌乱或不连贯,动作杂乱,无目的性,偶有冲动。有的可出现片断心因性幻觉。数小时后意识恢复,事后可有部分或全部遗忘。

2.反应性木僵状态(reactive stupor
state) 临床主要表现为以精神运动性抑制为主。患者受到打击后,表现为目光呆滞,表情茫然,情感迟钝,呆若木鸡,不言不语,呼之不应,对外界刺激毫无反应,呈木僵状态或亚木僵状态。此型历时短暂,多数持续几分钟或数天,但不超过1周。大多有不同程度的意识障碍,有的可转入兴奋状态。

3.反应性兴奋状态(reactive excitement
state) 此时临床以精神运动性兴奋为主。患者受精神打击后,表现为伴有强烈情感体验的精神运动性兴奋,表现为话多、呼喊,内容往往与精神创伤有密切联系,且伴有相应情感反应,情绪激越,情感暴发,有时会冲动伤人、毁物行为。此型历时短暂,一般在1周内缓解。

4.急性应激性精神病(acute stress
psychosis) 也称为急性反应性精神病(acute reactive
psychosis)。这是急性应激障碍的一种表现形式,是强烈且持续一定时间的精神创伤事件直接引起的精神病性障碍。临床上以妄想或严重情感障碍为主,反应内容与应激源密切相关,易被人理解。急性或亚急性起病,历时短暂,一般在1个月内恢复,经治疗预后良好。

\subsubsection{诊断和鉴别诊断}

1.诊断 急性应激性障碍的诊断主要依靠病史、临床特征,通常实验室检查及其他辅助检查多无阳性发现。CCMD-3诊断要求包括以下内容:

【\textbf{症状标准}
】 以异乎寻常的和严重的精神刺激为原因,并至少有下列1项:

(1)有强烈恐惧体验的精神运动性兴奋,行为有一定盲目性;

(2)有情感迟钝的精神运动性抑制(如反应性木僵),可有轻度意识模糊。

【\textbf{严重标准} 】 社会功能严重受损。

【\textbf{病程标准}
】 在受刺激后若干分钟至若干小时发病,病程短暂,一般持续数小时至1周,通常在1月内缓解。

【\textbf{排除标准}
】 排除器质性精神障碍、非成瘾物质所致精神障碍、抑郁症及癔症。

2.鉴别诊断

(1)器质性精神障碍:如某些非成瘾物质中毒、中枢神经系统感染、躯体疾病引起的精神障碍等。急性期常出现谵妄,患者表现为精神运动性兴奋、恐惧、意识障碍,有些患者还可追溯到发病前有某些应激事件,应注意鉴别。一般来讲,详细的病史和体格检查、实验室检查确定有无器质性病因是最重要的。其次,器质性精神障碍患者即使病前有应激事件,但程度不强烈,与症状的关系不密切。

(2)心境障碍:多数心境障碍的起病也与某些应激事件相关,主要症状可表现为精神运动性兴奋或抑制,需与急性应激障碍相鉴别。心境障碍的精神运动性兴奋或抑制为协调性,且情感障碍占优势,病程一般较长,常循环发作。抑郁症的抑郁心境涉及较广,包括兴趣、日常喜好、个人前途等各方面,没有固定的应激事件,消极、自卑或自杀企图也较常见,整个临床相有晨重夕轻的变化规律,而应激性障碍无上述特征。

(3)癔症:也常在精神应激性事件后发病,且症状表现短期内有时难与急性应激障碍区别。但癔症表现更为多样化,带有夸张或表演性,并给人做作的感觉,病前个性有自我中心、富于幻想、外向等特点,其中很重要的一点为暗示性较强,病性反复多变。

\subsubsection{治疗}

由于急性应激障碍多起病急、症状持续时间短暂,因此很多患者可能首先被送到综合性医院的急诊室,由内、外科医生或全科医生进行诊断和处理。

急性应激障碍的治疗因患者和创伤性事件的特点不同而有所不同。基本原则是及时、就近、简洁、紧扣重点。除帮助患者尽快脱离创伤性情境外,主要是减轻情绪反应,学习面对应激事件,使用有效的应付技能,调动有效资源帮助解决其他相关问题。

1.药物治疗 对表现为精神运动性兴奋的患者,可以使用抗精神病药物,如氟哌啶醇针剂等,使症状迅速缓解,帮助患者度过这一时期。对表现为精神运动性抑制甚至木僵的患者,要注意营养支持以及耐心的照料。对有焦虑或抑郁症状的患者,可使用抗焦虑药或抗抑郁药治疗。治疗中注意药物剂量不宜过大,疗程不宜过长。

2.心理治疗 治疗的目的是降低情绪反应和帮助患者调整认知以便更有效地应对环境。由于急性应激障碍通常是短暂的反应,支持性的心理治疗往往有效。

有研究表明,短疗程的认知行为治疗对治疗急性应激障碍以及预防其发展成为创伤后应激障碍有效。认知行为治疗的步骤包括:①对创伤性事件所引起的反应进行解释;②渐进性肌肉放松训练;③逐渐延长时间的暴露;④对与恐惧相关的信念进行认知重建;⑤逐级的现场暴露。

\subsection{创伤后应激障碍}

\subsubsection{概述}

创伤后应激障碍(post-traumatic stress disorder,
PTSD)是指由于受到异乎寻常的威胁性、灾难性心理创伤,导致延迟出现和长期持续的精神障碍。本病最早出现在1980年美国的精神障碍诊断标准中(DSM-Ⅲ),早期主要研究的对象是退伍军人、战俘、集中营的幸存者,后逐渐扩展到遭遇各种天灾人祸的人群。发病率因研究对象不同而差别很大,国内外采用不同方法、在不同人群的社区进行流行病学调查结果发现,创伤后应激障碍的患病率为1%~14%,对高危人群(如参与战争后的退伍军人、火山爆发或空难的幸存者)的研究发现,患病率为3%~58%。创伤后应激障碍可发生于任何年龄,包括儿童和老人,最常见于青年人。关于创伤后应激障碍发病的性别差异问题,有研究发现,对同一创伤事件,女性发病的可能性是男性的2倍,女性的患病率(10%~12%)高于男性(5%~6%)。一般的,临床症状出现于创伤发生后3个月内,部分患者可以在数月甚至数年后起病。

异乎寻常的精神创伤性事件是PTSD发生的必备条件。这类事件包括地震、洪水等巨大的自然灾害、战争、严重的突发事故、被强奸或受到严重的躯体攻击等。其强度几乎能使所有经历这类事件的人都会感到巨大的痛苦,常引起个体极度恐惧和无助感。形成所谓“创伤性体验”。但是最终只有部分人出现PTSD,因此疾病的发生还与个体易感素质有关。现有研究发现,所谓“创伤性体验”应该具备两个特点:第一,对未来的情绪体验具有创伤性影响,例如,被强奸者在未来的婚姻生活或性生活中可能反复出现类似的体验;第二,是对躯体或生命产生极大的伤害或威胁。当然,老年和儿童、女性、社会支持缺乏、躯体健康状况不良等都是发生PTSD的危险因素。

\subsubsection{临床表现}

PTSD多于遭受创伤后数日甚至数月后才出现,病程可长达数年。症状的严重程度可能有波动。临床上主要有以下三种表现:

1.创伤性体验的反复出现 PTSD最特征性的表现是在重大事件发生后,患者有各种形式的反复发生的闯入性地出现以错觉、幻觉(幻想)构成的创伤性事件的重新体验,称症状闪回(flash
back)。创伤性事件在患者的意识中反复涌现、萦绕不去,反复体验到创伤时的体验,令患者痛苦不已。它是和过去创伤性记忆有关的强烈的闯入性体验。闪回经常占据患者整个意识,仿佛此时、此刻又重新生活在那些创伤性事件中。闪回不同于强迫观念,因为它来自于对过去体验的记忆,而不是与以前体验无关的内容。在闪回期间,患者常并未意识到自己的行为在当前是不适当的,但常常触景生情,任何与创伤性事件有关的线索,如遭遇相似的环境、人物、事件等时候,患者就表现出紧张不安,头脑里进而就浮现创伤时的情景。如一个空难幸存者事后常常会重新体验到尸体横在自己的面前,无数人在呻吟,空气里弥漫着烧焦的气味,自己躺在又冷又湿的地上,等待救援等场景。

2.持续性的回避 患者表现为尽量回避与创伤有关的人、物及环境,回避有关的想法、感觉和话题,不愿提及相关的话题。同时患者还表现出不能回忆起有关创伤的一些重要内容。患者对一些重要的活动明显失去兴趣,不愿与人交往,与外部世界疏远,对很多事情都索然无味,对亲人表现冷淡,难以表达和感受一些细腻的感情,对工作、生活缺乏打算,变得退缩,让人感觉患者性格孤僻,难于接近。

3.持续性的警觉性增高 患者表现为睡眠障碍,易发脾气,很难集中注意力,容易受惊吓。遭遇与创伤事件相似的情境时,会出现明显的自主神经系统症状,如心悸、出汗、肌肉震颤、面色苍白或四肢发抖。此外,此类患者多数有焦虑或抑郁情绪,少数甚至出现自杀企图。有人报道,多数患者常继发抑郁障碍和物质滥用。

\subsubsection{诊断和鉴别诊断}

1.诊断 主要依靠病史和临床特征,实验室及其他辅助检查无特异性。CCMD-3的诊断标准包括以下内容:

【\textbf{症状标准} 】

(1)遭受对每个人来说都是异乎寻常的创伤性事件或处境(如天灾人祸);

(2)反复重现创伤性体验(病理性重现),并至少有下列1项:①不由自主地回想受打击的经历;②反复出现有创伤性内容的恶梦;③反复发生错觉、幻觉;④反复发生触景生情的精神痛苦,如目睹死者遗物、旧地重游,或周年日等情况下会感到异常痛苦和产生明显的生理反应,如心悸、出汗、面色苍白等;

(3)持续的警觉性增高,至少有下列1项:①入睡困难或睡眠不深;②易激惹;③集中注意困难;④过分地担惊受怕;

(4)对与刺激相似或有关的情境的回避,至少有下列2项:①极力不想有关创伤性经历的人与事;②避免参加能引起痛苦回忆的活动,或避免到会引起痛苦回忆的地方;③不愿与人交往、对亲人变得冷淡;④兴趣爱好范围变窄,但对与创伤经历无关的某些活动仍有兴趣;⑤选择性遗忘;⑥对未来失去希望和信心。

【\textbf{严重标准} 】 社会功能受损。

【\textbf{病程标准}
】 精神障碍延迟发生(即在遭受创伤后数日至数月后,罕见延迟半年以上才发生),符合症状标准至少已3个月。

【\textbf{排除标准}
】 排除情感性精神障碍、其他应激障碍、神经症、躯体形式障碍等。

2.鉴别诊断

(1)与急性应激障碍的鉴别:两者发病都和应激因素紧密相关,主要区别是起病时间和病程。

(2)与抑郁障碍的鉴别:抑郁障碍是以持久的情绪低落、兴趣下降为主要表现,常无严重的创伤性事件作为发病的主要原因,没有与创伤性事件相关联的闯入性回忆和梦境,也没有对特定事物或场景的回避。抑郁情绪与精神创伤内容没有密切关联。抑郁症状常有全面的精神运动性抑制、有晨重夕轻及早醒的规律。同时可以参考家族史、过去发作史、病前人格特点等。

(3)与强迫障碍的鉴别:强迫障碍的患者也可能出现反复挥之不去的强迫思维,但强迫思维通常与过去的严重创伤性经历没有关系。

(4)与适应障碍的鉴别:创伤后应激障碍的应激源通常是异常强烈的、威胁生命的,几乎每个人都会觉得害怕;而适应障碍的应激源可以是任何程度的,疾病的发生与个体的适应能力有关。创伤后应激障碍的诊断要求有特征性的症状。

\subsubsection{治疗}

1.早期干预 类似急性应激障碍,包括鼓励患者冷静地面对痛苦经历、表达相关的情绪体验和帮助患者调整情绪反应到接近正常水平,提供心理支持,减轻患者的内疚感(如患者的亲人在同一事故中死亡)和改变患者对创伤性事件的态度等。少量、短期应用抗焦虑药和镇静催眠药有助于缓解焦虑和调整紊乱的睡眠。这些简单的干预方法要尽早实施,以阻断创伤性事件心理痕迹的保持。

2.后期干预 确诊的或慢性的PTSD患者的治疗比较棘手,一般应由精神科专科医生和临床心理学家来处理。根据患者的临床表现、严重程度和病程选择心理治疗,并适当使用药物。

(1)心理治疗

①认知-行为治疗:大量的研究认为,认知-行为治疗对创伤后应激障碍有效。与患者讨论对创伤性事件的认识是认知-行为治疗的重点之一。认知-行为治疗是让个体反复暴露于与创伤性事件有关的刺激下,缓解焦虑和恐惧。可以进行想象中的暴露练习,也可以进行现场暴露,如让车祸的幸存者重新回到车祸的发生地,鼓励患者面对创伤性事件,表达、宣泄相应的情感。同时找出并纠正对创伤性事件及后果的负性评价,改变患者不合理的认知,如强烈的内疚和自责,学习新的应对方式,以更好地面对今后的生活。

②其他形式的心理治疗:包括心理动力学治疗、眼动脱敏治疗、催眠治疗等。眼动脱敏治疗(eye-movement
desensitization reprocessing,
EMDR)是一种相对较新且有争议的治疗。患者在注视前后移动的治疗师的手指的同时,让患者睁眼想象与创伤有关的情境。在数次眼动后,患者将和治疗师一起讨论有关的认知和情绪反应。该治疗机制还不清楚,但有假说认为,快速眼动可以产生一种拮抗恐惧的状态,因此和系统脱敏中的放松练习有对等的作用。

(2)药物治疗:药物治疗最多的是抗抑郁药。多数研究表明,SSRIs类药物如帕罗西汀、舍曲林等能有效地治疗创伤后应激障碍的回避、警觉性增高、麻木等症状,优于其他的药物治疗。其他还可以选择抗焦虑药物对症治疗。有研究表明,认知-行为治疗合并使用SSRIs类药物是较好的治疗选择。

\subsection{适应障碍}

\subsubsection{概述}

适应障碍(adjustment
disorders)是因长期存在应激源或困难处境,加上病人有一定的人格缺陷,产生以烦恼、抑郁等情感障碍为主,同时有适应不良的行为障碍或生理功能障碍,并使社会功能受损。病程往往较长,但一般不超过6个月。通常在应激性事件或生活改变发生后1个月内起病。随着时过境迁,刺激的消除或者经过调整形成了新的适应,精神障碍随之缓解。

国外认为适应障碍较常见,有研究报道占精神科门诊的5%~20%。可发生于任何年龄,青少年最常见,成年人中单身女性的患病危险性最高。

适应障碍的发生与应激源和个体适应能力有关。与急性应激障碍和创伤后应激障碍不同的是,适应障碍的应激源强度较弱,多为日常生活中常见的应激事件,应激源可以是单一的或多重的。青少年中最常见的应激源是父母不和或离婚、学习环境的改变;成年人中最常见的应激源是婚姻冲突、经济问题或残疾子女出生等;老年人最常见的应激源是离、退休、社会角色的变化及丧失子女等。

\subsubsection{临床表现}

发病多在应激性生活事件发生后的1~3个月内出现,临床表现多样,包括抑郁心境、焦虑或烦恼、失眠。感到不能应对当前的生活或无法计划未来,躯体功能障碍(如头疼、腹部不适、胸闷、心慌),社会功能受损。有些患者可出现暴力行为。

临床症状表现多种多样,按主要精神症状可分以下类型:以情绪低落、忧伤易哭、对日常生活丧失兴趣、自责、无望无助感等为主的抑郁型(严重者可出现自杀行为);以焦虑、烦恼、害怕、敏感多疑、紧张颤抖、愿向别人倾诉痛苦等为主的焦虑型;以逃学、旷工、斗殴、粗暴、破坏公物、目无法纪和反社会行为等为主的品行障碍型;以孤独、离群、不参加社会活动、不注意卫生、生活无规律等为主的行为退缩型;以影响学习或工作、效率下降(成绩不佳)为主的工作学习能力减弱型。许多病人出现的症状是综合性的。假如无突出症状则为混合型。病人也常伴有生理功能障碍如睡眠障碍、食欲不佳、心慌、呼吸急促、窒息感等。

青少年以品行障碍为主者,表现为逃学、斗殴、盗窃、说谎、物质滥用、离家出走、性滥交等。儿童适应性障碍主要表现为尿床、吸吮手指等退行性行为,以及无原因的腹部不适等含糊的躯体症状。

适应性障碍的临床表现比较复杂,变化较大。以占优势的临床相进行分类,在临床上可作参考。

\subsubsection{诊断与鉴别诊断}

1.诊断 诊断主要依靠病史和临床特征,实验室及其他辅助检查无阳性结果。需要注意的是要有强有力的证据表明,如果没有应激就不会出现障碍。如果因正常沮丧反应就诊,而且出现的反应在个人所在文化中是恰当的,且持续时间不超过6个月,则不诊断适应障碍。根据CCMD-3的诊断标准,具体包括以下内容:

【\textbf{症状标准} 】

(1)有明显的生活事件为诱因,尤其是生活环境或社会地位的改变(如移民、出国、入伍、退休等)。

(2)有理由推断生活事件和人格基础对导致精神障碍均起着重要的作用。

(3)以抑郁、焦虑、害怕等情感症状为主,并至少有下列1项:①适应不良的行为障碍,如退缩、不注意卫生、生活无规律等;②生理功能障碍,如睡眠不好、食欲不振等。

(4)存在见于情感性精神障碍(不包括妄想和幻觉)、神经症、应激障碍、躯体形式障碍,或品行障碍的各种症状,但不符合上述障碍的诊断标准。

【\textbf{严重标准} 】 社会功能受损。

【\textbf{病程标准}
】 精神障碍开始于心理社会刺激(但不是灾难性的或异乎寻常的)发生后1个月内,符合症状标准至少已1个月。应激因素消除后,症状持续一般不超过6个月。

【\textbf{排除标准}
】 排除情感性精神障碍、应激障碍、神经症、躯体形式障碍,以及品行障碍等。

2.鉴别诊断

(1)抑郁症:由于适应障碍临床上以抑郁、焦虑等症状为主,故需与抑郁症相鉴别。一般来讲,抑郁症的情绪低落较重,常常伴有自责、自罪、轻生等消极念头或行为,内心体验缺乏兴趣,有睡眠障碍、以早醒居多、情绪变化昼重夕轻这些临床特点。

(2)焦虑症:焦虑症患者除了临床表现为焦虑、紧张不安等情感症状外,常伴有明显的自主神经功能失调的症状,患者因感到痛苦而积极要求治疗。本病病程较长,往往无明显的应激因素存在。

(3)人格障碍:虽然适应障碍的发生与人格基础有一定的关系,但临床上不仅表现为人格缺陷,而且有明显的抑郁、焦虑等情感症状。随着应激源的消失,适应障碍的症状可缓解或消失,而人格障碍是在未成年时已很明显。

应激性事件的存在及其与发病在时间上的关联,加上症状的严重程度或症状组合不符合任何特定障碍或综合征的临床特征,这两个方面使适应障碍区别于其他精神障碍。

\subsubsection{治疗}

适应障碍的病程限定为1~6个月。换言之,随着时间的推移,适应障碍可以自行缓解,或转化为更为特定、更严重或更持续的其他障碍。因此,适应障碍治疗的根本目的是帮助患者提高处理应激境遇的能力,早日恢复到病前的功能水平,防止病情恶化或慢性化。

治疗重点以心理治疗为主。首先要评定患者症状的性质及严重程度,了解诱因、患者人格特点、应对方式等因素在发病中的作用,应注意应激对患者的意义,避免单纯根据医生的理解或常识匆忙作出判断。尤其要考虑是否存在不利于预后的危险因素,如同时面临多种问题或应激事件持续存在,缺乏支持性的人际关系,存在躯体健康问题,病前功能欠佳等因素。

1.心理治疗 重点在于减轻或消除应激源,增强应对能力进而消除或缓解症状。心理治疗的方式包括精神动力学治疗、认知行为治疗、家庭心理治疗、团体心理治疗和支持性心理治疗等。可根据患者的特点和要求以及治疗者的专长,选择相应的治疗。认知行为治疗是比较实用而有效的方法,主要是通过对自动思维的监测,帮助患者识别对应激源和应对能力的不合理认知,重建适应性的行为,从而有效地克服适应障碍。

2.药物治疗 是对症治疗,可加快缓解患者的症状,提高患者的生活质量,因此也是很必要的。抗焦虑药物:可解除患者焦虑、紧张不安、失眠等症状,可用地西泮、阿普唑仑、艾司唑仑等;抗抑郁药物:有明显情绪低落的患者,可选用氟西汀、帕罗西汀等药物。在药物治疗时,应注意用药剂量,以小剂量为宜,疗程不宜过长。



