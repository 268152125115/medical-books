\chapter{精神疾病的诊断}

精神疾病的临床诊断包括收集资料、分析资料和作出诊断这三个步骤。其中最重要的步骤是收集资料,它包括完整的病史采集,系统的体格检查、精神检查和辅助检查。但是由于精神科的一些疾病,如精神分裂症、情感性精神障碍、神经症等,迄今在体格检查和辅助检查方面均没有固定的有助于诊断的阳性发现,所以目前精神疾病临床诊断的主要根据是病史和精神检查。病史采集和精神检查在精神科统称为精神病学检查。其主要方法是与患者和知情人面谈。这是一种特别的检查技巧,需要通过不断的临床实践和积累经验。精神病学检查是每一个精神科医生必须掌握的基本功。

\section{病 史 采 集}

\subsection{采集方法}

病史主要源于患者和知情者,对于缺乏自知力、不能客观而正确地叙述病史的患者,其病史主要由知情者提供。病史提供者包括与患者共同生活的配偶、父母、子女等,与患者共同学习或工作的同学、同事、领导等,与患者关系密切的朋友、邻居等,以及既往为患者诊治过的医生。采集方法有面谈、书面介绍等方式。面谈灵活机动,易于深入,最常用。另外,还可以收集患者的日记或其他书写材料作为补充。

采集病史前最好先观察一下患者。患者的表情、言语、动作姿势、步态以及躯体状况等,往往能为诊断提供线索,询问病史就有的放矢;还应向提供病史的人提出明确的要求,使之有条理地介绍病史。医生应耐心、仔细地倾听,适当地引导提问,有时还需给予必要的解释,以解除病史提供者的顾虑。在门诊,由于患者和家属最关心的是现病史,且时间有限制,一般先从现病史问起;而住院病史的采集则先问家族史、个人史和既往史,对发病背景有了充分了解后更有利于病史的收集。记录病史应如实、条理清楚、简明扼要,能清楚地反映疾病的发生发展过程以及各种精神症状的特点。对一些具有代表性的事例和患者的原话可适当加以摘录,最好保留患者原话的语气。

\subsection{病史内容}

\subsubsection{一般资料}

包括姓名、性别、年龄、婚姻状况、民族、籍贯、职业、工作单位、文化程度、宗教信仰、住址、联系电话、入院日期,以及联系人的姓名、工作单位、与患者的关系和联系电话。还有病史提供者的姓名、与患者的关系,医生对病史完整性和可靠性的评估。

\subsubsection{主诉}

主诉包括主要临床表现和病程,尽量简明扼要。

\subsubsection{现病史}

现病史为病史中的重要部分。按发病时间先后描述疾病的起始及其发展的各阶段的临床表现。包括以下内容:

1.发病条件及原因 询问患者发病的环境背景及与患者有关的生物、心理、社会因素,探索与发病有关的原因和诱因。如有社会心理因素,应了解其与精神症状的关系,是发病原因还是诱因。躯体疾病、颅脑外伤、手术、妊娠、乙醇、某些药物等均可引起或促发精神症状的出现。如能及时发现这些致病因素或诱因,对于诊断和处理有很大帮助。

2.起病形式和病期 由于起病急缓和病程长短对诊断和预后有重要意义,因此要深入询问和正确估计。家属介绍病史往往只从症状已经明朗化谈起,若仔细询问,有时可发现在此之前患者就有一些反常现象,如个性改变、情感淡漠、生活懒散等已存在较长一段时间了,这说明起病缓慢,病期远比家属提供的要长。

3.疾病发展及演变过程 可按时间先后逐年、逐月甚至逐日分段地纵向描述。内容包括疾病的首发症状、具体表现及持续的时程、症状间的关系、症状的演变及其与生活事件、应激源、心理冲突、所用药物之间的关系;患者病后社会功能的变化;病程特点,是进行性、发作性或迁延性等。对于能说明疾病性质的症状要做详细而具体的描述。病程中如有伤人、毁物、自伤、自杀、走失等情况应注明,以利于护理和防止意外事件的发生。此外还应注明必要的鉴别诊断内容。

4.既往诊疗情况 应从患病后首次就医开始记录,包括就医单位、辅助检查、诊断、门诊治疗还是住院治疗、具体用药情况、治疗效果及不良反应等。

\subsubsection{家族史}

着重了解家庭成员的年龄、职业、性格特征,以及家庭结构、经济状况、家庭成员之间的关系,父母两系三代有无神经、精神疾病及近亲婚配等。如有家族遗传史,应注明具体发病情况、症状特点和结局。

\subsubsection{既往史}

询问有无发热、抽搐、昏迷、药物过敏史,有无感染、中毒、外伤及躯体疾病,特别是有无中枢神经系统疾病史,有无其他精神疾病史。

\subsubsection{个人史}

采集个人史可因患者年龄不同有所侧重。对未成年患者应着重了解母孕期、出生、生长发育和学习情况,成年初发病则应就生活经历、婚姻、工作以及与社会集体的关系等做详细询问,老年患者的幼年情况可略。母孕期应了解母亲的营养情况,有无感染、中毒、外伤、腹部X线照射及其他躯体疾病,有无饮酒、用药或物质滥用史;出生时是否足月,有无产伤或窒息;幼年生长发育状况,有无不良遭遇;在教育方面需了解教育方式、入学年龄、学习成绩、在校表现及与同学之间的关系等;工作的性质、能力、表现、人际关系以及工作变动情况;生活中有无特殊遭遇,生活习惯及有无特殊嗜好;性格的倾向性、稳定性如何,有无怪僻之处;婚姻及夫妻生活情况;女性的月经、分娩、绝经期的情况等。

\subsection{注意事项}

1.采集病史前首先应建立良好的医患关系。医生应保持外观整洁、态度庄重、和蔼可亲,并尊重患者,尽可能使用与患者或知情者文化水平相应的语言进行交谈,不要随意打断对方的谈话,保持交谈的连续性。

2.采集病史时患者应回避,使病史提供者无所顾虑、畅所欲言,也可免去患者当场争辩,影响病史采集。

3.采集病史应尽量客观、全面和准确。可以通过不同的知情者了解患者不同时期、各个侧面的情况,相互核实,相互补充。事先应向知情者说明病史准确与否将直接关系到诊治结果,提醒病史提供者注意资料的真实性。还应了解病史提供者与患者接触是否密切,对病情的了解程度是否掺杂了个人的情感,或因种种原因有意无意地隐瞒或夸大了一些重要情况,从而对病史的可靠程度给以恰当的评估。如家属之间或家属与单位之间对病情的看法有严重分歧,则应分别加以询问,了解分歧的原因何在。如病史提供者对病情不了解,则应请知情者补充病史,必要时通过信函或社会调查采集病史,使之不断补充和完善。

4.由于病史提供者往往缺乏精神病学专业知识,接触患者有局限性,有时可能带有主观或某些偏见,甚至接受了患者的病态观点,同时受文化水平的影响,因此他们提供的资料可能是不完整和不准确的。这时医生不应只是在倾听,还应观察病史提供者的心理状态,要善于启发诱导,将谈话内容引导到需要了解的问题上,以期获得全面而真实的资料。

5.由于不同的文化背景可产生不同的症状,因此要充分了解患者的文化背景,如宗教信仰、风俗习惯等等,以便弄清精神症状与文化背景的关系。

6.精神疾病的病史常常涉及个人隐私或法律问题,因此病史的记录应实事求是、全面而详尽,避免使用医学术语,同时医护人员还负有保密之责,不得作为闲谈资料。

\section{精 神 检 查}

\subsection{检查方法}

精神检查是临床精神疾病诊断的基本手段,用以系统了解和掌握患者当前的精神状态,存在哪些精神症状,精神症状的特点以及相互之间的关系,从而为诊断提供依据。常规的精神检查包括与患者交谈和对患者进行观察两种方式。此外,还可以借助于患者在病中所写的书面材料、临床心理测验结果等,从各个不同角度了解患者的精神状态。

\subsection{检查内容}

\subsubsection{一般表现}

1.意识状态 注意意识的清楚程度和范围,有无意识障碍以及意识障碍的程度和内容。

2.定向力 包括时间、地点、人物及自我定向能力,有无双重定向。

3.接触情况 对周围人或事物是否关心,主动接触与被动接触怎样,合作情况。

4.日常生活 包括仪表、饮食、睡眠、个人卫生及大小便能否自理,女性患者月经情况。

\subsubsection{认知活动}

1.知觉 是否有错觉、幻觉、感知觉综合障碍,具体类型、出现时间、频度、内容、性质、出现时患者的情绪和行为反应。

2.思维 思维联想的速度及结构上有无异常,思维逻辑推理及概念形成是否有障碍,思维内容方面是否存在妄想以及妄想的种类、内容、牵涉范围、出现时间、原发或继发、系统性以及对患者行为的影响。

3.注意 观察注意力及注意范围的改变。

4.记忆 检查记忆过程和内容有无障碍。

5.智能 检查内容包括一般常识、计算力、理解力、分析综合及抽象概括能力等。

6.自知力 有无自知力,还是存在部分自知力。

\subsubsection{情感活动}

情感活动包括客观表现和主观体验两方面。客观表现可根据患者的面部表情、姿势、动作以及面色、呼吸、脉搏、出汗等自主神经反应来判定;主观体验可通过交谈了解患者的内心体验。应观察情感反应的性质、强度、稳定性、协调性以及持续时间。

\subsubsection{意志行为活动}

主要观察有无欲望减退、增强或异常,动作行为增多或减少,与周围环境有无联系,姿势是否自然或奇特,有无意志减退或增强。

\subsection{注意事项}

1.检查之前应熟悉病史,根据患者的特点事先略作筹划,做到有目的、有计划地进行检查。

2.检查环境要安静,避免外来的干扰。检查时患者亲友不宜在场,以免患者出现不安、争辩、沉默不语或有意隐瞒等情况。

3.医生的态度要亲切、诚恳,尊重患者,建立良好的医患关系,从容自然地与患者交谈,创造一种无拘束的融洽气氛,使患者自由地讲述自己的思想和看法。

4.询问不要公式化,要灵活机动、因人而异。询问时不应带有先入为主的观点,要避免对患者的回答给予暗示,做到客观地观察和记录。

5.应随时做好记录,确保内容真实和完整。有些存在幻觉、妄想的患者不宜当面记录,以免引起患者的怀疑、警惕和反感,影响检查的进行,但检查结束后应立即补记。

6.对兴奋、木僵、不合作的患者,要着重观察他们的言行表情。对脑器质性精神病患者,检查重点是意识状态、记忆、智能和人格变化。

\section{体格检查和辅助检查}

\subsection{体格检查}

体格检查对精神疾病的诊断及鉴别诊断十分重要。住院患者应按全面体格检查要求系统地进行,门急诊患者应根据病史重点地进行体检。只重视精神症状而忽略体格检查往往会出差错,应绝对避免。神经科与精神科是两个关系十分密切的学科,许多神经疾病可伴有精神症状,反之亦然。因此,所有精神疾病患者都必须进行详细的神经系统检查。

\subsection{辅助检查}

实验室辅助检查可为精神疾病的诊断和鉴别诊断提供必要的依据。一般常规检查有血常规、肝功能检查、胸部X线透视和心电图检查。根据病情还应进行以下各项检查:脑电图、头颅X线平片、脑超声波、头颅CT或MRI、脑血管造影和脑脊液检查等。另外,通过心理测验和特定的评定量表,可对患者的智力、人格、社会功能以及疾病严重程度等进行评定。

\section{临床资料分析与诊断}

当临床资料收集完备之后,下一步就要进行全面系统的分析,以便确立诊断。由于目前精神疾病的诊断主要依据病史和精神症状,因此这两方面的临床资料分析在精神疾病的诊断中十分重要。其包括对疾病的发病基础、可能的发病原因、疾病发生发展过程、临床表现及特点等进行系统全面的分析。

\subsection{发病基础}

分析患者的性别、年龄、职业、生活环境、病前人格特征、既往史、家族史以及个人史等,可以判定疾病是在什么样的基础上发生发展的,并可为疾病性质的确定得到某些启发。如老年初发病例需首先考虑脑器质性精神病,从事有毒工种工作的患者可考虑中毒性精神病,躯体疾病所致精神障碍往往在躯体疾病的最严重时出现精神症状,躁狂症患者多好交际、热情、脾气急躁,精神分裂症患者病前性格多内向,而性格改变往往又是起病的界限和早期表现。因此,这些基础因素的分析对精神疾病的诊断具有一定的参考价值。

\subsection{起病及病程}

应重点分析起病形式和病程特点,这在诊断上具有一定的意义。起病分为急性(不超过两周)、亚急性(2周到3个月)和慢性(3个月以上)三种形式。病程发展可表现为发作性、周期性、间歇性、持续性、进行性等几种。急性起病常为感染、中毒所致精神障碍,以及癔病和心因性精神障碍;精神分裂症起病多隐袭,而病程多为持续性的;情感性精神障碍的病程特点多为间歇性;癔病、癫痫
及某些躯体疾病所致精神障碍的病程可呈发作性或周期性波动。

\subsection{临床表现}

精神症状是精神疾病临床诊断的重要依据。因此首先应确定患者有哪些精神症状,其中哪些是主要症状。在分析时不能仅从症状的表面现象来看,应从各症状的特点、症状间的相互关系、症状的变化和发展以及整个精神状态与外界环境的联系来分析。要抓住患者的主要病态心理活动,并与发病基础、发病因素以及病程发展结合起来考虑诊断。

\subsection{病因}

精神疾病的病因分析要考虑躯体因素、心理因素和原因不明三类情况。由躯体因素引起的精神疾病,体格检查及实验室检查方面可有相应的阳性发现,如脑器质性精神病和躯体疾病所致精神障碍等。心理因素引起的精神障碍必然有明显的精神创伤,如心因性精神障碍和癔病等。迄今原因不明的精神疾病主要有精神分裂症和情感性精神障碍等,但在发病前可找到诱发因素。需要注意的是,有的人错误地认为心理因素必然与精神疾病有联系,因而心理因素常被病史提供者强调为致病因素。因此在分析所谓的心理因素时,首先要判定心理因素是否客观存在,分析心理因素在发病时所起的作用以及与发病的关系。

\subsection{诊断及鉴别诊断}

在获得完整的病史资料以及经过详尽的精神检查、体格检查和实验室检查后,首先要确定患者存在的主要症状或综合征,然后根据其临床特点作出疾病分类学诊断。按等级诊断原则,应先重后轻。首先考虑脑器质性精神障碍、躯体疾病所致精神障碍、精神活性物质所致精神障碍,排除以上器质性精神障碍之后再考虑功能性精神疾病。其中又要先考虑较重的有精神分裂症、情感性精神障碍等,再考虑较轻的有心因性精神障碍、神经症、心理生理障碍和人格障碍等。诊断指导治疗,而治疗也可验证诊断,这是临床医学的一般规律,这一规律在精神病学中也同样适用。为了避免或减少诊断上的误差,还要以实践、认识、再实践、再认识的原则,在临床过程中继续观察,包括当病情缓解后进一步核实病情,即使患者已经出院也应随访观察,从实践中检验诊断。只有这样才能不断提高诊断水平。


\section{标准化精神检查和心理量表的应用}

人类的精神活动非常复杂,当出现精神活动异常时也难以作出客观一致的观察和判断,世界卫生组织曾有研究显示,不同文化背景的医师或同一文化背景的不同临床医师在精神疾病的诊断上存在一些差异,关键问题在于对精神活动缺乏客观一致的检查和测量工具,因此标准化的精神检查程序和心理测试工具的发展成为精神病学的一个重要发展方向,通过近50年的努力,在标准化检查和量化评价方面都取得了长足的进展,并在精神科临床和科研中得到广泛的应用。在精神科临床和科研中,标准化检查和量化评价主要包括标准化结构式精神检查、精神症状的量化评定、认知功能和人格测量以及其他认知方式、应对方式、应激和社会功能的评定,现分述如下。

\subsection{标准化精神检查}

为了提高精神疾病临床诊断的一致性,许多精神疾病诊断系统都发展了标准化或定式访谈和精神检查程序,精神现状检查第10版(PSE-10)、复合性国际诊断交谈检查表(CIDI)和国际人格障碍检查表(IPDE)都与国际精神疾病诊断分类系统(ICD-10)和美国精神疾病诊断分类系统(DSM-IV)相对应。近年来的一个重要进展是上述标准化访谈和检查工具都编制了相应的计算机软件,使这些工具的使用更简便、更广泛、经济效益更高,这里对PSE-10和CIDI核心版做简要介绍。

1.精神现状检查第10版(PSE-10) PSE-10是神经精神疾病临床评定表的核心部分,是一套半结构式的访谈问卷,每个条目或症状都附有操作定义,评定患者最近一个月的精神状况。1992年的PSE-10,有相应的计算机软件。PSE-10与ICD-10和DSM-IV相配套,在研究和临床上获得广泛应用,能明显提高临床诊断的准确性和一致性。PSE-10包含两部分,第一部分含14节,主要评定非精神病性障碍,能提供一般躯体疾病、躯体形式障碍、神经症、应激和适应障碍、生理心理障碍、情感障碍、酒精和药物滥用等疾病诊断所需要的资料;第二部分含10节,主要评定精神病性障碍和认知障碍,能提供各种精神障碍诊断所需要的资料。

2.复合性国际诊断交谈检查表(CIDI-C) CIDI-C是WHO(1990)推出的标准化定式精神检查工具,包含一系列工具,如检查者用表、研究者用表、使用者用表、训练手册、模拟手册及CIDI-C/ICD-10/DSM-Ⅳ计算机诊断手册。CIDI-C内容与ICD-10相呼应,包含:A节人口学资料;B节吸烟问题;C节躯体形式障碍(F45)和转换分离障碍(F44);D节惊恐发作(F40)/广泛性焦虑(F41);E节抑郁障碍(F32/F33)、心境恶劣(F34);F节躁狂(F30)和双相情感障碍(F31);G节精神分裂症和其他精神病性障碍(F20、F22、F23和F25);H节进食障碍(F50);I节饮用酒精所致的障碍(F10);K节强迫性障碍(F42);L节使用精神活性物质所致的障碍(F11、F16、F18和F19);M节器质性障碍(F0);N节病理性赌博(F63.0);O节性心理障碍(F52);以及检查者的观察和评定、追问流程图及附件。共计380题。通过检查可获得症状及其严重度、病程、发病次数和发病(始发和近发)年龄等资料。将CIDI-C的评分输入CIDI-C/ICD-10/DSM-Ⅳ计算机程序可显示主要和次要的疾病分类学诊断。

\subsection{标准化量表}

精神活动的客观测量和量化评定是精神医学发展的一个重要方向,在过去半个世纪中也取得了重大发展,在精神科临床、科研和教学获得广泛应用,为精神疾病的病因调查、科学研究、疾病诊断和疗效评价提供了科学客观的量化工具。心理测验或评定量表是用一些有代表性任务或条目对人行为表现或典型症状做定性或定量的描述,所有测验或量表都必须接受信度和效度考验,并建立特定文化群体的常模(不同形式的正常值),个体的测验或评定结果只有同常模比较才有意义。精神科常用的测验或量表根据用途可分为以下几类,现按分类做简要介绍。

1.一般能力测验或量表 一般能力指个体生存、生活、学习、工作和适应社会所必备的基本心理能力,一般包括认知能力(记忆和智力)和社会适应能力。这类测验一般用于精神发育迟滞、痴呆和脑器质性疾病的诊断和康复效果的评定以及脑损害者的司法鉴定和劳动能力或残疾鉴定。目前国内常用的智力测验有韦氏智力量表修订版和中国比内智力测验,韦氏智力量表有幼儿、儿童和成人等版本,分别用于相应年龄段的人群,不过这些测验修订的时间已相当久远,常模标准已失去时效性。近年来有人编制了一些本土化的智力测验,如姚树桥编制的《中华成人智力量表》(2007年)适用于16岁以上成人,赵介城编制的《中国少年智力量表》(2007年)适用于10~15岁少年,程灶火编制的《华文认知能力量表》(2006年)适用于5~80岁的人群。有些单位用瑞文测验和画人测验代替成套智力测验,笔者认为不妥,因为这些单项测验只能测量某种认知能力,不能对个体智力水平做全面准确的估计。在智力低下或残疾评定的同时还必须评定社会功能状况,国内也有这方面的量表,如姚树桥编制的《儿童适应行为量表》,龚耀先编制的《智残评定量表》。记忆测验在神经精神科中也有广泛的用途,尤其是老年痴呆的早期诊断,国内常用的成套记忆测验有龚耀先修订的《韦氏记忆量表》、许淑连编制的《临床记忆量表》和程灶火编制的《多维记忆评估量表》。除上述成套认知功能测验外,还有一些简易认知功能测验或评定量表,如简易精神状况检查(MMSE)、长谷川痴呆评定量表和临床痴呆评定量表。

2.人格测验或量表 人格是一个人的总体精神面貌,决定个体的行为方式和生活态度,许多精神疾病的发生发展与个体的人格特征有密切的关系,许多疾病可以导致个体的人格改变,因此使人格测验或量表成为精神科最常用的心理测验。人格测验分两类:问卷式的人格测验和投射式的人格测验,前者包括明尼苏达多项个性调查表(MMPI)、艾森克个性问卷(EPQ)、十六种人格因素调查表(16PF)和加州心理调查表(CPI),后者包括罗夏测验(RT)、主题统觉测验(TAT)和语句填充测验,这些测验在国内都有相应的修订本。近年也有国内学者编制的本土化人格问卷,如王登峰编制的《中国人个性问卷》。

3.症状评定量表 症状评定量表主要是对精神症状进行量化评定,客观地反映症状的严重程度,可以协助临床诊断、评定病情的严重程度和评定各种治疗的效果。症状评定的数量很多,但可以按某些属性分类,按评定者性质可分成自评量表和他评量表,前者为患者本人对照量表条目报告自己的行为表现和内心感受,实施方便经济,而且还能反映别人观察不到的症状,如焦虑自评量表;后者是医务人员对照量表条目,结合与患者和知情人访谈资料和观察资料对症状出现的频度和严重程度做出量化评定,评定结果更客观全面,如汉密尔顿焦虑量表。按内容可分为综合评定量表和专项评定量表,前者是对多方面的心理问题或精神症状进行评定,如90项症状自评量表(SCL-90),简明精神症状评定量表(BPRS),康奈尔医学指数(CMI);后者是对某一特定领域的心理问题或症状进行评定,如评定抑郁症状的量表有抑郁自评量表(SDS)、Beck抑郁调查表(BDI)、流调中心用抑郁量表(CES-D)、汉密尔顿抑郁量表(HAMD)和老年抑郁量表(GDS)等,评定焦虑症状的量表有焦虑自评量表(SAS)、社交焦虑量表、汉密尔顿焦虑量表(HAMA)、状态-特质焦虑问卷(STAI)和Beck焦虑调查表(BAI)等;还有其他许多评定特定领域问题或症状的量表,如自杀态度问卷(QSA)、饮酒问卷、Bech-Rafaelsen躁狂量表(BRMS)、儿童孤独量表和多伦多述情障碍量表等。

4.社会功能评定量表 社会功能评定量表主要评定个体的社会功能,如婚姻功能、家庭功能、人际关系、生活质量和学习工作能力等,这些都是评定精神疾病严重程度的重要指标,也是评定治疗效果和康复状况的重要指标。这类量表很多,如生活满意度评定量表、中国人婚姻质量问卷、家庭功能评定和大体功能评定量表等,可以根据需要选择使用。

5.其他 有调查发病因素的量表,如生活事件量表、养育方式问卷和社会支持量表等,有调查发病中介因素的量表,如应付方式问卷、防御方式问卷和认知偏差问卷等。