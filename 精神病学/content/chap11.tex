\chapter{神经症及癔症}

\section{概  述}

神经症(neuroses),旧称神经官能症,是以焦虑、抑郁、恐惧、强迫、疑病症状或神经衰弱症状为突出症状的,多种症状组合的一组精神障碍。患者有多种躯体或精神上的不适感,没有可以证实的客观器质性病变,与患者的现实处境不相称,但患者对存在的症状感到痛苦和无能为力,无持久的精神病性症状,自知力完整或基本完整,求治心切。患者病前多有一定的易患素质和个性特征,疾病的发生与发展常受心理社会(环境)因素的影响,病程多迁延,进入中年后症状常可缓解或部分缓解。

\subsection{共性}

虽然它不是指某一特定的疾病单元,而是包括各自不同的病因、发病机制、临床表现、治疗反应、病程与预后的一组精神障碍,但也存在一些共性。

\subsubsection{起病常与社会心理因素有关}

许多研究表明,神经症患者在病前较他人更多或更易遭受应激性生活事件。其特点为:强度常不十分强烈,但为多个事件反复发生,持续时间很长;应激性事件对患者往往具有某种独特的意义或患者对此特别敏感,且社会心理因素多为“变形”的,或是通过个性放大、变形了的;患者对心理困境或冲突有一定的认识,但常不能将自己解脱出来;应激性事件不仅来源于外界,更多地源于患者内在的心理欲求与对事件的不良认知。他们常常忽略和压抑自己的需求以适应环境,但又总是对他人和自己的作为不满,总是生活在遗憾和内心冲突之中。

\subsubsection{患者病前常有一定的易患素质和人格特征}

其个性特征常损害人际交往过程,导致生活中产生更多的冲突与应激。患者的个性特征一方面决定着个体罹患神经症的难易程度。如巴甫洛夫认为,神经类型为弱型,或强而不均衡型者易患神经症;Eysenck等认为,个性古板、严肃、多愁善感、焦虑、悲观、保守、敏感、孤僻的人易患神经症。另一方面,不同的个性特征可能与所患神经症亚型有关。如有强迫型人格特征者易患强迫症,有表演型人格特征者易患癔症,有A型行为倾向者易患焦虑症等。

\subsubsection{症状没有相应的器质性病变为基础}

神经症分类已不再包括存在“器质性”依据的“神经症样综合征”。当然,没有相应的器质性病变基础只是相对的,绝对的功能性症状是不存在的,异常的精神活动必然有异常的物质活动为基础。所谓的“功能性”是指目前科学技术水平还未能发现的、肯定的、相应的病理学和组织形态学变化。

\subsubsection{社会功能相对较好}

多数神经症患者的社会功能是较好的,他们一般生活能自理,勉强坚持工作或学习,他们的言行通常都保持在社会规范允许的范围内。但与正常人或与病前相比,其社会功能只能是相对较好,他们的工作、学习效率和适应能力均有不同程度的减退。有些神经症患者,社会功能受损可能相当严重,如严重的疑病症患者、某些慢性强迫症患者等。

\subsubsection{一般没有明显或持续的精神病性症状}

极少数患者可能短暂出现牵连观念、幻听等症状,但绝非主要临床相,明显或持续的精神病性症状罕见。个别强迫症患者的强迫行为可能显得非常古怪,但患者能就此作出心理学上的合理解释;某些疑病症患者的疑病观念可能达到妄想的程度。

\subsubsection{一般自知力完整,有求治要求}

多数神经症患者在疾病发作期亦保持较好的自知力,他们的现实检验能力通常不受损害。患者能识别他们的精神状态是否正常,哪些属于病态。他们常对病态体验有痛苦感,有摆脱疾病的求治欲望,一般能主动求治。

\subsection{分类}

国际疾病分类第十版(ICD-10)和《美国精神疾病诊断与统计手册第四版》(DSM-Ⅳ)抛弃了神经症这一术语。但将与神经症这一总的概念有相对稳定关系的几种神经症亚型,通过改变名称或类别,实质上在分类系统中保留了下来。我国的精神疾病分类体系中,保留了神经症这一疾病单元,但将抑郁性神经症归类于心境障碍,并将癔症单列出来。

CCMD-3将神经症分为:

1 恐惧症

1.1 场所恐惧症

1.2 社交恐惧症(社会焦虑恐惧症)

1.3 特定的恐惧症

2 焦虑症

2.1 惊恐障碍

2.2 广泛性焦虑

3 强迫症

4 躯体形式障碍

4.1 躯体化障碍

4.2 未分化躯体形式障碍

4.3 疑病症

4.4 躯体形式自主神经紊乱

4.4.1 心血管系统功能紊乱

4.4.2 高位胃肠道功能紊乱

4.4.3 低位胃肠道功能紊乱

4.4.4 呼吸系统功能紊乱

4.4.5 泌尿生殖系统功能紊乱

4.5 持续性躯体形式疼痛障碍

4.6 其他或待分类躯体形式障碍

5 神经衰弱

6 其他或待分类的神经症

\subsection{流行病学资料}

国内外的调查均显示,神经症是一组高发疾病。WHO根据各国和调查资料推算:人口中5%~8%有神经症或人格障碍,是重性精神病的5倍。我国1982年进行的12个地区精神疾病流行病学调查资料显示:神经症的总患病率为2.2%;女性高于男性;以40~44岁年龄段患病率最高,初发年龄最多为20~29岁年龄段;文化程度低、经济状况差、家庭气氛不和睦者患病率较高。我国1993年7地区的调查结果为:神经症患病率为1.5%。神经症的总患病率国外报道在5%左右,比国内高,差异的原因可能与样本的构成、诊断标准、东西方社会文化差异等因素有关。

\subsection{诊断与鉴别诊断}

\subsubsection{CCMD-3诊断标准}

神经症是一组主要表现为焦虑、抑郁、恐惧,强迫、疑病症状,或神经衰弱症状的精神障碍。本病有一定人格基础,起病常受心理社会(环境)因素影响。症状没有可证实的器质性病变作基础,与患者的现实处境不相称,但患者对存在的症状感到痛苦和无能为力,自知力完整或基本完整,病程多迁延。

1.症状标准 至少有下列1项:①恐惧;②强迫症状;③惊恐发作;④焦虑;⑤躯体形式症状;⑥躯体化症状;⑦疑病症状;⑧神经衰弱症状。

2.严重标准 社会功能受损或无法摆脱的精神痛苦,促使其主动求医。

3.病程标准 符合症状标准至少3个月,惊恐障碍另有规定。

4.排除标准 排除器质性精神障碍、精神活性物质与非成瘾物质所致精神障碍、各种精神病性障碍如精神分裂症、偏执性精神障碍、心境障碍等。

神经症的诊断标准包括总标准与各亚型的诊断标准。在作出各亚型的诊断之前,任一亚型首先必须符合神经症总的诊断标准。

\subsubsection{鉴别诊断}

神经症的症状在精神症状中特异性最差,几乎可以发生于任一种精神疾病和一些躯体疾病中,因此在作出神经症的诊断之前,常需排除以下疾病。

1.器质性精神障碍 各种神经症的症状均可见于感染、中毒、物质依赖、代谢或内分泌障碍及脑器质性疾病等多种躯体疾病之中,尤其在疾病的早期和恢复期最常见,此时不能诊断为神经症,称为神经症样综合征。器质性精神障碍的神经症样综合征具备:①生物源性的病因,如脑的器质性病变,躯体疾病的存在及其引起的脑功能性改变,依赖或非依赖性精神活性物质应用等;②脑器质性精神障碍的症状,如意识障碍(最常见为谵妄)、智能障碍、记忆障碍、人格改变等;③可有精神病性症状,如幻觉、妄想、情感淡漠等。通过详细询问病史、系统的体格检查和必要的实验室检查可以鉴别。

2.精神病性障碍 精神病性障碍中最常需要鉴别的是精神分裂症。一些精神分裂症患者早期常表现为神经症样症状,如头痛、失眠、学习工作效率下降、一些情绪变化,或出现一些强迫症状,易误诊为神经症。鉴别的要点是,对有神经症症状的患者,要仔细辨别有无精神分裂症的症状,尤其是易忽略的阴性症状,如懒散、孤僻、情感变淡漠、意志力减退等;分裂症患者常漠视自身症状,缺乏治疗要求或求治心不强烈;分裂症患者常缺乏现实检验能力,社会功能损害相对较重,而幻觉、妄想等阳性症状的存在则更使分裂症的诊断易于确定。

3.心境障碍 尤其是抑郁发作的患者,常伴有焦虑、强迫以及其他神经症的症状。此时的鉴别要点是心境障碍患者以抑郁(或躁狂)为主要临床相,其他症状大多继发于抑郁(或躁狂),而且情感症状程度严重,社会功能受损明显;而神经症的患者虽然也可有抑郁情绪,但大多程度轻、持续时间较短,不是主要临床相,多继发于心因或其他神经症症状。

4.应激相关障碍 神经症症状的发生与发展常常不完全取决于精神应激的强度,而与患者的素质和人格特征有关。起病与生活事件之间不一定有明显关联,因而其致病因素常不为患者所意识,病程常迁延或反复发作。而应激相关障碍的致病因素常为重大的生活事件,症状则是个体对应激性事件的直接反应,患者常能意识到症状的发生和发展与事件有关,病程多短暂,少有反复发作。

5.人格障碍 神经症的发生与发展常经历一个疾病过程,健康与疾病两个阶段明显不同;而人格障碍则是自幼人格发展的偏离常态,没有正常与异常的明显分界。人格障碍不是神经症发生的必备条件,如果神经症症状继发于人格障碍,可以下两个诊断。

应与各亚型重点鉴别的疾病见各亚型部分。

\subsection{治疗}

神经症的治疗以心理治疗为主,在一定治疗阶段可以有选择地配合药物治疗。一般来说,药物治疗对于控制神经症的症状是有效的。但由于神经症的发生与心理社会应激因素、个性特征有密切关系,可因生活事件的出现而反复发作,病程常迁延波动,成功的心理治疗可能更重要。心理治疗不但可以缓解症状、加快治愈过程,而且还能帮助患者学会新的应付应激的策略和处理未来新问题的技巧。这种结局显然对消除病因、巩固疗效是至关重要的,也是药物治疗所无法达到的。心理治疗方法的选择取决于患者的人格特征、疾病类型以及治疗者对某种心理治疗方法的熟练程度与经验。

药物治疗系对症治疗,可针对患者的症状选药。药物治疗的优点是控制靶症状起效较快,尤其是早期与心理治疗联合应用,有助于缓解症状,提高患者对治疗的信心,促进心理治疗的效果。但用药前需向患者说明所用药物的起效时间,及治疗过程中可能出现的不良反应,使其有充分的心理准备,以增加治疗的依从性;同时,也应该强调对于神经症的治疗以心理治疗为主,药物治疗为辅,否则会降低患者自我改变和调整动机。

\section{焦 虑 症}

\subsection{概述}

焦虑症(anxiety disorder)又称焦虑性神经症(anxiety
neurosis),是一种以广泛和持续性焦虑或反复发作的惊恐不安为主要特征,常伴有自主神经紊乱、肌肉紧张与运动性不安等症状的神经症。临床分为广泛性焦虑(generalized
anxiety disorder, GAD)与惊恐障碍(panic disorder, PD)两种临床类型。

自Freud(1895)首次提出焦虑性神经症的概念以来,焦虑症曾有许多别称,如心脏神经官能症、战士心脏(soldier's
heart)、神经循环衰弱、血管运动性神经症等,这是医学发展的不同阶段及不同的角度对焦虑症的理解。随着有效的抗焦虑药物问世,以及对焦虑状态的深入研究,在上世纪80年代初,逐渐形成了广泛性焦虑和惊恐障碍的诊断和分类体系。

\subsection{流行病学}

由于诊断标准和调查方法的不同,各国、各地区及不同时代的焦虑症发病率和患病率均不相同。1982年全国12地区的调查显示,在15~59岁的人口中,焦虑症的时点患病率为0.148%,占全部神经症病例的6.7%。1993年7地区的调查显示,焦虑症的时点患病率小于0.134%。2001年浙江省15岁以上的人群中惊恐障碍的时点患病率为0.176%,其中女性为0.27%,男性为0.076%;广泛性焦虑为0.489%,其中女性为0.68%,男性为0.286%。美国国家共病调查(National
Comorbidity Survey,
NCS)的数据表明,广泛性焦虑的终生患病率为5.1%,其中女性为6.6%,男性为3.6%;惊恐发作的终生患病率为3.5%,其中女性为5.1%,男性为1.9%。

广泛性焦虑症大多起病较早,平均21岁,早发者常与儿童期恐惧、婚姻/性生活紊乱有关,晚发者多与应激性事件、单身、失业有关;而惊恐障碍的起病年龄呈双峰分布,15~24岁为最高峰点发病率年龄段,45~54岁为第二峰点年龄段,65岁以后起病者罕见。两种临床类型的女性发病率均高于男性,约为2:1。

\subsection{病因}

\subsubsection{生物学因素}

1.遗传因素 ①家系调查:焦虑症具有家族聚集性。Noyes(1987)与Mendenhvicz(1993)分别报道广泛性焦虑与惊恐障碍先证者的一级亲属中本病的患病率远高于对照组。当然,这种家族聚集性,一方面与遗传有关,另一方面与环境因素(父母的性格和家庭教养方式等)有关。②双生子研究:早期的研究表明同卵双生子的焦虑障碍同病率远高于异卵双生子(Slater等,1969,1972)。Kendler等(1992)的一项对1033对女性双生子研究证实广泛性焦虑具有遗传倾向,但遗传度仅为30%(惊恐障碍的遗传度为30%~40%),认为女性患者中环境因素可能决定了广泛性焦虑遗传易感性的表达。③分子遗传学研究:最近,在95%的惊恐障碍人群(7%整体人群)中发现的15号染色体(15q24~q26,
DUP25)或许是惊恐障碍遗传易感性的有力证据(Gratacós, 2001)。

2.生化因素 惊恐障碍和广泛性焦虑存在多种神经递质、细胞因子及其相关受体的异常改变和脑功能的变化,但对于这些异常改变,至今尚不能明确与疾病的因果关系。对焦虑的生物学本质的理解,部分得益于抗焦虑药物的作用机制。这些作用机制主要涉及三大方面:①γ-氨基丁酸(GABA)受体-苯二氮䓬
受体-氯化物通道复合物;②去甲肾上腺素蓝斑核团和相关的脑干核团;③5-羟色胺系统,特别是中缝核及其投射系统(Ebert,
2002)。

关于焦虑形成的模式和假说主要有:

(1)乳酸盐假说:乳酸盐可引起异常的代谢活动。静脉注射乳酸钠、吸入5%~35%CO\textsubscript{2}
或应用重碳酸盐可引起惊恐发作。

(2)5-羟色胺能假说:焦虑的发生可能与5-HT突触后受体过度反应及5-HT\textsubscript{1A}
受体的低敏感性有关。主要影响5-HT能神经递质系统的药物(如氯丙咪嗪及SSRI等)对焦虑有较好的疗效,而促进5-HT释放的物质(芬氟拉明)可加剧和诱发焦虑,丁螺环酮主要是通过作用于海马的5-HT\textsubscript{1A}
受体使5-HT功能下调而产生抗焦虑效应。

(3)去甲肾上腺素能假说:焦虑症患者存在去甲肾上腺素(NE)能活动的增强现象。蓝斑含有整个中枢神经系统50%以上的NE神经元,焦虑状态时脑脊液中NE的代谢产物(MHPG)显著增加;儿茶酚胺(肾上腺素和NE)能诱发焦虑,并能诱发有惊恐发作史病人的惊恐发作。NE水平由蓝斑核的胞体及α\textsubscript{2}
受体调节,α\textsubscript{2}
受体拮抗剂如育亨宾(yohimbine)能使NE增加而致焦虑,而α\textsubscript{2}
受体激动剂可乐定对焦虑治疗有效。

(4)γ-氨基丁酸(GABA)假说:焦虑患者可能是由于苯二氮䓬
受体功能不足或缺乏内源性配体所致。苯二氮䓬
类药物是苯二氮䓬
受体激动剂,具有良好的抗焦虑作用,而苯二氮䓬
受体的拮抗剂(flumazenil)和逆转激动剂(β-carbdine)可诱发焦虑。

3.神经解剖 与广泛性焦虑联系最紧密的脑区是边缘系统。Gray(1982,
2000)确定了一个大脑环路,称之为行为抑制系统(BIS)环路,这个环路始于边缘系统的脑隔膜区和海马回区域,并延伸到额叶。它会被具有威胁性事件所产生的信号激活,使人产生焦虑、警觉性增高等反应。German等(1989)基于Klein的现象学模型提出惊恐障碍的神经解剖假说:惊恐发作与脑干特别是蓝斑密切相关;预期焦虑与边缘叶的功能损害有关;恐惧性回避与皮层的认知和意识活动有关。

\subsubsection{社会心理学因素}

1.应激性事件 应激性事件会触发产生焦虑的生物和心理易感性,威胁性应激事件尤与焦虑障碍有关(Finlay-Jones
\& Brown,
1981)。约80%的惊恐障碍患者起病之前常存在一个重要的应激性事件(Rappe,
1990);经历预料不到的负性事件(如父/母早亡、强奸、战争)、慢性应激源(家庭/婚姻功能紊乱)、缺乏温暖和回应的过度保护与广泛性焦虑的发病有关(Semple等,2005)。

2.个性特征 应激性事件可诱发焦虑反应,但反应的强弱程度与个体的个性特征有关。部分焦虑症患者存在敏感、易紧张、不安全感、过分自责及自卑等焦虑倾向的个性特征,与这种性格特征关系密切的焦虑称为特质性焦虑。此外,胆小羞怯、缺乏自信或躯体情况不佳者,对心理社会应激的应对能力较差,也易发生焦虑。

3.精神分析理论 精神分析理论认为焦虑源于内在的心理冲突,是童年或少年期被压抑在潜意识中的冲突在成年后被激活,从而形成焦虑。多项研究表明,早年的负性经历是其后产生惊恐发作和广泛性焦虑的原因之一(Brown
\& Harris, 1993)。

4.行为主义理论 行为主义理论认为焦虑是对某些环境刺激的恐惧而形成的一种条件反射,而继发的行为反应(如回避行为)是焦虑得以持续的重要原因。

5.认知理论 Beck(1963)认为焦虑是对面临危险的一种反应,相应的自动思维与认知歪曲是导致患者产生焦虑情绪的重要原因。惊恐障碍患者最常见的认知歪曲是选择性关注身体上的负性感觉(选择性负性关注,negative
filter)和灾难化思维(catastrophizing),在躯体症状的参与下,形成恶性循环,最终导致惊恐发作。而广泛性焦虑患者存在对潜在威胁的敏锐觉知,当面临事件时,灾难化的歪曲认知促发他们的警觉和担心,伴随着生理改变而最终焦虑形成(J.
S. Beck, 1995; Barlow, 1996, 2002)。

\subsection{临床表现}

任何人都可以体验到焦虑的情绪,适度的焦虑对个体的生存具有积极意义,但焦虑症患者的焦虑与正常人的焦虑不同,其具有以下特点:①是一种情绪状态:表现为害怕、惊恐或恐惧、提心吊胆;②痛苦情绪(濒死感、失常感);③指向未来,但危险实际不存在;④与诱发因素不相称;⑤伴躯体不适、精神运动、自主神经症状(Lewis,
1967)。根据临床症状和病理特点,CCMD-3将焦虑症分为广泛性焦虑与惊恐障碍两种临床类型。

\subsubsection{广泛性焦虑}

广泛性焦虑又称慢性焦虑症,是焦虑症最常见的表现形式。常缓慢起病,以经常或持续存在的焦虑为主要临床相。具有以下表现:

1.精神性焦虑 精神上的过度担心是焦虑症状的核心,表现为对未来可能发生的、难以预料的某种危险或不幸事件的经常担心。有的患者不能明确意识到他担心的对象或内容,而只是一种提心吊胆、惶恐不安的强烈的内心体验,称为自由浮动性焦虑(free-floating
anxiety,又称漂浮焦虑)。有的患者担心的也许是现实生活中可能发生的事情,但其担心、焦虑和烦恼的程度与现实很不相称,称为忧虑性期待(apprehensive
expectation),此为广泛性焦虑的核心症状。如担心家人出门会遭遇车祸等,患者常有恐慌的预感,终日心烦意乱、忧心忡忡,坐卧不宁,似有大祸临头之感。

2.躯体性焦虑 表现为运动不安与多种躯体症状。运动不安:可表现搓手顿足,不能静坐,不停地来回走动,无目的的小动作增多。有的病人表现舌、唇、指肌的震颤或肢体震颤。躯体症状:胸骨后的压缩感(胸闷)是焦虑的一个常见表现,常伴有气短、呼吸困难。肌肉紧张:表现为主观上的一组或多组肌肉不舒服的紧张感,严重时有肌肉酸痛,多见于胸部、颈部及肩背部肌肉,紧张性头痛也很常见。自主神经功能紊乱:表现为心动过速、皮肤潮红或苍白,口干,便秘或腹泻,出汗,尿意频繁等症状。

3.觉醒度增高 表现为过分的警觉,对外界刺激敏感,易出现惊跳反应,在入睡前与醒觉前尤易出现;注意力难于集中,易受干扰;难以入睡、睡中易惊醒;情绪易激惹;感觉过敏,有的病人能体会到自身肌肉的跳动、血管的搏动、胃肠道的蠕动等。

4.其他症状 患者感到头昏、行走不稳、虚弱或头晕;有的患者可出现早泄、阳痿、月经紊乱等症状。

\subsubsection{惊恐障碍}

惊恐障碍又称急性焦虑障碍。其特点是发作的不可预测性和突然性,反应程度强烈,病人常体会到濒临灾难性结局的害怕和恐惧,终止亦迅速。

1.惊恐发作 典型的表现为:患者正在进行日常活动时,突然感到心悸,胸闷,呼吸困难或过度换气,喉头堵塞感,窒息感;同时出现强烈的惊恐体验,如濒死感、失控感或短暂的人格解体体验等;伴有头昏、眩晕、四肢麻木和感觉异常、出汗、肉跳、全身发抖或全身无力等自主神经功能紊乱症状。发作一般历时5~20分钟,很少超过1个小时,但不久又可突然再次发作。发作期间意识始终清晰,高度警觉,发作后仍心有余悸,担心再发,此时的焦虑体验不再突出,代之以虚弱无力的表现,需数小时到数天才能恢复。

2.预期焦虑(anticipatory
anxiety) 患者在反复出现惊恐发作之后的间歇期常担心再次发病,因而惴惴不安,也可出现自主神经活动亢进的症状。

3.求助和回避行为 惊恐发作时由于强烈的恐惧及对死亡的担心,患者常立即要求给予紧急帮助,故患者有反复多次的急诊经历。在发作间歇期,60%的患者因担心发病时得不到帮助,而主动回避一些活动与场所,如不愿单独出门或独自在家,不愿到人多的热闹场所等。

\subsection{诊断和鉴别诊断}

\subsubsection{CCMD-3的广泛性焦虑诊断标准}

广泛性焦虑指一种以缺乏明确对象和具体内容的提心吊胆,及紧张不安为主的焦虑症,并有显著的自主神经症状、肌肉紧张及运动性不安。患者因难以忍受又无法解脱,而感到痛苦。

1.症状标准

(1)符合神经症的诊断标准;

(2)以持续的原发性焦虑症状为主,并符合下列2项:①经常或持续的无明确对象和固定内容的恐惧或提心吊胆;②伴自主神经症状或运动性不安。

2.严重标准 社会功能受损,患者因难以忍受又无法解脱而感到痛苦。

3.病程标准 符合症状标准至少已6个月。

4.排除标准

(1)排除甲状腺机能亢进、高血压、冠心病等躯体疾病的继发性焦虑;

(2)排除兴奋药物过量、催眠镇静药物,或抗焦虑药的戒断反应,强迫症、恐惧症、疑病症、神经衰弱、躁狂症、抑郁症或精神分裂症等伴发的焦虑。

\subsubsection{CCMD-3的惊恐障碍诊断标准}

惊恐障碍是一种以反复的惊恐发作为主要原发症状的神经症。这种发作并不局限于任何特定的情境,具有不可预测性。

1.症状标准

(1)符合神经症的诊断标准;

(2)惊恐发作需符合以下4项:①发作无明显诱因、无相关的特定情境,发作不可预测;②在发作间歇期,除害怕再发作外,无明显症状;③发作时表现强烈的恐惧、焦虑及明显的自主神经症状,常有人格解体、现实解体、濒死恐惧或失控感等痛苦体验;④发作突然开始,迅速达到高峰,发作时意识清晰,事后能回忆。

2.严重标准 患者因难以忍受又无法解脱,而感到痛苦。

3.病程标准 在1个月内至少有3次惊恐发作,或在首次发作后继发害怕再发作的焦虑持续1个月。

4.排除标准

(1)排除其他精神障碍,如恐惧症、抑郁症或躯体形式障碍等继发的惊恐发作;

(2)排除躯体疾病如癫痫
、心脏病发作、嗜铬细胞瘤、甲状腺功能亢进或自发性低血糖等继发的惊恐发作。

\subsubsection{鉴别诊断}

1.躯体疾病所致焦虑 多种躯体疾病可出现焦虑症状。如库兴综合征、甲状腺功能亢进、甲状旁腺功能减退及嗜铬细胞瘤等内分泌系统疾病,心肌梗死、冠状动脉供血不足、阵发性心动过速及二尖瓣脱垂等心脏疾病,哮喘、慢性阻塞性肺病、高通气综合征等呼吸系统疾病,脑血管疾病、颞叶癫痫
等神经系统疾病,系统性红斑狼疮、贫血等都易于出现焦虑症状。临床上对初诊、年龄大、无心理应激因素、病前个性素质良好的患者,要高度警惕焦虑是否继发于躯体疾病。焦虑症的焦虑症状是原发的;凡继发于高血压、冠心病、甲状腺功能亢进等躯体疾病的焦虑称为焦虑综合征。

2.药源性焦虑 许多药物在中毒、戒断或长期应用后可致典型的焦虑障碍。如某些拟交感药物苯丙胺、可卡因、咖啡因;某些致幻剂及阿片类物质;长期应用抗高血压药、抗心律失常药、抗胆碱能药、抗帕金森病药、甲状腺素、激素、抗生素、抗抑郁药、抗精神病药物、镇静催眠药的撤药等等。根据服药史可资鉴别。

3.精神疾病所致焦虑 精神分裂症病人可伴有焦虑,只要发现有分裂症症状,就不考虑焦虑症的诊断;抑郁症是最多伴有焦虑的疾病,当抑郁与焦虑严重程度主次难以分清时,二者的鉴别依据是症状的严重程度和发生的先后次序。就目前的诊断体系而言,不论焦虑有多严重,只要达到抑郁症的诊断标准,就应该首先诊断为抑郁症;其他神经症伴有焦虑时,焦虑症状在这些疾病中常不是主要的临床相或属于继发症状。

4.共患病 焦虑症患者常合并多种疾病,美国国家共病调查发现的广泛性焦虑共病率较其他焦虑障碍高,不同时共患其他精神障碍的“单纯”广泛性焦虑仅占总广泛性焦虑的1/3。90%的终生广泛性焦虑患者具有另一种终生精神病学诊断。广泛性焦虑:常见的是抑郁症和恶劣心境,其次是物质滥用、单纯恐惧、社交恐惧及“其他”躯体情况。广泛性焦虑患者当前共患抑郁症的发生率为39%,恶劣心境为22%。同样,具有终生广泛性焦虑诊断的患者中,共患抑郁症占62%,共患恶劣心境占39%。惊恐障碍:常见的是社交恐惧症、抑郁障碍、其他焦虑障碍(如社交恐惧症、强迫症等)、酒精和物质滥用及内科疾病。

\subsection{治疗}

\subsubsection{治疗原则}

焦虑症的治疗原则:①综合治疗:综合药物和心理治疗,有助于全面改善患者的预后;②全程治疗:焦虑症是一种慢性且易复发的疾病,应当采取全程治疗的原则。急性期控制症状,尽可能达到临床痊愈。巩固及维持期恢复患者社会功能和预防复发;③个体化治疗:全面考虑患者的年龄特点、躯体状况、既往药物治疗史、有无并发症等,因人而异地个体化合理治疗。

\subsubsection{药物治疗}

1.药物治疗策略 焦虑症的急性期治疗持续4~12周,巩固期至少2~6个月,广泛性焦虑需要维持治疗至少12个月以防止复发,而惊恐障碍需要维持更长的时间,常推荐为2年。尽可能单一用药,足量、足程治疗。

2.药物种类 我国食品药品监督管理局(SFDA)批准治疗广泛性焦虑的药物有文拉法辛缓释胶囊、丁螺环酮、曲唑酮、多塞平;冶疗惊恐障碍的药物有帕罗西汀、艾司西酞普兰、氯米帕明。

与三环类药物相比,SSRIs(帕罗西汀、西酞普兰和艾司西酞普兰等)、SNRIs(文拉法辛、度洛西汀)类药物的副反应较轻,常被推荐为治疗广泛性焦虑与惊恐障碍的一线药物。

苯二氮䓬
类药物治疗广泛性焦虑与惊恐障碍的疗效已经在早期多项研究中得到证实。但此类药物对广泛性焦虑共病的抑郁症状没有疗效,且容易出现过度镇静、记忆受损、精神运动性损害等不良反应,以及容易出现依赖或滥用,并在停药后易出现戒断症状,目前不推荐为一线药物。通常建议:在治疗初期一线药物疗效尚未表现出来时可以考虑合并苯二氮䓬
类药物,但最长使用2~3周,随后逐渐减药、停药。

其他药物:5-TH\textsubscript{1A}
受体部分激动剂(丁螺环酮、坦度螺酮)、5-HT受体拮抗和再摄取抑制剂(曲唑酮)、三环类和杂环类药物(TCAs)、β受体阻滞剂(普萘洛尔)、小剂量的非典型抗精神病药(惊恐障碍患者应避免使用)等,但这类药物皆非一线抗焦虑治疗药物。

\subsubsection{心理治疗}

焦虑症的心理治疗有多种,最常使用的是认知-行为治疗(cognitive-behavioral
therapy, CBT)、精神动力性心理治疗(psychodynamic psychotherapy,
PPT)、支持性治疗与生物反馈治疗等。

1.认识-行为治疗 众多对照研究显示,认识-行为治疗对广泛性焦虑与惊恐障碍的疗效至少与丙咪嗪相当,并且在预防复发方面具有药物治疗无可比拟的优势(Barlow等,2000)。完整的治疗分为引入治疗、治疗及结束治疗三个阶段。治疗中,治疗师主要教会患者识别引起焦虑的自动思维、中间规则及核心信念,并通过认知重建与行为矫正的方法调整患者的认知偏差,缓解焦虑症状,获得更持久的疗效。治疗一般持续12~16次(1次/周),对于伴有轴Ⅱ诊断(人格障碍)的患者,治疗时间可能会有所延长。

2.精神动力性心理治疗 这是推荐用于治疗广泛性焦虑与惊恐障碍的短程疗法。它主要针对患者过度的内心冲突,帮助协调超我、自我及本我的关系,从而能够自我控制情感症状和异常行为,同时能更好地处理一些应激性境遇。治疗一般持续10~20次(1次/周),少数患者可达40次。在治疗结束前一般安排2~3个月的随访,其间逐步拉长会谈见面的间歇期。

\subsection{病程与预后}

焦虑症一般缓慢或亚急性起病,病程为慢性并呈波动性,常在应激时病情加重。焦虑症的预后在很大程度上与个体素质有关,如处理得当,大多数患者能在半年内好转。一般来说,病程较短、症状较轻、病前社会适应能力完好、病前个性缺陷不明显及无共患疾病者,预后较好,反之预后不佳。也有人认为,有晕厥、激越、现实解体、癔症样表现及自杀观念者,常提示预后不佳。

\section{恐 惧 症}

\subsection{概述}

恐惧症(phobia),又称恐惧焦虑障碍,曾称恐惧症,是一种以过分和不合理地惧怕外界客体或处境为主的神经症。患者明知这种恐惧反应是过分的或不合理的,但在相同场合下仍反复出现,难以控制。恐惧发作时常伴有明显的焦虑和自主神经症状。患者极力回避恐惧的客观事物或情境,或是带着畏惧去忍受,因而影响其正常的社会功能。临床分为场所恐惧症(agoraphobia)、社交恐惧症(social
phobia,或社交焦虑症)和特定的恐惧症(specific phobia)三种临床类型。

在古代,人们就已认识到恐惧现象,如罗马帝国的奥古斯都大帝就害怕独自一人呆在黑暗的地方,但最早对恐惧现象进行系统性医学研究的可能是18世纪的Le
Camus(Errera,
1962)。现代精神病学关于恐惧概念的形成,则归功于Westphal(1872)。在20世纪60年代,根据行为治疗对恐惧症的不同疗效反应,建立起目前的三种类型恐惧症的诊断和分类体系(Marks,
Gelden, 1966; Klein, 1964)。

\subsection{流行病学}

由于应用的诊断标准和调查方法的不同,各国、各地区及不同时代的恐惧症患病率差异较大。1982年全国12地区的调查显示,在15~59岁的人口中,恐惧症的时点患病率为0.059%,占全部神经症病例的2.7%。1993年7地区的调查显示,恐惧症的时点患病率小于0.134%。2001年浙江省15岁以上的人群中,无惊恐障碍的场所恐惧症3个月的时点患病率为0.35%,其中女性为0.677%;社交恐惧症为0.045%;特定的恐惧症为1.201%,其中女性为1.839%,男性为0.512%(石其昌等,2005)。美国国家共病调查(NCS)的数据表明,场所恐惧症的终生患病率为6.7%,其中女性为9.0%,男性为4.1%;社交恐惧症为13.3%,其中女性为15.5%,男性为11.1%;特定的恐惧症为11.3%,其中女性为15.7%,男性为6.7%(Merikangas,
2005)。

特定的恐惧症的平均发病年龄为16.1岁,且发病年龄因不同的恐惧对象而不同。动物恐惧症常起病于少儿期,恐高症开始于青春期,情境恐惧发病晚,大多在青春后期。社交恐惧症多起病于童年后期或少年早期,5岁和11~15岁有两个峰点而呈双峰分布,但患者常到30多岁才就诊。场所恐惧症的发病年龄较其他恐惧症晚,首次发作年龄段较宽(15~35岁)且呈双峰分布,晚年的场所恐惧症状可能继发于身体虚弱,且与害怕躯体问题恶化或发生事故有关。除社交恐惧症男女比例相当外,其他恐惧症女性发病率均高于男性(Davidson等,1993)。

\subsection{病因}

\subsubsection{生物学因素}

1.遗传因素 恐惧症可能与遗传因素有关。Fyer(1993,
1995)等研究发现,31%的特定的恐惧症患者的一级亲属有同样的问题;社交恐惧症患者亲属中社交恐惧症的发病率比一般人群高。Kendler等(1992a)的一项对2000多对女性双生子的研究提出不同类型的恐惧症具有共同的病因模式:一定的遗传易感性与非特异性环境因素的相互作用。

2.生化因素 与其他焦虑障碍不同,迄今为止,人们对恐惧症的生理生化因素所知甚少。随着对社交恐惧症有效药物治疗的发现,激发了对这一障碍的生物学研究。

Tancer等(1993)研究发现:约50%的社交恐惧症患者出现恐惧症状时有血浆肾上腺素含量的升高,但在一般情况下,给患者快速静脉滴注肾上腺素并不引起社交恐惧现象;同时,乳酸盐静脉滴注可引起惊恐障碍患者的惊恐发作,对社交恐惧症患者几乎毫无影响,表明社交恐惧症与广泛性焦虑与惊恐障碍存在不同的生化背景。Levin等(1986)及Rapee等(1992)研究发现,社交恐惧症存在多巴胺能异常而提出多巴胺能假说。另外,Liebowitz等(1992)研究发现,肾上腺素能的过度活动可能参与操作性焦虑的形成和发展,而对广泛性社交焦虑障碍却无影响。

\subsubsection{社会心理因素}

1.人格因素 场所恐惧症患者常具有依赖性强、倾向于回避问题的特征,这种依赖性可能与幼年期的过度保护有关。回避型人格特征在社交恐惧症患者中普遍存在,Heimberg(1993)甚至认为回避型人格障碍也许是极端严重社交恐惧症导致的结果。

2.创伤性经验 ①自身的经验:很多特定的恐惧都源于特殊的创伤性事件,亲身经历的危险和痛苦经验容易导致个体对特定事物和情境的高度警觉反应。②替代性经验:观察到他人的创伤性事件或承受强烈恐惧的经历,也会形成恐惧症。③学习的经验:家长不断告诫儿童某种动物的危险性,可能引起儿童对这种动物的恐惧(成年期特定的恐惧症常常是童年恐惧的延续)。另外,报纸、广播、电视等媒体报道及网络也是重要的“学习”途径。

3.精神分析理论 精神分析认为特定的恐惧症的恐惧与明显的外界刺激无关,而是与内在的焦虑有关。此内在的根源通过压抑而被排除在意识之外,并通过置换将其附于外界物体上。对于场所恐惧症最初的焦虑发作,其产生是由潜意识的冲突所致,而冲突常与压抑的性或攻击冲动有关,这种冲动被起病环境间接促发。

4.行为理论 行为主义认为人类的许多行为(包括恐惧)都是在条件反射的基础上产生的。Watson(1920)通过著名的“Albert病例”提出条件反射理论来解释恐惧症的发生机制,认为恐惧症状的扩展和持续是由于症状的反复出现使焦虑情绪条件化,而回避行为则阻碍了条件化的消退。这也是行为治疗的理论基础。

5.认知理论 认知理论认为情绪困扰是由患者自身的不合理信念导致的,恐惧症患者对恐惧对象和情境存在多种不合理信念。社交恐惧症患者常常认为自己的行为表现不恰当或缺乏吸引力(贴标签,Labeling),并且坚信别人也能认识和注意到这一点(读心术,Mind
reading),患者同时又选择性注意到支持自己判断与评价的种种证据(选择性负性关注,negative
filter),而导致相应的焦虑发作;与场所恐惧症和特定恐惧症有关的不合理信念还有灾难化(catastrophizing)、情感推理(emotional
reasoning)等。

\subsection{临床表现}

正常人对某些事物或场合也会有恐惧心理,如毒蛇、猛兽、黑暗而静寂的环境等;同时,在一个人正常发育过程中,也都会经历短暂的社交羞怯和焦虑。但恐惧症患者的恐惧症状与正常明显不同,其具有以下特征:①某种客体或处境常引起强烈的恐惧,恐惧与处境不相称。②恐惧时常伴有明显的自主神经症状,如头晕、晕倒、心悸、心慌、战栗、出汗等。③对恐惧的客体或处境极力回避,或是带着畏惧去忍受。④患者明知这种恐惧是过分的、不合理的和不必要的,但无法控制。根据临床症状和病理特点,恐惧症分为场所恐惧症、社交恐惧症和特定的恐惧症三种临床类型。

\subsubsection{场所恐惧症}

场所恐惧症曾称旷野恐怖、广场恐怖,最早由Westphal在1872年描述的一种对大而开阔空间的病态恐惧。场所恐惧症的主要表现为对某些特定环境的恐惧。这些特定环境具有远离家、拥挤和受到限制的特点,具体包括空旷的广场、人多拥挤的公共场所、密闭的空间、公共交通工具上以及难以立即离开的场所等。当患者进入这类场所或处于这种状态时便感到紧张、不安,出现明显的头昏、心悸、胸闷、出汗等自主神经症状,严重时可出现惊恐发作、晕厥等。与其他恐惧症相比,场所恐惧症还具有其特有的症状:

1.惊恐发作 惊恐发作在场所恐惧症患者中出现极为频繁。超过1/3的场所恐惧症患者同时伴有惊恐发作,而30%~50%的惊恐障碍患者同时伴有场所恐惧,以至于在DSM-Ⅳ中,惊恐障碍的诊断价值甚至超过场所恐惧症本身。

2.头晕 在公共场所的头晕有时会成为患者的突出症状(Benedikt, 1870)。

3.回避 由于强烈的害怕、不安全感或痛苦的体验,常随之出现回避行为。严重者行为退缩,长期社会隔离。但是,有时甚至是一个小孩或一条小狗陪伴时,焦虑会明显减轻,甚至不出现对恐惧情境的回避。

4.预期焦虑 预期焦虑很常见,症状明显者可在进入恐惧场所前数小时就出现焦虑。

5.伴发症状 恐惧发作时还常伴有抑郁、强迫、人格解体等症状。

\subsubsection{社交恐惧症}

社交恐惧症又称社交焦虑障碍(social anxiety disorder,
SAD),其主要表现为对社交场合和人际接触的恐惧。社交恐惧症患者的核心症状为:在处于被关注并可能被评论的情境下,可产生不适当的焦虑。症状发作时可伴有脸红、发抖、恶心或尿急等自主神经症状,严重时可达到惊恐发作的程度。症状的存在可严重影响患者的生活质量及有效生命年。社交恐惧症患者常与回避性人格障碍共患,表现为自我评价低、害怕被批评、担心被人瞧不起等。部分患者常伴有突出的场所恐惧与抑郁障碍;部分患者还可能通过酒精等物质滥用来缓解焦虑而最终导致物质依赖。根据临床表现可分为:

1.广泛性社交焦虑障碍(generalized social anxiety
disorder) 占临床表现的多数,指患者在大多数社交场合都焦虑,包括社交场合操作性焦虑和与人交往的焦虑,操作性焦虑通常指对操作性事件的恐惧,例如面对公众讲话、在他人的注视下签署重要文件或支票、在公共场合吃东西等。与人交往的焦虑指怕赴约会、参加聚会等需要与人接触的社交场合的焦虑。此类患者常常害怕出门,不敢与人交往,甚至长期社会隔离。

2.非广泛性社交焦虑障碍(nongeneralized social anxiety
disorder) 指单纯的社交场合操作性焦虑。这些患者常常在非正式的社交场合很自在,但要在公共场合讲话或操作时就会感到窘迫或产生严重的焦虑。

有两种特定的社交恐惧症值得注意:①排泄恐惧:此类患者要么在公共厕所感到焦虑而无法排尿,要么常常感到急于排尿而害怕失禁。因此,他们将自己的活动限定在离厕所不远的范围内。少数患者的类似症状是关于排粪。②呕吐恐惧:此类患者担心自己会在公共场合呕吐,通常在公共汽车或火车上,当他们置身于这些环境时便感到焦虑和恶心。少数患者是担心别人会在这些地方呕吐。

\subsubsection{特定的恐惧症}

特定的恐惧症是指对广场恐惧症和社交恐惧症未包括的、特定的客体或情境感到强烈的、不合理的害怕或厌恶。当面对恐惧的物体或情境时,患者可出现明显的焦虑不安、恐慌,甚至出现惊恐发作,预期性焦虑也较为常见,回避行为依然突出。患者害怕的往往不是与这些物体直接接触,而是担心接触之后会产生可怕的后果。根据恐惧对象及临床特点,特定的恐惧症分成以下几种类型:

1.动物恐惧 害怕昆虫、蜘蛛、老鼠、蛇等。

2.自然环境恐惧 害怕雷电、黑暗、高处、临水等。

3.血液-注射-损伤恐惧 鲜血,暴露的伤口,接受注射、手术等情境引起的恐惧,但伴随的自主神经反应与其他恐惧症不同,在最初的心动过速之后便是心动过缓、面色苍白、眩晕、恶心甚至晕厥等血管迷走神经反应。

4.其他恐惧 有几类特殊的恐惧较为常见:①牙科治疗恐惧:约有5%的成人对牙科诊疗椅感到恐惧,严重者甚至回避所有的牙科治疗,以致满嘴龋齿。②飞行恐惧:与幽闭恐惧不同,患者对飞行而不是飞机的恐惧。有的飞行员在遭遇空中事故后可出现对飞行的恐惧。③疾病恐惧:患者反复担心自己可能患上癌症或其他严重的疾病,这种担心伴有对医院的回避,此与躯体形式障碍不同。

5.在DSM-IV系统中,特定的恐惧症有一个情境性恐惧类型,患者主要的害怕对象为一些特定的场景,如公共交通工具、隧道、桥梁、电梯或封闭的空间等,但在CCMD-3系统中,此类型归于场所恐惧症。

\subsection{诊断和鉴别诊断}

\subsubsection{CCMD-3恐惧症的诊断标准}

1.符合神经症的诊断标准。

2.以恐惧为主要临床相,符合以下各点:①对某些客体或处境有强烈恐惧,恐惧的程度与实际危险不相称;②发作时伴有自主神经症状;③有回避行为;④知道恐惧过分、不合理、不必要,但无法控制。

3.对恐惧情境和事物的回避必须是或曾经是突出症状。

4.排除焦虑症、精神分裂症、疑病症。

\subsubsection{场所恐惧症的诊断标准}

1.符合恐惧症的诊断标准。

2.害怕对象主要为某些特定环境,如广场、闭室、黑暗场所、拥挤的场所、交通工具(如拥挤的船舱、火车车厢)等。其关键的临床特征之一是过分担心处于上述情境时没有即刻能用的出口。

3.排除其他恐惧障碍。

\subsubsection{社交恐惧症的诊断标准}

1.符合恐惧症的诊断标准。

2.害怕对象主要为社交场合(如在公共场合进食或说话、聚会、开会,或怕自己做出一些难堪的行为等)和人际接触(如在公共场合与人接触,怕与他人目光对视,或怕在与人群相对时被人审视等)。

3.常伴有自我评价低和害怕批评。

4.排除其他恐惧障碍。

\subsubsection{特定的恐惧症的诊断标准}

1.符合恐惧症的诊断标准。

2.害怕对象是场所恐惧和社交恐惧未包括的特定物体或情境,如动物(如昆虫、鼠、蛇等)、高处、黑暗、雷电、鲜血、外伤、打针、手术或尖锐锋利物品等。

3.排除其他恐惧障碍。

\subsubsection{鉴别诊断}

1.正常人的恐惧 正常人对某些事物或场合也会有恐惧心理,如毒蛇、猛兽、黑暗而静寂的环境等。从个体发展的角度来看,合理的恐惧对于个体生存有重要意义,这能使个体避免接触到有危害的事物或情境。恐惧情绪正常与否的判断,一方面要看引起恐惧情境本身的性质和频率,另一方面要综合考虑患者出现这种恐惧的合理性、发生的频率、恐惧的程度、是否伴有自主神经症状、是否明显影响社会功能及是否存在回避行为等。一般而言,如果恐惧情绪影响到一个人的社会适应能力并给其带来痛苦,那就成为一种病态了。

2.与其他类型神经症的鉴别

(1)恐惧症与广泛性焦虑:两者都以焦虑为核心症状,但恐惧症的焦虑由特定的对象或处境引起,呈境遇性和发作性,而广泛性焦虑的焦虑常没有明确的对象,常持续存在,而且通常不伴有回避行为。

(2)社交恐惧症与惊恐障碍:虽然社交恐惧症有可能会出现惊恐发作,而惊恐障碍的发作也可能在社交情境下发生,但两者存在多方面的不同。认知方面:社交恐惧症所恐惧的是由于自己不适当的操作或他人过分关注而引起的窘迫;惊恐障碍恐惧的常常是身体上的伤害或精神上的失控。回避行为:社交恐惧症回避与人交往,宁愿独处;惊恐障碍回避独自外出或独自在家,常需人陪伴。躯体症状:社交恐惧症的脸红最为常见;惊恐障碍多见心慌、胸闷、呼吸困难等,较少出现脸红。诊断较为困难的是两者的重叠或共病。

(3)强迫症与恐惧症:强迫症的强迫性恐惧源于自己内心的某些思想或观念,怕的是失去自我控制或带来的伤害等,并非对外界事物恐惧。再有恐惧症不具有强迫症的强迫与反强迫的特点。

(4)疑病症与疾病恐惧:疑病症患者由于对自身状况的过分关注而可能表现出对疾病的恐惧,但这类患者认为他们的怀疑和担忧是合理的,所恐惧的只是自身的身体状况而非外界客体或情境,恐惧情绪通常较轻;疾病恐惧患者常认为他们的担心和恐惧是不合理的,并伴有对与疾病有关的场所回避,恐惧情绪通常较重。

3.颞叶癫痫
 可表现为阵发性恐惧,但其恐惧并无具体对象,发作时的意识障碍、脑电图改变及神经系统体征可资鉴别。

4.共患病 恐惧症与其他精神障碍有较高的共患病率。场所恐惧症常见的是惊恐障碍、抑郁症、其他焦虑性障碍、酒精和物质滥用;社交恐惧症常见的是特定的恐惧症、场所恐惧症、惊恐障碍、广泛性焦虑、PTSD、抑郁/心境恶劣以及物质滥用;特定的恐惧症患者终生经历至少一种其他终生的精神障碍的风险超过80%,尤其是其他焦虑障碍(惊恐障碍、社交恐惧症)和心境障碍(躁狂、抑郁、心境恶劣),但物质滥用的发生率远少于其他焦虑障碍(美国国家共病调查)。

\subsection{治疗}

\subsubsection{治疗原则}

恐惧症的治疗原则是:①综合治疗:采取药物联合心理治疗的原则。②长程治疗:恐惧症患者常在患病后多年后才开始寻求治疗,病情无典型的急性期与缓解期之分。长程治疗可分为前期治疗和维持治疗。前期治疗(包括CBT等心理治疗和药物治疗)在于控制症状,改善患者的错误认知,减少恐惧性回避行为,尽可能达到临床痊愈;维持治疗在于改善患者的社会功能,提高患者的生活质量和预防复发。③个体化治疗:全面考虑患者的年龄特点、躯体状况、既往药物治疗史、有无并发症,因人而异地个体化合理治疗。

\subsubsection{药物治疗}

1.药物治疗策略 恐惧症的前期治疗通常持续8~12周,药物应从小剂量起始,根据治疗反应调节剂量。经过前期治疗有效后,至少维持6~12个月后,再根据临床特征考虑逐渐减药。

2.药物种类 我国食品药品监督管理局(SFDA)批准治疗恐惧症的药物有帕罗西汀、丁螺环酮、曲唑酮、多塞平等。

SSRIs与SNRIs类药物常作为恐惧症的一线治疗药物,一般情况下,药物治疗4~12周有效。

苯二氮䓬
类药物常与SSRIs与SNRIs类等一线药物合并使用来减轻焦虑,通常不主张长期使用,一般治疗时间掌握在2~4周,4周以后开始逐步减量,到最后停用。

其他药物:5-HT受体拮抗和再摄取抑制剂(曲唑酮)、三环类和杂环类药物(TCAs)、β受体阻滞剂(普萘洛尔)、单胺氧化酶抑制剂等药物对恐惧症有效,目前均不作为一线治疗药物。

\subsubsection{心理治疗}

1.行为疗法 越来越多的证据表明,行为治疗对恐惧症有效,尤其是特定的恐惧症和部分场所恐惧症,单用行为治疗即能奏效。具体方法可采用系统脱敏疗法或逐级暴露疗法,并与进行性放松训练结合起来。其治疗的基本原则:一是消除恐惧对象与焦虑恐惧反应的条件性联系;二是对抗回避反应。另需注意的是,对血液-注射-损伤恐惧而言,其伴随的自主神经反应与其他恐惧症不同,不宜使用肌肉放松疗法,而是在暴露练习中患者必须收紧肌肉,以保持足够高的血压来完成练习。

2.认知-行为治疗 对社交恐惧症常用的心理治疗方法为认知-行为治疗(CBT),它采用暴露、认知转变、放松训练和社交技巧训练等方法的整合,不同技术的整合有助于提高治疗效果,而且与药物治疗合用疗效更好。由Heimberg(1990,
2001)创立的小组形式的认知-行为治疗(CBGT)或许是治疗社交恐惧症最佳的心理社会干预方法,已有的资料表明,其有效率一般为45%~70%,且长期预后良好(Heimberg,
1998)。

3.其他治疗 精神分析(通过精神分析治疗试图寻找患者潜意识下的心理冲突)、催眠疗法以及支持性心理治疗,或具有中国特色的气功治疗、道家治疗等也具有一定的疗效。

\subsection{病程与预后}

恐惧症多数病程迁延,有慢性化发展的趋势。特定的恐惧症的病程最长,而且最为稳定。临床经验提示,起源于幼年的特定的恐惧症可持续多年,而在成年期应激事件后出现的预后较好。社交恐惧症的平均病程约20年,如共患抑郁障碍、酒精及物质滥用可导致较高的自杀率(Davidson,
1993; Schneier,
1992)。一般而言,各类恐惧症中的预后以特定的恐惧症相对较好,社交恐惧症次之,而场所恐惧症相对较差。起病的年龄晚、病程短、症状较轻、发病前存在诱因、病前社会适应能力完好、病前个性缺陷不明显及无共患疾病者预后较好,反之预后不佳。

\section{强 迫 症}

\subsection{概述}

强迫症(obsessive compulsive disorder,
OCD)又称强迫性神经症或强迫性障碍,是指一种以强迫症状为主要临床相的神经症。其特点是有意识的自我强迫和反强迫并存,两者强烈冲突使患者感到焦虑和痛苦;患者体验到观念或冲动系来源于自我,但违反自己意愿,虽极力抵抗,却无法控制;患者也意识到强迫症状的异常性,但无法摆脱。病程迁延者可以仪式动作为主而精神痛苦减轻,但社会功能严重受损。

强迫症的特点包括:①患者体验到这种反复进入患者意识领域的思想、表象或意向是自己的主观活动的产物,毫无意义,有受强迫的体验;②主观上感到必须加以有意识的抵抗,这种反强迫与自我强迫是同时出现的;③有症状自知力,患者感到这是不正常的,甚至是病态的,至少患者希望能消除这些症状,但似乎无能为力。

\subsection{流行病学}

Woodruff等(1964年)报道,充分发展的强迫症在人口中患病率为0.05%。美国国立精神卫生研究所报道(1984年)的患病率为1.3%~2.0%。Nemiah等(1985年)调查显示,普通人群的患病率为0.05%。Marvin等(1988年)对18500人的调查,患病率为1.2%~2.4%。我国(1982年)12地区精神疾病流行学调查,本病在15~59岁人口中患病率为0.03%,占全部神经症的1.3%

Goodwin等(1969年)发现,约65%的强迫性障碍患者起病年龄是在25岁以前,只有不到15%的人是在35岁以后发病,无论男女的平均发病年龄都是20岁。我国张守杰(1991年)的研究结果与此非常相近,而Beech(1974年)对354名个案的研究报告中显示,其中有大约1/3的人是在15岁之前开始发病。儿童与成年强迫症之间并非一定有联系。事实上相当一部分童年患强迫症者预后良好,成年后无强迫症状,而成年患者童年期并无强迫症病史。在性别上总体上女性发病率略高。

\subsection{病因}

\subsubsection{遗传因素}

有研究提示,强迫症一级亲属中焦虑性障碍的发病危险性显著高于对照组的一级亲属,但他们的强迫症的发病危险性并不多于对照组。如果把患者一级亲属中有强迫症状但达不到强迫症诊断标准的病例包括在内,则患者组的父母强迫症状的危险率为15.6%,显著高于对照组父母(2.9%)。Black等(1992年)研究也发现,这种强迫特征在单卵双生子中的同病率多于双卵双生子的同病率。另有一些研究报道表明,强迫症可与精神分裂症、抑郁症、惊恐障碍、恐惧症、进食障碍、孤独症和抽动秽语综合征同时存在。

\subsubsection{生理、生化因素}

巴甫洛夫的理论认为强迫症在强烈的情感体验影响下,大脑皮质兴奋或抑制过程过度紧张或相互冲突,形成了孤立的病理惰性兴奋灶。由于条件反射的形成,使强迫症状固定下来,持续存在,而强迫性对立思维则与超反常相有关。

生化方面,患者5-HT系统功能异常,证据为:①氯米帕明治疗强迫症有效,其他具有选择性5-HT回收抑制作用的药物也对强迫症具有较好的疗效,而缺乏抑制5-HT重摄取的其他三环类抗抑郁剂如阿米替林、丙咪嗪、去甲丙米嗪等对强迫症的治疗效果不佳;②强迫症状减轻常伴有血小板5-HT含量和脑脊液5-羟吲哚醋酸(5-HIAA)含量下降;③治疗前血小板5-HT和脑脊液中5-HIAA基础水平较高的病例,用氯米帕明治疗效果较佳;④给强迫症患者口服选择性5-HT激动药MCPP可使强迫症状暂时加剧等。以上均提示5-HT系统功能增强可能与强迫症发病有关。另外,强迫症患者基础血浆皮质醇和基础血催乳素含量均高于对照组,但地塞米松抑制实验(DST)未见脱抑制现象,不同于抑郁症,显示强迫症患者神经内分泌也存在异常。

\subsubsection{神经解剖与脑影像学}

有一些证据提示,强迫症存在一定的神经解剖学改变,强迫症发病可能与选择性基底核功能失调有关。如正电子成像技术(PET)显示额叶眶面皮质,包括尾状核与扣带回前部代谢活性与脑血流增加,难治性强迫症切断额叶或环绕纤维有效;MRI提示患者的右额白质T\textsubscript{1}
相延长;与基底核功能障碍密切相关的抽动秽语综合征患者中15%~18%患有强迫症状;脑CT检查可见到有些强迫症患者双侧尾状核体积缩小。

\subsubsection{社会心理因素}

生活事件在强迫症的发生、发展和转归中均发挥着一定的作用,它们既可能是诱发因素,也可能成为持续因素。人格特征在强迫症的发生中也起着一定的作用,以偏内向为主。但强迫性人格与强迫症并不存在必然关系,有强迫性人格特点的人并不一定就发展成为强迫症。

\subsection{临床表现}

强迫症的基本症状是以强迫观念和强迫动作或行为为基本特征。

\subsubsection{强迫观念}

强迫观念(obsession)是患者不想要的思想、想象和冲动,至少在早期阶段患者努力抵抗,企图减少这些思想出现的强度和频率。患者认识到这些思想是他们自己的,但与他们的愿望和人格不符合,为此感到痛苦。强迫观念主要有:

1.强迫性怀疑 是对自己言行的正确性及可靠性产生了怀疑,即自我怀疑(selfdoubt)。与其相联系的是犹豫不决和摇摆不定,严重者真实感发生障碍。患者总怀疑自己是否确实说过或做过某事,怀疑自己说错了或做错了,如怀疑门窗、煤气是否关了,投寄的信是否贴了邮票,反复检查仍不放心。刚说过的话或做过的事,总怀疑是否说过或做过,是否正确;出门时怀疑门窗是否关好,寄信时怀疑邮票是否贴好,虽然检查了很多遍,还是不放心、焦虑不安。

2.强迫性穷思竭虑 患者对日常生活中的一些事情或自然现象反复思索,寻根究底。患者可以相当一段时间内常固定在某一件事或某一个问题上,也可能碰到什么想什么,患者诉述脑子总是不能闲着。如反复思索“到底是先有鸡,还是先有蛋?”“花为什么会开?”“人为什么要分男女?”“为什么1+1等于2,而不等于3?”等。患者明知没有必要,但不能自控,无休止地想下去,无法解脱。

3.强迫性对立思维 患者脑子内出现一个观念,马上出现一个与其完全对立的另一观念。如听说某人去世了,认为死者真不幸,同时却想到他该死;想到“和平”,立刻联想到“战争”;脑子内出现“万岁”时,立即又出现“打倒”。当对立观念涉及父母、老师、公认的伟人时,患者会觉得“思想总是跟真理唱反调”。患者十分痛苦,紧张恐惧,苦恼不堪。

4.强迫性回忆 患者经历过的事件,不由自主地、频繁地在意识中反复呈现,无法摆脱。有时这种回忆是一种生动的、鲜明的形象,且往往是令患者难堪的或厌恶的。例如患者经常回忆起过去见过的一个乞丐肮脏的形象,使他感到十分厌恶。

5.强迫情绪 主要表现为对某些事物的担心或厌恶,对自己情感的恐惧,包括强迫性害怕失控。患者害怕自己丧失自控,害怕会发疯,会干坏事,内心极度紧张不安。明知不必要或不合理,自己却无法摆脱,无法克制。例如,担心自己会伤害别人,担心自己说错话,担心自己出现不理智的行为,或担心自己受到毒物的污染或细菌的侵袭,或看到棺材、出丧、某个人时,立即产生强烈的厌恶感。与强迫意向的区别是患者并没有马上要行动的内在驱力或冲动。

6.强迫意向 患者反复体验到想要做某种违背自己意愿的动作或行为的强烈内心冲动,但实际上并不直接转变为行动。患者强烈地感到意志失控,明知是非理性的、荒谬不可能的,并努力控制自己不去做,但无法摆脱。想要做的可以是无关紧要的动作,也可以是拿起刀来砍自己或砍别人的严重行为,把心爱的孩子丢在河里。患者感到强烈不安,感到他的意志失控。即使患者感到所想要做的完全是无伤大雅的小动作,他还是感到强烈的不安。

\subsubsection{强迫动作或行为}

强迫动作或行为(compulsive
behavior)与强迫观念有联系,患者不由自主地采取相应行为,以期减轻强迫观念引起的焦虑。但它的作用往往是轻微和短暂的,并不能有效地减少患者的焦虑,有时反而增加了克服强迫动作和行为的焦虑。

1.强迫性检查 患者为减轻强迫性怀疑引起的焦虑情绪而采用的措施,包括强迫核对。如出门时反复检查门窗是否关好,寄信时反复核对信中的内容,看是否写错了名字、地址等。

2.强迫性询问 强迫症患者常常不相信自己,为了消除疑虑或强迫性穷思竭虑带来的焦虑,患者常要求他人不厌其烦地给予解释或保证,一遍又一遍,患者仍不能释怀。

3.强迫性清洗 患者为了消除对受到污物毒物或细菌污染的担心,常反复洗手、洗澡、洗餐具和衣服,有的甚至把家具等不能清洗的东西也清洗。有的患者不仅自己反复清洗,而且要求与他一道生活的人,如配偶、子女、父母等也必须按照他的要求彻底清洗。患者对卫生的概念并非完全明确,有的患者可能集中于某一物特别怕脏,面对其他物品一概不管。如怕手受到污染,反复洗手,但对家中其他卫生可以不管。

4.强迫性仪式动作 通常是为了对抗某种强迫观念引起的焦虑而逐渐发展起来的。是一系列重复出现的动作,他人看来是不合理的,或荒谬可笑的,但对患者来说却很重要,可减轻或防止强迫观念引起的紧张不安。由于确有暂时缓解之效,动作逐渐被强化而难以摆脱。一种动作实行久了,不新奇了,转移注意和控制强迫观念的作用随之下降,患者只好增添新花样,逐渐形成复杂的动作程序,并且这套程序本身也具有强迫性。

例如,一位患者上床前脱衣服要先脱左边袖子,后再脱右边袖子,绕床3圈,再脱左边鞋子,脱后鞋尖朝一定方向放好。患者要求自己按严格的方式和程序行事,稍有差错,便一切作废,穿上衣服鞋子,从头开始再脱一遍,直到符合程序为止;出门前先向前走两步,再向后退一步,然后再出门。患者有时也可以表现为强迫性计数,如计数台阶、窗格、电线杆等,或强迫性地默念一些词语,目的都是为了解除某种担心或避免焦虑的出现。

5.强迫性缓慢 这些患者可能否认有任何强迫观念,缓慢的动机是努力使自己所做的一切非常完美。由于以完美、精确、对称为目标,所以常常失败,因而增加时间。患者举止行动缓慢,具明显仪式化特征,例如早晨起床后反复梳洗,使患者迟迟不能出门。严重时刷牙也许要花1小时,从房间门口走到桌子边也许要花半个小时,甚至可以在浴室一站就是3个多小时等。患者常考虑自己在思考行动的计划和步骤是否恰当,此时焦虑情绪不明显,除非被强行阻断。

多数强迫症患者强迫观念和强迫行为混合存在。

\subsection{诊断和鉴别诊断}

\subsubsection{CCMD-3强迫症的诊断标准}

1.症状标准

(1)符合神经症的诊断标准,并以强迫症状为主,至少有下列1项:①以强迫思想为主,包括强迫观念、回忆或表象,强迫性对立观念、穷思竭虑、害怕丧失自控能力等;②以强迫行为(动作)为主,包括反复洗涤、核对、检查或询问等;③上述的混合形式。

(2)患者称强迫症状起源于自己内心,不是被别人或外界影响强加的。

(3)强迫症状反复出现,患者认为没有意义,并感到不快,甚至痛苦,因此试图抵抗,但不能奏效。

2.严重标准 社会功能受损。

3.病程标准 符合症状标准至少已3个月。

4.排除标准

(1)排除其他精神障碍的继发性强迫症状,如精神分裂症、抑郁症或恐惧症等;

(2)排除脑器质性疾病,特别是基底核病变的继发性强迫症状。

\subsubsection{鉴别诊断}

1.精神分裂症 精神分裂症患者无论是早期还是发作缓解期都可合并有强迫症状。

如L.
Rosen(1957年)研究了英国2所精神病院的848例精神分裂症患者,发现其中30%当时存在或既往有强迫症状。近年来国内外这方面的报道也多不胜举。同时也有报道一部分强迫症患者经数十年后最后诊断为精神分裂症。

两者的区别主要在于:①强迫症患者往往有症状自知力,能感到是本人意志的产物,具有强烈的自我控制意向,要求治疗,而精神分裂症患者则相反;②强迫症患者往往怀疑自己,对自己的信念行为要反复考虑检查,而精神分裂症患者对自己的信念却坚信不疑;③强迫症患者是跟自己作对,而精神分裂症患者大多是针对他人;④精神分裂症患者往往会有一些精神分裂症的其他症状长期存在,而慢性强迫症患者病情加剧时可出现短暂的精神病性症状,但不久可恢复;⑤强迫症患者强迫症状出现可有明显的心理诱因,伴明显的焦虑情绪,但精神分裂症的强迫症状为分裂症症状表现的一部分,内容离奇,形式多变,起病缺乏明显的心理诱因,且常不伴有明显的焦虑情绪等。

2.抑郁症 对强迫症认识之初,曾将归之于情感性疾病。近年来,抗抑郁药对强迫症的明显疗效更促发了人们重新关注两者的联系。与此同时,抑郁症与强迫症之间有联系的生物学、遗传学研究也不断有报道,如抑郁症患者地塞米松抑制试验和睡眠脑电图的变化在强迫症患者身上也可见;强迫症患者父母、同胞躁狂抑郁的患病率与焦虑症、歇斯底里以及正常对照组比起来均明显增高。

但两者之间仍然存在一些不同:①抑郁症为发作性病程,而强迫症往往为持续慢性病程;②电休克治疗对内源性抑郁症疗效确切,而对强迫症疗效不肯定;③抑郁症与躁狂症常属于同一临床实体,而强迫症与躁狂症无明确联系;④强迫症患者病前有一定性格特点,而抑郁症与人格特征往往无显著相关;⑤两者的首发症状也不同,且随着抑郁情绪症状的改善;⑥两者的强迫症状结局不同,抑郁症缓解后强迫症状往往消失,而强迫症的强迫症状在情绪改善后往往只有部分缓解。两者常有交叉重叠的症状,在诊断时应根据首先出现的优势症状诊断,如果两组症状都存在且不占优势,则应该按等级诊断的原则诊断抑郁症。

3.恐惧症 强迫症可出现惊恐发作或伴有轻微的惊恐症状,但不妨碍强迫症的诊断。恐惧症患者也常常有类似强迫症患者的回避行为,但恐惧症的恐惧对象常来源于客观现实。恐惧症回避行为更突出,焦虑情绪常与客观现实可能产生的后果有关,而强迫症的强迫观念和行为常起源于患者的主观体验,有具体内容,且有意识的自我强迫和反强迫并存。焦虑是两者强烈冲突的产物。

4.Tourtte综合征、其他脑器质性精神障碍(特别是基底节病变)患者出现强迫症状,这些症状常仅为这类疾病的伴发症状,并不影响主要诊断。

5.强迫性人格障碍 患者多注重细节、追求完美、刻板固执等,患者往往习惯于自身的行为方式,并不认为有任何异常,缺乏自知力,极少主动就诊。该症患者缺乏明确的强迫性思维或动作,往往有较好的社会功能。

\subsection{治疗}

强迫症的病因错综复杂,虽然症状类似,但引起的原因可能大相径庭。因此,治疗的原则和目标也就可能完全不同。治疗的目标包括:①原发性强迫症状,这是患者精神痛苦或心理冲突的核心;②继发的强迫动作,如各种仪式化行为等;③强迫症伴有的各种生理功能障碍,如头痛、失眠及自主神经紊乱症状等;④社会适应不良;⑤抑郁、先占观念、恐惧等并发症症状;⑥患病引起的各种问题,如学习工作问题、个人生活的料理、夫妻和家庭关系问题等。

\subsubsection{心理治疗}

1.认知-行为治疗 是对强迫症治疗最有效的心理治疗方法。认知治疗主要针对患者的认知歪曲模式;行为治疗则重点在于改变患者应对强迫的策略和方法。

2.家庭治疗 强调通过分析和借助对家庭人际结构的思考解决问题,强调在人际关系的系统水平上进行治疗。

3.精神分析 通过顿悟、改变情绪经验以及强化自我的方法分析和解释各种心理现象之间的矛盾冲突,以达到治疗目的,对部分患者有效。

4.森田疗法 对强迫症治疗亦有效,特别是静卧期结束时患者症状改善幅度较大。

\subsubsection{药物治疗}

药物治疗以具有5-HT再摄取阻滞作用的氯米帕明和SSRI等疗效较好。焦虑明显者可并用苯二氮䓬
类药物类如氯硝西泮等。

1.5-HT回收抑制药 经典的为氯米帕明,这类药对强迫症状和伴随的抑郁均有较好的效果,临床疗效约为70%。本品宜从小剂量开始逐渐加量,治疗时间不应少于3~6个月,以免复发,有些较严重的患者需终生服药。另外,新型的选择性5-HT回收抑制药亦有很好的疗效。

2.其他药物 苯二氮䓬
类药,如氯硝西泮、阿普唑仑等,这类药对强迫症患者伴随的严重焦虑和激动不安,睡眠障碍有较好的疗效。必要时可加用锂盐、丁螺环酮等,或者抗精神病药氟哌啶醇、利培酮以及喹硫平等,以提高疗效。强迫症需要较长的治疗时间,一般需应用治疗剂量治疗10~12周。

\subsubsection{精神外科治疗}

对极少数慢性强迫症患者,药物治疗和心理治疗均失败,而患者又处于极度痛苦之中,目前国内外已开展了手术治疗,通过立体定向或γ刀破坏目前与强迫症成因有关的部位以达到治疗效果,但长期疗效如何尚需进一步观察。精神外科手术的指征为:症状严重、药物与心理治疗失败、自愿等,但必须从严掌握。

\subsection{病程与预后}

强迫症多在少年期发病,1/3病例症状首发于10~15岁,3/4的患者起病于30岁前,无明显原因,缓慢起病。就诊时病程已达数年之久,半数以上病例逐渐发展,病程波动,11%~14%的病例有完全缓解的间歇期,常伴有中度及重度社会功能障碍。病前人格健康、发作性病程、不典型症状,尤其伴显著焦虑或抑郁、病程短者,预后好。病前有严重的强迫性人格障碍、症状严重而且弥散、童年起病、病程长、从未明显缓解过,预后不良。

综合研究报道,约3/4的患者呈慢性化趋势,1/3左右的患者的症状持续存在,变化不明显,1/4的患者为波动性,1/10的患者为一过性,总之本病预后不容乐观。

\section{躯体形式障碍}

\subsection{概述}

躯体形式障碍(somatoform
disorders)是一类以持久地担心或相信各种躯体症状的先占观念为特征的神经症。其主要特征是患者反复陈述躯体不适,不断要求医学检查,对于反复检查的阴性结果以及医生关于其症状并无躯体疾病基础的解释表示怀疑。即使患者有时存在某种躯体障碍,却不能用躯体障碍来解释其躯体症状的性质、严重程度以及患者的先占观念和痛苦体验,以致影响到日常的学习、工作和人际交往等。患者认为其疾病的本质是躯体性的,即使症状的出现和持续与不愉快的生活事件、困难或心理冲突密切相关,也拒绝探讨心理病因的可能。患者的躯体症状可累及全身多个系统,反复就诊于各大医院的诸多科室,不断更换医生,进行过大量不必要的或者重复的检查,服用过多种药物,甚至做过探查性手术,易形成药物依赖和药物滥用,延误诊断和治疗,医疗花费高,常引起医患关系问题,预后不佳。

\subsection{流行病学调查}

由于调查依据的诊断标准ICD-10与DSM-Ⅳ有所不同,躯体形式障碍及分类的患病率有很大的出入。美国一般人群中躯体形式障碍的患病率为4.4%,其中躯体化障碍的患病率为0.2%~2.0%,终身患病率为0.13%。Karvonen报道1966年出生的1598名人群中躯体化障碍的患病率为1.1%,女:男为5:1。Gabe等在一个4075名的普通人群样本的调查中,按DSM-Ⅲ-R诊断标准躯体形式疼痛的患病率为33.7%,按DSM-Ⅳ诊断标准患病率则降至12.3%。Fink在392名内科住院病人中,用ICD-10诊断标准,5%患躯体化障碍,而用DSM-Ⅳ标准则为1.5%。与此相反的是,在ICD-10标准下仅有0.7%患未分化躯体形式障碍,而DSM-Ⅳ标准下则为10%。Mde
Waal报道为16.1%。

Mullick在112名患躯体形式障碍的儿童及青少年中发现,女孩比男孩有更多的躯体化障碍和躯体形式疼痛的诊断,儿童未分化躯体形式障碍和躯体形式自主神经功能失调类型较多,青少年躯体化障碍较多。躯体形式疼痛障碍也多见于女性,常起病于成年或成年早期,也可发生于任何年龄。但一项对630名60岁以上的老年人的调查中发现,躯体形式疼痛很常见,71.8%的人报告了一个症状,50.5%的人报告了4个症状,23.4%的人报告了8个以上症状,与年轻组报告不一致,老年妇女没有报告比老年男性更多的躯体形式疼痛。疑病障碍男女比率相当,男性多起病于30~40岁,女性多于40~50岁起病,很少在50岁以后首次发病。在普通人群中,躯体变形障碍的患病率为1.19%。躯体变形障碍病例中男女比例大约为1.3:1。躯体变形障碍的发病年龄较小。大多数躯体变形障碍的患者未婚,病例报道中85%的患者大于19岁且单身。发病年龄从青春期早期到20多岁;19岁为已报道病例的平均发病年龄,但是患者通常平均在6年后才到精神科医师处就治。

国内崔利军等调查发现,河北省一般人群中躯体形式障碍的患病率为0.692%,躯体化障碍0.03%、未分化的躯体形式障碍0.24%、疼痛障碍0.56%、疑病症0.18%。女性(1.43%)明显高于男性(0.13%)。孟凡强等利用ICD-10诊断标准,发现综合医院门诊就诊患者18.2%为躯体形式障碍,躯体化障碍占门诊总就诊数的7.4%。一般认为,躯体化障碍患者以女性多见,女性人口中的患病率约为1%,起病多在30岁之前。

\subsection{病因}

\subsubsection{遗传因素}

有关遗传对躯体形式障碍发病的影响尚缺乏精确的结论性证据。有报道认为,女性更多地以躯体化障碍表达遗传倾向。躯体形式障碍具有家族聚集性,亲属患该病的风险比较高。男性患者一级亲属中社会病理性人格和嗜酒者显著较多,20%左右的女性患者一级亲属中患有“癔症”,比一般女性群体高10倍。

\subsubsection{个体因素}

多项研究发现,不少躯体形式障碍的患者存在人格障碍,其中以表演性人格障碍、反社会型人格障碍、强迫型人格障碍多见。患者敏感、多疑、主观、固执、自我中心、自怜、谨小慎微、对安全过分关注、对周围事物缺乏兴趣、要求十全十美,对自身健康状况过分关注等性格特点,为躯体形式障碍的发病提供了重要条件。男性患病前常具有强迫人格,女性则与癔症性格有关。患者常表现人际交往困难,婚姻生活不协调,有寻求注意的倾向,过分关注和夸大自身症状,暗示性高。一些证据表明,躯体形式障碍患者对躯体的感觉比一般人敏感,他们对躯体不适耐受的阈值较低。患者可能有认知功能缺陷,他们过分关注躯体感觉,将它们夸大,给它们错误解释,并被这些不适所困扰。对于文化程度较低的患者,其沟通表达能力受限,易于出现情绪障碍,情绪表达常以躯体不适来体现。而女性患病率受其生理功能特殊性的影响,明显高于男性,有时月经紊乱成为起病的信号之一。

\subsubsection{心理社会因素}

1.医源性因素 患者在错误的传统观念影响下,如对一些躯体疾病过分不恰当的宣传,亲友死于某种严重的疾病,以及医生不恰当的解释、检查,不恰当的态度,对疑病观念的形成都可产生不良影响,特别是医源性影响。当然不必要的重复大量检查和治疗,甚至进行探查性手术,不仅会造成药物滥用,还会因出现药物不良反应、药物依赖或手术后遗症,增加了躯体症状,强化了患者角色,使患者症状迁延难愈。

2.继发性获益 使得躯体症状迁延不愈,一方面,患者躯体症状的出现能缓解冲突;另一方面则可以回避不愿承担的责任,使其不良经历得到合理化的解释,并获得他人的关心和照顾,甚至改变人际关系。

3.其他 约1/3患者起病之前有躯体疾病,但躯体形式障碍的表现及症状严重程度与原有疾病不相符。处于青春期或更年期,较易出现自主神经症状,老年人生活孤独,机体各脏器机能衰退,都可促使疑病观念的产生。

\subsection{临床分型与临床表现}

CCMD-3将躯体形式障碍分成:躯体化障碍、未分化躯体形式障碍、疑病症、躯体形式自主神经紊乱、持续性躯体形式疼痛障碍和其他或待分类躯体形式障碍六种亚型。其共同特点是:①对身体健康或疾病过分担心,其严重程度与实际健康状况很不相称;②对通常出现的生理现象和异常感觉作出疑病性解释;③患者有牢固的疑病观念,缺乏充足的根据,但未达到妄想的程度,不是妄想;④由于疑病观念,患者反复就医或反复要求医学检查;⑤有生物、心理、社会环境等诱发因素,其中心理因素在医生启发下可能会充分暴露出来;⑥症状繁多,但含糊不清,涉及多系统,病程至少3个月,有的长达数年,病者为此不安;⑦不断拒绝多位医生关于其症状没有躯体病变解释的忠告和保证;⑧症状和其所致行为造成一定程度的社会和家庭功能损害;⑨患者常借这些症状应付精神压力,表达困扰,而家庭、学校、社会常间接地、不自觉地扮演了支持的角色;⑩患者可获得“社会性收益”,而另一方面却又增强了原先的心理生理症状。

各亚型的临床表现为:

\subsubsection{躯体化障碍}

躯体化障碍(somatization
disorder)又称Briquet综合征,主要特征为多种多样、反复出现变化的躯体症状。在转诊到精神科或心理科之前,大多数患者已有在综合性医院或其他专科医院反复就诊的经历,进行过多项检查,服用过多种药物,甚至进行过多次探查性手术,均未发现明显的器质性疾患,常有多个诊断。患者症状可累及身体的多个系统或部位,较常见的是胃肠不适(疼痛、打嗝、反酸、恶心、呕吐、饱胀等),皮肤症状(酸痛、刺痛、痒感、麻木感、烧灼感等),性功能障碍(性冷淡、阳痿、月经紊乱),多部位、多性质的疼痛等。患者过分关心和担心自己的主观症状,往往有夸大,有时出现戏剧性变化。同时,常伴有明显的抑郁、焦虑,甚至有自杀倾向,易形成药物依赖。患者的社会功能、人际交往及家庭职能均有可能长期存在严重的障碍。女性远多于男性,常在成年再起发病。

\subsubsection{未分化躯体形式障碍}

未分化躯体形式障碍(undifferentiated somatoform
disorder)可视为未充分发展的躯体化障碍,躯体主诉同样具有多样性、变异性和持续性,症状相对较少,涉及的部位不如躯体化障碍广泛,缺乏典型性。常见症状为:疲乏无力、食欲减退、胃肠不适、尿频、尿急等,病程多在半年以上、两年以下。

\subsubsection{疑病症}

疑病症(hypochondriasis)包括两个主要内容,一是疑病性障碍(hypochondriacal
disorder);另一种是躯体变形障碍(body dysmorphic disorder, BDD)。

疑病性障碍是指以担心或相信患严重躯体疾病的持久性优势观念为主要临床相。患者为此常反复就医,但各种医学检查阴性和医生的解释均不能打消其疑虑。患者认为检查结果阴性可能受医学发展水平的限制,对小概率事件紧追不放,常片面寻求诊断,对治疗并不关心,害怕药物及其不良反应,不遵医嘱。患者诉说的症状可限于某一器官,也可涉及全身,主要为各种异常的感受,如头颈部、下背部、右下腹部等疼痛,但对疼痛的描述通常含糊不清,其他还有胃肠不适、心悸、呼吸困难感、对血压的担忧、胸闷、尿频、尿急、恶心、吞咽困难、反酸、口腔异味、腹部胀气和腹痛等。患者的疑病观念比较牢固,但达不到妄想的程度。即使患者有时存在某种躯体障碍,也不能解释所诉症状的性质、程度,或患者的痛苦与优势观念,常伴有焦虑或抑郁。本障碍男女均有,无明显家庭特点,社会功能受损。

躯体变形障碍是指患者过分关注和放大身体上尤其是面部细微缺陷,坚信自己的外表如鼻子、嘴唇、皮肤皱纹、身高等存在严重缺陷,常就诊于整形或美容外科,要求实行矫正手术。医学干预往往难以纠正患者的先占观念,常伴有明显的抑郁情绪,患者感到自卑,甚至出现自杀倾向,有发展为精神病的风险,社会功能受损,预后不佳。常见于青少年或成年早期。

\subsubsection{躯体形式自主神经功能紊乱}

躯体形式自主神经功能紊乱(somatoform autonomic
dysfunction)主要表现为受自主神经支配的器官、系统出现各种躯体形式自主神经功能紊乱的症状,临床表现多样化,主要表现为:

1.转换性症状或假性神经系统症状 吞咽困难,失音,失聪,失明,复视,视物模糊,昏倒或意识丧失,记忆缺失,癫痫
样发作或抽搐,行走困难,肌肉乏力或麻痹,尿潴留或排尿困难,异常皮肤感觉等。

2.消化道症状 腹痛,恶心,呕吐(妊娠期除外),不能耐受某些食物,腹泻,便秘等。

3.生殖系统症状 性欲冷淡,性交时缺乏快感,性交疼痛,阳痿等;痛经、月经不规则、月经过多;整个妊娠期出现严重呕吐,不得不住院等。

4.疼痛 背、关节、四肢、生殖器等部位疼痛及排尿疼痛等。

5.呼吸及心血管症状 气促、气短、心悸、胸痛、头晕等。

6.过分担心、多虑过分担心年龄、体重、皮肤、瘢痕、水肿及性功能等。

患者可有个体特异性主诉,如部位不定的疼痛、沉重感、紧束感、肿胀感、搅拌感、烧灼感。反复医学检查均未发现器质性疾病依据,但患者坚持将症状归咎于某一特定的器官或系统,对存在的心理冲突和人际困难避而不谈,包括心脏神经症、心因性呃逆、胃神经症、心因性肠激惹综合征、心因性过度换气和咳嗽、心因性尿频和排尿困难等障碍。

\subsubsection{持续性躯体形式疼痛障碍}

持续性躯体形式疼痛障碍(persistent somatoform pain
disorder)主要表现为不能用生理过程或躯体障碍予以合理解释的、持久性、严重的疼痛。患者感到非常痛苦,常见疼痛部位是头面部、腰背部、盆腔等,身体其他任何部位皆可发生疼痛,疼痛的性质常为胀痛、钝痛、酸痛。情绪冲突或心理社会因素是导致疼痛发生的主要原因,患者有寻求注意的倾向,存在潜在的继发性获益,并由此强化症状,导致症状迁延不愈,常持续6个月以上,社会功能受损。患者反复求治,接受过多种药物或其他相关治疗,为明确病因甚至进行过手术探查,易形成药物依赖,同时伴有抑郁、焦虑等,社会功能受损。发病年龄在30~50岁,女性多见,有家族聚集倾向。

\subsubsection{其他或待分类躯体形式障碍}

其他或待分类躯体形式障碍(other or unspecified somatoform
disorders)指其他不是由客观性躯体疾病引起的,在时间上与负性生活事件密切相关的躯体不适,难以用上述疾病归类的躯体形式障碍,如癔症、心因性斜颈等。

\subsection{诊断与鉴别诊断}

CCMD-3躯体形式障碍诊断标准为:

1.症状标准

(1)符合神经症的诊断标准。

(2)以躯体症状为主,至少有下列1项:①对躯体症状过分担心(严重性与实际情况明显不相称),但不是妄想;②对身体健康过分关心,如对通常出现的生理现象和异常感觉过分关心,但不是妄想。

(3)反复就医或要求医学检查,但检查结果阴性和医生的合理解释均不能打消其疑虑。

2.严重标准 社会功能受损。

3.病程标准 符合症状标准至少已3个月。

4.排除标准 排除其他神经症性障碍(如焦虑、惊恐障碍或强迫症)、抑郁症、精神分裂症、偏执性精神病。

\subsubsection{躯体化障碍诊断标准}

1.症状标准

(1)符合躯体形式障碍的诊断标准。

(2)以多种多样、反复出现、经常变化的躯体症状为主,在下列4组症状之中,至少有2组共6项:①胃肠道症状:如腹痛,恶心,腹胀或胀气,嘴里无味或舌苔过厚,呕吐或反胃,大便次数多、稀便或水样便;②呼吸循环系症状:如气短、胸痛;③泌尿生殖系症状:如排尿困难或尿频、生殖器或其周围不适感、异常的或大量的阴道分泌物;④皮肤症状或疼痛症状:如瘢痕,肢体或关节疼痛、麻木或刺痛感。

(3)体检和实验室检查不能发现躯体障碍的证据,能对症状的严重性、变异性、持续性或继发的社会功能损害作出合理解释。

(4)对上述症状的优势观念使患者痛苦,不断求诊,或要求进行各种检查,但检查结果阴性和医生的合理解释,均不能打消其疑虑。

(5)如存在自主神经活动亢进的症状,但不占主导地位。

2.严重标准 常伴有社会、人际及家庭行为方面长期存在的严重障碍。

3.病程标准 符合症状标准和严重标准至少已2年。

4.排除标准 排除精神分裂症及其相关障碍、心境精神障碍、适应障碍或惊恐障碍。

\subsubsection{未分化躯体形式障碍诊断标准}

1.躯体症状的主诉具有多样性、变异性的特点,但构成躯体化障碍的典型性不够,应考虑本诊断。

2.除病程短于2年外,符合躯体化障碍的其余标准。

\subsubsection{疑病症诊断标准}

1.症状标准

(1)符合神经症的诊断标准。

(2)以疑病症状为主,至少有下列1项:①对躯体疾病过分担心,其严重程度与实际情况明显不相称;②对健康状况,如通常出现的生理现象和异常感觉作出疑病性解释,但不是妄想;③牢固的疑病观念,缺乏根据,但不是妄想。

(3)反复就医或要求医学检查,但检查结果阴性和医生的合理解释,均不能打消其疑虑。

2.严重标准 社会功能受损。

3.病程标准 符合症状标准至少已3个月。

4.排除标准 排除躯体化障碍、其他神经症性障碍(如焦虑、惊恐障碍或强迫症)、抑郁症、精神分裂症、偏执性精神病。

\subsubsection{躯体形式自主神经紊乱诊断标准}

1.症状标准

(1)符合躯体形式障碍的诊断标准。

(2)至少有下列2项器官系统(心血管、呼吸、食管和胃、胃肠道下部、泌尿生殖系统)的自主神经兴奋体征:①心悸;②出汗;③口干;④脸发烧或潮红。

(3)至少有下列1项患者主诉的症状:①胸痛或心前区不适;②呼吸困难或过度换气;③轻微用力即感过度疲劳;④吞气、呃逆、胸部或上腹部烧灼感等;⑤上腹部不适或胃内翻腾或搅拌感;⑥大便次数增加;⑦尿频或排尿困难;⑧肿胀感、膨胀感或沉重感。

(4)没有证据表明患者所忧虑的器官系统存在结构或功能的紊乱。

(5)并非仅见于恐惧障碍或惊恐障碍发作时。

\subsubsection{持续性躯体形式疼痛障碍诊断标准}

1.症状标准

(1)符合躯体形式障碍的诊断标准。

(2)持续、严重的疼痛,不能用生理过程或躯体疾病作出合理解释。

(3)情感冲突或心理社会问题直接导致疼痛的发生。

(4)经检查未发现与主诉相应的躯体病变。

2.严重标准 社会功能受损或因难以摆脱的精神痛苦而主动求治。

3.病程标准 符合症状标准至少已6个月。

4.排除标准

(1)排除检查出的相关躯体疾病与疼痛。

(2)排除精神分裂症或相关障碍、心境障碍、躯体化障碍、未分化的躯体形式障碍、疑病症等。

鉴别诊断:

1.躯体疾病 躯体形式障碍的患者躯体症状涉及范围较广泛,躯体主诉多且程度重,感到痛苦,但缺乏相应的体征和客观检查阳性的依据。但当患者年龄大于40岁,症状相对单一、固定且日趋加重时,需持慎重态度,完善各项检查,坚持严密观察,充分排除严重的器质性疾患,如多发性硬化、甲状腺和甲状旁腺疾病、全身性红斑狼疮等。应通过详细询问病史,躯体、神经系统检查和必要的辅助检查加以鉴别。

2.焦虑症及其他神经症 躯体形式障碍的患者常伴有明显的躯体性焦虑和精神性焦虑,亦可出现明显的自主神经功能紊乱,但其焦虑常继发于躯体不适及迁延不愈,对医生合理的解释难以接受,总是试图寻找器质性疾病。焦虑症患者常常感到莫名地焦虑,对医生合理的解释能够接受,愿意寻求心理病因,积极配合治疗,预后较好。神经衰弱、强迫症等均可有疑病观念,但在临床表现中并不占主导地位,且各自有其特殊的病程和主要的临床表现。

3.抑郁症 抑郁症患者常伴有躯体症状,多集中在胃肠系统,三低症状明显,抑郁症尚有一些生物学方面的症状,昼重夜轻的昼夜节律改变,体重减轻,精神运动迟滞,自罪等。抑郁症的疑病症状往往与其抑郁背景相联系,患者对疑病症状常作出自责自罪的解释。而躯体形式障碍的患者抑郁症状为继发,无典型的“三低”症状。40岁以后起病的患者需仔细问诊,排除有无抑郁症的可能。

4.精神分裂症及其他精神病性障碍 精神分裂症早期可出现疑病症状,内容多为离奇、较荒谬或不固定。这些患者对于疑病症状并不要求反复检查,亦无积极治疗的要求,并缺乏相应的情感体验。

5.诈病 诈病多发生在监狱、工伤、交通事故及司法鉴定的过程中,当事人为获取一定的利益,有目的、有意识地伪装疾病症状,需严加防范,密切观察。躯体形式障碍的患者其躯体不适是客观存在的,继发性获益受无意识支配,患者自己常意识不到这一点,是非自愿的。

\subsection{治疗}

躯体形式障碍的患者常首先辗转于各大综合性医院或基层医院,专科医生却很难遇到。提高综合性医院各临床科室医生对该疾病的认识相当重要,否则不仅增加患者的经济负担,不良的医源性暗示反而加重疾病,导致迁延不愈。本病治疗以心理治疗为主,辅以抗焦虑药等治疗。

\subsubsection{心理治疗}

1.支持性心理治疗 支持性心理治疗是治疗本病的基础。应该耐心和反复地用科学常识进行讲解,以肯定的态度说明患者的疾病性质,指导患者正确对待疾病。切勿迁就患者,给予过多的检查和随便开药以满足患者的要求,使疑病观念强化或固定下来。要逐步引导患者从对自身的关注转移到外界,通过参加各种社交或工娱疗活动,使之逐步摆脱疑病观念。

2.认知治疗 让患者充分认识心理社会因素,诸如应激性生活事件、人际关系冲突等,认识与躯体症状的关系,改善对疾病的不良认知,接受疾病系心因性而非器质性的说法,转移对躯体症状的过分关注,减少继发性获益,改善对应激源的应对策略。

3.行为治疗 躯体形式障碍的患者病程为慢性波动性,常形成特有的行为模式,可进行行为分析与行为治疗,达到行为矫正的目的,以减缓躯体症状。可采取暴露疗法、行为强化训练等。

4.婚姻和家庭治疗 改善夫妻关系、家庭关系,争取家庭成员的理解与支持,调动患者本身的积极资源。

5.其他 可采取精神分析治疗、催眠暗示治疗、生物反馈治疗。

治疗的形式可采取个体治疗、团体治疗等。

\subsubsection{药物治疗}

躯体形式障碍的患者往往伴有明显的抑郁、焦虑、失眠等症状。在心理治疗的基础上合并抗抑郁药、抗焦虑药及苯二氮䓬
类药物可有效改善上述症状。随着目前抗抑郁药的快速发展,选择性5-HT回收抑制药(SSRIs)因其疗效肯定、不良反应轻微,已经成为首选药物。5-HT、NE双受体回收抑制药(SNRIs),NE能与5-HT能抗抑郁药(NaSSAs)等药物亦证明有效。此类药物不良反应小、用药方便,患者易接受,用药依从性好。对经济能力相对较弱的农村地区,亦可应用三环类药物(阿米替林、多塞平),但因其不良反应大,宜小剂量起用,药量个体化,让患者对可能出现的不良反应做到心中有数,避免增加不必要的心理负担。

在治疗的早期,合并苯二氮䓬
类药物可尽快减少患者的焦虑,增强信心,在疗效稳定后需缓慢减量至停用,避免形成药物依赖。

\subsection{预后}

短暂的疑病反应在一般人中并不少见,当自己关系密切的亲友死于或患有某种严重疾病,正常人往往会有短哲的疑病反应;当自己大病初愈之后,往往也暂时残留疑病观念。这种情况往往不持久,因外在应激产生的疑病反应大多随应激事件的消失而消失。但如果受到周围人或者医生的不适当影响(医源性因素),疑病反应就可成慢性化,致残率高,预后差。

躯体形式障碍患者病情呈慢性波动性,反复求治,反复检查,花费大量的人力、物力,却难以得到规范化的专科治疗。本病起病常在30岁以前,症状持续2年以上,慢性波动性病程,女性多于男性,预后不佳。一般急性或亚急性起病无人格缺陷者,预后较佳。长期随访发现,疑病患者中约20%呈慢性化或波动病程,预后不佳。

与预后有关的因素:明显诱发因素、患者满怀信心努力求治者、伴有焦虑和抑郁、急性起病、不伴有人格障碍、社会经济地位较高、年轻、不伴有器质性疾病、在其他疾病的基础上发生预后好;具有疑病性格,慢性病程,信心不足者,预后不佳。在儿童期发病的疑病症大多在接近成年期或成年早期缓解。

\section{神 经 衰 弱}

\subsection{概述}

CCMD-3神经衰弱的概念为:一种以脑和躯体功能衰弱为特征的神经症,表现为精神易兴奋,但易疲劳,以及紧张、烦恼、易激惹等情绪症状和肌肉紧张性疼痛、睡眠障碍等生理功能紊乱症状。症状不是继发于躯体和脑的疾病,也不是其他任何精神障碍的一部分。

现今国际上影响最大且为很多国家所采用的国际精神疾病分类第10版(ICD-10)虽然保留了这一名词,但限制使用。在国际上影响很大的另一分类系统美国精神疾病诊断手册(DSM)从第三版开始取消了这一分类。中国的精神疾病分类(CCMD)到目前为止仍然保留这一名词,而且中国的精神病学家们同意以往在中国神经衰弱的诊断确实存在诊断泛化、扩大化的现象,但仍然认为在临床实践中确实存在这样一组疾病。

神经衰弱(neurasthenia)这一术语有着悠久的历史。1869年Beard
GM在有关神经衰弱的文章中提出神经衰弱系神经力量耗损和衰弱的结果,是和美国的工业化过程等有关的“文明病”,多见于社会中、上层脑力劳动者中。之后Mitchell
SW提出了一个详细而富有诱惑力的方案,即在优越舒适的条件下静养,辅以丰富的营养、按摩等休息疗法治疗。到了19世纪中后叶神经衰弱的概念才被大家认识和使用,并广泛流传开来,逐渐被医生及平民接受,甚至变成了“睡眠障碍”及伴失眠的其他疾病的代名词。

现在神经衰弱的名词已被美国和欧洲等大多数西方国家所抛弃。大致原因有:①由于Beard当时概括的神经衰弱症状有50余种,几乎概括了所有神经症的症状以及许多精神疾病(如抑郁症、精神分裂症等)的早期症状,特异性很差,神经衰弱的概念不明确、分界不清;②随着时代的发展,对焦虑和抑郁等问题的关注,精神疾病的治疗手段有了长足的发展,原来诊断为神经衰弱的部分患者经抗抑郁、抗焦虑治疗后病情明显好转,诊断分别改为焦虑症、抑郁症等;③精神疾病的分类逐渐细化,到目前为止无法确切地找到“衰弱”的证据;④确有神经衰弱概念泛化,诊断范围扩大,甚至变成一种流行的用语等原因。西方国家近半个世纪以来抛弃了这一概念和术语,改头换面以“疲劳综合征”来描述那些确有相应症状但又无法用其他疾病解释的现象。

虽然目前国际上许多发达国家已经取消了神经衰弱的诊断和分类学地位,但是,仍然有许多国家坚持认为存在这样一组以疲乏、失眠、精神活力下降等表现为主的疾病,其中尤其以东方国家及发展中国家为主。在我国经久不衰地使用的大致原因:①中医和传统文化习惯的影响;②概念名词的可接受性;③前苏联尤其是巴甫洛夫学派的影响等等。

\subsection{流行病学调查}

我国成都1959年对7种不同职业的10830人进行调查发现,神经衰弱的患病率为59‰。由于对神经衰弱有了新的认识,调查标准和方法随之改变,患病率也明显下降。1982年全国12个地区精神疾病流行病学调查神经衰弱的患病率为13.03‰,1993年全国7个地区精神疾病流行病学调查神经衰弱的患病率为8.39‰。ICD-10精神障碍现患病率见表\ref{tab11-1}\footnote{材料引自D. P. Goleberg and Lecrubier},各国ICD-10精神障碍的现患率见表\ref{tab11-2}。

\begin{table}
\centering
\caption{ICD-10精神障碍现患病率}
\label{tab11-1}
\begin{tabular}{cc}
\toprule
诊断 & 患病率(\%) \\
\midrule
抑郁症 & 10.4\\
广泛焦虑症 & 7.9\\
神经衰弱 & 5.4\\
有害使用酒类 & 3.3\\
酒精依赖 & 2.7 \\
躯体化障碍& 2.7 \\
心境恶劣 & 2.1 \\
惊恐发作 & 1.1 \\
广场恐怖伴惊恐发作 & 1.0 \\
疑病症 & 0.8 \\
广场恐惧症 & 0.5\\
两种或以上合病 & 9.5\\
\bottomrule
\end{tabular}
\end{table}

\begin{table}
    \centering
    \caption{各国ICD-10精神障碍的现患率}
    \label{tab11-2}
    \begin{tabular}{ccc}
    \toprule
    调查地区 & 精神障碍总患病率(\%) & 估计神经衰弱症状群患病率 \\
    \midrule
土耳其安哥拉 & 17.6 & 4.1(23.3\%)\\
希腊雅典& 22.1 & 4.6(20.8\%)\\
印度班加罗尔& 23.9 & 2.7(11.3\%)\\
德国柏林& 25.2 & 7.4(29.4\%)\\
德国美因茨& 30.6 & 7.7(25.2\%)\\
荷兰格罗宁根& 29.0 & 10.5(36.2\%)\\
尼日利亚伊巴丹& 10.4 & 1.1(10.6\%)\\
英国曼彻斯特& 26.2 & 9.7(37.0\%)\\
日本长崎& 14.8 & 3.4(23.0\%)\\
法国巴黎& 31.2 & 9.3(29.8\%)\\
巴西里约热内卢& 38.0 & 4.5(11.8\%)\\
智利圣地亚哥& 53.5 & 10.5(19.6\%)\\
美国西雅图& 20.4 & 2.1(10.3\%)\\
中国上海& 9.7 & 2.0(20.6\%)\\
意大利维罗纳& 12.4 & 2.1(16.9\%)\\
    \bottomrule
    \end{tabular}
    \end{table}

神经衰弱的患病率高低不仅存在着时代的特点,同样由于与对疾病本质、分类的认识及民族、文化差异,不同国家之间神经衰弱的患病率差距较大,存在着明显的地域及文化性。

\subsection{病因和发病机制}

神经衰弱的病因和发病机制至今尚无定论。

\subsubsection{素质、躯体、心理、社会和环境等的综合作用是病因}

1.生理基础 巴甫洛夫学派认为神经衰弱存在一定的生理基础,即神经活动类型属于弱型或中间型。其特点是神经系统的抑制过程较弱,而相对兴奋性较高,对刺激反应迅速,兴奋性阈值较低,但由于兴奋频繁且过度,易出现疲劳,最终由于疲劳及耗竭神经系统出现保护性抑制,神经系统调节功能削弱,出现自主神经功能紊乱。

2.心理基础 心理社会因素是否能成为神经衰弱的致病因素,决定于这些因素的性质、强度、作用持续时间,更重要的还取决于思考对这些因素的态度和情感体验等。神经衰弱者的性格存在一定的特点,如孤僻、胆怯、自卑、敏感、多疑、依赖性强、缺乏自信、主观、任性、好强、急躁、自制力差、易冲动等。

3.应激 生活事件引发的长期心理冲突和精神创伤,大脑和精神过度紧张和疲劳,易诱发神经衰弱。①社会心理因素:如工作学习生活不适应、恋爱婚姻家庭不顺利、亲人的丧亡及人际关系的紧张等负性情绪,以及各种突发生活事件、生活节奏改变等促发因素。②机体状态:如体质、感染、中毒、脑外伤、慢性躯体疾病等因素对机体状态的影响可能成为诱发条件。③影响程度:如促发因素和诱发条件的性质、强度、持续时间及累积作用等。④认知评价及应对策略:如对各种生活事件的认知评价导致不同的态度及情感反应,应对策略影响着应对效果及对机体影响程度等。

\subsubsection{发病机制}

早在1869年Beard
GM就提出了神经衰弱系神经力量耗损和衰弱结果,而杨德森等用“能量”消耗导致“疲劳”,即“疲劳”是“能量”消耗的结果,反过来又阻止机体过度消耗等解释神经衰弱的发病机制。

神经系统与机体其他各个系统或器官一样,均有较大的功能储备,代偿及容忍范围较宽,在一定范围内的负荷与消耗不至于导致功能失调。但当消耗和磨损持续存在,且超出一定范围时,系统或器官的功能储备、代偿及容忍范围会逐渐丧失,如进一步遭遇负荷和压力,就可能会出现功能紊乱及平衡失调。如果某人既存在一定的生理或心理基础,又在一定的时间范围内遭遇一定强度的应激,且修复、缓解不好,那么出现问题的可能性就大大增加。然而究竟出现哪种问题,是心身疾病、神经症,还是抑郁症、心因性精神障碍、精神分裂症等,则取决于各种因素的作用强度、相互作用、生理遗传性特点等。神经衰弱与此机制有关。

\subsection{临床表现}

1.紧迫感 感到任务迫在眉睫或事情太多而时间不够用。患者整天忙碌,好像天天都在赶任务,总有做不完的事,从来没有事情告一段落暂时松一口气的体验。其可以是陷于紧张状态后才有的,也可以是病前人格的一个特性。

2.负担感 感到肩负的责任重大,力不从心,也担心失职;感到名誉地位太高,能力和品德都不能胜任;感到任务太困难,尤其是人际关系太复杂,“内耗”太大,怎么努力也很难干好。因此,原有的工作乐趣和成就的满足感或喜悦都消失了,工作成了纯粹的精神负担。

3.效率下降感 感到工作、学习进展缓慢,质量不佳,失误太多,尤其是和过去比感到尽管更加努力成绩却下降了。实际上,成绩并未下降,至少别人还看不出来。

4.精神功能下降感 感到脑子不像过去好使了,注意力集中难于持久,容易分心,杂念多,记忆力变坏了。

5.感到过敏 感到自己变得不冷静了,容易着急、急躁和生气,容易因刺激(如突然的关门声、闪过一个人影子等)引起惊跳反应,常常因为一点小事而情绪久久不能平息,或由于情绪反应过于强烈而后悔,会伤了和气。

6.自控失灵感 感到必须加强自我控制,如果放松控制,似乎工作、学习和人际关系就会出大问题,担心自己会情绪暴发或有过激的言语行动。

7.缺乏轻松愉快的体验 总是不踏实,放心不下,似乎有重大的疏忽或像做了什么亏心事一样,完全体会不到轻松愉快的心情,过去喜欢的休闲活动和赏心乐事现在完全不能享受了。

除紧张状态外,神经衰弱主要以精神活力下降、情绪症状和生理功能紊乱等三组症状为主。

精神活力下降分为衰弱和兴奋两个方面。衰弱表现为精力不济,萎靡不振,用脑困难,注意力不能集中,四肢乏力,困倦嗜睡,工作效率下降等;而兴奋则表现为不自主的回忆与联想增多,思维活跃,但缺乏有效的指向性,怕光、怕吵等(这不同于情感性精神障碍躁狂的兴奋,没有相应的活动增多),伴有相应的不愉快感,兴奋症状在夜晚和入睡前常更加明显,严重影响睡眠。

情绪症状以烦恼和易激惹为主要表现,一方面感到困难重重,另一方面又感到能力不济,遇事容易激动、烦躁、易怒,常后悔,可伴有一定的焦虑和抑郁。此外还可能存在紧张不安、担心多虑,整天愁眉苦脸等。

生理功能紊乱表现为紧张性疼痛包括头痛,四肢、腰背部酸痛,还有头昏、头重,伴有紧缩感,头胀等;睡眠障碍,以入睡困难、多梦、易惊醒、自觉睡眠浅不解乏、无睡眠感、白天嗜睡等为主。睡眠问题常常是患者最关心及引起重视、迫切要求解决的问题,他们常认为只要能睡好,问题都可以得到解决,因此特别看重睡眠,越是睡不着越着急,常形成恶性循环。其他生理功能障碍还包括心悸、耳鸣、心慌、眼花、胸闷、多汗,腹胀、消化不良、性功能障碍等。

\subsection{诊断及鉴别诊断}

神经衰弱的症状表现多种多样,但缺乏特异性,常可见于各类神经症、抑郁症、精神分裂症的早期、某些脑器质性疾病的早期和恢复期等,诊断上一定要严格掌握,避免误诊。

\subsubsection{CCMD-3的诊断标准}

1.符合神经症的诊断标准。

2.以脑和躯体功能衰弱症状为主,特征是持续和令人苦恼的脑力易疲劳(如感到没有精神,自感脑子迟钝,注意力不集中或不持久,记忆差,思考能力下降)和体力易疲劳,经过休息或娱乐不能恢复,并至少有下列2项:

(1)情感症状,如烦恼、心情紧张、易激惹等,常与现实生活中的各种矛盾有关,感到困难重重,难以应付。可有焦虑或抑郁,但不占主导地位。

(2)兴奋症状,如感到精神易兴奋(如回忆和联想增多,主要是对指向性思维感到费力,而非指向性思维却很活跃,因难控制而感到痛苦和不快),但无言语运动增多。有时对声光很敏感。

(3)肌肉紧张性疼痛(如紧张性头痛、肢体肌肉酸痛)或头晕。

(4)睡眠障碍,如入睡困难、多梦、醒后感到不解乏,睡眠感丧失,睡眠觉醒节律紊乱。

(5)其他心理生理障碍,如头晕眼花、耳鸣、心慌、胸闷、腹胀、消化不良、尿频、多汗、阳痿、早泄或月经紊乱等。

3.严重标准 患者因明显感到脑和躯体功能衰弱,影响其社会功能,为此感到痛苦或主动求治。

4.病程标准 符合症状标准至少已3个月。

5.排除标准 排除任何一种神经症亚型;排除精神分裂症、抑郁症。

6.说明

(1)神经衰弱症状若见于神经症的其他亚型,只诊断其他相应类型的神经症。

(2)神经衰弱症状常见于各种脑器质性疾病和其他躯体疾病,此时应诊断为这些疾病的神经衰弱综合征。

\subsubsection{鉴别诊断}

神经衰弱的名词曾变成一种社会流行用语,人们往往把睡眠不好都归入神经衰弱范畴。精神医学界在20世纪80年代以前,诊断神经衰弱的比例也相当高。神经衰弱的鉴别诊断见表\ref{tab11-3}。

\begin{table}
    \centering
    \caption{神经衰弱的鉴别诊断}
    \label{tab11-3}
    \begin{tabular}{ccccc}
    \toprule
    & 神经衰弱 & 躯体形式障碍 & 心境恶劣 & 抑郁症 \\
    \midrule
    人格特征 & 衰弱、敏感 & 内激敏感、暗示性高 & ? & ?\\
    应激诱因 & 2+ & 2+ & $\pm$ & $\pm$\\
    抑郁程度 & $\pm\rightarrow+$ & $\pm$ & $+\rightarrow 2+$ &  $+\rightarrow 3+$\\
    紧张与担心 & + & $\pm$ & $\pm$ & $\pm$\\
    易激惹、克制力差 & 2+ & - & + & +\\
    注意力不集中 & 2+ & - & + &\\
    记忆力差  & 2+ & - & + &\\
    脑力易疲劳  & 2+ & - & + &\\
    躯体不适主诉 & 2+(恒定) & 2+(多变) & + & +\\
    睡眠障碍 & 2+ & $\pm$ & + & +\\
    \bottomrule
    \end{tabular}
    \end{table}

在诊断的方面应该充分认识到:

(1)以往神经衰弱的诊断确实存在泛化和扩大的过程,有滥用之嫌,今后应严格掌握标准。

(2)强调三组突出症状的作用和地位。

(3)按照等级诊断的原则,尤其是基层精神科医生和其他科医生使用时要特别注意,诊断时需充分排除神经、精神科的其他疾病,特别是脑器质性疾病、抑郁症、各类其他神经症及精神分裂症的早期后,方能诊断本病。

\subsection{治疗}

1.神经衰弱的治疗策略

(1)确立长期治疗和恢复的指导思想:由于神经衰弱系长期消耗和磨损的结果,那么治疗和恢复也必定是一个漫长的过程,不可能在短时间内达到治疗和恢复的目的。在治疗上,无论医生和患者均必须确立长期治疗和恢复的指导思想,切勿操之过急。

(2)以提高功能储备为首要目的:由于神经衰弱系消耗和磨损导致功能储备、代偿及容忍范围丧失后出现的功能紊乱及平衡失调,那么治疗的根本就在于重新恢复功能储备、代偿及容忍范围。治疗既要立足长期,更要注重功能储备、代偿及容忍范围的恢复。

(3)了解自身特点:每个人都有自己的生活和行为节奏、能力和容忍范围,在自己的节奏、能力和容忍范围内适应性良好,但一旦超出自己固有的节奏、能力和容忍范围,就会感到压力。因此,对自身特点的了解有助于根据实际情况采取适当的生活、行为方法,避免压力过大。

(4)适当改变行为模式(节奏)及摆脱应激:尽管每个人都有自己的生活和行为节奏、能力和容忍范围,但是人们始终生活在真实的社会环境中,逃避和寄希望于环境改变是不现实的。改变环境不如适应环境。因此,在充分了解自身特点的基础上,面对自己不能控制的环境,适当改变自己不良的行为模式、应对方式,将有助于适应环境,摆脱应激。

(5)不应拒绝用药:固然长期、大量用药不是好办法,会导致药物依赖,增加药物不良反应。但是由于神经衰弱发病机制是消耗、磨损与失衡,疲劳是机体自我保护的一种机制,治疗的根本在于重新恢复功能储备、代偿及容忍范围。对于神经衰弱患者而言,这种机体的自我保护已经变得苍白无力,必须用切实有效的方法加强保护、促进恢复,这时药物的作用是毋可置疑的。此外,如果拒绝用药或间断用药,不仅起不到保护和治疗的作用,相反会错过治疗的最佳时机,造成将来治疗上的困难。有些患者开始因各种原因拒绝用药或不规则用药,最后常不得不用药。所以,正确的治疗方法应是在治疗初期坚决用药,待症状缓解后再逐步考虑药物剂量等问题。

(6)治疗与预防并举:由于神经衰弱的发生存在一定的生理和心理基础,而这些基础有一些经过努力可以改变,还有一些则改变起来较困难。因此,在治疗神经衰弱的同时要对“基础”问题有一个清醒的认识,治疗应与预防并举,立足长远,通过锻炼、自我矫正等方法改变发病基础,根本治愈神经衰弱。

2.治疗方法

(1)药物治疗:到目前为止,治疗神经衰弱的各类各种药物很多,但尚未发现有哪类哪种药物有独特疗效。一般采取对症治疗。以抗焦虑药为主,酌情小剂量使用抗抑郁药、镇静催眠药、促脑代谢药等。

(2)心理治疗:认知疗法、森田疗法、放松训练、生物反馈等。

(3)中医治疗:中药、针灸、气功、艾灸等。

(4)其他:如按摩、水疗、脑功能保健、电磁治疗、光电子治疗、音乐治疗等。

\subsection{病程与预后}

神经衰弱常起病于成年早期,也有可能起病于青壮年。神经衰弱的治疗存在着相当的难度。其主要原因是早期症状不严重,对患者的社会功能损害不大,患者常未引起足够的重视,耽误了治疗的最佳时期,使疾病逐渐进入慢性状态;另一方面由于神经衰弱是一种耗损及失衡性疾病,而患者常一直生活在原有的环境及心理条件下,应激作用仍持续存在,加之缺乏特效及迅速的治疗方法,神经衰弱常迁延不愈。

\section{癔  症}

\subsection{历史与概述}

癔症(hysteria)又称歇斯底里,指一种有癔症性人格基础和起病常受心理社会因素影响的精神障碍。其主要表现为解离症状(CCMD-3称为癔症性精神症状)和转换症状(CCMD-3称为癔症性躯体症状)。这些症状没有可证实的器质性病变基础,并与病人的现实处境不相称。本症除癔症性精神病或癔症性意识障碍有自知力障碍外,自知力基本完整,病程多反复迁延。

癔症是精神病学诊断术语中最为古老的病名之一。在公元前已被描述,希腊人认为癔症是妇女特有的一种疾病,是子宫在腹腔内游走而致病。希波克拉底认为,怀孕和婚姻是最好的治疗方法。在中世纪,或者更晚一些时期,癔症很容易与魔法混淆,至19世纪后期才开始对本症进行科学的研究。Charcot强调暗示与自我暗示在本症发病机制中的作用;Janet认为本症是人格解离造成的;Freud从癔症研究开创精神分析理论,认为本症是由性创伤、特别是婴儿期本能被压抑引起,受到创伤的性本能通过转换机制,产生麻痹、痉挛等症状。Freud把具有这类症状者称为转换性癔症(conversion
hysteria),转换性障碍的名称也由此而来。

癔症的归属一直争论不休。历来将解离(转换)性障碍(dissociative
disorder)归入癔症的不同类型。ICD-10和DSM-Ⅳ都取消了“癔症”这一名称,理由是其含义太多且不确定,建议最好尽量避免用“癔症”一词。我国CCMD-2-R和CCMD-3虽然保留了“癔症”一词,但在临床描述、分型等方面注意和ICD-10的分类一致。近年来把癔症划出神经症的意见已占上风,CCMD-3把癔症单列一病。

解离(转换)性障碍的共同特点是部分或完全地丧失了对过去的记忆、身份意识、即刻感觉以及身体运动控制四方面的正常整合。我国学者将对过去经历、当今环境和自我身份的认知不符,称为解离症状;而将生活事件或处境影响下出现的躯体症状,称为转换症状,意指个人内在的冲突引起的不愉快情感以某种方式变形为躯体症状。转换症状的确立需排除器质性病变。

\subsection{病因和发病机制}

\subsubsection{心理、生理、社会和环境等的综合作用是病因}

解离(转换)性障碍的病因与精神因素关系密切,各种不愉快的心境,愤怒、惊恐、委屈等精神创伤常是初次发病的诱因,以后因联想或重新体验初次发作的情感可再发病,且多由于暗示或自我暗示引起。

有易感生理、心理素质者易发本症,发生本症具癔症性格特征者约占49.8%。其性格的主要特点为:

1.情感丰富,但肤浅,似一名蹩脚演员,看得出其在表演。凭情感分辨好恶,所谓情感逻辑,好者欲其生,恶者欲其死。

2.以自我为中心。

3.暗示及自我暗示性强。

4.丰富想象,甚至以幻想代替现实。

部分患者有遗传素质,而躯体疾患也可引起自我暗示,削弱神经系统功能,可成为发病的客观条件。

社会文化素质如风俗习惯、宗教信仰、生活习惯等,对本症的发生与发作形式及症状表现等也有一定影响。

\subsubsection{癔症的发病机制尚无定论}

根据文献报道,躯体疾病与癔症性症状有5种方式相关。

(1)独立的情绪紧张可能使患者不断地诉说器质性损害造成的不幸后果,而此器质性损害已经扰乱了正常的功能。因此,功能减弱的手可能会瘫痪。

(2)过去的或间断出现的躯体症状,如癫痫
,为癔症性症状提供了模式。心理刺激则更容易产生患者所了解的某种疾病的形式。曾发生在亲戚或熟人中的疾病也会起到这种作用。

(3)躯体疾病的不愉快的心理含义,如与躯体疾病有关的不适或恐惧,可能会使患者不断诉说已经存在的症状或产生的新症状。

(4)躯体疾病引起行为的衰退产生疾病角色。这种疾病角色的好处可能是收到祝福康复的卡片、水果或鲜花等礼物,也可能逃避痛苦的责任和心灵中进退两难的困境。

(5)脑损害可能以我们尚不知道的方式特异性地产生转换性症状。不管内在机制是什么,躯体疾病,尤其器质性脑病,是导致癔症性症状发展的重要因素。

癔症的发生和流行随着时间、地点及文化的不同而异。例如,在第一次世界大战期间,前线战士中可见到很高的发病率,而当今的精神临床中这种情况很罕见。美国妇女、非白人、社会经济地位低者发病率较高(Stefannson等,1976年)。瑞典对普通人群调查发现,癔症的发病风险为0.5%(Lazare,
1981年)。在精神科的会诊临床中,转换性症状的发病率高,为5%~16%(Lazare,
1981年)。疼痛是最常见的症状,转换性症状出现在器质性疾病患者中。多种躯体化症状在精神科住院患者中约为10%(Blbb和Guze,1972年),在精神科门诊患者中为5%~11%(Guze等,1971年)。在一些发展中国家,门诊报道的这类症状出现率更高。例如Hafeiz(1980年)发现在苏丹为10%。有转换症状和具有多种躯体化主诉的患者之间有相当一部分的重叠,因为后者较常见。有资料显示,最近几十年癔症的患病率不断下降。

\subsection{临床表现}

1.癔症性精神障碍 也称为解离性障碍,其临床表现主要为意识及情感障碍。DSM-Ⅲ和Ⅳ根据其临床表现分为心因性遗忘、心因性神游、多重人格、人格解体障碍及非典型解离性障碍几类。

意识障碍以意识狭窄、朦胧状态为多见,意识范围缩小,有的呈梦样状态或酩酊状态。意识障碍时,各种防御反射始终存在,并与强烈的情感体验有关,可以有哭笑打滚、捶胸顿足、狂喊乱叫等情感暴发症状。有时呈戏剧样表现,讲话内容与内心体验有关,因此容易被人理解。

解离性障碍有以下一些特殊形式:

(1)童样痴呆:较多见,其表情、行为、言语等精神活动都回到童年,稚气十足,且表现过分,看得出其做作色彩,装出两三岁无知孩子的样子。

(2)Ganser综合征:对问题能正确领悟,答案与标准近似,但不正确,给人以故意做作或开玩笑的形象。如问一患者:“2+2等于几?”,其答“3”或“5”,而在有些行为方面却不能显示痴呆。缓解后,其谓刚才似在梦中。

(3)假性痴呆:向其提简单问题,均回答“不知道”,或借口搪塞;相反,对复杂问题的回答,却能正确无误。

(4)癔症性遗忘症:主要表现为突然出现的、不能回忆自己个人重要的事情(例如姓名、职业、家庭等)。主要特点是记忆丧失,通常是重要的近期事件,不是由器质性原因所致。遗忘范围之广也不能用一般的健忘或疲劳加以解释。遗忘可以是部分性的和选择性的,一般都是围绕创伤性事件,如意外事故或意外亲人亡故。遗忘的程度和完全性每天均有不同,遗忘的阶段常与所受创伤的时间吻合,故为阶段性的,伴茫然的表情。不同检查者所见也不一样,但总有一个固定的核心内容在醒觉状态下始终不能回忆。该症患者在首次进入一种不同的意识状态时,也可能正在进行某项复杂的活动,为此他可能疑惑不解,常感到糊涂而迷失方向。这种状态可持续数分钟至数小时,随后意识到自己不能回忆一段时间内所发生的事情,有时甚至不能记得自己的身份。在这种时候,有些患者为自己的遗忘感到悲哀,有些则漠然视之。遗忘一般突然缓解,且很少复发。

(5)癔症性漫游症或神游症:表现为遗忘和身体的逃走,往往是离开一个不能耐受的环境。从某一地方向另一地方游荡,时间可达几天或更长些。在这期间的行为相当,过后完全遗忘。典型的神游极为少见。患者离开家去旅行,以一个新的身份,而对原来的自己不能回忆。可以是很短的一段生活,仅是一段不长的旅行;但也可以持续较久,甚至在另一个地方开始全新的生活(例如,一位男子可能离开家到了另一个城市,有一个新的工作,有一群新朋友)。该病往往出现在面对重大的创伤事件时,如自然灾害或战争。有些报道认为患者的病理心理素质和成长于破损家庭的历史可以使其易出现解离性漫游。许多有这种障碍的患者,具有强烈的解离性焦虑和自杀或杀人的冲动意念。

(6)癔症性身份障碍:以往称为多重人格障碍,主要表现为患者存在两种或更多种完全不同的身份状态,在一个人身上先后或交替出现,每种身份均有自己独有的记忆、观点和社会关系。每种身份都很突出并可决定患者在不同时间的行为。从一种身份向另一种身份的转换常常是突然的。这些身份侧面的表现常常截然不同,却是代表了患者身份中不能整合的各个方面。例如,一个患者可能有一个害羞的、拘束的身份;另一个却是热衷社交、男女关系随便的身份;第三个可能是充满敌意和怀疑的身份。每种身份均有相应的称呼。每种身份状态可有不同的声音、姿态、面部表情、手势,甚至对药物不同的反应或过敏性;心理测验得到不同的结果。各个身份之间并不意识到其他身份的存在,只是在另一身份活动时,该身份感到好像失去了一段时间的存在。有些患者各身份间彼此清楚对方的存在,可为工资争斗,亦可彼此妥协。这是一种特殊意识障碍,可以由特殊的环境、应激性处境或精神内部的冲突引发。

解离性身份障碍大多数患者(近98%)具有幼年性或身体虐待史。有的孩子适应了这种虐待性生活,成年后也成为有虐待行为的人。有可能某些孩子有进行解离的先天素质。这种素质加上不恰当的社会支持系统(例如离开家),应对创伤性事件缺乏其他防御方法,如此一来,会出现持久、广泛地使用解离性防御机制,从过分的、无法接受的环境逃开。

(7)癔症性精神病:在受到严重精神创伤后突然发病,症状多变,主要表现为明显的行为紊乱、哭笑无常、表演性矫饰动作、幼稚与混乱的行为、短暂的幻觉、妄想和思维障碍及人格解体等。本病女性多见,病程很少超过3周,可突然痊愈而无后遗症,但可再发。

2.癔症性躯体障碍 也称为转换性障碍。本组疾患临床表现有以下特点:

(1)临床症状主要表现为感觉、运动功能丧失或部分障碍。找不到可解释症状的躯体疾患,但患者的表现却恰似真“有病”。

(2)所见症状常反映出患者对躯体疾病的概念,即是其想象中的应该如此表现。与客观的生理或解剖机制不相符合。

(3)病前往往存在应激性事件。通过对患者精神状态和社会环境的评定,常可发现。功能障碍所致的残损有助于患者逃避不愉快的冲突,甚至有继发性获益。虽然他人能清楚地看到存在的问题已经解决,但患者对此一概否认。

(4)症状严重程度与周围环境有关,暗示性言语或行为可以造成症状的波动。

(5)患者不愿探究躯体症状的心理性病因,但对已有残疾表现出惊人的冷静接受,即“漠然置之”。通过观察,可发觉其关注与应激事件有关的问题。

其可分为以下三类:

①转换性运动障碍:多表现为一个或几个肢体的全部或部分丧失功能;瘫痪可为部分性的,即运动减弱或运动缓慢,也可为完全性的;可有各种形式和程度的共济失调,尤以双腿多见;也可有一个或多个肢端或全身的夸张震颤。总之,可以表现为近似以下疾病的任何形式:共济失调、失用症、运动不能症、构音困难、异常运动、瘫痪。

②转换性感觉障碍:主要表现为转换性感觉麻木和感觉丧失。症状表现不符合感觉神经的分布。其中,视觉障碍很少是完全性的,多为视野模糊或“管状视野”。感者虽有视觉丧失的主诉,却惊人地保留着完好的活动能力和运动表现。

③躯体化障碍:以多种多样、经常变化的躯体症状为主,可涉及任何系统和部位。其最重要的特点是应激引起的不快心情,以转化成躯体症状的方式出现。在此基础上,又附加了症状主诉的主观性,常坚持把症状归咎于某一特定器官和系统,尽管查体和各种辅助检查不能发现器质性病变。

其临床常见的表现有:

①抽搐大发作:发病前常有明显的心理诱因。抽搐发作无规律性,没有强直及阵挛期,常为腕关节、掌指关节屈曲,指骨间关节伸直,拇指内收,下肢伸直或全身强硬,肢体阵发性乱抖、乱动。发作时可伴哭叫,呼吸呈阵发性加快,脸色略潮红,无尿失禁,不咬舌。发作时瞳孔大小正常,角膜反射存在甚至反而敏感。意识虽似不清,但可受暗示使抽搐暂停。发作后期肢体不松弛,而大多为有力的抵抗被动运动;无病理反射,如发作后期出现阳性跖反射者,提示有器质性病变。一般发作可持续数分钟或数小时。例如,某市郊外一厂领导干部,自厂部乘车赴市区,途遇一车迎面而来,为避开对方,不幸双双翻车,患者未受伤,还在现场指挥抢救工作。可当他再次想起当时危险情境时,突然发生全身抽搐,神志欠清,经送医院救治方愈。以后每当他走过出事地点时就有同样的发作,只得绕道而行。

②瘫痪:可表现为单瘫、偏瘫、截瘫、四肢瘫痪(下肢多见),但不符合解剖特点,常以关节为界;要求瘫痪肢体运动时,可发现拮抗肌肉收缩。将瘫痪肢体上抬,检查者突然放手时,瘫痪肢体徐徐落下,而不与中枢性瘫痪远端重于近端、周围性瘫痪近端重于远端的特点相符。下肢瘫痪,腿拖着走,而不是借髋部力量先将腿甩到前面。虽走路歪斜,但会支撑,很少跌倒。下肢瘫痪者卧位时,下肢活动自如,但不能站立行走,如扶之行走,则比真正器质性患者还要困难。但当患者确信旁边无人时,则行走很好,没有提示器质性病变的肌张力及腱反射改变或阳性病理反应。

③各种奇特的肌张力紊乱、肌无力、舞蹈样动作,但不能证实有器质性改变。例如一青年男子,因儿子夭亡,哀伤不已,之后经常有手舞足蹈的怪异动作,有时日发数次。送医院注射一支葡萄糖酸钙溶液后即愈,以后改用氯化钠注射液注射并予暗示,均即愈。

④失音、失语,但没有声带、舌、喉部肌肉麻痹,咳嗽时发音正常,能轻声耳语。

⑤视、听、嗅如有功能性障碍,也均无病理改变。

⑥皮肤感觉障碍,痒、烧灼感、刺痛、麻木感、酸痛,但不符合神经分布特点,且有矛盾出现。如一患者可用“无感觉”的手凭借视觉扣扣子;针刺“麻木”的皮肤时均答“没有感觉”。

⑦若有转换性痛觉,可从患者夸张的言辞及表情,病变部位的弥散,所说的语意不详,局部封闭治疗不起作用,佐以既往病史、心理因素等予以诊断。

⑧躯体化障碍表现最常见的是胃肠道症状,如疼痛、打嗝、反酸、呕吐、恶心、食欲不佳等。性和月经方面的主诉也很常见。常存在明显的抑郁和焦虑。

⑨中医所谓“卒然无音”、“气厥”、“梅核气”等的症状,大多归属于此。

\subsection{诊断与鉴别诊断}

\subsubsection{诊断}

诊断本症的主要依据为:

1.有解离性障碍与躯体功能障碍,特别是神经系统功能障碍,有充分证据排除器质性病变。

2.心理需要和心理矛盾有关的精神刺激,与症状的发生、发展或恶化具有暂时性联系。

3.症状妨碍社会功能。

4.可有模拟症状及淡漠处之的表情。

5.不能以躯体疾病的病理生理机制解释,甚至和神经解剖生理相矛盾。

6.不是其他精神病。

据文献报道,原先诊断为转换性精神障碍者经追踪随访,其中13%~30%的患者系器质性疾病,大多是神经系统疾病。

\subsubsection{鉴别诊断}

本症特别要与下列疾病鉴别:

1.额叶病变者 精神症状出现较早,有的欣快,有的情绪低沉,50%的患者可有全身抽搐发作,有强握、摸索动作。

2.多发性硬化 早期易与转换性障碍相混。

3.脑震荡或严重脑外伤后的遗忘 通常是逆行性的,在严重病例也可见顺行性遗忘。解离性遗忘也常是逆行性的,但可经催眠或发泄加以改变。

4.癫痫
 尤为精神运动性癫痫 。

5.诈病 其动机是在意识上,只欺骗别人,不欺骗自己;而转换性障碍者既欺骗别人,又骗了自己。

6.精神分裂症、反应性精神病、躯体化障碍等。

7.反应性精神障碍 患者在遭到强烈的精神刺激后立即发病,表现有明显的意识障碍与狭窄,不能正确感知周围事物,对时间、地点或人物定向发生障碍,理解困难,同时常伴有表情迷惑与注意力涣散,言语凌乱,不连贯,使人难以理解。其精神症状的内容多与精神因素引起的情绪体验有一定的联系,当精神因素解除后症状很快消失。

8.拘禁反应 又称监狱精神病。患者在拘禁情况下,表现回答问题的方式及其行为具有荒谬特征,给人以严重痴呆的印象。例如患者答不出自己的姓名、年龄,不知有几只手、几个手指,分不清早和晚、左和右、白天黑夜。有时叫错或叫不出日常生活用品的名称,反应迟钝,动作不灵活,表情呆板,但拘禁情况解除后,患者一切表现正常。

\subsection{治疗}

\subsubsection{一般治疗}

以心理治疗为主,如说理开导、疏泄鼓励、支持保证、自我松弛、催眠暗示、行为疗法等。给患者以心理治疗时,需得其家属配合。有不少家属在患者发病时大惊失色,这样反而加重其症状。

如焦虑或抑郁症状严重者,可给予抗焦虑、抗抑郁药物。有时药物暗示也可起到一定的效果。

诊断过程即是治疗的开始,在采集病史及体检中发展的医患关系非常重要。偶尔初次交谈的发泄作用会彻底打消症状。不幸的是,在初次会谈结束时,有人试图劝说患者他们的躯体症状是“思想问题”,而这样的说法会给患者以误导。患者总是抗拒这种信息,精神科医生的任务是找到有效的方法来解决这个问题。直接向患者解释症状是无意识的,或告诉患者换个角度想,症状可能就会被治愈,对患者的这些说教都没有任何效果,甚至起反作用。在与患者第一次会谈时,不要唐突地作任何评论,在没有充分的支持证据时也不对患者说出任何定论,这是十分重要的。更为有用的是给患者以不确定和亲切的印象,让患者认为医生需要更多的时间来考虑出正确的观点,而不要向患者提出不能说服他们的非实质性的结论,以免遭到患者的反对。

对癔症性症状的治疗要求耐心、全面、逐步,但坚定地进步。我们很容易对患者说:“症状在你的思想中,我们需要改变你对疾病的认识”。患者会回答:“对,我也希望改变,但请你告诉我怎么做。”还有的患者可能一点也不相信他们的瘫痪实际上是不存在的。更好的做法是不直接向患者解释症状,直接的解释经常导致医患间的互不理解,有时甚至是直接的敌视态度。

与患者亲属的会谈可能会得到重要信息,有时会使亲属的态度转变,对患者很有帮助。在医生的询问下,患者可能会慢慢地改变看起来固定的态度或想法,尤其当他们开始讨论令人烦恼的人际关系和其他敏感问题时。

必需识别和排除伴随的器质性疾病。作者在临床实践中发现,表现为典型转换性症状的患者中约60%的人伴随躯体障碍。最常见的器质性原因是抗抽搐药物和其他药物达到中毒水平,常表现为吐词不清或神经反射异常,减少药物剂量后很容易治愈。转换性症状也可由抑郁症,偶尔由精神分裂症引起,对这些疾病的适当治疗也会治愈癔症性症状。

\subsubsection{特殊性心理治疗}

1.理解症状 如上所述,医生努力理解和减轻患者的症状会加强医患联系。另外,在治疗中逐步应用暗示方法,使患者减轻症状。症状缓解的部分原因是疏泄机制,但也可能仅仅因为患者的生活有所改变。对人际关系的重视可能通过精神动力学过程使症状缓解。

有些临床医生认为癔症性症状是发出者和接受者之间的信息所致,他们努力使患者明白交往的意义和原因,希望随着更直接和更恰当的行为被患者采纳,患者会放弃症状。

2.内省性心理治疗 长期的心理分析性治疗通常不是最好的方法,因为心理问题或者是太强烈,或者是根本看不到。但是,治疗者应安排6~10次会谈,一般性地探索情感问题,而不是仅仅集中在症状上。在其他简短的心理治疗中,治疗者应起到更主动的作用。在这些会谈中,应谨慎地处理甚至完全回避转移性的解释。这是因为患者的愤怒和敌视经常是表浅的,而集中在这些情感会使情况恶化而不会更好。如果谈论到敌视,最好是限制在患者的“情感”“精神紧张状态”、易激惹的情绪、对情景的担忧和他们对治疗的期待中,而不是医生对症状的象征意义做出的大胆解释。

麻醉分析既是诊断性的,也是一种心理治疗。通过注射异戊巴比妥钠或其他镇静药的麻醉分析被广泛地应用于刚从战争前线下来的士兵。对这种情况下的患者会收到很好的效果。通过口服镇静药及交谈可能也会得到同样的效果。Stengel认为麻醉分析与一系列的会谈所起到的效果相当,现代的证据也证明了这一点(Piper,
1994年)。

3.暗示 暗示是主要的传统治疗,现在仍在广泛应用。间接暗示也经常应用于心理治疗中。直接暗示包括再训练和行为技术。

引导患者说“ah”,再尝试更复杂的音节如“baa,
paa”直到恢复正常的语言。在心理和社会支持的背景下这种方法经常获得成功。暗示也可使瘫痪缓解。一个容易成功的方法是:躯体治疗者提供一系列难度进行性增加的躯体治疗程序,旨在提高运动功能,这样就不需要再担心患者的内心想法及能力。

有些治疗者喜欢催眠术,可能因为这是普遍接受的方法,且提供一个保留脸面的做法。但这可能并不优于暗示的方法。关于催眠状态是否为特殊的状态还有争议。催眠可能伴随意识的改变,在生理上不同于正常的清醒和注意。理论学者相信上述情况,认为催眠状态是特殊的生理改变。因为没有生理学证据,故这种观点也受到挑战(Merskey,
Bafber,
1969年),已证明任务动机性暗示与催眠术产生催眠性指导或暗示的过程常可以解除急性症状,但很少改变慢性症状。在没有给予患者持续性的间接或直接的心理支持保证的情况下,不主张进行催眠及用精神动力学方法进行治疗。

\subsection{病程与预后}

样本最大的随访调查(Ljungberg,
1957)发现,1年后43%男性及35%女性患者仍有症状,5年后仍有症状者分别为25%和22%,10~15年后没有改变。在已经恢复者中,大约7%在1年的随访中复发。年龄和智力对于复发没有影响,但发病前不良的人格对发病有不利作用。

Lewis(1975年)调查的预后令人较乐观,他的随访为7~12年,75例中7例死亡,其中3例有中枢神经系统疾病,1例自杀。存活者中2/3完全康复,能够工作和从症状中解脱。其余少数好转但仍有症状,也有的恶化。在战争中发生的癔症性症状或者流行性癔症均完全康复。在急性转换性症状较常见的发展中国家,也报道了乐观的预后(Hafeiz,
1980年)。

转换性与解离性症状的结局与疾病的严重性及缓急有很大关系。严重、表演性的症状可能对治疗反应良好。而许多其他障碍症状容易持续存在。在所有病例中,病程越短,预后越好。多数初次发病者恢复迅速。如果病程超过一年,可能要持续多年才能恢复。一般预后良好,恢复不理想的患者有癔症性人格障碍和社会适应困难。

歇斯底里一语至少有以下10种不同的含义或用法:

1.转换症状 转换(conversion)是S.
Freud的一种理论构想,用以说明歇斯底里躯体症状发生的机制。转换症状的描述性定义是:某一生活事件或生活处境引起了患者一定的情绪反应,通常是明显的,看上去是强烈的,接着出现某种躯体功能障碍(例如瘫痪、失明、失聪等),而躯体症状一出现,情绪反应便消失,并且患者往往不能回忆发生过的情绪,甚至连引起情绪反应的生活事件也不能回忆,这样的躯体功能障碍便叫做转换症状。其重点在于,先有情绪、后有躯体症状,且躯体症状一出现,情绪便消失。这与心理生理障碍根本不同。

2.与解剖生理学不符合,并且有明显直接相矛盾的躯体症状。例如,患者躺在床上,双下肢活动自如,神经系统检查无任何阳性体征,但患者不能站立和行走,有手套形或袜形感觉缺失,界线分明。

3.可以用暗示引起或消除症状 单纯或主要根据暗示性高而诊断歇斯底里是危险的。近几十年来,国内已经发生过多起因使用这种诊断方法造成严重医疗事故。

4.法国学者描述的对身体症状漠不关心或泰然处之的态度(la belle
indifference)。患者对表面上很严重的身体症状(如双下肢完全瘫痪)满不在乎,不主动求治,不主动提及,即使是患者力所能及的事也不主动配合医生治疗,甚至拒绝尝试。与此同时,患者对他的患者角色却相当重视,要求周围人对他在物质生活和精神生活上给予特殊的关怀与照顾,否则,患者会表现出不满和抱怨。如果医护人员或家属对患者关心体贴,患者的情感反应是生动的和合作的。这就是说,除了身体症状以外,患者并没有情感淡漠。

5.游离症状 游离(dissociation)现一般译作解离,因Separation也译作解离,而解离焦虑是儿童常见的一种情绪障碍,故改译成游离。游离是Janet
P首先提出的一个概念,带有Janet所特有的理论含义,似乎还无人给出过令人满意的描述性定义。这里只能满足于列举几种比较公认的精神症状或精神病理状态,如发作性身份障碍、附体体验、发作性意识改变状态(altered
state of
consciousness)、心因性遗忘症(指对精神创伤性事件或经历的遗忘,如结婚已年余的妻子忘记自己结过婚,视丈夫为陌生人,但近一年多发生的与婚姻无关的事都能回忆)。

6.一种特殊的情绪障碍 情绪暴发或短路反应带有明显发泄的性质;表面显得生动强烈但给人以肤浅、缺乏真情实感和作夸张印象的情绪;情感逻辑,即凭一时的情感评断别人的好坏,一下子把人家捧上天,一下子又把人家说得一文不值,同时以受骗上当者自居;缺乏稳定的心情,情绪几乎完全是反应性的等等。

7.反应的原始性 在精神打击下立即出现僵住不动、假死、机械地模仿别人的言语动作、盲目的躁动、非癫痫
性全身抽动或童样痴呆等。

8.反应目的性 行为具有满足愿望的性质或有明显的目的,但显然是异常的。例如,死了婴儿的母亲把枕头当孩子抱在怀里,喂奶把尿,像对待活着的孩子一样忙个不停;逃避现实困难处境的患者每次面临某种处境便犯病,使患者无法履行职责义务,而在其他场合下却不犯病。

9.自我戏剧化 用幻想代替现实,用想象激发情绪使自己感到满意,“进入角色”,“假戏真做”,沉溺于体验戏剧化的主观效应。

10.引人注意 经常把自己放在生活舞台的中心和聚光灯之下,极力引人注意,受重视时洋洋自得,不被注意时则十分不快或感到空虚无聊,也可能产生强烈的嫉妒和仇恨;喜欢凑热闹,赶时髦,出风头,追求刺激,热衷于激动人心的场面;为了引人注意,不惜说谎,捏造传奇式的经历;扮演“英雄”或小丑,可以不顾面子,甚至伤害自己的身体;行为的动机和设计几乎完全为了“剧场效果”。

在上述10个不同的定义中,有的只限于身体症状,有的只限于精神症状,有的只限于人格,但也有些兼指两者或三者。当然,这些定义涉及的病例事实上常常有重叠。然而,不论是哪一种定义,都与其他神经症的典型症状和临床相大不相同,格格不入。

许多精神病学家对歇斯底里与其他神经症之间的显著不同这一事实熟视无睹,主要有以下三方面的原因:

①随着19世纪神经病学的进步,各种器质性疾病陆续从神经症里区分了出去,神经症逐渐被公认为神经系统的功能性障碍。由于这一共同的阴性特征(无器质性病变作为基础)歇斯底里与其他神经症的差异被掩盖了。

②神经症的心因性学说首先且主要的是关于歇斯底里研究的产物。理论上设想的共同病因掩盖了临床事实和症状的巨大差异。

③自20世纪以来,Freud
S的精神分析学说在神经症领域里长期居于统治地位。正是Freud的无意识概念掩盖了歇斯底里与其他神经症在临床事实和症状上的巨大差异。

描述性定义要求我们把有争议的病因学说和病理机制学说暂时搁置一旁,而只考虑对患者行为的客观观察和患者对体验的叙述,也就是只考虑临床事实和症状。换言之,包括歇斯底里在内的神经症是症状学上或现象学上无法概括的一个杂类。摆在我们面前的有两条路。第一条路是坚持把歇斯底里看成一种神经症。第二条路是把歇斯底里从神经症里划分出去。这对歇斯底里和神经症(不包括歇斯底里的神经症)的理论和实践都大有好处。在理论上,两者描述性定义的发展将促进病因病理等许多问题的解决,在实践上,也将促进歇斯底里与诈病和多种神经科疾病的鉴别诊断、歇斯底里的特殊治疗的发展。

\protect\hypertarget{text00016.html}{}{}

