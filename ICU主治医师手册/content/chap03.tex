\chapter{超声在休克和循环功能监测及支持中的应用}

\section{前沿学术综述}

超声心动图是目前能够在床旁提供实时有关心脏结构和功能信息的唯一影像工具。多普勒心脏超声技术可以更加详细地评估患者的血流动力学改变,因而更有助于快速明确导致急性循环衰竭的机制与原因。由于可以在很短的时间内准确评估血流动力学状态,心脏超声对于休克或存在循环衰竭的重症患者,无论是早期识别与评估,还是整个诊疗过程中都有理由成为适合的理想的监测工具。另外,随着科学技术和电子技术的快速进步、经食管的多平面探头的出现,使心脏超声的图像质量大幅提高,使一些过去经胸心脏超声很难获得满意图像的患者也可以获得可靠的相关信息。目前许多研究表明,心脏超声在重症患者的应用,可以促使患者的治疗产生有益的改变
\protect\hyperlink{text00009.htmlux5cux23ch1-8}{\textsuperscript{{[}1{]}}}
。同时,值得关注的是,肺部超声、肾脏超声在重症监测的快速发展进一步丰富了超声在休克和循环功能监测及支持中的应用,因此超声作为有前途的重症监测与支持工具在重症医学科的应用中逐渐走向成熟与普及。

\subsubsection{心脏超声在重症医学科中应用的发展与特点}

早期的综合重症医学科,心脏超声检查大多由通过资质认证的心脏专科医生来进行,主要目的是快速准确获得图像,帮助诊断心血管疾病,如心包填塞、急性心肌梗死的并发症、自发的主动脉夹层和创伤性主动脉损伤等。而对于血流动力学的无创评估仅仅是应用二维技术联合多普勒模式来测量每搏输出量和每分心脏输出量。事实上,当时的重症医学科医生对心脏超声的潜力和作用缺乏全面的认识。直到20世纪80年代中期,一些重症医学科医生中的先行者开始拓展应用心脏超声对血流动力学的全面而详尽的评估。首先推荐用于感染性休克和急性呼吸窘迫综合征患者,应用心脏超声替代右心漂浮导管进行血流动力学评估,并且率先开始自己进行心脏超声检查,尤其是可以24小时随时床旁进行重复检查和评估,并且指导治疗。随后由于在循环衰竭诊断与评估应用的扩展、随着监测和测量经验的积累,尤其经食管超声(TEE)准确度的增加,重症患者床旁超声的应用价值逐步得到认识和肯定,有研究表明其对治疗支持的影响和预测病死率有重要作用。

但直到上世纪90年代,重症医学科医生对心脏超声的兴趣才真正开始明显增加,主要原因有:心脏漂浮导管研究出现大量阴性甚至负面结果;与传统有创血流动力学评估手段相比心脏超声无创、实用;大量相关研究文献发表和大量相关重症医学科医生心脏超声培训课程出现使得重症医学科医生的超声应用技能得以明显提高。在这一时期,一些官方组织开始推荐经食管超声作为急性循环衰竭的一线评估手段。

近年来,功能血流动力学评估概念的提出,再次间接推动了心脏超声在重症医学科循环衰竭患者中的应用。越来越多证据显示,超声检查参数可准确评估重症医学科机械通气的感染性休克患者的心功能和液体反应性,而这些参数丰富了重症医学科时刻存在的心功能和液体反应性评估指标,同时大大激发了重症医学科医生对心脏超声的兴趣
\protect\hyperlink{text00009.htmlux5cux23ch2-8}{\textsuperscript{{[}2{]}}}
。

\subsubsection{心脏超声在评估心脏前负荷及容量反应性方面的作用}

众所周知,在重症医学科管理血流动力学不稳定的患者时,最常见的临床行为就是实现以提高心输出量和组织灌注为目的的血管内容量和心脏前负荷的最佳化调节。而在此调节过程中,无论是让患者处于容量不足还是容量过负荷状态均会产生严重的后果,评估患者的容量状态极为重要。所以在有指征给患者输液时,进行容量反应性的评估尤为重要,而心脏超声给我们提供了更多更准确更便捷的选择。

心脏超声能够评估患者的容量状态,是传统有创血流动力学监测评估的有益补充,更有可能更加可信可靠。一般情况下,经胸心脏超声已经可以提供足够可用的信息。当经胸超声图像欠理想时,经食管超声检查可以提供理想图像,用于比经胸心脏超声更准确的评估心内流量、心肺相互作用、上腔静脉的扩张变异度等。

心脏超声对容量状态的评估可采用静态或动态指标,静态指标即单一的测量心脏内径大小和流量快慢;动态指标用来判断液体反应性,包括自主或机械通气时呼吸负荷的变化、被动抬腿试验和容量负荷试验等。其中,动态指标临床使用更多。心肺相互作用的指标如上腔静脉塌陷率、下腔静脉扩张指数、左室射血的呼吸变化率等,用于预测窦性心律、无自主呼吸机械通气患者的容量反应性;被动抬腿试验相当于内源性容量负荷试验,通过超声观察抬腿前后左室射血流速增加情况来预测容量反应性,无论患者自主呼吸或机械通气、任何心律情况下,均可应用。临床治疗中,可动态和静态指标联合应用进行评估。如严重低血容量时评估的超声征象:功能增强但容积很小的左室;自主呼吸时下腔静脉吸气塌陷非常小;机械通气患者呼气末下腔静脉呼吸变化非常小。

评估容量反应性时,必须考虑以下因素:①容量反应性的评估需要测量多个参数,综合分析;②左室或右室内径大小的变化对容量反应性的预测不可靠;③评估容量反应性时,必须考虑自主呼吸与正压通气对采用指标的不同影响,当患者存在心律失常或自主呼吸时,应用心肺相互作用的指标评估容量反应性并不准确,可选择被动抬腿试验;④非心脏超声获得的心肺相互作用评估容量反应性(如脉压呼吸变化率)的假阳性原因(尤其严重右心衰)易于通过心脏超声检查明确。

总之,心脏超声在评估心脏前负荷及容量反应性方面可用、有效且极具前景。

\subsubsection{心脏超声在评估心功能中的作用}

重症患者心功能的改变非常常见,如心功能衰竭或心肌抑制,此时心室收缩、舒张功能的定量分析对于病情监测、指导治疗和判断预后具有十分重要的临床意义。心脏超声通过二维心脏超声、M型心脏超声、利用几何模型的容量测定、辛普森法、组织多普勒技术、Tei指数和三维心脏超声等方法对心脏功能进行评估,无创且便捷。心功能测定包括左(右)心室收缩和舒张功能测定,其中,左心室功能检测在临床病情评估和治疗中最为重要
\protect\hyperlink{text00009.htmlux5cux23ch3-8}{\textsuperscript{{[}3{]}}}
。

射血分数是目前研究最多,且最为临床所接受的心脏功能指标,具有容易获得(甚至有经验的操作者目测的结果与实测结果相差很小,相关系数达0.91)、可重复性好以及能够较早评价全心收缩功能等优点(不同于环周纤维缩短率,在有节段异常时,也经常发生改变)。目前研究表明,射血分数是与预后最相关的心功能指标。射血分数的测量方法很多,其中Simpson最准确,被美国超声学会所推荐。但最大的缺陷在于对心内膜边缘的确认水平要求足够高,两腔像与四腔像要求垂直,而且操作略显繁杂费时。射血分数值作为一个最重要的评价心脏收缩功能指标,也具有明显的局限性,受前后负荷的影响非常明显。前负荷增加通过Frank-Starling机制增加射血分数值,而后负荷增加抑制射血分数值,如在没有血管活性药物支持、仅扩容治疗的感染性休克患者,前负荷稳定或增加,同时血压/外周阻力明显下降都会导致射血分数测量值不能代表心肌的真实收缩功能。另外一个重要的心功能指标是平均环周纤维缩短率,最大优点在于不依赖于前负荷改变,同时,经过心率纠正后的指标心率纠正的平均环周纤维缩短率,由于去除了心率的影响,似乎比射血分数能更好地反映心肌收缩功能。

有研究显示组织多普勒技术测定的心肌收缩速度可以代表全心室功能,尤其可反映二尖瓣环心肌收缩速度。另外,有研究表明,尽管存在对前后负荷的依赖,在肥厚性心肌病和舒张功能不全的患者,运用组织多普勒技术测定的心肌收缩速度指标可以在显性心肌肥厚和心脏收缩功能不全之前即发现渐进的心肌收缩功能受损,同时,这些指标对受心脏前后负荷的影响不大
\protect\hyperlink{text00009.htmlux5cux23ch4-8}{\textsuperscript{{[}4{]}}}
。

综上所述,近年来,在心脏超声多普勒技术领域,评估左心室收缩功能的进展主要集中在两个方向。首先是探索对负荷依赖程度低的指标,即接近心肌内在性能的指标,如左心室等容收缩压力增加速率,不依赖后负荷而对前负荷轻度依赖,同时,已有许多研究表明这些指标有助于预后判断;其次是研究心肌本身的指标,以往的许多指标大多依赖于血容量(腔室的大小)和血流(多普勒流速和压力的变化)进行测量,而随着超声多普勒技术的进步,尤其是组织多普勒的发展,最近的研究则侧重于应用无创技术测量心肌本身或其内在的机能。目前可测得的主要指标包括心肌收缩速度、左室质量、应变和应变率以及与应力的关系等。这些指标对患者预后影响的研究尚少,尤其缺乏大规模研究,仅发现充血性心衰患者心肌收缩速度<5厘米/秒可预测心脏不良事件的发生。

组织多普勒技术测定的Tei指数又称为心肌做功指数,心肌做功指数=(心室等容收缩时间+心室等容舒张时间)/心室射血时间。该指数于1995年由日本学者Tei提出,无创、敏感,能综合反映心室收缩及舒张功能,是可行的评价左室功能的指标,是对常规测定的血流多普勒参数的重要补充。目前尚无公认的正常值。

实时三维心脏超声全面、快速准确地测定左室功能,一直是心脏超声工作者的梦想。有人研究应用这一新的技术,测定正常人和心脏病患者的左室射血分数,并与常规双平面二维改良Simpson's法测定左室射血分数进行对照,证明实时一次心动周期三维超声即能准确、快速测定左室射血分数。实时三维心脏超声可以产生实时三维的心脏图像及左室容积时间曲线,克服了二维超声的限制,在测量心室容积时不需要几何形状的假定,不受心脏几何形态的影响,因而测量的结果更为准确,能全面实时地观察和测量动态心室的整体及局部容积大小、运动及功能状态,从而提高心功能评估的可靠性,是一种无创的新方法。

\subsubsection{心脏超声对外周血管阻力的评估}

心脏超声多普勒技术可以直接测量外周血管阻力,但不易方便和简单使用,因此在临床工作当中,经常根据临床和心脏超声的检查结果进行排除诊断,如在心脏负荷足够同时左右心脏收缩功能均满意的情况下,仍然存在低血压则提示外周血管阻力低。

\subsubsection{心脏超声在特殊情况下的应用}

严重感染和感染性休克是常见病、多发病,与急性心肌梗死发病率相当,甚至高于许多肿瘤的发病率,是住院患者最常见的死亡原因之一,且病死率随着年龄增加而增加,甚至大于急性心肌梗死,达到30%~60%。其中,早期出现心功能异常的患者若表现为低心排,死亡率>80%。另有研究提示,合并出现心血管损害的全身性感染患者,死亡率由20%升至70%~90%。

临床上常见严重感染和感染性休克时,心输出量并不降低或反而增加,但合并心肌功能不全。这种心功能不全多出现于感染性休克早期,往往难以早期发现及处理,造成的危害极大。随着心脏超声在评估左室心脏功能应用的进展,目前已被应用于感染性休克相关的心肌抑制的早期发现与指导支持治疗
\protect\hyperlink{text00009.htmlux5cux23ch5-8}{\textsuperscript{{[}5{]}}}
。目前常用指标有射血分数、环周纤维缩短率、心肌收缩速度等,而应用应变和应变率以及与应力的关系等对于早期发现与感染相关的心肌抑制及指导正性肌力药物应用具有更好的前景。

无论是围手术期还是严重创伤患者,缺血性心脏病非常常见。局部心肌缺血导致局部心肌运动异常。临床实际中,局部心肌缺血的评估最常用到的方法是对二维超声显像室壁运动和室壁增厚率进行目测。与心肌节段的室壁增厚率相比较,二维超声应变成像对心肌缺血的变化更加敏感。急性心肌梗死后可出现多种舒张期充盈异常即左心室舒张功能异常,表现为二尖瓣血流频谱E峰峰值速度减低,A峰峰值速度增高,E/A比值<1,E峰减速时间延长,等容舒张时间延长,肺静脉血流频谱S/D峰值比值增加等。另外,随着彩色多普勒心脏超声在临床的广泛运用,急性心肌梗死后左室舒张功能得到更全面深刻的认识,对临床治疗方案的制定和调整也起到重要作用。心肌应变测量的是心肌各节段的变形,在定量评价心肌各节段的收缩和舒张功能时,心肌应变与心肌的收缩和舒张功能密切相关,因此能准确评估心肌收缩和舒张功能。

急性肺血栓栓塞是临床上一种危重心肺疾病,心脏超声对其病变程度、治疗效果及预后评估有重要作用,已经普遍应用于临床。超声检查急性肺血栓栓塞一般包括心脏超声检查及下肢深静脉检查。尤其对于确诊的急性肺血栓栓塞患者,超声探测到中度、重度右室功能障碍者,其近期及长期病死率均明显升高,而不伴有右室负荷过重的患者,近期预后良好。因此超声能够根据右室功能状态进行危险度分层及预后判断。心脏超声可以动态、无创、重复估测肺动脉压力,因此可以判断治疗效果,可以作为随访追踪的一种快速、简便的检查手段。

\subsubsection{肺部超声在循环监测与支持中的作用}

最近几年来,随着肺部超声的进步与推广,成为能够发现与评估不同肺部与胸腔病变的有力技术。肺部超声常见征象与特点包括:①正常通气,胸膜线下平行排列的A线;②肺间质肺泡综合征,彗星尾征,根据B线的间隔不同分为B7线(B线间隔大约7mm,主要是肺小叶间隔增厚)和B3线(B线间隔3mm);③肺实变征,包括组织样征、碎片征和支气管气象;④胸腔积液,静态征象为四边形征,动态征象为水母征和正弦波征;⑤气胸,肺点消失。

以上是常见肺部病变的超声表现。对于肺水肿患者,肺水含量的评估非常重要,肺部超声获得B线可以早期发现在血气分析改变之前的肺水肿,而且超声具有简单、无创、无放射性和实时性等优点。超声监测导向诊断的难点在于急性心源性肺水肿与ARDS肺水肿的鉴别,最新有研究表明,循环支持过程中,肺部超声的A-优势型表现提示肺动脉嵌顿压<13mmHg的可能性大;而在B-优势型时,提示肺动脉嵌顿压>18mmHg可能性大
\protect\hyperlink{text00009.htmlux5cux23ch6-8}{\textsuperscript{{[}6{]}}}
。

\subsubsection{重症肾脏超声在循环监测及休克支持中的作用}

肾脏是休克时最容易受损或最早受损的器官之一,重症患者病变过程中易并发急性肾损伤。术后患者发生率1%,重症患者达到35%,尤其感染性休克患者发生率在50%以上。因此预测、发现和评估急性肾损伤非常重要。重症肾脏超声能够床旁及时无创监测肾脏改变,能够同时关注和监测肾脏大循环与微循环情况,为休克循环监测和支持提供了新的重要思路。

总之,重症超声包括超声心动图、肺部超声和重症肾脏超声在血流动力学评估,尤其对于心脏功能、容量反应性等血流动力学评估的作用越来越重大;在重症医学科常见的重症疾病如休克的监测与支持等诸多方面都开始发挥举足轻重的作用,已经被众多重症医学科医生所接受和掌握。因此,全世界范围内的重症医学科医师的重症超声培训和认证正在如火如荼地进行。

\section{临床问题}

\subsection{超声评价血流动力学的作用}

\subsubsection{为什么超声是评价重症患者血流动力学的重要方法?}

在重症患者中,血流动力学不稳定(急性或慢性)是很常见的问题。长期低血压可能导致器官缺血、功能紊乱等不良后果。相反,快速的诊断和早期干预可以避免血流动力学的进一步恶化。然而,仅仅依靠临床常规检查尚不足以做出正确的诊疗决策。对于不常见的临床问题,临床疑诊是建立鉴别诊断和灵活应用诊疗技术来做出诊疗决策的关键。超声心动图就是能够在不同疾病的快速诊断中发挥重要作用的技术之一。因此,对于患者血流动力学不稳定的原因和监测,超声心动图能够发挥强大作用,可以用于评估前负荷、后负荷和心肌收缩力。各类研究表明,超声心动图的应用使至少1/4的重症患者的治疗有所改变。

应该强调的是,应用超声心动图来评估重症患者,能快速而可靠地排查像肺栓塞和心包填塞等能引起患者血流动力学不稳定的主要病因,而这些操作可由经过简易超声心动图检查训练的重症医学科医生或者急诊医生完成,并且是血流动力学不稳定重症患者评估的关键一步。

在排除了一些主要病因之后,需评价患者的容量状态和心功能。最重要也是最常使用的评价左心室整体或者局部室壁运动的方法,是多切面的定性评估。这种方法快速而有效,并且与核素扫描结果具有很好的一致性。超声心动图的检查结果不仅能评估局部室壁运动,还能通过估计射血分数来评估左心室整体功能。心室功能的定量评估能提供可测量性更好的、误差更少的评价方法。但需要警惕的是,所有有效的评估方法都既有长处,又有各自局限性。

\subsubsection{如何看待经胸壁超声心动图、经食管超声心动图和手持设备在重症医学科的作用?}

在重症医学科中,经食管超声心动图经常被认为比经胸壁超声心动图更有优势,因为后者常常由于下列原因得到的图像质量欠佳:比如术后患者由于机械通气(呼气末正压>15cm
H\textsubscript{2}
O)无法调整体位、缺乏合作耐心、胸壁水肿以及由于伤口敷料、胸腔引流管、胸腹壁开放而使视野阻断。经胸壁超声心动图在被检患者中的成功率为50%~80%,而经食管超声心动图的成功率高达90%。但近年来,更多研究表明经胸壁超声心动图有助于诊疗的超声切面获得率在86%以上。另外,经胸壁超声心动图的常规实施过程面临很多问题。与经胸壁超声心动图相比,经食管超声心动图耗时更长,对专业知识要求更高,而且经食管置入探针有误入气道而阻塞气道的风险。另外,虽然经食管超声心动图会产生像食管穿孔这样的严重并发症,但其可能性较小,大约只有0.01%。

手持式可移动设备轻巧、简单而且方便,能提供定性评估。手持式设备在经超声引导下胸穿以及中心静脉置管等操作中作用明显。新一代的电池供电的检查设备也已出现,这些设备在血流动力学不稳定的重症医学科患者中的地位和应用在进一步加强。

不管检查形式怎样,检查过程本身必须是完整的,并且跟从业人员在训练中要求的一样全面。如果初期检查因为不同原因有所限制,或者结果存在疑问,要求更加有经验的从业人员及早进行更全面的检查。全面检查就是尽量避免罕见疾病的漏诊。经过反复练习之后,完整的检查过程应该在数分钟内完成。合理的检查程序应该是在体格检查的基础上定位于可疑病变部位或结构。一旦解决了直接问题,接下来应该做更加全面的检查,对于可疑病变部位能够有更加充分的检查时间。目前的指南上有经食管超声心动图和经胸壁超声心动图检查的标准图像,以确保所有结构都是从多角度去查看的,而单个结构能被完整而准确的评估并且根据需要被记录下来。标准切面能保证任何结构不被遗漏,还能为从业人员的相互交流提供有效的媒介。

\subsection{超声在容量及容量反应性监测中的作用}

\subsubsection{什么是容量状态与容量反应性?超声检查在其中有什么作用?}

血管内容量和心脏前负荷的最佳化调节是提高心输出量和改善组织灌注的重要环节,通常是血流动力学支持最早期的临床行为。在此调节过程中,评估患者的容量状态极为重要。因为无论是让患者处于容量不足还是容量过负荷状态均会导致严重的后果。所以在有指征给患者输液时,进行容量反应性的评估尤为重要。

目前对容量治疗有反应定义为给予液体治疗后,心输出量指数或每搏输出量指数较前增加≥15%。心脏对容量治疗有反应的生理机制是基于Frank-Starling机制:当心功能处于心功能曲线上升支时,增加前负荷,则可以显著增加心输出量,改善血流动力学,提高氧输送,从而改善组织灌注;而心功能处于平台期时,提高前负荷的潜能有限,扩容则难以进一步增加心输出量,反而可能带来肺水肿等容量过多的危害。

提出容量反应性近20年来,大量研究力图寻找简单可靠并且敏感快捷的指标或方法来预测,进而指导液体治疗,如何选择和应用这些指标也一直是研究的热点。目前预测容量治疗反应的指标或方法,主要包括传统的静态前负荷参数(前负荷压力指标及前负荷容积指标)的监测、容量负荷试验,以及近来研究较多的经心肺相互作用的动态前负荷参数(收缩压变异度、脉搏压变异度、每搏输出量变异度等)和被动腿抬高试验等。

心脏超声能够评估患者的容量状态和容量反应性,是传统有创血流动力学监测评估的有益补充,更有可能比之更加可信可靠。当经胸超声图像欠理想时,经食管超声可以提供理想图像,用于比经胸心脏超声更准确地评估心内流量、心肺相互作用、上腔静脉的变异度等。当然,一般情况下,经胸心脏超声已经可以提供足够可用的信息。心脏超声对容量状态和容量反应性的评估一般包括静态指标和动态指标,静态指标即单一的测量心脏内径、面积及容积大小和流量的快慢;动态指标,广义包括流量和内径大小对于动态手段的变化(自主或机械通气时呼吸负荷的变化、被动腿抬高试验、容量负荷试验等),狭义即指心肺相互关系引导的动态指标。

\subsubsection{根据临床经常面临的容量和容量反应性问题,超声临床判断评估的流程与思路及评估的指标与方法是什么?}

(1)严重容量不足或输液有明显限制时液体反应性的评估 当患者没有进行容量状态和容量反应性评估的指征时,首先可以快速判断是否存在严重容量不足或输液有明显限制及容量过负荷,此时应用的大多为静态指标。

严重低血容量时,预测容量反应性阳性结果的可能非常大。超声评估指标包括:功能增强但容积很小的左室,左心室舒张末期面积<5.5cm\textsuperscript{2}
/m\textsuperscript{2}
体表面积;在自主呼吸时下腔静脉内径小且吸气塌陷非常明显;在机械通气患者呼气末下腔静脉内径非常小,常见<9mm,并且容易随呼吸变化。

容量过负荷或输液限制明显,预测容量反应性阴性可能很大时的超声评估指标包括:在无心包填塞时上下腔静脉有明显充盈表现(扩张或固定);严重右室功能不全及过负荷(右室大于左室的超声证据);心脏超声估测有很高的左室充盈压,如很高的E/E'值。

类似的这些静态指标在评估容量反应性时,有多种影响因素。所以单纯根据一个静态指标评估容量反应性可靠性很差,但对于评估容量明显缺乏和明显过负荷时,却较为可靠,即尽管不敏感,但特异性很强。

(2)既不是严重容量不足、也不是容量过负荷时容量反应性的评估 当患者既不是严重容量不足、也不是容量过负荷,即容量反应性判断比较困难时,此时包括完全机械通气和自主呼吸两种不同的情况,选择的指标和方法如下。

1)完全机械通气容量反应性的评估:在完全机械通气的无心律失常患者,选择心肺相互作用相关的动态指标可以预测容量反应性,如主动脉流速和左室每搏射血的呼吸变化率以及上腔静脉塌陷率、下腔静脉扩张指数等,并且研究证明同非超声获得的动态指标一样,上述指标均明确优于静态指标。

近年来,随着对心肺相互作用认识的进步,在机械通气的患者,左室每搏输出量的呼吸变化率可以作为容量反应性的指标,但由于床旁左室每搏输出量的测量依然复杂而相对困难,所以一些左室每搏输出量呼吸变化率的替代指标被应用和研究,包括动脉监测的脉压呼吸变化率和脉搏轮廓推导的每搏输出量变化率。当然随着心脏超声在重症医学科的更广泛应用,尤其对于血流动力学不稳定患者评估的应用,一些超声检查可以获得的左室每搏输出量呼吸变化率的替代指标被认识和研究应用。2000年前后,Feissel等应用经食管超声测量主动脉瓣环的主动脉血流速的呼吸变化率判断容量反应性,2005年Monnet和Teboul等应用食管多普勒探头直接测量降主动脉峰流速的呼吸变化率来预测容量反应性,均取得理想结果;在儿童相关的研究中,进一步证明经胸超声获得的主动脉峰流速呼吸变化率在预测液体反应性、评估心脏前负荷储备时优于脉搏压变异度和收缩压变异度。另外,在动物研究(阶梯失血兔子模型)中,无论应用经食管超声测量主动脉流速还是经胸超声测量的主动脉血流速度积分呼吸变化率,均可高度准确预测容量反应性。

须说明的是,主动脉流速的测量无论经食管还是经胸,都存在一定的技术问题。而外周的动脉血管,包括桡动脉、肱动脉和股动脉等,其超声血流图像易于获得,因此,近年来研究显示肱动脉峰值血流速的呼吸变化率可预测患者的容量反应性,其敏感度和特异度都达到了90%以上,不亚于脉搏压变异度等动态指标,尤其优于一些静态指标。当然优点还在于完全无创,同时简单易学,甚至于需要培训的时间很短且不需要经验的积累。

对于非外周动脉流速的测量有限性在于需要减低操作者依赖性和进行可重复性可靠性研究,而对于外周动脉,仅仅需要关注局部肌肉收缩对测量的影响。另外尤其要注意这些指标只适用于没有自主呼吸及心律失常的机械通气患者。

使用具有心内膜自动描记功能的超声诊断仪时,可以用左室每搏射血面积呼吸变化率来预测液体反应性。

尽管大规模的荟萃综述分析认为脉搏压变异度是最理想的判断容量反应性的动态指标,但研究对比的对象是收缩压变异度和每搏输出量变异度。在应用超声进行评估时,由于主动脉流速甚至外周动脉的流速变化早于每搏输出量,因此,未来的研究需进一步明确其优越性。

以往的研究多以机械通气的休克患者为研究对象,最近一个关于自主呼吸志愿者的研究证实,在一些较单纯的情况下,如仅仅低血容量时,在自主呼吸状态下主动脉流速的呼吸变化率也可以预测液体反应性,不过此研究需要进一步验证。

另外,还可以通过判断腔静脉的变异度判断容量反应性,如下腔静脉呼吸扩张率和上腔静脉呼吸塌陷率。有研究表明,感染性休克患者下腔静脉扩张率为18%时,预测液体反应性的敏感性和特异度均在90%以上,而上腔静脉呼吸塌陷率的预测值为36%,预测容量反应性的敏感性和特异度也均在90%以上。但需要关注的是,影响腔静脉变异度的因素除了容量状态外还有右心功能和静脉顺应性。下腔静脉呼吸扩张率提出较早,但直到近年,随着对正压通气对下腔静脉影响认识的进步才被广泛接受和应用;而上腔静脉呼吸塌陷率的认识得益于经食管超声在重症患者中的广泛应用,尤其用于对血流动力学不稳定患者的评估
\protect\hyperlink{text00009.htmlux5cux23ch7-8}{\textsuperscript{{[}7{]}}}
。最近,针对失血性休克、全身性感染、蛛网膜下腔出血的患者,尤其慢性肾衰接受肾脏替代治疗患者的研究,进一步显示出腔静脉变异度的临床意义,但依然没有统一的预测值,仍需扩大研究规模。

2)自主呼吸或存在心律失常时容量反应性评估:对于存在自主呼吸或心律失常患者容量反应性的评估,可选择应用被动腿抬高试验相关的超声指标,相当于内源性的容量负荷试验,被动腿抬高试验产生300~450ml血浆快速输入。有研究表明,可应用超声观察每搏输出量的替代指标如被动腿抬高试验前后左室射血流速和流速积分变化来预测容量反应性,并且已经证明其敏感性和特异度均优于收缩压力和心率等;而在具有心内膜自动描记功能的超声诊断仪时,可以用左室每搏射血面积在被动腿抬高试验前后变化情况来预测液体反应性。

除应用左室射血流速和流速积分变化来预测容量反应性,最新有研究发现对于全身性感染和重症胰腺炎患者,在被动腿抬高试验前后应用外周动脉如股动脉峰值流速的变化与每搏输出量、脉压变化都可以用来预测液体反应性,前后变化分别为8%、10%和9%,同时研究还发现用心率来代表被动腿抬高试验前后自主神经功能时,前后没有变化,使得临床可操作性明显增强,当然除了选择股动脉还可以考虑其他外周动脉,如桡动脉和肱动脉等。

最近的一项包括9个相关研究的被动腿抬高试验荟萃分析认为,被动腿抬高试验相关的心指数和每搏输出量变化优于脉搏压的变化来预测液体反应性,可喜的是,其中6个研究应用了超声技术,入选患者数居多,所以随着未来有关主动脉流速和外周动脉流速的研究的增加,或许会有不同结论产生。

当然,在完全机械通气时和任何心律情况下,无论此时能不能合理应用动态指标,也可选择应用被动腿抬高试验相关的超声指标。

3)选择容量负荷试验进行容量反应性评估:当以上的方法依然不能合理预测容量反应性时,最终在谨慎考虑输液限制情况下,还可以选择容量负荷试验。此时,可选择超声测量每搏输出量、心输出量和左心室舒张末期面积变化以及多普勒测量左室充盈压变化判断容量负荷试验。最近的研究表明,容量负荷试验前后应用外周动脉流速变化如股动脉流速变化同样可以预测容量反应性,应该说除需要承担液体过负荷风险外,在评估容量反应性上完全与被动腿抬高试验接近,甚至于更可靠些
\protect\hyperlink{text00009.htmlux5cux23ch8-8}{\textsuperscript{{[}8{]}}}
。

\subsubsection{超声容量反应性评估时的注意事项是什么?}

在评估容量反应性时,一定要认真考虑以下因素:①液体反应性的评估需要测量多个参数,因为没有任何一个指标是绝对和排他的,临床上应该结合具体临床情况联合应用,最终有助于准确评估容量反应性;②心脏超声获得的心肺相互作用评估容量反应性的动态指标不但有助于评估容量反应性,同时心脏超声易于发现非超声获得的动态指标的假阳性(尤其严重右心衰),但依然需要更多的研究来证明临床价值。

总之,心脏超声在评估前负荷及容量反应性方面可用、有效,且极具前景。在应用心脏超声时,无论评估的流程还是指标的选择均有一定科学内涵,应该在应用时进一步设计合理的临床研究来证实临床有效性,期待能够对死亡率和致残率以及并发症发生率产生深远的影响。

\subsection{左心室功能的超声心动图评估}

\subsubsection{左心室功能评估的要点是什么?}

心室收缩与舒张功能及其随时间变化的评价在重症患者中作用很大。由于超声心动图以二维图像来展示三维结构,所以在诊断或者治疗之前,每个结构至少要得到相互垂直的两个切面的图像。新出现的或者进一步恶化的室壁运动异常可能提示急性心肌缺血或者缺血所致损伤,而像重症感染等多种重症疾病所导致室壁运动异常并非心室局部的功能障碍,而是心室的整体功能异常,因此全心室收缩功能评估十分重要。

心室收缩功能同时依赖于心脏的前负荷和后负荷,所以必须在不同负荷状态下评估收缩功能才能确保得到真实结果。另外还要注意连续评估的重要性,不能仅仅依赖某一次评估的结果得出结论。压力容积关系是不依赖于容量状态的左心室心肌收缩力的评估方法。超声心动图中用来评估整个左心室收缩功能的定性和半定量测量指标有射血分数、缩短分数、面积变化分数、左心室功能评估的Simpson法、二尖瓣环运动、用二尖瓣反流束计算等容收缩压力增加速率、使用标准17-节段模型和应变率来评估局部室壁运动异常。最常用的方法是射血分数。

\subsubsection{左心室收缩功能定性评估的首要问题是什么?}

评价左心室的收缩功能时,首先要明确以下问题:心室充盈如何?心肌有足够的收缩力吗?在冠脉分布的范围内心肌收缩一致吗?

\subsubsection{如何运用左心室标准的17节段分法进行视觉评估左室功能?}

左心室功能评估的形式多种多样,如心脏MRI、超声心动图、核素扫描、血管造影等。为了能统一术语,美国心脏学会达成共识,将左心室分成17个不同的节段。沿心脏长轴左心室分为基底段、中段和心尖段,基底段和中段又各自进一步分为6个节段,尖段分为4个节段,再加上第17节段的心尖帽部。相应的冠脉分布为:左前降支提供心脏的前壁和前间壁前2/3的血供,左回旋支提供左心室侧壁的血供,右冠状动脉提供室间隔后1/3和左心室下壁的血供。室壁运动评分和指数可以用来进行半定量评估。左心室收缩力依赖心脏从基底部到心尖部的运动、室壁的厚度和左心室螺旋挤压和旋转运动。心室壁的切面厚度以及左心室局部心内膜运动幅度对心室壁运动的评估十分重要。室壁运动评分描述如下:

正常(>30%心内膜运动幅度,>50%室壁厚度);

轻度运动功能减退(10%~30%心内膜运动幅度,30%~50%室壁厚度);

严重运动功能减退(<20%心内膜运动幅度,<30%室壁厚度);

运动不能(心内膜运动幅度为零,<10%室壁厚度);

运动障碍(收缩期反常运动)。

室壁运动评分指数是指局部的室壁运动分数,是一种主观评估方法,分数之间没有真正意义的线性关系。缺乏血流灌注的心肌将表现为异常的室壁运动。只有多个切面的图像才能真正反映左心室受损情况和相应冠脉分布情况。仅仅是心内膜运动幅度的改变可能是心肌栓塞造成的,而室壁厚度改变是缺血的确切指征。经过多次室壁厚度的测量可以得出以下结论:沿长轴平面很难获得连续的室壁厚度数据;多角度多平面测量可以减小误差;确定边界、方位和角度值。

\subsubsection{什么是射血分数及测量方法?}

每搏输出量等于舒张末容积与收缩末容积之差。射血分数等于每搏输出量除以舒张末容积。可以在经胸壁超声心动图的左室长轴和短轴不同平面测量,但美国超声心动图学会建议使用修改后的Simpson法,计算两个平面的射血分数然后取平均值。该方法可通过经食管超声心动图的经中段食管切面、四腔切面、二腔切面进行计算。局限性在于测量时要求心内膜边界能清晰显示,而二尖瓣环的钙化通常会干扰心内膜边界的探查;在四腔切面中,因为超声束与心室侧壁平行,所以会出现侧壁信号丢失的情况;左心室内小梁形成也会干扰心内膜边界的探查。在这种情况下,使用造影剂能提高边界成像的清晰度。左心室尖部常因为透视原理而缩小。

\subsubsection{怎样进行左心室收缩功能的超声心动图定量评估?}

(1)心输出量的计算

心输出量=心率×每搏输出量

在重症医学科中,肺动脉导管可以用来测量心输出量。但目前的证据显示,肺动脉导管的使用并没有明显优势,所以超声心动图对心输出量的测定具有重要作用。左右心室的心输出量都可以通过超声心动图来测量。左心室心输出量测量的可重复性和准确性更高:

左心室流出道面积=左心室流出道半径\textsuperscript{2} ×3.14。

心率可以通过心电图测量,或者从一个速度-时间积分到另一个速度-时间积分进行推算。每搏输出量等于左心室流出道面积乘以左心室主动脉瓣收缩期射血速度-时间积分。当血液从左心室射进圆柱体形的主动脉,每搏输出量就可以通过圆柱体血液的高度来计算,而这个高度就是速度-时间积分。圆柱体形的底是左心室流出道,而流出道面积能够很容易进行计算。圆柱体的高,也就是速度-时间积分,是通过经胸壁超声心动图时的心尖五腔切面、经食管超声心动图(TEE)时经胃主动脉瓣切面或者经胃主动脉瓣长轴切面运用脉冲多普勒测量左心室流出道的血流得出。该参数的准确测定基于左心室流出道面积在收缩期恒定不变的基础之上。左心室流出道半径的测量误差将使面积计算的误差放大。为了使误差最小化,图像的灰度要减小,而左心室流出道要尽量大;另一个假设是通过左心室流出道的血流是层流。这个假设通过脉冲多普勒上的窄流速带和平滑的光谱信号来证实。将样本体积的液体流通过两个互相垂直的切面来解释液体流的中心流速和边缘流速相等,以此证实平均流速分布图的存在。需要强调的是,多普勒射束应该与血流平行或者<20°。多普勒信号记录的是与血流平行的拦截角,所以能准确测量血流速度。左心室流出道直径和脉冲多普勒应该在同一解剖位置进行测量以保持脉冲多普勒的空间与即时关系。选择某一个靠近动脉瓣的位置当作常规测量点可以减小误差。因为在不同心率下血流动力学有所不同,因此这些测量应该在同一时间点进行,当在不同时间点评估心输出量时,所有的测量都要重复进行。

(2)不同部位每搏输出量的测量 使用经食管超声心动图时,一般选择左心室流出道作为最主要的测量点,然后就是肺动脉和右心室流出道。经食管超声心动图测量每搏输出量时,可以选择在主动脉瓣瓣叶尖端或者升主动脉。升主动脉直径是从胸骨旁长轴切面测量的,从胸骨上切迹或者心尖部的经胸壁超声心动图切面测出。二尖瓣口每搏输出量也可以通过脉冲多普勒在二尖瓣瓣叶尖端测得。因为二尖瓣的复杂几何特征和大量的假设,一般不选择该处作为心输出量的常规测量点。在心脏的右侧,可以选择三尖瓣或者肺动脉来测量每搏输出量。右心室心输出量也可以测量得出。然而,大的肺动脉直径不是固定的,而是依赖于切面的不同而不同;另外,并非时时都能取到与右心室射出血流平行的多普勒图。

\subsubsection{左心室收缩功能的超声心动图半定量测量方法如何采用?}

(1)测量缩短分数 缩短分数是一种评价左心室整体收缩功能的一维测量方法。经左心室乳头肌短轴的M型超声能测量出该参数的值。M型超声的定格分析用来计算缩短分数。缩短分数=(左心室舒张期内径-左心室收缩期内径)/左心室舒张期内径×100(正常值>25%)。正常值在25%~45%之间。

缩短分数的测量是一种基本的粗糙的左心室整体收缩功能的评估方法,优点是快捷而且可重复性高,M型超声检查可以节约很多时间,而且心内膜边界显示非常清晰。在测量过程中需注意,如果局部心室壁存在异常运动,容易产生误差;一维平面的斜切可能导致长度测量的误差。因此,在这个半定量测量中加入另外维度的测量可以增加整体功能评估的准确性。

(2)测量面积变化分数 面积变化分数是测量左心室收缩功能的二维参数。测量的准确性依赖于获得足够清晰的心内膜边界,边界显示不清晰时进行描记是十分困难和耗时的。面积变化分数可以定量评估射血分数。面积变化分数=(左心室舒张末面积-左心室收缩末面积)/左心室舒张末面积×100%。正常值>50%~75%。

面积变化分数高度依赖后负荷,也一定程度依赖前负荷。其中,经胃乳头肌短轴切面计算的面积变化分数与放射性核素血管造影术测量有很好的相关性。

(3)等容收缩压力增加速率的测量 评价左心室功能指标在射血期很容易得到,但这些指标的负荷依赖性明显影响心室功能的客观和准确评估。等容收缩压力增加速率对心肌收缩能力的变化较为敏感,受前后负荷变化影响较小,对左室心肌收缩力的评估较为准确,可用来反映心肌收缩力的变化。测量方法如下:连续波超声多普勒测定二尖瓣反流的速度,测量从1m/秒增加到3m/秒所需时间。根据简化的伯努利方程(压力=4×速度\textsuperscript{2}
),等容收缩压力增加速率(dP/dt)可以表示为:dP/dt=32/Δt;即运用简化的伯努利方程,速度为3m/秒时,压力为4×3\textsuperscript{2}
=36mmHg;速度为1m/秒时,压力为4×1\textsuperscript{2}
=4mmHg,压力差为32mmHg。用压力差除以速度从1m/秒增加到3m/秒所需的时间Δt,等容收缩压力增加速率即可计算出来。正常值>1200mmHg/秒,小于1000mmHg/秒则为异常。左心室功能良好的状态下,该时间可以大大缩短。值得注意的是,测量该指数时患者必须存在二尖瓣反流。

(4)运用组织多普勒成像评估心室功能 组织多普勒成像是一种量化测量左心室整体和局部功能的手段。组织多普勒显示的二尖瓣环下行速度可以评估左心室的收缩功能。心肌组织速率一般在二尖瓣环的室间隔、侧壁、下壁、前壁、后壁和前间隔部位测量。从上述部位得到的二尖瓣环下行平均峰速度可以衍生出以下计算方程:左心室射血分数=8.2×二尖瓣环平均峰速+3%。

该方程可以评估心内膜边界显示欠佳患者的整体左心室功能,缺点在于不能鉴别真正的心肌运动与心肌被动牵拉运动或者心室的整体位移运动。这些参数能从节段性应变成像模式中获得。

(5)比较少用的左心室收缩功能半定量测量工具

1)压力-容积环 压力0容积环的Y轴代表压力,X轴代表容量,压力-容积环的斜率反映心肌收缩能力,不受心脏前后负荷的影响。左室收缩功能增强压力-容积环向左上移动,反之,收缩功能下降时移向右下。将心室不同前负荷所对应的不同环的收缩末压点相连,即可得到反映收缩末期压力-容积环的变化关系,也被称作弹量。该方法测定需要足够的时间,而且前负荷的改变易于影响患者病情的稳定,因此,不具有实用性,尤其不适用于重症患者。

2)室壁应力和左心室质量 室壁应力是指施加在单位心肌面积上的力,取决于心腔容积、压力和室壁厚度。室壁应力包括圆周、子午或径向三个方面。通常计算收缩末期的圆周及子午室壁应力。将心肌体积乘以特异的心肌密度即可计算出左室心肌质量。超声心动图可以通过评估左室流出道的收缩速度加速度以及心肌收缩的应变率得到收缩末弹量。心肌做功指数(Tei)是另一种心肌收缩功能的评估方法,通过等容收缩期与等容舒张期之和除以射血时间得到,然而心肌做功指数的临床实用性仍有争议。

(6)左心室收缩功能半定量测量新技术

1)运用组织多普勒、应变和应变率评估心功能 多普勒组织成像和斑点追踪成像是新近发明的测量局部心肌功能的重要方法。组织速度信号是一种低速信号,它通过除去室壁过滤,并使用低增益放大,使得心肌组织速率测定成为可能。放置在心肌特定部位获得的脉冲多普勒或定向的M超声都可以用来展示心肌组织速率。当室壁运动异常与标准评估相混淆时,可用组织多普勒来鉴别。多普勒组织成像的常见缺陷包括:只能测量与超声束平行的运动成分;不能鉴别心室平行的位移运动;不能鉴别被邻近组织牵拉的运动与正常收缩运动。应变和应变率可用来测量在超声扫描线上出现的变形。传感器定位十分敏感,比多普勒的角度依赖性更敏感。心肌峰速度、应变率以及应变能识别静息状态以及应激状态下的局部心肌功能异常。斑点追踪成像可避免角度依赖性,能得到更准确的组织速度、应变率和应变力,用于测量两个维度的变形。在静息、应激(应力)、局部缺血等状态下的局部功能是运用多普勒组织成像或者斑点追踪成像进行应变率和应变评估的指征,将其与三维斑点追踪成像技术相结合是评估左心室功能的有力工具。

2)有利于分辨心内膜边界评估心室功能的新技术 心内膜边界的清晰度在左心室功能评估中十分重要。处于不同状态时,如肥胖或者肺气肿的患者,心内膜边界不太清晰。超声心动图造影技术在这些患者中有重要作用。彩色室壁运动技术通过声学定量原理能够将组织和血液区分开来,自动勾勒出心内膜边界,能够动态定量分析左室功能。在有室壁瘤或者其他心室不对称等异常情况下,该方法的有效性需要进行校正。这种情况下,三维超声能够真实反映左心室功能。

\subsection{左心舒张功能评估}

\subsubsection{如何应用跨二尖瓣左心室充盈评估左心舒张功能?}

左心室的舒张功能与收缩功能同等重要,舒张功能正常可防止肺静脉淤血和心源性肺水肿。超声心动图检查可通过测定跨二尖瓣左心室充盈、肺静脉血流模式和二尖瓣环侧壁心肌速度来评估左心室舒张功能。

将脉冲多普勒取样窗放置在二尖瓣瓣叶尖端可以获得舒张早期最大流速E和心房收缩期最大流速A。正常左心室E峰一般大于A峰。左心室肥厚或老年患者,E/A比值<1,反映舒张功能受损。E峰加速度与左心房压力除以τ的比值成正比,其中τ是等容期左心室压力下降的指数时间常数。为了保证每搏输出量,在有进行性舒张功能障碍的患者中存在进行性左心房压力增高的代偿,以将受损的舒张形态逆转到假性正常化。当左心室功能严重受损,在很短的充盈时间内出现左房压的极度上升,表现为经典的减速时间减少和高E/A比值。这些参数都是随着前负荷的变化而改变,单凭这种评估方法不能鉴别舒张功能不全的所有形式,还可能造成一些病例的漏诊。一些特定方法像Valsalva试验等可以帮助鉴别假性正常化的形态和进行性左心室舒张功能障碍。

\subsubsection{如何应用肺静脉血流脉冲多普勒评估左心舒张功能?}

肺静脉血流脉冲多普勒是一种通过评估跨二尖瓣充盈来诊断心室舒张功能障碍的辅助手段。将脉冲多普勒放置在肺静脉入左心房开口的远心端,能得到收缩波S、舒张波D和心房波A。在心房收缩产生的心房逆转波大小和形态最有临床应用价值。跨二尖瓣时间与肺静脉A波时间的差值有助于预测左心室舒张末压。

\subsubsection{如何应用M型彩色多普勒测量血流加速度?}

舒张期通过二尖瓣血流的时空图与左心室舒张有关,而这个时空图就是血流加速度。将彩色多普勒取样窗放置在左心室流入道,再将M型取样线穿过此窗口即可获得血流加速度。将色彩基线调整至最大二尖瓣口流速的30%~40%,然后计算红蓝渐变斜率即可计算血流加速度。与跨二尖瓣口血流充盈评估相比,血流加速度一般不会出现假性正常化,当其<45cm/秒提示左室舒张功能障碍。该方法的主要局限性是可重复性不高。当跨二尖瓣血流充盈和肺静脉脉冲多普勒相结合在左室舒张功能不全的诊断中不明确时,多普勒组织成像在外侧二尖瓣环获得的E峰、A峰以及血流加速度等附加标准有助于鉴别舒张功能障碍的程度。

\subsubsection{如何进行左心室充盈压评估?}

肺动脉导管可以用来测量左心室充盈压。在没有任何远端梗阻情况下的肺小动脉嵌顿压近似于舒张末期左心室压力,在左心室顺应性正常的情况下,该压力可以间接反映左心室舒张末期容积,也就是左心室前负荷。而在高龄或者高血压患者中,左心室肥厚以及左心室顺应性降低较常见,导致舒张末期左心室压力与左心室舒张末期容积关系发生改变。此时,超声心动图检查有助于评估左心室舒张末期压力和舒张功能。常用的指标为左心室的被动跨二尖瓣充盈(E峰)和与之相对应的侧面二尖瓣环移位(E'峰)关系及比值,比值>15,提示左心室舒张末压>15mmHg;比值<8,提示左心室舒张末压<15mmHg。E'速度<5cm/秒则提示心室顺应性减低。

\subsubsection{如何对左心室容积进行半定量评估?}

通过压力测量来评估左心室容量状态是临床常用的方法。然而,对于部分特定的患者,特别是机械通气患者,压力与充盈容积的对应关系并不准确,因此,压力指标不能准确反映患者容量状态。而超声心动图中有很多方法评估左心室容积和压力,既可以单次使用,也可以重复应用以监测患者对补液的反应。因此,在临床的应用逐步得到推广。左心室具有对称性,有两个相对相等的短轴,而长轴从心底指向心尖。长轴方向心尖较圆钝,近心尖侧左心室为半椭圆形,而心底侧为圆柱形,所以在短轴切面呈圆形。因此,在测量和计算左心室容积时,可假设为M型超声或者二维切面时的形状。但使用这些参数来评估正常或者异常形状的左心室时仍需要谨慎分析。

左心室舒张末容积、左心室舒张末表面积、上腔静脉塌陷率、下腔静脉宽度、容量反应性等都可用来评估左心前负荷。低血容量的诊断指标包括舒张末直径<25mm、左心室腔收缩闭塞和左心室舒张末表面积<55cm\textsuperscript{2}
。在经食管超声心动图的经胃乳头肌短轴平面可以比较容易得出这些参数。存在基础心脏疾病或者左心室低顺应性的患者,左心室的压力容积关系都将改变,最适左心室舒张末表面积将比正常人的更大。这就突出了对于既定的左心室舒张末表面积与每搏输出量测量的匹配关系。相对于单次测量结果,连续测量左心室舒张末容积更加可靠,但非常耗时,同时在实践中很难实现。追踪容量状态变化能证实与左心室舒张末表面积测量的相关性,左心室舒张末表面积是通过追踪上述切面的左心室舒张末静态轮廓来计算的。此过程可以通过使用自动声学定量边界监测系统来简化。收缩末与舒张末的容积都应进行检测,随着时间的推移,还可以追踪容量状态的变化。收缩末心室腔闭塞或者叫“乳头肌亲吻征”是低血容量的征象,预测心室收缩末表面积减少的敏感性达100%,但特异性只有30%。

二尖瓣环(E')的组织多普勒成像与二尖瓣口E波血流模式相结合可以预测左心室顺应性和平均舒张压。E/E'比值<8表示心室顺应性良好,>15表示左心室平均充盈压高,顺应性低。中间值的评估还需要结合其他参数,比如肺静脉血液回流和二尖瓣流入减速时间。

\subsection{左心室功能评估的新技术}

\subsubsection{如何利用三维技术进行左心室功能评估?}

实时图像重建能获得左心室图像。当进行三维图像重建时,通过一个固定的传感器在3°或5°标准下可获得一系列的二维图像。平面的数量和二维图像的质量共同决定三维图像的质量。矩阵阵列传感器的发明使得多线图像同时用于重建一组超声数据。但对于左心室,需要将连续心动周期获得的数据组整合起来进行评估。

左心室容量和功能也能通过三维方法来计算。与MRI相比,该方法观察者之间的主观误差少,图像重建的假设成分少,因而能够更加准确评估左心室的前负荷和射血分数。随着图像分析时间的进一步减少以及更多先进科技的出现,三维未来将成为评估重症患者左室容积和功能的最好方法,但该方法也有一定的局限性:三维容积中的线条密度比二维图像低,所以经常需要填描;当图像是从垂直于很多器官的固定传感器得到的,那么结果会是质量欠佳的图像。另外,随着呼吸运动和心律失常会出现结果的伪像。

\subsection{右心功能的评估}

\subsubsection{如何评估右室收缩功能?}

因为右室缺乏特殊的形态,心脏超声很难定量评估右室功能。因此,在正常和疾病状态下,通常仅能对右室形态大小与功能进行定性评估。判断右室扩张程度、室间隔左向偏移及运动情况是定性评估右室功能常用的基本方法。近年来,有研究逐步探索定量评估右心大小及功能的指标和方法,这些指标包括面积变化分数、三尖瓣环位移、组织多普勒三尖瓣环心肌收缩速度和心肌做功指数。最近三维超声技术的发展将进一步有助于临床准确评估右室大小及功能。其他的复杂技术如应变与应变率等目前仅在有经验的实验室作为特殊临床或试验研究应用,尚未应用于临床。

\subsubsection{如何评估右室舒张功能?}

对于右室功能障碍的患者,应测定右室舒张功能。三尖瓣E/A比、E/E'比及右房大小,已被证明均是有效的指标。右室舒张功能的分级如下:三尖瓣E/A比<0.8,提示松弛不良;三尖瓣E/A比0.8~2.1、同时E/E'比>6或肝静脉舒张期流量显著,提示假性充盈;三尖瓣E/A比>2.1、结合减速时间<120毫秒提示限制性充盈。进一步的研究需要针对上述指标的敏感性及特异性进行探讨,并研究分级与患者预后间的关系。

\subsection{超声在感染性休克循环支持中的作用}

\subsubsection{感染性休克的血流动力学特点是什么?}

感染性休克是重症患者转入重症医学科的常见原因之一。感染性休克的分子病理生理学机制复杂,以外周血管阻力降低、有效循环血量减少和组织灌注不足为特征的血流动力学改变是其显著的临床特点,因此超声心动图在感染性休克患者的病情监测和床旁管理中逐步得到应用。感染性休克的病理生理学特点包括低血容量、左室收缩和舒张功能障碍、右室收缩功能障碍及外周血管麻痹。超声心动图使重症医学科医师能识别这些过程,监控其发展,并采取相应的治疗性干预。

\subsubsection{感染性休克的容量特点是什么?}

感染性休克患者的容量特点是有效循环血量不足。表现为绝对或相对低血容量。绝对低血容量是指总循环血量减少,常为感染性休克早期的表现,需要立即纠正,常见的原因包括:非显性丢失,如由于发热、出汗和过度通气经皮肤和呼吸道丢失所致;经胃肠道丢失,如腹泻和呕吐;经第三间隙丢失,如胰腺炎、烧伤、软组织损伤、血管渗漏、低胶体渗透压、腹水、胸水;液体摄入过少,如精神状态改变、身体虚弱、医院内液体复苏不足。

相对低血容量由血液在外周和中心腔室内异常分布所致。相对血容量不足在感染性休克中常见,并可在初步液体复苏后持续存在。这类患者总血容量可能正常,但血容量分布在中心腔室以外。血管扩张是由于外周血管收缩机制障碍和血管扩张机制的异常激活所致。

无论低血容量是绝对、相对还是混合性,导致的后果一致,均表现为组织氧供减少和组织缺氧。液体复苏通过增加静脉回流、前负荷、心输出量和动脉压(收缩压、平均压和脉压)来改善感染性休克的容量状态。识别并纠正低血容量状态是感染性休克治疗的一个重要目标。

\subsubsection{感染性休克时左室收缩功能障碍的特点是什么?}

感染性休克患者常出现心肌收缩障碍。实验和临床研究表明多种因素共同作用导致感染性休克产生心肌功能抑制,如心肌水肿、心肌细胞凋亡、细胞因子作用(尤其是白介素-1、白介素-6和肿瘤坏死因子-α)以及一氧化氮激活。虽然无冠脉灌注和心肌能量代谢异常,但肌钙蛋白水平升高却很常见。

由于传统的左室收缩功能的超声心动图参数受左室前后负荷的影响,因此,超声心动图识别左室收缩功能障碍很难。如心室前负荷降低而血管扩张导致的低血压患者射血分数可以正常。在容量复苏和使用血管加压药物调整合适的前后负荷前提下,再进行超声心动图检查才能真正显示心室收缩功能的改变。而前后负荷的进一步变化又可以改变超声心动图的结果。因此,射血分数正常并不能排除左室功能障碍。临床和实验研究显示,感染性休克发生早期出现可逆的左室功能抑制,表现为左室压力容积曲线右移,射血分数下降,警示临床医生可能需要控制后负荷和给予强心治疗。感染性休克中左室收缩功能障碍的改善与生存率变化的关系仍存在争议。Parker等的研究首先显示两者具有相关性;但Vieillard-Baron等进行的研究没有得出类似的结果。有假说认为感染性休克左室扩张与收缩功能受到抑制有关,是心脏为维持心输出量而做的适应性改变,该假说已被部分超声心动图检查所证实。

\subsubsection{感染性休克时左室舒张功能障碍特点是什么?}

感染性休克常伴有左室舒张功能障碍,并与死亡率增加相关。这主要与肌钙蛋白水平升高、细胞因子活性(肿瘤坏死因子-α、白介素-8、白介素-10)增加有关。舒张功能障碍常与收缩功能障碍同时发生,但约20%的病例单独出现。

\subsubsection{超声心动图在感染性休克管理中的应用特点是什么?}

有效循环血量降低在感染性休克患者中很常见,而早期足够的容量复苏与患者的预后显著相关。因此,临床治疗中不能因等待超声心动图检查而延迟液体复苏。入院前和急诊的临床评估有助于获得初步的信息来决定容量复苏的补液量。入住重症医学科后需要关注的问题是患者是否还需要进一步进行容量复苏、是否需要继续调整血管活性药物的使用。在这种情况下,超声心动图是评估容量状态和心功能的理想工具,有助于识别低血容量、评价左室收缩期和舒张期功能障碍和右室功能障碍。最初的评估结果有助于制定治疗计划,而后续治疗过程中的监测有助于评估治疗效果、疾病变迁并识别新问题的出现
\protect\hyperlink{text00009.htmlux5cux23ch9-8}{\textsuperscript{{[}9{]}}}
。

\subsubsection{超声心动图如何评估感染性休克患者的容量反应性?}

对感染性休克患者进行容量复苏是初始复苏的重要部分,但容量复苏过度则导致相反的后果。利用床旁超声心动图检查可评估容量状态和容量反应性,常选用动态容量指标来进行评价。

下腔静脉直径的呼吸变异是判断容量反应性的有效方法,但要求患者必须有机械通气支持并完全没有自主呼吸。此外,超声心动图显示感染性休克患者小的高动力左室(收缩末左室腔消失)或小的下腔静脉直径(<10mm)提示患者存在容量反应性。

具有高级重症超声心动图检测能力的重症医学医师能通过多种多普勒方法来了解感染性休克患者是否需要进一步容量复苏。对于无自主呼吸、窦性心律的机械通气患者,可用经食管超声心动图测得的上腔静脉直径的呼吸变异测定容量反应性,也可通过多普勒测得的每搏输出量的呼吸变异进行判断。对于有自主呼吸和心律不齐的患者,可采用被动抬腿前后用多普勒测量每搏输出量和心输出量判断容量反应性。

\subsubsection{超声心动图如何评估感染性休克患者的左室收缩功能?}

感染性休克早期,常出现左室收缩功能受损,且通常在感染性休克恢复后7~10天完全恢复。感染性休克患者血流动力学改变呈“高动力”状态的高排低阻表现。对心脏功能非容量依赖性指数的研究显示,即使心输出量和射血分数正常或升高,但患者仍表现为收缩功能损害。超声心动图检查结果易于将高动力的左室收缩误读为左室充盈不足和后负荷过低。进行容量复苏和血管活性药物治疗调整后负荷后,超声心动图检查可以确切显示左室收缩功能受损
\protect\hyperlink{text00009.htmlux5cux23ch10-8}{\textsuperscript{{[}10{]}}}
。

左室收缩功能的评价依赖于射血分数的测定。超声心动图检查可以通过多种方法测定射血分数值。M型超声依赖于左室直径的测量。Teichholz方法测量的技术要求较高,要求在心室中央水平和胸骨旁长轴测量左室的直径,M型探头与左室壁垂直,重症患者往往心脏难以朝向适合测量的方向,加上由机械运动周期导致的平移运动伪影和用直径测量来定义复杂的三维结构所导致的内在的几何假设,M型射血分数测量方法可能不是测量重症患者射血分数的可靠方法。另外,该方法不能用于有室间隔异常的患者,机械通气的重症患者是否有效尚未得到证实。另一种方法是面积测量法。在胸骨旁短轴的乳头肌水平(使用经食管超声心动图)测量舒张末和收缩末左室腔的面积。尽管在理论上该方法优于基于直径的测量方法,面积测量法仍然易受室间隔异常和平移运动伪影的影响。准确测定射血分数可以采用Simpson方法,通过2个直角平面的顶面观来测量左室舒张末和收缩末面积(顶面四腔和顶面二腔视图)。该方法测定费时、需要明确心内肌边界、较高的测量技术(如理想的轴线和避免平移运动伪影)以及高质量的设备。

射血分数测定有助于评价左室收缩功能,但不能反映每搏输出量和心输出量。低灌注高动力的左室可以表现为射血分数正常,而每搏输出量和心输出量可能不足。同样,扩张而收缩功能下降的左室射血分数虽低,每搏输出量和心输出量可能并不降低。因此,临床治疗中往往需要测量每搏输出量和心输出量,这需要使用多普勒进行测定。在经胸壁超声心动图心尖五腔切面或经食管超声心动图胃深部视图测量,多普勒探头的脉冲波置于左室流出道,超声波束与血流方向平行。主动脉收缩期血流速度曲线下面积与每搏输出量成正比。主动脉收缩期血流速度时间积分乘以左室流出道面积即得到每搏输出量和心输出量。射血分数反映左室收缩功能,而每搏输出量和心输出量反映氧输送。感染休克早期的检查可能显示射血分数显著下降。恢复期检查可显示左室功能完全正常,这为患者的临床管理提供了重要信息。如果没有再次检查,患者可能被视为有慢性左心衰,从而进行不恰当的长期治疗。

\subsubsection{超声心动图如何评估感染性休克的左室舒张功能?}

感染性休克患者常出现左室舒张功能异常。舒张功能的测定非常重要,有助于评估左室舒张压和左房压,评价左心室对容量的耐受性,以尽早采取有效的治疗手段防止左室舒张压升高导致肺动脉压升高和肺水肿。一旦发现左室舒张末期压力升高可以及时采取治疗性干预,如限制液体输注和利尿,以保证在改善组织灌注的同时降低肺水肿发生风险。

传统测量方法依赖于多普勒分析负荷依赖性的二尖瓣流入量,也可以通过非负荷依赖性方法测量二尖瓣环组织的纵向运动多普勒速度(E')。另外,多普勒超声心动图检查可通过多种方法评估肺动脉嵌顿压。采用多普勒脉冲在顶面四腔视图上测量跨二尖瓣舒张期流速,E/A>2与肺动脉嵌顿压力>18mmHg显著相关,其阳性预测值为100%;收缩期前向运动速度/收缩期和舒张期速度<45%提示肺动脉嵌顿压力>12mmHg,其阳性预测值为100%;肺静脉反向A波时间大于二尖瓣流入A波时间提示肺动脉嵌顿压力>15mmHg,阳性预测值为83%;二尖瓣环组织多普勒测量二尖瓣E波速度比E'(E/E')>9提示肺动脉嵌顿压力>15mmHg。

\subsubsection{超声心动图如何评估感染性休克的右室功能?}

感染病原菌、毒素、炎症介质、感染性休克并发症等同样可损害右室功能。急性肺损伤、缺氧性肺血管收缩和正压通气都可能增加右室后负荷而导致急性肺心病。超声心动图有助于识别急性肺心病,从而有利于采取措施降低右室后负荷,缓解右室扩张。

\subsubsection{如何利用超声心动图对血管外周阻力进行评估?}

心脏超声多普勒技术可以直接测量外周血管阻力,但不易方便和简单使用,因此在临床工作当中,较少应用超声心动图检查评价外周血管阻力,而常根据临床和心脏超声的检查结果进行除外诊断,如在心脏前负荷充足的同时左右心脏收缩功能均满意的情况下仍然存在低血压,提示外周血管阻力降低。

\subsubsection{超声心动图在感染性休克管理中的临床应用流程是什么?}

低血容量、有效循环血量降低导致组织灌注不足是感染性休克的最主要特点。除立即使用有效抗生素抗感染治疗,早期的治疗应给予足量的容量复苏合并使用血管活性药物以改善组织灌注。该治疗常在重症医学科外已开始执行。患者转入重症医学科后,医师需要进行评估并进一步制定治疗计划。初步超声心动图检查首先有助于排除其他或并存的导致休克的原因,如早期未发现的心包填塞、严重瓣膜疾病、室间隔异常、缺血性心肌病或肺栓塞;其次有助于进行血流动力学评估,以指导进一步容量管理和血管活性药物的调整。

超声心动图检查显示以下特征性的改变时,提示需要继续进行容量复苏:①显示下腔静脉直径小或高动力的左室、收缩末室腔消失;②没有自主呼吸的机械通气患者,下腔静脉直径或每搏输出量随呼吸发生显著的变异;③有自主呼吸的机械通气患者,测量的被动腿抬高试验变异度>12%。

超声心动图检查有助于评价左心功能,指导血管活性药物的使用和调整。感染性休克患者常合并左室收缩功能下降,但并不说明患者一定需要使用血管活性药物。通过超声心动图检查有助于判断患者是否需要应用正性肌力药物。最常用的方法为直接测量每搏输出量和心输出量。超声检查即使显示左室收缩功能降低,但如果每搏输出量和心输出量在正常范围,没有必要使用强心治疗;如果每搏输出量和心输出量降低以至氧供减少,则有使用正性肌力药物的指征。如果无法进行量化的每搏输出量和心输出量测量,需要综合临床表现来决定是否使用正性肌力药物。

超声心电图检查有助于识别患者有无急性肺心病。多种因素可导致感染性休克患者出现急性肺心病。如细菌毒素、炎症介质、不恰当的机械通气治疗等。右室扩张和室间隔运动障碍,对急性肺心病有重要诊断意义。急性肺心病的识别有助于临床医师及时采取有效措施降低右室后负荷。

\subsection{超声心动图与重症相关心肌梗死}

\subsubsection{超声如何早期发现重症相关心肌梗死?}

无论是围手术期还是严重创伤的重症患者,缺血性心脏病常见,心肌局部缺血导致局部心肌运动异常。临床实际中,超声检查评估局部心肌缺血最常用的方法是进行二维超声显像检查,目测室壁运动和室壁增厚率。与心肌节段的室壁增厚率相比,二维超声应变成像对心肌缺血的变化更加敏感。

心肌应变是指心肌各节段的变形,与心肌的收缩和舒张功能密切相关,因此超声检查心肌应变可用于评估心肌收缩和舒张功能。

随着彩色多普勒心脏超声在临床的广泛运用,使急性心肌梗死后心脏功能、包括左室舒张功能异常得到全面深入的认识,对临床治疗方案的制定也起到重要作用。急性心肌梗死后可出现左心室舒张功能异常,表现为二尖瓣血流频谱E峰峰值速度减低,A峰峰值速度增高,E/A比值<1,E峰减速时间延长,等容舒张时间延长,肺静脉血流频谱S/D峰值比值增加等。

\subsection{超声心动图与急性肺动脉栓塞}

\subsubsection{超声心动图如何早期发现急性肺动脉栓塞?}

急性肺血栓栓塞是临床上一种危重的心肺疾病,超声心动图检查对其病变程度、治疗效果及预后评估有重要作用,已经普遍应用于临床。超声检查急性肺血栓栓塞应心脏超声检查及下肢深静脉检查。心脏超声可以从直接征象及间接征象为诊断急性肺血栓栓塞提供重要辅助诊断依据,其中,直接征象包括肺动脉和左右肺动脉主干内血栓;右心内血栓伴右心扩大、肺动脉高压;血栓到达肺动脉以前,可被腔静脉入右房处的Eustachil瓣、三尖瓣或右心耳阻截,如果同时伴有右心室扩大或肺动脉高压,则可以直接诊断急性肺血栓栓塞。

心脏超声检测急性肺血栓栓塞的间接征象包括肺动脉高压及肺源性心脏病征象。具体表现在以下几方面:栓子栓塞肺动脉,受机械、神经反射和体液因素的综合影响,肺血管阻力升高,右心后负荷增大,导致右心系统扩大;右室壁运动幅度减低;室间隔与左室后壁运动不协调,在左室短轴切面,室间隔向左心室膨出,左心室呈“D”字形改变;由于右心扩大,导致三尖瓣瓣环扩大,可引起不同程度三尖瓣反流及肺动脉压力增高,频谱多普勒可以测得三尖瓣反流压差,并可据此估测肺动脉压力;此外,还可见多普勒改变、肺动脉血流流速曲线发生特征性改变,主要表现为加速、减速时间缩短及频谱形态发生改变,如果伴有肺动脉高压,则血流频谱表现为收缩早期突然加速,加速支陡直,峰值流速前移至收缩早期,而后提前减速,呈直角三角形改变,有时可于收缩晚期血流再次加速,出现第二个较低的峰。

心脏超声可通过上述直接征象来直接诊断急性肺血栓栓塞,但临床检查发现直接征象的概率往往较低,主要原因为:当肺栓塞栓子位于肺动脉外周血管时,往往难以检出;新鲜的血栓回声多较低,超声不易识别;而机化的血栓与血管壁融合,也不易区分。间接征象可以提示诊断,更重要的是对具有胸痛、呼吸困难、心悸、气短等症状的患者进行鉴别诊断,主要与冠心病、急性心肌梗死、主动脉夹层、心包积液等疾病鉴别。对于确诊的急性肺血栓栓塞患者,如超声探测到中度、重度右室功能障碍,则其近期及长期病死率明显升高,而不伴有右室负荷过重的患者,近期预后良好。可见,除辅助诊断外,心脏超声检查还能够根据右室功能状态进行疾病危险度分层及预后判断。由于心脏超声可以动态、无创、重复估测肺动脉压力,因此也是疗效判断、随访追踪的一种快速、简便的检查手段。

\subsection{肺部超声在循环监测与支持中的作用}

\subsubsection{常见的肺部超声征象包括哪些?}

最近几年来,随着肺部超声的进步与推广,超声检查成为肺部和胸腔疾病诊疗的重要手段。正常和疾病状态下肺部超声常见的特征性的表现有:①正常通气征象------胸膜线下平行排列的A线;②肺间质肺泡综合征------彗星尾征,根据B线的不同间隔分为B7线(B线间隔大约7mm,主要是肺小叶间隔增厚)和B3线(B线间隔3mm);③肺实变征------组织样征和碎片征,可见支气管气象;④胸腔积液征象------静态征象为四边形征,动态征象包括水母征和正弦波征;⑤气胸征象------平流征,超声诊断气胸的优势是快速、直接。

\subsubsection{如何认识肺部超声对血流动力学性肺水肿的评估作用?}

血流动力学性肺水肿患者通常需要进行肺水含量的评估。肺部超声检查获得的B线提示患者出现肺水肿,该表现往往出现在血气分析改变之前。另外,超声具有简单、无创、无放射性和实时性等特点,可以实时监测肺水肿的改变。例如,随着肺水肿的增加,由肺间质水肿发展为肺泡水肿,肺部超声检查的B线也相应发生变化
\protect\hyperlink{text00009.htmlux5cux23ch11-8}{\textsuperscript{{[}11{]}}}
\textsuperscript{,}
\protect\hyperlink{text00009.htmlux5cux23ch12-8}{\textsuperscript{{[}12{]}}}
。

\subsubsection{如何利用超声监测鉴别急性心源性(血流动力学性)肺水肿与急性呼吸窘迫综合征肺水肿?}

肺部超声监测导向诊断的难点在于鉴别急性心源性(血流动力学性)肺水肿和急性呼吸窘迫综合征肺水肿。最新有研究对比急性呼吸窘迫综合征与急性心源性(血流动力学性)肺水肿超声征象的不同。研究纳入7个征象:肺泡间质综合征、胸膜线异常征象、胸膜滑动征消失、存在未受损伤的区域、肺部实变、胸腔积液和肺搏动征。研究结果表明:由于两种疾病发病的病理生理机制不同,肺部超声表现也不同。心源性肺水肿时,超声肺彗星尾征的绝对数量与血管外肺水含量明显相关,甚至随着肺部含水量的增加从黑肺到黑白肺直至白肺发展;急性呼吸窘迫综合征时,早期CT能发现的所有特点包括肺部及胸腔改变均可由肺部超声检查发现,包括不均匀的含有未受损伤区域的肺部间质综合征、胸膜线异常改变及肺实变和胸腔积液等。可见肺部超声有助于床旁即时鉴别诊断急性呼吸窘迫综合征肺水肿与急性心源性(血流动力学性)肺水肿
\protect\hyperlink{text00009.htmlux5cux23ch13-8}{\textsuperscript{{[}13{]}}}
\textsuperscript{~}
\protect\hyperlink{text00009.htmlux5cux23ch15-8}{\textsuperscript{{[}15{]}}}
。

\subsubsection{肺部超声如何估测肺动脉嵌压?}

在循环支持的过程中,有研究表明,肺超的A-优势型表现提示肺动脉嵌压<13mmHg的可能性大,而在B-优势型时,提示肺动脉嵌压>18mmHg的可能性较大。

\subsection{重症肾脏超声在循环监测及休克支持中的作用}

\subsubsection{肾脏超声在休克循环监测中也具有重要作用吗?}

肾脏是休克时最容易受损或最早受损的器官之一,术后患者发生率达到1%,而在重症患者则达到35%,尤其感染性休克患者发生率在50%以上。因此肾功能的评估和急性肾损伤的早期诊断非常重要
\protect\hyperlink{text00009.htmlux5cux23ch16-8}{\textsuperscript{{[}16{]}}}
。重症肾脏超声能够床旁及时、无创监测肾脏大循环与微循环的改变,为休克循环监测提供新的诊断依据。

\subsubsection{在循环监测及休克支持中如何应用肾脏超声?}

近年来,应用超声多普勒技术监测肾脏阻力指数成为评估肾脏灌注的重要工具。过去的研究表明,肾脏阻力指数与疾病的进展明确相关,建议肾脏阻力指数用于监测肾脏移植后功能不全、尿路梗阻等。近年,由于超声监测肾脏阻力指数无创、简单、可重复性强,成为重症患者首选监测急性肾损伤发生发展的重要工具,尤其有益于调整休克的血流动力学策略。另外,由于超声造影技术的进展,使床旁定量实时监测大血管与微血管血流成为可能,尤其对于休克时肾脏灌注的变化,包括对于治疗干预的变化均有重要的监测价值。

重症超声是重症医学科中指导血流动力学监测和治疗的有效方法,它为重症医学提供了连续动态管理重症患者的重要床旁工具。

\begin{center}\rule{0.5\linewidth}{\linethickness}\end{center}

参考文献

\protect\hyperlink{text00009.htmlux5cux23ch1-8-back}{{[}1{]}} .Morris
C,Bennett S,Burn S,et al.Echocardiography in the intensive care
unit:current position,future directions.JICS,2010,11:90-97.

\protect\hyperlink{text00009.htmlux5cux23ch2-8-back}{{[}2{]}} .Danilo
T,Marcelo L,Tatiana M,et al.Echocardiography for hemodynamic
evaluation in the intensive care unit.Shock.2010,34S(1):59-62.

\protect\hyperlink{text00009.htmlux5cux23ch3-8-back}{{[}3{]}} .Price
S,Nicol E,Gibson DG,et al.Echocardiography in the critically
ill:current and potential roles.Intensive Care Med,2006,32:48-59.

\protect\hyperlink{text00009.htmlux5cux23ch4-8-back}{{[}4{]}} .Gerstle
J,Shahul S,Mahmood F.Echocardiographically derived parameters of
fluid responsiveness.Int Anesthesiol Clin.2010,48(1):37-44.

\protect\hyperlink{text00009.htmlux5cux23ch5-8-back}{{[}5{]}}
.Vieillard-Baron A,Caille V,Charron C,et al.The actual incidence of
global left ventricular hypokinesia in adult septic shock.Crit Care
Med,2008,36:1701-1706.

\protect\hyperlink{text00009.htmlux5cux23ch6-8-back}{{[}6{]}} .Price
S,Via G,Sloth E,et al.World Interactive Network Focused On Critical
UltraSound ECHO - ICU Group:Echocardiography practice training and
accreditation in the intensive care:document for the World Interactive
Network Focusedon Critical Ultrasound(WINFOCUS).Cardiovasc
Ultrasound,2008,6:49.

\protect\hyperlink{text00009.htmlux5cux23ch7-8-back}{{[}7{]}} .Vincent
Caille1,Jean-Bernard Amiel,Cyril Charron,et al.Echocardiography:a
help in the weaning process.Critical Care,2010,14:R120.

\protect\hyperlink{text00009.htmlux5cux23ch8-8-back}{{[}8{]}} .Salem
R,Vallee F,Rusca M,et al.Hemodynamic monitoring by echocardiography
in the ICU:the role of the new echo techniques.Current Opinionin
Critical Care,2008,14(5):561-568.

\protect\hyperlink{text00009.htmlux5cux23ch9-8-back}{{[}9{]}}
.王小亭,刘大为,张宏民,等.扩展的目标导向超声心动图方案对感染性休克患者的影响.中华医学杂志,2011,91(27):1879-1883.

\protect\hyperlink{text00009.htmlux5cux23ch10-8-back}{{[}10{]}}
.王小亭,刘大为.重视心脏多普勒超声在重症医学领域中的应用.中华内科杂志,2011,50(07).

\protect\hyperlink{text00009.htmlux5cux23ch11-8-back}{{[}11{]}}
.Bellani G,Mauri T,Pesenti A.Imaging in acute lung in jury and acute
respiratory distress syndrome.Curr Opin Crit
Care,2012,18(1):29-34.

\protect\hyperlink{text00009.htmlux5cux23ch12-8-back}{{[}12{]}} .Rajan
GR.Ultrasound lung comets:a clinically useful sign in acute
respiratory distress syndrome/acute lunginjury.Crit Care
Med,2007,35(12):2869-2870.

\protect\hyperlink{text00009.htmlux5cux23ch13-8-back}{{[}13{]}}
.Jambrik Z,Gargani L,Adamicza A,et al.B-lines quantify the lung
water content:a lung ultrasound versus lung gravimetry study in acute
lung injury.Ultrasound Med Biol,2010,36(12):2004-2010.

{[}14{]}.Copetti R,Soldati G,Copetti P.Chest sonography:a useful
tool to differentiate acute cardiogenic pulmonary edema from acute
respiratory distress syndrome.Cardiovasc
Ultrasound,2008,29(6):16.

\protect\hyperlink{text00009.htmlux5cux23ch15-8-back}{{[}15{]}}
.王小亭,刘大为.超声监测导向的ARDS诊断与治疗.重症医学年鉴,2012.

\protect\hyperlink{text00009.htmlux5cux23ch16-8-back}{{[}16{]}} .Le
Dorze M,Bouglé A,Deruddre S,et al.Renal Doppler Ultrasound:A New
Tool to Assess Renal Perfusion in Critical
Illness.Shock,2012,37(4):360-365.

\protect\hypertarget{text00010.html}{}{}

