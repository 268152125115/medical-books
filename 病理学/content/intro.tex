\mychapter{绪论}

\subsection*{一、病理学的概念和任务}

病理学(pathology)是研究疾病的病因、发病机制、病理变化(包括代谢、功能、形态结构)、转归、结局的一门医学基础学科。

疾病是在致病因子的作用下机体局部或全身所发生的代谢、功能和形态结构的变化。病理学的主要任务有二:一是应用科学的方法研究疾病发生、发展和转归的规律,从而阐明疾病的本质,为防治疾病提供理论依据;二是根据患病机体的病理形态学改变对疾病作出诊断。由此可见,病理学是医学中居于核心地位的学科。作为医生,要认识疾病防治疾病就必须有坚实的病理学基础。

\subsection*{二、病理学在医学中的作用}

病理学在医学教学体系中居于核心地位。人们把病理学形象地比喻为基础医学和临床医学之间的桥梁,是因为在医学教学体系中,解剖学、组织胚胎学、生理学、生物化学、病原生物学和免疫学等基础医学课程是让医学生了解和掌握正常人体形态结构、代谢、功能及其调节机制,而病理学的教学目的是引导医学生用上述基础知识来辨别患病机体所出现的各种病理现象并掌握其发展规律,为后续临床学科(主要阐述疾病的诊断、治疗和预防)的学习打下基础,其桥梁作用就体现在这里。因此,一名医学生只有很好地掌握病理学基础知识,才能学好临床各学科的课程。医学生毕业后诊治疾病的能力高低,也与病理知识掌握的好坏密切相关,只有那些对疾病的病因、发病机制、病理变化等有深刻理解、融会贯通的医学生才能成为高水平的临床医生。

病理学在临床诊疗中发挥至关重要的作用。尽管临床影像学、生化检验技术的发展突飞猛进,大大提高了疾病诊断的时效性和准确率,但活体组织的病理诊断仍然是临床诊断的金标准,而通过尸体解剖可对死因作出准确回答。尤其是分子病理诊断技术的进步,一大批肿瘤标志物、病原标志物的发现和检测为临床靶向治疗提供了重要靶点,从而推动了精准医学的发展。

病理学在疾病的研究中扮演重要角色。这主要体现在三个方面:一是病理工作者不仅从事教学和医疗工作,而且通过细胞培养、动物试验对疾病的发生发展机制和防治进行研究;二是从日常工作中积累的大量组织标本着手,开展旨在提高临床诊疗水平的系列研究工作;三是借助病理形态学知识和技术为其他学科开展的研究工作提供帮助。

\subsection*{三、如何学好病理学}

要学好病理学应该注意以下几个方面:首先必须有正常人体形态结构、代谢、功能及其调节机制的知识基础。其次是了解教科书的内容编排和学习要领。一般来说,病理学均包括总论和各论两部分内容,本书也不例外。总论阐述细胞和组织的适应、损伤与修复,局部血液循环障碍,炎症和肿瘤等基本病理变化,也就是疾病的普遍规律(共性)。各论,如循环系统疾病、消化系统疾病等章节则介绍各系统常见疾病的原因、发病机制、病变及其发生、发展的特殊规律(个性)。学好总论是学习各论的必要基础,学习各论必须联系、运用总论的知识,两者之间有着密切的内在联系。第三,学会运用病理知识去解释临床现象,可以起到巩固病理知识的作用。第四,及时地复习和总结也很重要。另外,病理学网络教学资源很丰富,在条件许可情况下,应该积极利用。

\subsection*{四、病理学的研究方法}

病理学的研究方法很多,有些是经典的,还有些是近30年发展起来的,概括起来,主要有以下几种:

{(一)尸体剖验}

尸体剖验(简称尸检或尸解,autopsy)可以直接观察疾病的病理改变,进行详细的组织学检查,结合临床资料明确诊断,查明死因,总结经验,提高临床医疗水平。此为病理学基本研究方法之一。通过大量尸体剖验资料的积累,不仅可以研究疾病发生、发展的规律,而且还能及时发现和确诊某些新的传染病、地方病、流行病,为防治疾病提供依据。尸体剖验积累的标本,也为培养医务工作者提供了大量有价值的资料。由于人们观念陈旧以及相关法规不健全,所以我国尸检率很低,对病理学及医学的发展极为不利。

{(二)活组织和细胞学检查}

从患者活体采取组织进行病理检查,以确定诊断,称为活组织检查(简称活检,biopsy)。这是临床广泛开展的病理检查方法。目前采取病变组织的方法有钳取、切取、穿刺、局部切除等。这些方法的优点是组织新鲜,不仅可供常规病理诊断,而且可用于各种细胞化学、组织化学、超微结构以及分子病理学研究。在临床上,活检对判断病变性质、确定治疗方案有重要意义。对性质不明的肿瘤,还可在手术过程中,切取病变组织作冰冻切片或快速石蜡切片,迅速确定肿瘤的良性、恶性,决定手术范围,这对肿瘤病人的治疗和预后尤为重要。

细胞学(cytology)检查又称脱落细胞学,是指采集病变处脱落细胞或细针吸取的细胞,涂片染色后进行诊断。其具有方法简便、创伤小、可重复等优点,适于对人群进行大规模普查。但由于没有组织结构、细胞常有变性,所以易出现假阴性结果,有时需结合其他检查结果综合判断。

{(三)动物试验}

在适宜动物体内复制某些人类疾病的模型,以了解该疾病或某一病理过程的发生、发展。

这种方法也可用以研究疾病的病因、发病机制及进行药物治疗试验,观察疗效及药物不良反应。但要注意,动物和人类之间存在种系差异,不能将动物实验的结果直接套用于人类。

{(四)组织培养与细胞培养}

将人体或动物的某种组织或细胞在体外培养,以观察组织或细胞病变的发生、发展,也可观察药物等外来因子对培养细胞的影响。这种方法周期短、见效快,但要注意体内和体外有差异,孤立的体外环境缺少体内存在的整体环境中众多因素间的相互影响,因此不能将体外研究结果与体内病变过程等同对待。

{(五)其他技术在病理研究中的应用}

除上述四种基本研究方法外,随着生物学和相关学科的发展,近数十年特别是近二十年来高新技术也相继进入病理领域,目前在病理研究中应用较多的有组织和细胞化学、免疫组织化学、电子显微镜技术。①组织和细胞化学:组织和细胞化学法是应用某些化学试剂,在组织及细胞上进行特异性化学反应,呈现出特异的颜色,从而了解和鉴定组织细胞中的各种蛋白质、脂类、糖、酶和核酸等化学成分的状况。②免疫组织化学法:是应用酶标抗体(或抗原)和相应的抗原(或抗体)接触,形成特异性抗原抗体复合物,催化底物后,可呈现颜色变化,在原位检测组织细胞内的抗原或抗体的技术。组织细胞中凡是能作为抗原或半抗原的物质,如蛋白质、多肽、氨基酸、多糖、磷脂、受体、酶、激素及病原体等都可用相应的特异性抗体进行检测。③电子显微镜技术(简称电镜):应用透射电镜或扫描电镜对细胞内部和表面的超微结构进行更细微的观察,即从亚细胞(细胞器)水平上认识和了解细胞的病变。除上述三种常用方法外,进入病理领域的其他高新技术还有流式细胞仪技术、图像分析技术、分子原位杂交、聚合酶链反应(PCR)、共聚焦显微镜技术、组织芯片技术、二代测序技术等,使病理学对疾病的研究从定性进入定量,从细胞水平进入分子水平,并使形态结构和代谢、机能的研究联系起来,其结果不仅加深了对疾病的理解和认识,又推动了病理学发展。

\subsection*{五、病理学发展简史}

在我国,早在公元前700年,《黄帝内经》中就有以阴阳、脏腑和经络之间功能失调作为疾病病因的论述,这是源于当时阴阳五行(金、木、水、火、土)的哲学思想,这种思想延续至今仍然是中医诊治疾病的理论基础。隋唐时代巢元方所著的《诸病源候论》对疾病的病因和征候的记载十分详细,可以说他是我国古代第一个病理学家。但是由于中西医研究疾病的角度不同,所以祖国医学中的病理学和现代病理学分属不同的理论体系,前者与古希腊名医Hippocrates建立的体液病理学相似。

现代病理学的建立源于尸体解剖,其发展与人们对疾病的认识息息相关,特别是与基础医学学科的发展和技术进步有密切联系,主要分为三个阶段:

初期即器官病理学阶段。这一阶段是建立在尸解的基础上的。该时期最有影响的代表人物是意大利名医Morgagni(1682---1771)。他根据700例尸解肉眼观察材料,结合临床资料,对照分析,著成《疾病的位置与原因》一书,提出了疾病的器官定位的观点,为病理学的发展奠定了基础。此后Rokitansky在掌握了大量尸解资料的基础上,于1843年完成了《病理解剖学》巨著,丰富了器官病理学的内容,但病理学向广度和深度发展主要得益于其他科学技术的发展。

中期即细胞病理学阶段。19世纪中叶,德国病理学家Rudolf
Virchow(1821-1902)在进行大量尸检的同时,借助显微镜对尸检材料进行观察研究,于1858年出版了著名的《细胞病理学》一书,提出了细胞形态和功能的变化是疾病的基础的观点,使病理学从器官的模糊阶段进入细胞的微观水平。他对近代病理学的发展作出了卓越的贡献,而且也为所有医学基础学科的建立和发展奠定了基础。

繁荣阶段即现代病理学阶段。20世纪40年代末期电子显微镜应用于医学生物领域,60年代开展免疫组织化学的研究,70年代分子生物学崛起,研究方法和技术日趋进步,使病理学取得突破性的进展,不仅极大丰富了细胞病理学的内容,而且使病理学从经典的形态学范畴进入亚分子和分子水平。随着研究内容的拓宽与深入,在病理学范畴内又出现了新的病理学科分支。从临床医学上分出了外科病理学、妇产科病理学、儿科病理学、神经病理学、皮肤病理学、眼科病理学、耳鼻喉病理学。随着边缘学科的兴起及研究方法的互相渗透,又出现了超微病理学、免疫病理学、实验病理学、定量病理学、遗传病理学及分子病理学等。

精准病理学阶段已经到来。为了在治疗过程中尽可能减少对正常组织的损伤,近年提出了精准医学的概念,介入治疗、靶向治疗即属此范畴。要真正实现精准治疗,必须有可靠靶标。病理科开展乳腺癌激素受体、癌基因HER-2/neu表达的检测,以及其他肿瘤的标志物检测,即可提供上述靶标,从而指导临床采用内分泌治疗、单克隆抗体靶向治疗等。可以说,病理学已进入了一个崭新的发展阶段。

20世纪初,现代病理学传入我国,经过几代病理学家的艰苦努力,造就了一大批优秀的病理学人才,积累了具有我国疾病特点的病理资料,编写了富有特色的病理教材。当前摆在新一代病理工作者面前的任务是不仅要继承老一辈病理学家的研究方法,还要将生命科学研究中的新方法、新技术用于病理学的研究中,承前启后,继往开来,为赶上世界先进水平、发展我国病理事业作出贡献。

\authorinfo{(陈平圣)}